\ifthenelse{\boolean{papier}}{\clearemptydoublepage}{}
\chapter{Problématique}
\section{Question envisagée en rapport avec la profession}\label{question}
La démarche d’investigation (cf. figure p.~\pageref{demarcheSci} de l’annexe~\ref{questionnaireWeb}), utilisée par les chercheurs, a été adaptée pour l’enseignement des sciences à l’école élémentaire et pour les années postérieures. La littérature nous montre l’existence de certaines dérives. En effectuant une enquête des pratiques enseignantes à partir d’un questionnaire, nous allons essayer faire le point sur la pratique enseignante lors de la mise en œuvre de la démarche d’investigation en maternelle. La question qui se pose est : \og Dans quelle mesure l’enseignement des sciences par la démarche d’investigation est-il envisageable en maternelle ? \fg{} 

Gauthier et al. (1986, cité dans \citeA[p.~23]{Long2004}) considèrent qu’\og un problème de recherche est considéré comme étant un écart ou un manque à combler dans le domaine de nos connaissances entre ce que nous savons et ce que nous devrions ou désirons savoir sur le réel. \fg{} Nous sommes conscients qu’il ne s’agit pas d’une question de recherche, puisqu’il n’y a pas d’\og écart \fg{}, il s’agit plus d’un état des lieux!% qui devrait permettre justement à mesurer cet écart. Ce qui pourrait déboucher sur une question de recherche.

\section{Réponses possibles}
À la question \og Dans quelle mesure l’enseignement des sciences par la démarche d’investigation est-il envisageable en maternelle ? \fg{} plusieurs réponses sont possibles :
\begin{itemize}
\item Les sciences ne sont pas enseignées.
\item Les sciences sont enseignées, mais pas en utilisant la démarche d’investigation.
\item Les sciences sont enseignées, en utilisant la démarche d’investigation.
\end{itemize}

Dans ces deux derniers cas, nous allons regarder plus précisément la démarche mise en place et comparer la position des enseignants aux différentes assertions que nous avons trouvées dans la littérature. Notamment l’affirmation de \citeA[p.~4]{Sanchez2012} : \og à de rares exceptions près n’importe quel concept scientifique peut être abordé avec chaque enfant, à n’importe quel âge, à partir de situations adaptées \fg{}, ainsi que la dérive du \og tout méthodologique \fg{} et la dérive du \og tout technologique \fg{} misent en avant par \citeA{Sarmant1999}. Nous allons également comparer les heures d’enseignement des sciences aux heures qui sont censées être enseignées au cycle~2 et voir comment la démarche d’investigation est adaptée.

De la même manière que nous avons remarqué à la section \ref{question} qu’il ne s’agissait pas d’une question de recherche, nous remarquerons qu’ici nous ne sommes pas en présence d’une série d’hypothèses envisageables, mais simplement d’un choix de réponses possibles !