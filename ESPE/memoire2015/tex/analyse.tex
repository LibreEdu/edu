\ifthenelse{\boolean{papier}}{\clearemptydoublepage}{}
\chapter{Analyse}
\section{Description du questionnaire}
Le questionnaire contient 21~questions obligatoires à choix multiples et 3~zones facultatives de texte libre. Avant d’analyser le questionnaire, voici une description succincte des 12~pages du questionnaire :
\begin{description}
\item[Page 1 : La démarche d’investigation en maternelle] Présentation du questionnaire et de son objectif.
\item[Page 2 : Population cible] Question vérifiant que la personne appartient bien à la population ciblée.
\item[Page 3 : Mise au point] Lien vers deux images détaillant la démarche d’investigation.
\item[Page 4 : Cadre] Questions par rapport à des variables qui peuvent influencer l’ensemble du questionnaire (nombre d’années d’expérience\dots)
\item[Page 5 : Enseignement des sciences] Questions sur la position de l’enseignant par rapport aux sciences. Pour les enseignants qui sont dans une classe où les sciences ne sont pas enseignées, le questionnaire se poursuit à la page~9  et pour les enseignants qui sont déchargés de l’enseignement des sciences le questionnaire se poursuit à la page~11.
\item[Page 6 : Démarche] Première série de questions sur les démarches mises en œuvre lors de l’enseignement des sciences. Pour les enseignants qui n’utilisent jamais la démarche d’investigation, le questionnaire se poursuit à la page~10 et il se poursuit à la page~11 pour ceux et celles qui ne répondent pas à la question sur la fréquence d’utilisation de la démarche d’investigation.
\item[Page 7 : Démarche d’investigation] Questions sur la démarche d’investigation.
\item[Page 8 : Activités de recherche] Quelles sont les différentes activités mises en place lors de la phase d’investigation ?
\item[Page 9] Zone facultative de texte libre pour permettre d’expliquer pourquoi les sciences ne sont pas enseignées.
\item[Page 10] Zone facultative de texte libre pour permettre d’expliquer pourquoi la démarche d’investigation n’est pas utilisée.
\item[Page 11] Zone facultative de texte libre afin de pouvoir ajouter des commentaires.
\item[Page 12] Page de remerciement.
\end{description}

\section{Présentation du questionnaire et de son objectif}
Le texte de cette première page du questionnaire est : \og Dans le cadre du master Métiers de l’enseignement, de l’éducation et de la formation, je rédige un mémoire professionnel ayant pour objectif de mieux connaitre la pratique enseignante par rapport à la démarche d’investigation. Ce questionnaire contient au maximum 21~questions, majoritairement fermées, ainsi qu’une zone finale de commentaires. Merci de votre participation. \fg{} L’objectif du questionnaire devait être clair afin que les enseignants interrogés puissent être dans une démarche réflexive.

\section{Population cible}
Le texte de cette première question est : \og Êtes-vous enseignante ou enseignant en maternelle dans une école appliquant les programmes du ministère français de l’Éducation nationale ? (Que cela soit en France ou à l’étranger.) \fg{} Le but était d’inclure non seulement les départements et régions d’outre-mer et collectivités d’outre-mer, mais également les écoles membres de l’Agence pour l’enseignement français à l’étranger. Soixante-quinze personnes ont répondu au questionnaire, dont sept n’appartenant pas à la population ciblée. L’effectif pris en compte pour le reste du questionnaire sera donc de 68~enseignants.

\section{Mise au point}
Lors du passage du premier questionnaire dont le titre était \og La démarche d’investigation en maternelle \fg{}, une des enseignantes s’est demandé ce que c’était la démarche d’investigation. Par la suite, lorsqu’elle a vu la figure sur la démarche d’investigation (page~4 du questionnaire de l’annexe~\ref{questionnairePapier}) elle a associé la démarche d’investigation à la démarche scientifique. Afin que le concept de démarche d’investigation soit le plus explicite, j’ai décidé de mettre un lien vers deux images représentant la démarche d’investigation (cf. figures du questionnaire de l’annexe~\ref{questionnaireWeb}) : une représentation avec des pictogrammes exploitable en maternelle et un schéma montrant la différence entre la démarche d’investigation et la démarche scientifique.

\section{Cadre}
La deuxième question (\og Depuis combien d’années enseignez-vous en maternelle ? \fg{}) permet de connaitre l’expérience de l’enseignant (Moins de 5~ans, Entre 5 et 10~ans, Plus de 10~ans). Cette variable sera utilisée lors de l’analyse d’autres questions. La moitié (34/68) des personnes qui ont répondu à cette question ont moins de 5~années d’expérience, ce qui peut s’expliquer par le fait que la promotion de ce questionnaire a été faite entre autres dans des groupes de professeurs des écoles stagiaires.

La troisième question concerne la durée moyenne hebdomadaire d’enseignement des sciences en maternelle que nous allons comparer au nombre d’heures d’enseignement censées être consacrées à la \og Découverte du monde \fg{} qui est en moyenne 2~h~15 par semaine selon les horaires de l’école élémentaire d’après le Bulletin officiel de 2008. Une très grande majorité des enseignants de maternelle (46/60) enseigne les sciences moins de deux heures par semaines, par contre il y a deux fois plus d’enseignants (30/46) qui font entre une et deux heures que moins d’une heure (16/46). Parmi ces seize enseignants, cinq n’enseignent pas les sciences : pour trois d’entre eux c’est quelqu’un d’autre qui s’en charge et pour les deux autres font partie de classe où les sciences ne sont pas enseignées, soit en tant qu’enseignant ou enseignante unique soit en tant que modulateur ou modulatrice.

La question suivante (\og Enseignez-vous également au cycle~2 ou au cycle~3 ? \fg{}) était destinée à voir s’il y avait une différence entre ceux et celles qui n’enseignent quand maternelle et ceux et celles qui enseignent sur plusieurs cycles. Cinquante-huit des enseignants n’exercent pas aux cycles~2 et 3. À première vue, il n’y aurait pas point commun pour les neuf autres personnes.

En ce qui concerne le nombre de classes enseignées (\og À combien de classes enseignez-vous en maternelle ? \fg{}), il s’agit surtout d’enseignants qui ont en charge la classe à plein temps (28/68) ou à mi-temps (20/68). En comparant cette variable avec d’autres critères, il n’y rien de significatif.

La dernière question (\og En quelle section enseignez-vous ? \fg{}) de cette section devait être une question à choix multiples où les enseignants devaient de prendre en compte pour cette question ainsi que pour le reste du questionnaire, seulement la classe ou le niveau où votre enseignement des sciences est le plus significatif. Seulement un peu plus de la moitié des classes (35/68) sont des classes à un seul niveau.

\section{Enseignement des sciences}
La première question de cette section est une question d’autoévaluation des enseignants par rapport à leur engagement dans l’enseignement des sciences. Une très forte majorité (52/68) se considère comme une personne engagée ou très engagée. Par contre, six enseignants ne se considèrent pas du tout engagés, deux d’entre eux font partie de classes où les sciences ne sont pas enseignées, et pour les quatre autres c’est quelqu’un d’autre qui s’en charge.

Ensuite, j’ai voulu tester leur position par rapport à l’affirmation \og À de rares exceptions près n’importe quel concept scientifique peut être abordé avec chaque enfant, à n’importe quel âge, à partir de situations adaptées \fg{} qu’on trouve dans \textit{Sciences à vivre : 23 séquences pour découvrir le monde du vivant et de la terre avec des enfants du cycle 1} de \citeA{Sanchez2012}. Les enseignants sont d’accord à plus de 80~\% avec cette assertion, et plus précisément 49~\% (33/68) sont plutôt d’accord et 32~\% (22/68) sont totalement d’accord.

Le but de la question suivante (\og Il est possible d’enseigner les sciences en maternelle en utilisant la démarche d’investigation. \fg{}) était de connaitre la prédisposition des enseignants par rapport à la démarche d’investigation. Il y plus de personnes qui sont en accord avec cette affirmation qu’avec l’affirmation précédente, 54~\% qui sont totalement d’accord et 38~\% qui sont plutôt d’accord.

La section suivant est destinée seulement à ceux qui enseignent les sciences, d’où l’intérêt de cette question (\og Enseignez-vous les sciences dans votre classe de maternelle ? \fg{}). Soixante personnes ont répondu par l’affirmative, ce qui sera l’effectif de référence pour la partie suivante du questionnaire.

\section{Démarche d’investigation}
La première série de questions, page~6 du questionnaire, ne fait pas spécifiquement référence à la démarche d’investigation, ce qui n’est pas le cas des deux questions de la page~7 du questionnaire.

La question \og Faites-vous des apports théoriques avant les expériences ? \fg{} a pour objectif d’être comparée à \og Adoptez-vous la démarche d’investigation lors de vos séances de sciences ? \fg{} L’idée de l’enseignement par la démarche d’investigation est d’apporter un apport théorique par l’investigation. Y a-t-il une relation entre les réponses entre ces deux questions, dont les réponses sont retranscrites à la table~\ref{theorie}.
\begin{table}[h!btp]
\centering
\caption{\label{theorie} Apports théoriques / Démarche d’investigation}
	\begin{tabular}{|c|c|c|c|c|}
	\cline{2-5}
	\multicolumn{1}{c|}{} & \multicolumn{4}{c|}{Adoptez-vous la démarche d’investigation?} \\ 
	\hline 
	\begin{minipage}[l]{3.7cm}\begin{flushleft}Faites-vous des apports théoriques avant les expériences ?\end{flushleft}\end{minipage} & Jamais & \begin{minipage}[c]{2cm}Occasion- nellement\end{minipage} & Souvent & Toujours \\ 
	\hline 
	Jamais & 1 & 5 & 9 & 7 \\ 
	\hline 
	Occasionnellement & 1 & 4 & 17 & 6 \\ 
	\hline 
	Souvent & 0 & 1 & 3 & 3 \\ 
	\hline 
	Toujours & 0 & 0 & 0 & 3 \\ 
	\hline 
	\end{tabular}
\end{table}
Le test du $\chi^2$ permet de savoir si les deux variables sont indépendantes. Pour ces données, nous avons $\chi^2=9,8$ qui est inférieur au seuil de référence $\chi^2_{9;0,05}=16,9$. Les deux variables sont indépendantes.

La question suivante est : \og Faites-vous des expériences sans questionnement initial ? \fg{} On a les données de la table~\ref{questionnement}.
\begin{table}[h!btp]
\centering
\caption{\label{questionnement} Questionnement initial / Démarche d’investigation}
	\begin{tabular}{|c|c|c|c|c|}
	\cline{2-5}
	\multicolumn{1}{c|}{} & \multicolumn{4}{c|}{Adoptez-vous la démarche d’investigation?} \\ 
	\hline 
	\begin{minipage}[l]{3.7cm}\begin{flushleft}Faites-vous des expériences sans questionnement initial ?\end{flushleft}\end{minipage} & Jamais & \begin{minipage}[c]{2cm}Occasion- nellement\end{minipage} & Souvent & Toujours \\ 
	\hline 
	Jamais & 0 & 2 & 12 & 13 \\ 
	\hline 
	Occasionnellement & 1 & 6 & 12 & 4 \\ 
	\hline 
	Souvent & 1 & 2 & 5 & 1 \\ 
	\hline 
	Toujours & 0 & 0 & 0 & 1 \\ 
	\hline 
	\end{tabular}
\end{table}
$\chi^2=12,8$, ce qui reste inférieur à $\chi^2_{9;0,05}$. Les deux variables sont indépendantes.

La question suivante, \og Faites-vous des activités exclusivement technologiques, c’est-à-dire de réaliser un objet sans autre problématique. \fg{}, correspond à la dérive du \og tout technologique \fg{} observée par \citeA{Sarmant1999}. Et on obtient les données  de la table~\ref{technologiques}.
\begin{table}[h!btp]
\centering
\caption{\label{technologiques} Activités technologiques / Démarche d’investigation}
	\begin{tabular}{|c|c|c|c|c|}
	\cline{2-5}
	\multicolumn{1}{c|}{} & \multicolumn{4}{c|}{Adoptez-vous la démarche d’investigation?} \\ 
	\hline 
	\begin{minipage}[l]{3.7cm}\begin{flushleft}Faites-vous des activités exclusivement technologiques ?\end{flushleft}\end{minipage} & Jamais & \begin{minipage}[c]{2cm}Occasion- nellement\end{minipage} & Souvent & Toujours \\ 
	\hline 
	Jamais & 0 & 3 & 11 & 9 \\ 
	\hline 
	Occasionnellement & 1 & 6 & 14 & 9 \\ 
	\hline 
	Souvent & 1 & 0 & 2 & 0 \\ 
	\hline 
	Toujours & 0 & 0 & 0 & 1 \\ 
	\hline 
	\end{tabular}
\end{table}
Cette fois-ci $\chi^2=12,9$, ce qui est également inférieur à $\chi^2_{9;0,05}$. Par conséquent, les deux variables sont indépendantes. On remarquera que 38~\% des enseignants (23/60) ne font jamais des activités exclusivement technologiques, et 50~\% (30/60) le font occasionnellement.

La dernière question de la page~6 (\og Adoptez-vous la démarche d’investigation lors de vos séances de sciences ? \fg{} ) est un filtre. Les questions suivantes font spécifiquement référence à cette démarche. Quarante-huit personnes sur  soixante qui enseignent les sciences déclarent l’utiliser souvent (29/60) ou toujours (19/60), ce qui représente 80~\% de ceux qui enseignent la science. Deux personnes déclarent ne jamais l’utiliser, l’effectif pour les questions suivantes ne sera plus de soixante, mais de cinquante-huit.

La première question spécifique à la démarche d’investigation concerne la deuxième dérive signalée par \citeA{Sarmant1999}, la dérive du \og tout méthodologique \fg{}. Cinquante-trois pour cent (31/58) des enseignants qui enseignent les sciences en maternelle privilégient souvent la démarche d’investigation à l’acquisition des connaissances et 19~\% (11/58) la privilégient toujours.

Une autre question concernant la démarche d’investigation était sur l’adaptation de la démarche d’investigation. L’intitulé de la question était : \og Adaptez-vous la démarche d’investigation en formulant vous-même la question problématique plutôt que de construire progressivement en classe le problème à partir d’un étonnement, d’une curiosité ou d’un questionnement ? \fg{} La majorité, soit 53~\% (31/58) le font occasionnellement et 26~\% (15/58) souvent.

La dernière série de questions concerne les différents types d’investigation pour les activités de recherche (cf. fig.~\ref{demarcheInvestigation} p.~\pageref{demarcheInvestigation}). Il fallait compléter la phrase suivante : \og Lorsque vous utilisez la démarche d’investigation, les activités de recherche réalisées par les élèves s’appuient sur\dots{} \fg{} Si l’on comptabilise les individus qui ont répondu par \og Toujours \fg{} ou \og Souvent \fg{}, on a :
\begin{itemize}
\item 67~\% (39/58) pour divers essais dont les résultats sont comparés (tâtonnement expérimental) ;
\item 55~\% (32/58) pour une observation directe ou assistée par un instrument (qui ne soit pas l’ordinateur) ou sur l’exploitation de documents (images, données, résultats d’expériences) ;
\item 47~\% (27/58) pour un dispositif où un seul facteur varie et où les résultats sont recueillis par l’observation ou la mesure (expérimentation directe ;
\item 36~\% (21/58) pour une réalisation matérielle (construction d’un objet, d’un modèle, recherche d’une solution technique) ;
\item 22~\% (13/58) pour la lecture de documents papier ou électroniques ou par l’interview de personnes compétentes (recherche documentaire).
\end{itemize}