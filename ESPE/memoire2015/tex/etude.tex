\ifthenelse{\boolean{papier}}{\clearemptydoublepage}{}
\chapter{Étude}
\section{Descriptif du dispositif}
Lors de la première partie du stage qui s’est déroulé d’octobre à décembre~2014, j’ai pu mettre en œuvre une partie d’une séquence d’enseignement en sciences en petite et moyenne section de maternelle. Cela m’a permis de me rendre compte de ce qu’il est possible et impossible de faire. Afin d’avoir une représentation plus précise de ce qu’il est possible de faire, j’ai préféré orienter mon étude à la pratique enseignante d’une manière générale, d’où l’idée du questionnaire. L’étude sera faite à partir d’un questionnaire qui permet de valider différentes affirmations exposées dans la section~\ref{cadre}, \og Cadre de l’étude \fg{} (p.~\pageref{cadre}).

\section{Mode de recueil de données}
L’étude se déroulera en deux étapes. La première étape correspond au questionnaire papier (cf. annexe~\ref{questionnairePapier}, p.~\pageref{questionnairePapier}) qui a été distribué à l’école maternelle Jean-Henri Fabre d’Avignon le 23~avril 2015 et qui a été rempli par six enseignantes. Cette étape m’a permis de tester le questionnaire et de modifier certaine question, principalement la question~4 : \og Enseignez-vous seulement en maternelle ? \fg{} qui a été remplacée par \og Enseignez-vous également au cycle~2 ou au cycle~3 ? \fg{} En effet, un professeur des écoles maitre formateur qui enseigne en maternelle et à l’université répondra \og non \fg{} à la première question, alors que l’objectif de la question était de savoir s’il enseigne à d’autres classes du primaire.

La deuxième étape correspond à un questionnaire web qui a été rempli par 67~enseignantes ou enseignants de maternelle du 25~avril au 4~mai 2015. C’est ce deuxième questionnaire qui se trouve à l’annexe~\ref{questionnaireWeb} (p.~\pageref{questionnaireWeb}) que nous allons analyser.

\section{Mode de traitement des données recueillies}
Les données sont récupérées dans une feuille de calcul exploitable par un tableur. Il s’agit essentiellement de questions fermées et de variables qualitatives. Il existe des logiciels d'analyse statistique. Malheureusement, par manque de connaissance et de temps, la possibilité d’utiliser PSPP une alternative libre à SPSS ne sera pas abordée. Le traitement des données recueillies prendra la forme d’une simple analyse statistique ainsi que de tableaux croisés.