\ifthenelse{\boolean{papier}}{\clearemptydoublepage}{}
\ifthenelse{\boolean{caco}}{%
	\mbox{}\newpage
	\ifodd\thepage ~\newpage~\fi
	}{\chapter*{Résumé}}
Le programme d’enseignement de l’école primaire spécifie la démarche d’investigation pour l’enseignement des sciences au cycle~3, le site internet du Ministère de l’Éducation nationale, de l’Enseignement supérieur et de la Recherche préconise une initiation à la démarche d’investigation dès la maternelle, et le socle commun de connaissances et de compétences qui se retrouve dans le code de l’éducation fait référence à l’opération \og La main à la pâte \fg{} pour l’enseignement des sciences. D'après notre enquête, l’enseignement des sciences se fait effectivement dès la maternelle en utilisant cette démarche, et le nombre d’heures consacrées est proche du nombre d’heures consacré aux sciences au cycle~2. Néanmoins les dérives soulignées par le \textit{Rapport sur l’opération \og La main à la pâte \fg{} et l’enseignement des sciences à l’école primaire} sont toujours d’actualité. Il s’agit de la dérive du \og tout méthodologique \fg{} où l’utilisation de la démarche est privilégiée à l’acquisition des connaissances, ainsi que la dérive du \og tout technologique \fg{} où le seul but est de construire un objet sans que cela réponde à une problématique.

\begin{description}
\item[Mots-clés:] Éducation, sciences, maternelle, démarche d’investigation, La main à la pâte
\end{description}
\vfill
\ifthenelse{\boolean{papier}}{\clearemptydoublepage}{}
\ifthenelse{\boolean{caco}}{}{\chapter*{Abstract}}
\selectlanguage{english}
The curriculum of primary school specify the investigative approach for cycle~III (8–10 years of age), the website of the Ministry of National Education, Higher Education and Research recommends an introduction to the investigative approach in pre-school (3–5 years of age), and the Common Base of Knowledge and Skills that is found in the Education Code refers to the ``Hands on'' operation for teaching Science. According to our survey, science teaching in pre-school is actually using this approach, and the number of hours spent is close to the number of hours devoted to science in cycle II (6–7 years of age). Nevertheless drifts highlighted by the \textit{Report on the ``Hands on'' operation and science education in primary school} are still relevant. It is the drift of ``all methodological'' where the use of the approach is preferred to the acquisition of knowledge and the drift of ``all technological'' where the only goal is to construct an object that is not consistent with a problem.

\begin{description}
\item[Keywords:] Education, science, pre-school, investigation approach, Hands on
\end{description}
\selectlanguage{french}
\vfill