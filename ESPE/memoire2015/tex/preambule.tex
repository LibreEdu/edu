\usepackage[utf8]{inputenc}
\usepackage[T1]{fontenc}
\usepackage{fourier}
\usepackage[svgnames]{xcolor}
\usepackage{pdfpages}
\usepackage{graphicx}
\usepackage{shorttoc}
\usepackage{apacite}
\usepackage{scrtime}%heure système
\usepackage{setspace}%interlignne
\usepackage{shapepar}
\usepackage{multirow}
\usepackage{amsmath}
\usepackage{microtype}%meilleur réglage espaces typo
\usepackage{ifthen}
%\usepackage[left,modulo]{lineno}
\usepackage{scrextend}%addmargin
\usepackage[english,french]{babel}
\graphicspath{{images/}}

%\usepackage{showframe}
%\usepackage{layout}
%\usepackage{lipsum}

\onehalfspacing

\newcommand{\corps}{12}

\newcommand{\clearemptydoublepage}{%
\newpage{\pagestyle{empty}\cleardoublepage}}

\newboolean{caco}
\newboolean{papier}

\newenvironment{caco}{%
	\begin{addmargin}[4em]{0em}
	\begin{singlespace}
%	\linenumbers*
	\vspace{.5em}
}{%
	\vspace{.5em}
	\end{singlespace}
	\end{addmargin}
}

\hypersetup{
    pdftitle={La démarche d’investigation en maternelle},
    pdfauthor={Pachot, Alexandre},
    pdfsubject={L’enseignement des sciences en maternelle},
    pdfkeywords={Éducation, sciences, maternelle, démarche d’investigation, La main à la pâte}
}