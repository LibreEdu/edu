\section{Mobiliser le langage dans toutes ses dimensions}
\begin{enumerate}
\item Communiquer avec les adultes et avec les autres enfants par le langage, en se faisant comprendre.
\item S’exprimer dans un langage syntaxiquement correct et précis. Reformuler pour se faire mieux comprendre.
\item Pratiquer divers usages du langage oral : raconter, décrire, évoquer, expliquer, questionner, proposer des solutions, discuter un point de vue.
\item Dire de mémoire et de manière expressive plusieurs comptines et poésies.
\item Comprendre des textes écrits sans autre aide que le langage entendu.
\item Manifester de la curiosité par rapport à l’écrit. Pouvoir redire les mots d’une phrase écrite après sa lecture par l’adulte, les mots du titre connu d’un livre ou d’un texte.
\item Participer verbalement à la production d’un écrit. Savoir qu’on n’écrit pas comme on parle.
\item Repérer des régularités dans la langue à l’oral en français (éventuellement dans une autre langue). 
\item Manipuler des syllabes.
\item Discriminer des sons (syllabes, sons-voyelles ; quelques sons-consonnes hors des consonnes occlusives).
\item Reconnaitre les lettres de l’alphabet et connaitre les correspondances entre les trois manières de les écrire : cursive, script, capitales d’imprimerie. Copier à l’aide d’un clavier.
\item Écrire son prénom en écriture cursive, sans modèle.
\item Écrire seul un mot en utilisant des lettres ou groupes de lettres empruntés aux mots connus.
\end{enumerate}

\section{Agir, s’exprimer, comprendre à travers l’activité physique}
\begin{enumerate}
\item Courir, sauter, lancer de différentes façons, dans des espaces et avec des matériels variés, dans un but précis. 
\item Ajuster et enchainer ses actions et ses déplacements en fonction d’obstacles à franchir ou de la trajectoire d’objets sur lesquels agir.
\item Se déplacer avec aisance dans des environnements variés, naturels ou aménagés. 
\item Construire et conserver une séquence d’actions et de déplacements, en relation avec d’autres partenaires, avec ou sans support musical.
\item Coordonner ses gestes et ses déplacements avec ceux des autres, lors de rondes et jeux chantés.
\item Coopérer, exercer des rôles différents complémentaires, s’opposer, élaborer des stratégies pour viser un but ou un effet commun. 
\end{enumerate}

\section{Agir, s’exprimer, comprendre à travers les activités artistiques}
\begin{enumerate}
\item Choisir différents outils, médiums, supports en fonction d’un projet ou d’une consigne et les utiliser en adaptant son geste.
\item Pratiquer le dessin pour représenter ou illustrer, en étant fidèle au réel ou à un modèle, ou en inventant. 
\item Réaliser une composition personnelle en reproduisant des graphismes. Créer des graphismes nouveaux.
\item Réaliser des compositions plastiques, seul ou en petit groupe, en choisissant et combinant des matériaux, en réinvestissant des techniques et des procédés.
\item Avoir mémorisé un répertoire varié de comptines et de chansons et les interpréter de manière expressive. 
\item Jouer avec sa voix pour explorer des variantes de timbre, d’intensité, de hauteur, de nuance. 
\item Repérer et reproduire, corporellement ou avec des instruments, des formules rythmiques simples. 
\item Décrire une image, parler d’un extrait musical et exprimer son ressenti ou sa compréhension en utilisant un vocabulaire adapté.
\item Proposer des solutions dans des situations de projet, de création, de résolution de problèmes, avec son corps, sa voix ou des objets sonores.
\end{enumerate}

\section{Construire les premiers outils pour structurer sa pensée}
\subsection{Découvrir les nombres et leurs utilisations}
\subsubsection{Utiliser les nombres}
\begin{enumerate}
\item Évaluer et comparer des collections d’objets avec des procédures numériques ou non numériques.
\item Réaliser une collection dont le cardinal est donné. Utiliser le dénombrement pour comparer deux quantités, pour constituer une collection d’une taille donnée ou pour réaliser une collection de quantité égale à la collection proposée.
\item Utiliser le nombre pour exprimer la position d’un objet ou d’une personne dans un jeu, dans une situation organisée, sur un rang ou pour comparer des positions.
\item Mobiliser des symboles analogiques, verbaux ou écrits, conventionnels ou non conventionnels pour communiquer des informations orales et écrites sur une quantité. 
\end{enumerate}

\subsubsection{Étudier les nombres}
\begin{enumerate}
\item Avoir compris que le cardinal ne change pas si on modifie la disposition spatiale ou la nature des éléments.
\item Avoir compris que tout nombre s’obtient en ajoutant un au nombre précédent et que cela correspond à l’ajout d’une unité à la quantité précédente.
\item Quantifier des collections jusqu’à dix au moins ; les composer et les décomposer par manipulations effectives puis mentales. Dire combien il faut ajouter ou enlever pour obtenir des quantités ne dépassant pas dix.
\item Parler des nombres à l’aide de leur décomposition. 
\item Dire la suite des nombres jusqu’à trente. Lire les nombres écrits en chiffres jusqu’à dix.
\end{enumerate}

\subsection{Explorer des formes, des grandeurs, des suites organisées}
\begin{enumerate}
\item Classer des objets en fonction de caractéristiques liées à leur forme. Savoir nommer quelques formes planes (carré, triangle, cercle ou disque, rectangle) et reconnaitre quelques solides (cube, pyramide, boule, cylindre).
\item Classer ou ranger des objets selon un critère de longueur ou de masse ou de contenance.
\item Reproduire un assemblage à partir d’un modèle (puzzle, pavage, assemblage de solides).
\item Reproduire, dessiner des formes planes.
\item Identifier le principe d’organisation d’un algorithme et poursuivre son application. 
\end{enumerate}

\section{Explorer le monde}
\subsection{Se repérer dans le temps et l’espace}
\begin{enumerate}
\item Situer des évènements vécus les uns par rapport aux autres et en les repérant dans la journée, la semaine, le mois ou une saison. 
\item Ordonner une suite de photographies ou d’images, pour rendre compte d’une situation vécue ou d’un récit fictif entendu, en marquant de manière exacte succession et simultanéité.
\item Utiliser des marqueurs temporels adaptés (puis, pendant, avant, après\dots) dans des récits, descriptions ou explications.
\item Situer des objets par rapport à soi, entre eux, par rapport à des objets repères.
\item Se situer par rapport à d’autres, par rapport à des objets repères.
\item Dans un environnement bien connu, réaliser un trajet, un parcours à partir de sa représentation (dessin ou codage). 
\item Élaborer des premiers essais de représentation plane, communicables (construction d’un code commun).
\item Orienter et utiliser correctement une feuille de papier, un livre ou un autre support d’écrit, en fonction de consignes, d’un but ou d’un projet précis.
\item Utiliser des marqueurs spatiaux adaptés (devant, derrière, droite, gauche, dessus, dessous\dots) dans des récits, descriptions ou explications.
\end{enumerate}

\subsection{Explorer le monde du vivant, des objets et de la matière}
\begin{enumerate}
\item Reconnaitre les principales étapes du développement d'un animal ou d'un végétal, dans une situation d’observation du réel ou sur une image.
\item Connaitre les besoins essentiels de quelques animaux et végétaux.
\item Situer et nommer les différentes parties du corps humain, sur soi ou sur une représentation.
\item Connaitre et mettre en œuvre quelques règles d'hygiène corporelle et d’une vie saine.
\item Choisir, utiliser et savoir désigner des outils et des matériaux adaptés à une situation, à des actions techniques spécifiques (plier, couper, coller, assembler, actionner\dots).
\item Réaliser des constructions ; construire des maquettes simples en fonction de plans ou d’instructions de montage. 
\item Utiliser des objets numériques : appareil photo, tablette, ordinateur.
\item Prendre en compte les risques de l'environnement familier proche (objets et comportements dangereux, produits toxiques).
\end{enumerate}