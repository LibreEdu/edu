\part{L’école maternelle : un cycle unique, fondamental pour la réussite de tous}
La loi de refondation de l’École crée un cycle unique pour l’école maternelle et souligne sa place fondamentale comme première étape pour garantir la réussite de tous les élèves au sein d’une école juste pour tous et exigeante pour chacun. Ce temps de scolarité, bien que non obligatoire, établit les fondements éducatifs et pédagogiques sur lesquels s’appuient et se développent les futurs apprentissages des élèves pour l’ensemble de leur scolarité. 

L’école maternelle est une école bienveillante, plus encore que les étapes ultérieures du parcours scolaire. Sa mission principale est de donner envie aux enfants d’aller à l’école pour apprendre, affirmer et épanouir leur personnalité. Elle s’appuie sur un principe fondamental : tous les enfants sont capables d'apprendre et de progresser. En manifestant sa confiance à l’égard de chaque enfant, l’école maternelle l’engage à avoir confiance dans son propre pouvoir d’agir et de penser, dans sa capacité à apprendre et réussir sa scolarité et au-delà.

\chapter{Une école qui s’adapte aux jeunes enfants}
L’enfant qui entre pour la première fois à l’école maternelle possède déjà des savoir-faire, des connaissances et des représentations du monde ; dans sa famille et dans les divers lieux d’accueil qu’il a fréquentés, il a développé des habitudes, réalisé des expériences et des apprentissages que l’école prend en compte.

\section{Une école qui accueille les enfants et leurs parents}
Dès l’accueil de l’enfant à l’école, un dialogue régulier et constructif s’établit entre enseignants et parents ; il exige de la confiance et une information réciproques. Pour cela, l’équipe enseignante définit des modalités de relations avec les parents, dans le souci du bien-être et d’une première scolarisation réussie des enfants et en portant attention à la diversité des familles. Ces relations permettent aux parents de comprendre le fonctionnement et les spécificités de l’école maternelle (la place du langage, le rôle du jeu, l’importance des activités physiques et artistiques\dots). 

L’expérience de la séparation entre l’enfant et sa famille requiert l’attention de toute l’équipe éducative, particulièrement lors de la première année de scolarisation. L’accueil quotidien dans la salle de classe est un moyen de sécuriser l’enfant. L’enseignant reconnait en chaque enfant une personne en devenir et un interlocuteur à part entière, quel que soit son âge.

\section{Une école qui accompagne les transitions vécues par les enfants}
L’école maternelle construit des passerelles au quotidien entre la famille et l'école, le temps scolaire et le temps périscolaire. Elle joue aussi un rôle pivot à travers les relations qu’elle établit avec les institutions de la petite enfance et avec l'école élémentaire.

L’équipe pédagogique organise la vie de l’école en concertation avec d’autres personnels, en particulier les Atsem (agents territoriaux spécialisés des écoles maternelles). L’articulation entre le temps scolaire, la restauration et les moments où l’enfant est pris en charge dans le cadre d'accueils périscolaires doit être travaillée avec tous les acteurs concernés de manière à favoriser le bien-être des enfants et constituer une continuité éducative. Tout en gardant ses spécificités, l’école maternelle assure les meilleures relations possibles avec les différents lieux d’accueil et d’éducation au cours de la journée, de la semaine et de l’année. Elle établit des relations avec des partenaires extérieurs à l’école, notamment dans le cadre des projets éducatifs territoriaux.

Elle travaille en concertation avec l’école élémentaire, plus particulièrement avec le cycle 2, pour mettre en œuvre une véritable continuité des apprentissages, un suivi individuel des enfants. Elle s’appuie sur le Rased (réseau d'aides spécialisées aux élèves en difficulté) pour comprendre des comportements ou une absence de progrès, et mieux aider les enfants dans ces situations. 

\section{Une école qui tient compte du développement de l’enfant}
Sur toute la durée de l’école maternelle, les progrès de la socialisation, du langage, de la motricité et des capacités cognitives liés à la maturation ainsi qu’aux stimulations des situations scolaires sont considérables et se réalisent selon des rythmes très variables. 

Au sein d’une même classe, l‘enseignant prend en compte dans la perspective d'un objectif commun les différences entre enfants qui peuvent se manifester avec une importance particulière dans les premières années de leur vie. L’équipe pédagogique aménage l'école (les salles de classe, les salles spécialisées, les espaces extérieurs\dots) afin d’offrir aux enfants un univers qui stimule leur curiosité, répond à leurs besoins notamment de jeu, de mouvement, de repos et de découvertes et multiplie les occasions d'expériences sensorielles, motrices, relationnelles, cognitives en sécurité. Chaque enseignant détermine une organisation du temps adaptée à leur âge et veille à l’alternance de moments plus ou moins exigeants au plan de l’implication corporelle et cognitive.

L’accueil, les récréations, l’accompagnement des moments de repos, de sieste, d’hygiène sont des temps d’éducation à part entière. Ils sont organisés dans cette perspective par les adultes qui en ont la responsabilité et qui donnent des repères sécurisants aux jeunes enfants.

\section{Une école qui pratique une évaluation positive}
L’évaluation constitue un outil de régulation dans l’activité professionnelle des enseignants ; elle n’est pas un instrument de prédiction ni de sélection. Elle repose sur une observation attentive et une interprétation de ce que chaque enfant dit ou fait. Chaque enseignant s’attache à mettre en valeur, au-delà du résultat obtenu, le cheminement de l’enfant et les progrès qu’il fait par rapport à lui-même. Il permet à chacun d’identifier ses réussites, d’en garder des traces, de percevoir leur évolution. Il est attentif à ce que l’enfant peut faire seul, avec son soutien (ce que l’enfant réalise alors anticipe souvent sur ce qu’il fera seul dans un avenir proche) ou avec celui des autres enfants. Il tient compte des différences d’âge et de maturité au sein d’une même classe.

Adaptée aux spécificités de l’école maternelle, l’évaluation est mise en œuvre selon des modalités définies au sein de l’école. Les enseignants rendent explicites pour les parents les démarches, les attendus et les modalités d'évaluation propres à l'école maternelle. 

\chapter{Une école qui organise des modalités spécifiques d’apprentissage}
Au sein de chaque école maternelle, les enseignants travaillent en équipe afin de définir une progressivité des enseignements sur le cycle. Ils construisent des ressources et des outils communs afin de faire vivre aux enfants cette progressivité. Ils constituent un répertoire commun de pratiques, d’objets et de matériels (matériels didactiques, jouets, livres, jeux) pour proposer au fil du cycle un choix de situations et d’univers culturels à la fois variés et cohérents.

L’enseignant met en place dans sa classe des situations d’apprentissage variées : jeu, résolution de problèmes, entrainements, etc. et les choisit selon les besoins du groupe classe et ceux de chaque enfant. Dans tous les cas et notamment avec les petits, il donne une place importante à l’observation et à l’imitation des autres enfants et des adultes. Il favorise les interactions entre enfants et crée les conditions d’une attention partagée, la prise en compte du point de vue de l’autre en visant l’insertion dans une communauté d’apprentissage. Il développe leur capacité à interagir à travers des projets, pour réaliser des productions adaptées à leurs possibilités. Il sait utiliser les supports numériques qui, comme les autres supports, ont leur place à l’école maternelle à condition que les objectifs et leurs modalités d’usage soient mis au service d’une activité d’apprentissage. Dans tous les cas, les situations inscrites dans un vécu commun sont préférables aux exercices formels proposés sous forme de fiches.

\section{Apprendre en jouant}
Le jeu favorise la richesse des expériences vécues par les enfants dans l'ensemble des classes de l’école maternelle et alimente tous les domaines d’apprentissages. Il permet aux enfants d’exercer leur autonomie, d‘agir sur le réel, de construire des fictions et de développer leur imaginaire, d’exercer des conduites motrices, d’expérimenter des règles et des rôles sociaux variés. Il favorise la communication avec les autres et la construction de liens forts d’amitié. Il revêt diverses formes : jeux symboliques, jeux d’exploration, jeux de construction et de manipulation, jeux collectifs et jeux de société, jeux fabriqués et inventés, etc. L’enseignant donne à tous les enfants un temps suffisant pour déployer leur activité de jeu. Il les observe dans leur jeu libre afin de mieux les connaitre. Il propose aussi des jeux structurés visant explicitement des apprentissages spécifiques. 

\section{Apprendre en réfléchissant et en résolvant des problèmes}
Pour provoquer la réflexion des enfants, l’enseignant les met face à des problèmes à leur portée. Quels que soient le domaine d’apprentissage et le moment de vie de classe, il cible des situations, pose des questions ouvertes pour lesquelles les enfants n’ont pas alors de réponse directement disponible. Mentalement, ils recoupent des situations, ils font appel à leurs connaissances, ils font l’inventaire de possibles, ils sélectionnent. Ils tâtonnent et font des essais de réponse. L’enseignant est attentif aux cheminements qui se manifestent par le langage ou en action ; il valorise les essais et suscite des discussions. Ces activités cognitives de haut niveau sont fondamentales pour donner aux enfants l’envie d’apprendre et les rendre autonomes intellectuellement.

\section{Apprendre en s’exerçant}
Les apprentissages des jeunes enfants s’inscrivent dans un temps long et leurs progrès sont rarement linéaires. Ils nécessitent souvent un temps d’appropriation qui peut passer soit par la reprise de processus connus, soit par de nouvelles situations. Leur stabilisation nécessite de nombreuses répétitions dans des conditions variées. Les modalités d’apprentissage peuvent aller, pour les enfants les plus grands, jusqu’à des situations d’entrainement ou d’auto-entrainement, voire d’automatisation. L’enseignant veille alors à expliquer aux enfants ce qu’ils sont en train d’apprendre, à leur faire comprendre le sens des efforts demandés et à leur faire percevoir les progrès réalisés. Dans tous les cas, les choix pédagogiques prennent en compte les acquis des enfants.

\section{Apprendre en se remémorant et en mémorisant}
Les opérations mentales de mémorisation chez les jeunes enfants ne sont pas volontaires. Chez les plus jeunes, elles dépendent de l’aspect émotionnel des situations et du vécu d’évènements répétitifs qu’un adulte a nommés et commentés. Ces enfants s’appuient fortement sur ce qu’ils perçoivent visuellement pour maintenir des informations en mémoire temporaire, alors qu’à partir de cinq-six ans c’est le langage qui leur a été adressé qui leur permet de comprendre et de retenir. 

L’enseignant stabilise les informations, s’attache à ce qu’elles soient claires pour permettre aux enfants de se les remémorer. Il organise des retours réguliers sur les découvertes et acquisitions antérieures pour s’assurer de leur stabilisation, et ceci dans tous les domaines. Engager la classe dans l’activité est l’occasion d’un rappel de connaissances antérieures sur lesquelles s'appuyer, de mises en relations avec des situations différentes déjà rencontrées ou de problèmes similaires posés au groupe. L’enseignant anime des moments qui ont clairement la fonction de faire apprendre, notamment avec des comptines, des chansons ou des poèmes. Il valorise la restitution, l’évocation de ce qui a été mémorisé ; il aide les enfants à prendre conscience qu’apprendre à l’école, c’est remobiliser en permanence les acquis antérieurs pour aller plus loin.       

\chapter{Une école où les enfants vont apprendre ensemble et vivre ensemble}
L’école maternelle structure les apprentissages autour d’un enjeu de formation central pour les enfants : \og Apprendre ensemble et vivre ensemble \fg{}. La classe et le groupe constituent une communauté d’apprentissage qui établit les bases de la construction d’une citoyenneté respectueuse des règles de la laïcité et ouverte sur la pluralité des cultures dans le monde. C’est dans ce cadre que l’enfant est appelé à devenir élève, de manière très progressive sur l’ensemble du cycle. Les enfants apprennent à repérer les rôles des différents adultes, la fonction des différents espaces dans la classe, dans l’école et les règles qui s’y rattachent. Ils sont consultés sur certaines décisions les concernant et découvrent ainsi les fondements du débat collectif. L’école maternelle assure ainsi une première acquisition des principes de la vie en société. L’accueil et la scolarisation des enfants handicapés participent à cet enjeu pour ces enfants eux-mêmes et contribuent à développer pour tous un regard positif sur les différences. L’ensemble des adultes veille à ce que tous les enfants bénéficient en toutes circonstances d'un traitement équitable. L’école maternelle construit les conditions de l’égalité, notamment entre les filles et les garçons. 

\section{Comprendre la fonction de l’école}
L’école maternelle est le lieu où l’enfant se familiarise progressivement avec une manière d’apprendre spécifique ; celle-ci s’appuie sur des activités, des expériences à sa portée, mais suppose qu’il en tire des connaissances ou des savoir-faire avec l’aide des autres enfants et de l’enseignant. Le langage, dans la diversité de ses usages, a une place importante dans ce processus. L’enfant apprend en même temps à entrer dans un rythme collectif (faire quelque chose ou être attentif en même temps que les autres, prendre en compte des consignes collectives) qui l’oblige à renoncer à ses désirs immédiats. L’école maternelle initie ainsi la construction progressive d’une posture d’élève. 

L’enseignant rend lisibles les exigences de la situation scolaire par des mises en situations et des explications qui permettent aux enfants -- et à leurs parents -- de les identifier et de se les approprier. Il incite à coopérer, à s’engager dans l’effort, à persévérer grâce à ses encouragements et à l’aide des pairs. Il encourage à développer des essais personnels, prendre des initiatives, apprendre progressivement à faire des choix.

Il aide à identifier les objets sur lesquels portent les apprentissages, fait acquérir des habitudes de travail qui vont évoluer au fil du temps et que les enfants pourront transférer. Pour ce faire, il s’attache à faire percevoir la continuité entre les situations d’apprentissage, les liens entre les différentes séances. Pour stabiliser les premiers repères, il utilise des procédés identiques dans ses manières de questionner le groupe, de faire expliciter par les enfants l’activité qui va être la leur, d’amener à reformuler ce qui a été dit, de produire eux-mêmes des explications pour d’autres à propos d’une tâche déjà vécue. 

L’enseignant exerce les enfants à l’identification des différentes étapes de l’apprentissage en utilisant des termes adaptés à leur âge. Il les aide à se représenter ce qu’ils vont devoir faire, avec quels outils et selon quels procédés. Il définit des critères de réussite pour que chacun puisse situer le chemin qu’il a réalisé et perçoive les progrès qu’il doit encore effectuer.

\section{Se construire comme personne singulière au sein d’un groupe}
Se construire comme personne singulière, c'est découvrir le rôle du groupe dans ses propres cheminements, participer à la réalisation de projets communs, apprendre à coopérer. C'est progressivement partager des tâches et prendre des initiatives et des responsabilités au sein du groupe. Par sa participation, l'enfant acquiert le gout des activités collectives, prend du plaisir à échanger et à confronter son point de vue à celui des autres. Il apprend les règles de la communication et de l’échange. L’enseignant a le souci de guider la réflexion collective pour que chacun puisse élargir sa propre manière de voir ou de penser. Ainsi, l’enfant trouve sa place dans le groupe, se fait reconnaitre comme une personne à part entière et éprouve le rôle des autres dans la construction des apprentissages.

Dans un premier temps, les règles collectives sont données et justifiées par l'enseignant qui signifie à l'enfant les droits (s’exprimer, jouer, apprendre, faire des erreurs, être aidé et protégé\dots) et les obligations dans la collectivité scolaire (attendre son tour, partager les objets, ranger, respecter le matériel\dots). Leur appropriation passe par la répétition d'activités rituelles et une première réflexion sur leur application. Progressivement, les enfants sont conduits à participer à une élaboration collective de règles de vie adaptées à l’environnement local. 

À travers les situations concrètes de la vie de la classe, une première sensibilité aux expériences morales (sentiment d’empathie, expression du juste et de l’injuste, questionnement des stéréotypes\dots) se construit. Les histoires lues, contes et saynètes y contribuent ; la mise en scène de personnages fictifs suscite des possibilités diversifiées d’identification et assure en même temps une mise à distance suffisante. Au fil du cycle, l’enseignant développe la capacité des enfants à identifier, exprimer verbalement leurs émotions et leurs sentiments. Il est attentif à ce que tous puissent développer leur estime de soi, s’entraider et partager avec les autres.

\part{Les cinq domaines d’apprentissage}
%\setcounter{chapter}{0}
Les enseignements sont organisés en cinq domaines d’apprentissage. Cette organisation permet à l’enseignant d’identifier les apprentissages visés et de mettre en œuvre leurs interactions dans la classe. Chacun de ces cinq domaines est essentiel au développement de l’enfant et doit trouver sa place dans l’organisation du temps quotidien. Dans la mesure où toute situation pédagogique reste, du point de vue de l’enfant, une situation riche de multiples possibilités d’interprétations et d’actions, elle relève souvent pour l’enseignant de plusieurs domaines d’apprentissage.

Le domaine \og Mobiliser le langage dans toutes ses dimensions \fg{} réaffirme la place primordiale du langage à l’école maternelle comme condition essentielle de la réussite de toutes et de tous. La stimulation et la structuration du langage oral d’une part, l’entrée progressive dans la culture de l’écrit d’autre part, constituent des priorités de l’école maternelle et concernent l’ensemble des domaines. 

Les domaines  \og Agir, s’exprimer, comprendre à travers l’activité physique \fg{} ; \og Agir, s’exprimer, comprendre à travers les activités artistiques \fg{} permettent de développer les interactions entre l’action, les sensations, l’imaginaire, la sensibilité et la pensée. 

Les domaines \og Construire les premiers outils pour structurer sa pensée \fg{} et \og Explorer le monde \fg{} s’attachent à développer une première compréhension de l’environnement des enfants et à susciter leur questionnement. En s’appuyant sur des connaissances initiales liées à leur vécu, l’école maternelle met en place un parcours qui leur permet d’ordonner le monde qui les entoure, d’accéder à des représentations usuelles et à des savoirs que l’école élémentaire enrichira. 

Le programme établit, pour chaque domaine d’apprentissage, une définition générale, énonce les objectifs visés et donne des indications pédagogiques de nature à fournir des repères pour organiser la progressivité des apprentissages. 

\chapter{Mobiliser le langage dans toutes ses dimensions}
Le mot \og langage \fg{} désigne un ensemble d’activités mises en œuvre par un individu lorsqu’il parle, écoute, réfléchit, essaie de comprendre et, progressivement, lit et écrit. L’école maternelle permet à tous les enfants de mettre en œuvre ces activités en mobilisant simultanément les deux composantes du langage :
\begin{itemize}
\item le langage oral : utilisé dans les interactions, en production et en réception, il permet aux enfants de communiquer, de comprendre, d’apprendre et de réfléchir. C’est le moyen de découvrir les caractéristiques de la langue française et d’écouter d’autres langues parlées.
\item le langage écrit : présenté aux enfants progressivement jusqu’à ce qu’ils commencent à l’utiliser, il les habitue à une forme de communication dont ils découvriront les spécificités et le rôle pour garder trace, réfléchir, anticiper, s’adresser à un destinataire absent. Il prépare les enfants à l’apprentissage de l’écrire-lire au cycle 2.
\end{itemize}

\section{L’oral}
L’enfant, quelle que soit sa langue maternelle, dès sa toute petite enfance et au cours d’un long processus, acquiert spontanément le langage grâce à ses interactions avec les adultes de son entourage. 

L’enseignant, attentif, accompagne chaque enfant dans ses premiers essais, reprenant ses productions orales pour lui apporter des mots ou des structures de phrase plus adaptés qui l’aident à progresser. L’enseignant s’adresse aux enfants les plus jeunes avec un débit ralenti de parole ; il produit des énoncés brefs, syntaxiquement corrects et soigneusement articulés. Constamment attentif à son propre langage et veillant à s’adapter à la diversité des performances langagières des enfants, il s’exprime progressivement de manière plus complexe. Il permet à chacun d’aller progressivement au-delà de la simple prise de parole spontanée et non maitrisée pour s’inscrire dans des conversations de plus en plus organisées et pour prendre la parole dans un grand groupe. Il sait mobiliser l’attention de tous dans des activités qui les amènent à comprendre des propos et des textes de plus en plus longs. Il met sur le chemin d’une conscience des langues, des mots du français et de ses unités sonores.

\subsection{Objectifs visés et éléments de progressivité} 
\subsubsection{Oser entrer en communication}
L’objectif est de permettre à chacun de pouvoir dire, exprimer un avis ou un besoin, questionner, annoncer une nouvelle. L’enfant apprend ainsi à entrer en communication avec autrui et à faire des efforts pour que les autres comprennent ce qu’il veut dire. Chacun arrive à l’école maternelle avec des acquis langagiers encore très hésitants. Entre deux et quatre ans, les enfants s’expriment beaucoup par des moyens non-verbaux et apprennent à parler. Ils reprennent des formulations ou des fragments des propos qui leur sont adressés et travaillent ainsi ce matériau qu’est la langue qu’ils entendent. Après trois-quatre ans, ils poursuivent ces essais et progressent sur le plan syntaxique et lexical. Ils produisent des énoncés plus complets, organisés entre eux avec cohérence, articulés à des prises de parole plus longues, et de plus en plus adaptés aux situations.

Autour de quatre ans, les enfants découvrent que les personnes, dont eux-mêmes, pensent et ressentent, et chacun différemment de l’autre. Ils commencent donc à agir volontairement sur autrui par le langage et à se représenter l’effet qu’une parole peut provoquer : ils peuvent alors comprendre qu’il faut expliquer et réexpliquer pour qu’un interlocuteur comprenne, et l’école doit les guider dans cette découverte. Ils commencent à poser de vraies questions, à saisir les plaisanteries et à en faire. Leurs progrès s’accompagnent d’un accroissement du vocabulaire et d’une organisation de plus en plus complexe des phrases. En fin d’école maternelle, l’enseignant peut donc avoir avec les enfants des conversations proches de celles qu’il a avec les adultes. 

Tout au long de l’école maternelle, l’enseignant crée les conditions bienveillantes et sécurisantes pour que tous les enfants (même ceux qui ne s’expriment pas ou peu) prennent la parole, participent à des situations langagières plus complexes que celles de la vie ordinaire ; il accueille les erreurs \og positives \fg{} qui traduisent une réorganisation mentale du langage en les valorisant et en proposant une reformulation. Ainsi, il contribue à construire l’équité entre enfants en réduisant les écarts langagiers.

\subsubsection{Comprendre et apprendre}
Les discours que tient l’enseignant sont des moyens de comprendre et d’apprendre pour les enfants. En compréhension, ceux-ci \og prennent \fg{} ce qui est à leur portée dans ce qu’ils entendent, d’abord dans des scènes renvoyant à des expériences personnelles précises, souvent chargées d’affectivité. Ils sont incités à s’intéresser progressivement à ce qu’ils ignoraient, grâce à l’apport de nouvelles notions, de nouveaux objets culturels et même de nouvelles manières d’apprendre. 

Les moments de réception où les enfants travaillent mentalement sans parler sont des activités langagières à part entière que l’enseignant doit rechercher et encourager, parce qu’elles permettent de construire des outils cognitifs : reconnaitre, rapprocher, catégoriser, contraster, se construire des images mentales à partir d’histoires fictives, relier des évènements entendus et/ou vus dans des narrations ou des explications, dans des moments d’apprentissages structurés, traiter des mots renvoyant à l’espace, au temps, etc. Ces activités invisibles aux yeux de tout observateur sont cruciales.

\subsubsection{Échanger et réfléchir avec les autres}
Les moments de langage à plusieurs sont nombreux à l’école maternelle : résolution de problèmes, prises de décisions collectives, compréhension d’histoires entendues, etc. Il y a alors argumentation, explication, questions, intérêt pour ce que les autres croient, pensent et savent. L’enseignant commente alors l’activité qui se déroule pour en faire ressortir l’importance et la finalité. 

L’école demande régulièrement aux élèves d’évoquer, c’est-à-dire de parler de ce qui n’est pas présent (récits d’expériences passées, projets de classe\dots). Ces situations d’évocation entrainent les élèves à mobiliser le langage pour se faire comprendre sans autre appui, elles leur offrent un moyen de s’entrainer à s’exprimer de manière de plus en plus explicite. Cette habileté langagière relève d’un développement continu qui commence tôt et qui ne sera constitué que vers huit ans. Le rôle de l’enseignant est d’induire du recul et de la réflexion sur les propos tenus par les uns et les autres.  

\subsubsection{Commencer à réfléchir sur la langue et acquérir une conscience phonologique}
Dès leur plus jeune âge, les enfants sont intéressés par la langue ou les langues qu’ils entendent. Ils font spontanément et sans en avoir conscience des tentatives pour en reproduire les sons, les formes et les structures afin d’entrer en communication avec leur entourage. C’est à partir de trois-quatre ans qu’ils peuvent prendre du recul et avoir conscience des efforts à faire pour maitriser une langue et accomplir ces efforts intentionnellement. On peut alors centrer leur attention sur le vocabulaire, sur la syntaxe et sur les unités sonores de la langue française dont la reconnaissance sera indispensable pour apprendre à maitriser le fonctionnement de l’écriture du français.

\paragraph{L’acquisition et le développement de la conscience phonologique}
Pour pouvoir lire et écrire, les enfants devront réaliser deux grandes acquisitions : identifier les unités sonores que l’on emploie lorsqu’on parle français (conscience phonologique) et comprendre que l’écriture du français est un code au moyen duquel on transcrit des sons (principe alphabétique).

Lorsqu’ils apprennent à parler, les enfants reproduisent les mots qu’ils ont entendus et donc les sons de la langue qu’on leur parle. S’il leur arrive de jouer avec les sons, cela se fait de manière aléatoire. À l’école maternelle, ils apprennent à manipuler volontairement les sons, à les identifier à l’oreille donc à les dissocier d’autres sons, à repérer des ressemblances et des différences. Pour pouvoir s’intéresser aux syllabes et aux phonèmes, il faut que les enfants se détachent du sens des mots. 

L’unité la plus aisément perceptible est la syllabe. Une fois que les enfants sont capables d’identifier des syllabes communes à plusieurs mots, de les isoler, ils peuvent alors s’attacher à repérer des éléments plus petits qui entrent dans la composition des syllabes. Parce que les sons-voyelles sont plus aisés à percevoir que les sons-consonnes et qu’ils constituent parfois des syllabes, c’est par eux qu’il convient de commencer sans vouloir faire identifier tous ceux qui existent en français et sans exclure de faire percevoir quelques sons-consonnes parmi les plus accessibles. 

Pour développer la conscience phonologique, l’enseignant habitue les enfants à décomposer volontairement ce qu’ils entendent en syllabes orales : en utilisant le frappé d’une suite sonore, en \og découpant \fg{} oralement des mots connus en syllabes, en repérant une syllabe identique dans des mots à deux syllabes, puis en intervertissant des syllabes, toujours sans support matériel, ni écrit ni imagé. Ces jeux phoniques peuvent être pratiqués en grand groupe, mais l’enseignant privilégie l’organisation en petits groupes pour des enfants qui participent peu ou avec difficulté en grand groupe. Dans le courant de la grande section, il consacre des séances courtes de manière régulière à ces jeux, en particulier avec les enfants pour lesquels il ne repère pas d’évolution dans les essais d’écriture. Pour ceux qui en sont capables, des activités similaires peuvent être amorcées sur des sons-voyelles -- notamment ceux qui constituent une syllabe dans les mots fréquentés -- et quelques sons-consonnes. Ces jeux et activités structurées sur les constituants sonores de la langue n’occupent qu’une part des activités langagières.
 
\paragraph{Éveil à la diversité linguistique}
À partir de la moyenne section, ils vont découvrir l’existence de langues, parfois très différentes de celles qu’ils connaissent. Dans des situations ludiques (jeux, comptines\dots) ou auxquelles ils peuvent donner du sens (DVD d’histoires connues par exemple), ils prennent conscience que la communication peut passer par d’autres langues que le français : par exemple les langues régionales, les langues étrangères et la langue des signes française (LSF). Les ambitions sont modestes, mais les essais que les enfants sont amenés à faire, notamment pour répéter certains éléments, doivent être conduits avec une certaine rigueur. 

\section{L’écrit}
\subsection{Objectifs visés et éléments de progressivité} 
Il appartient à l’école maternelle de donner à tous une culture commune de l’écrit. Les enfants y sont amenés à comprendre de mieux en mieux des écrits à leur portée, à découvrir la nature et la fonction langagière de ces tracés réalisés par quelqu’un pour quelqu’un, à commencer à participer à la production de textes écrits dont ils explorent les particularités. En fin de cycle, les enfants peuvent montrer tous ces acquis dans leurs premières écritures autonomes. Ce seront des tracés tâtonnants sur lesquels s’appuieront les enseignants de cycle 2. 

\subsubsection{Écouter de l’écrit et comprendre}
En préparant les enfants aux premières utilisations maitrisées de l’écrit en cycle 2, l’école maternelle occupe une place privilégiée pour leur offrir une fréquentation de la langue de l’écrit, très différente de l’oral de communication. L’enjeu est de les habituer à la réception de langage écrit afin d’en comprendre le contenu. L’enseignant prend en charge la lecture, oriente et anime les échanges qui suivent l’écoute. La progressivité réside essentiellement dans le choix de textes de plus en plus longs et éloignés de l’oral ; si la littérature de jeunesse y a une grande place, les textes documentaires ne sont pas négligés.

\subsubsection{Découvrir la fonction de l’écrit}
L’objectif est de permettre aux enfants de comprendre que les signes écrits qu’ils perçoivent valent du langage : en réception, l’écrit donne accès à la parole de quelqu’un et, en production, il permet de s’adresser à quelqu’un qui est absent ou de garder pour soi une trace de ce qui ne saurait être oublié. L’écrit transmet, donne ou rappelle des informations et fait imaginer : il a des incidences cognitives sur celui qui le lit. À l’école maternelle, les enfants le découvrent en utilisant divers supports (livres variés, affiches, lettres, messages électroniques ou téléphoniques, étiquettes, etc.) en relation avec des situations ou des projets qui les rendent nécessaires ; ils en font une expérience plus précise encore quand ils sont spectateurs d’une écriture adressée et quand ils constatent eux-mêmes les effets que produisent les écrits sur ceux qui les reçoivent. 

\subsubsection{Commencer à produire des écrits et en découvrir le fonctionnement}
C’est l’enseignant qui juge du moment où les enfants sont prêts à prendre en charge eux-mêmes une partie des activités que les adultes mènent avec l’écrit. Et comme il n’y a pas de pré-lecture à l’école maternelle, cette prise en charge partielle se fait en production et largement avec l’aide d’un adulte. Toute production d’écrits nécessite différentes étapes et donc de la durée avant d’aboutir ; la phase d’élaboration orale préalable du message est fondamentale, notamment parce qu’elle permet la prise de conscience des transformations nécessaires d’un propos oral en phrases à écrire. La technique de dictée à l’adulte concerne l’une de ces étapes qui est la rédaction proprement dite. Ces expériences précoces de productions génèrent une prise de conscience du pouvoir que donne la maitrise de l’écrit.

\subsubsection{Découvrir le principe alphabétique}
L’une des conditions pour apprendre à lire et à écrire est d’avoir découvert le principe alphabétique selon lequel l’écrit code en grande partie, non pas directement le sens, mais l’oral (la sonorité) de ce qu’on dit. Durant les trois années de l’école maternelle, les enfants vont découvrir ce principe (c’est-à-dire comprendre la relation entre lettres et sons) et commencer à le mettre en œuvre. Ce qui est visé à l’école maternelle est la découverte de ce principe et non l’apprentissage systématique des relations entre formes orales et écrites. 

La progressivité de l’enseignement à l’école maternelle nécessite de commencer par l’écriture. Les enfants ont en effet besoin de comprendre comment se fait la transformation d’une parole en écrit, d’où l’importance de la relation qui va de l’oral vers l’écrit. Le chemin inverse, qui va de l’écrit vers l’oral, sera pratiqué plus tard quand les enfants commenceront à apprendre à lire. Cette activité d’écriture ne peut s’effectuer que si, dans le même temps, l’enfant développe une conscience phonologique en devenant capable d’identifier les unités sonores de la langue.

La découverte du principe alphabétique rend possible les premières écritures autonomes en fin d’école maternelle parce qu’elle est associée à des savoirs complexes et à de nouveaux savoir-faire :
\begin{itemize}
\item la découverte de la fonction de l’écrit et les productions avec l’aide d’un adulte ;
\item la manipulation d’unités sonores non-signifiantes de la langue qui produit des habiletés qui sont utilisées lorsque les enfants essaient d’écrire ; 
\item parallèlement, à partir de la moyenne section, l’initiation aux tracés de l’écriture ;
\item la découverte des correspondances entre les trois écritures (cursive, script, capitales) qui donne aux enfants une palette de possibles, en tracé manuscrit et sur traitement de texte.
\end{itemize}

L’écriture autonome constitue l’aboutissement de ces différents apprentissages et découvertes.

\subsubsection{Commencer à écrire tout seul}
\paragraph{Un entrainement nécessaire avant de pratiquer l’écriture cursive : des exercices graphiques}
Il faut plusieurs années aux enfants pour acquérir les multiples habiletés nécessaires à l’écriture : utiliser leur regard pour piloter leur main, utiliser de façon coordonnée les quatre articulations qui servent à tenir et guider l’instrument d’écriture (épaule, coude, poignet, doigts), contrôler les tracés, et surtout tracer volontairement des signes abstraits dont ils comprennent qu’il ne s’agit pas de dessins mais de lettres, c’est-à-dire d’éléments d’un code qui transcrit des sons. Les exercices graphiques, qui permettent de s’entrainer aux gestes moteurs, et l’écriture proprement dite sont deux choses différentes. L’enseignant veille à ce qu’elles ne soient pas confondues.

En petite section, les exercices graphiques, en habituant les enfants à contrôler et guider leurs gestes par le regard, les entrainent à maitriser les gestes moteurs qui seront mobilisés dans le dessin et l’écriture cursive, à prendre des repères dans l’espace de la feuille. En moyenne et grande sections, ils s’exercent régulièrement à des tâches de motricité fine qui préparent spécifiquement à l’écriture. Ils s’entrainent également aux gestes propres à l’écriture et ils apprennent à adopter une posture confortable, à tenir de façon adaptée l’instrument d’écriture, à gérer l’espace graphique (aller de gauche à droite, maintenir un alignement\dots). L’enseignant varie les modèles et accorde du temps aux démonstrations qui permettent l’apprentissage de leur reproduction. 

L’écriture en capitales, plus facile graphiquement, ne fait pas l’objet d’un enseignement systématique ; lorsqu’elle est pratiquée par les enfants, l’enseignant veille au respect de l’ordre des lettres et met en évidence les conséquences du respect ou non de cet ordre sur ce qui peut ensuite être lu. L’écriture cursive nécessite quant à elle un entrainement pour apprendre à tracer chaque lettre et l’enchainement de plusieurs lettres, en ne levant qu’à bon escient l’instrument d’écriture. Cet entrainement ne peut intervenir que si les enfants ont acquis une certaine maturité motrice : s’il peut avec certains être commencé en moyenne section, c’est en grande section qu’il a le plus naturellement sa place, et souvent en deuxième partie d’année. Il devra être continué de manière très systématique au cours préparatoire. L’écriture régulière du prénom fournit une occasion de s’y exercer, les enfants ayant un moindre effort de mémoire à fournir et pouvant alors se concentrer sur la qualité du tracé. 

À partir de la moyenne section, et régulièrement en grande section, l’enseignant explique la correspondance des trois écritures (cursive, script, capitales). Les enfants s’exercent à des transcriptions de mots, phrases, courts textes connus, à leur saisie sur ordinateur. Travaillant alors en binôme, ils apprennent nombre de relations entre l’oral et l’écrit : un enfant nomme les lettres et montre, le second cherche sur le clavier, ils vérifient ensemble sur l’écran, puis sur la version imprimée.

L’objectif étant de construire la valeur symbolique des lettres, l’enseignant veille à ne jamais isoler les trois composantes de l’écriture : la composante sémantique (le sens de ce qui est écrit), la composante symbolique (le code alphabétique) et la composante motrice (la dextérité graphique).

\paragraph{Les essais d’écriture de mots}
Valoriser publiquement les premiers tracés des petits qui disent avoir écrit, c’est mettre toute la classe sur le chemin du symbolique. S’il s’agit de lignes, signes divers ou pseudo-lettres, l’enseignant précise qu’il ne peut pas encore lire. À partir de la moyenne section, l’enseignant fait des commandes d’écriture de mots simples, par exemple le nom d’un personnage d’une histoire. Le but est que les enfants se saisissent des apports de l’enseignant qui a écrit devant eux, ou des documents affichés dans la classe qui ont été observés ensemble et commentés. Leurs tracés montrent à l’enseignant ce que les enfants ont compris de l’écriture. Une fois les tracés faits, l’enseignant lit, ou bruite ou dit qu’il ne peut pas encore lire. Il discute avec l’enfant, il explique lui-même les procédés utilisés et écrit la forme canonique en faisant correspondre unités sonores et graphèmes. L’activité est plus fréquente en grande section.

L’enseignant ne laisse pas croire aux enfants que leurs productions sont correctes et il ne cherche pas non plus un résultat orthographique normé : il valorise les essais et termine par son écriture adulte sous l’essai de l’élève. 

\paragraph{Les premières productions autonomes d’écrits}
Lorsque les enfants ont compris que l’écrit est un code qui permet de délivrer des messages, il est possible de les inciter à produire des messages écrits. En grande section, les enfants commencent à avoir les ressources pour écrire, et l’enseignant les encourage à le faire ou valorise les essais spontanés. L’enseignant incite à écrire en utilisant tout ce qui est à leur portée. Une fois qu’ils savent exactement ce qu’ils veulent écrire, les enfants peuvent chercher dans des textes connus, utiliser le principe alphabétique, demander de l’aide. Plus ils écrivent, plus ils ont envie d’écrire. L’enseignant accepte qu’ils mêlent écriture en capitales pour résoudre des problèmes phonographiques et écriture en cursive. Lorsqu’ils ne se contentent plus de recopier des mots qu’ils connaissent, mais veulent écrire de nouveaux mots, ils recourent à différentes stratégies, en les combinant ou non : ils peuvent recopier des morceaux pris à d’autres mots, tracer des lettres dont le son se retrouve dans le mot à écrire (par exemple les voyelles), attribuer à des lettres la valeur phonique de leur nom (utiliser la lettre K pour transcrire le son /ca/). La séparation entre les mots reste un problème difficile à résoudre jusqu’au CE1. 

Les premiers essais d’écriture permettent à l’enseignant de voir que les enfants commencent à comprendre la fonction et le fonctionnement de l’écriture, même si ce n’est que petit à petit qu’ils en apprendront les règles. Il commente ces textes avec leurs auteurs (ce qu’ils voulaient dire, ce qu’ils ont écrit, ce qui montre qu’ils ont déjà des savoirs sur les textes écrits), puis il écrit en français écrit normé en soulignant les différences. Il donne aussi aux enfants les moyens de s’entrainer, notamment avec de la copie dans un coin écriture aménagé spécialement (outils, feuilles blanches et à lignes, ordinateur et imprimante, tablette numérique et stylets, tableaux de correspondance des graphies, textes connus). Un recueil individuel de ces premières écritures peut devenir un dossier de référence pour chaque élève, à apporter pour leur rentrée au CP.

\section{Ce qui est attendu des enfants en fin d’école maternelle}
\begin{itemize}
\item Communiquer avec les adultes et avec les autres enfants par le langage, en se faisant comprendre.
\item S’exprimer dans un langage syntaxiquement correct et précis. Reformuler pour se faire mieux comprendre.
\item Pratiquer divers usages du langage oral : raconter, décrire, évoquer, expliquer, questionner, proposer des solutions, discuter un point de vue.
\item Dire de mémoire et de manière expressive plusieurs comptines et poésies.
\item Comprendre des textes écrits sans autre aide que le langage entendu.
\item Manifester de la curiosité par rapport à l’écrit. Pouvoir redire les mots d’une phrase écrite après sa lecture par l’adulte, les mots du titre connu d’un livre ou d’un texte.
\item Participer verbalement à la production d’un écrit. Savoir qu’on n’écrit pas comme on parle.
\item Repérer des régularités dans la langue à l’oral en français (éventuellement dans une autre langue). 
\item Manipuler des syllabes.
\item Discriminer des sons (syllabes, sons-voyelles ; quelques sons-consonnes hors des consonnes occlusives).
\item Reconnaitre les lettres de l’alphabet et connaitre les correspondances entre les trois manières de les écrire : cursive, script, capitales d’imprimerie. Copier à l’aide d’un clavier.
\item Écrire son prénom en écriture cursive, sans modèle.
\item Écrire seul un mot en utilisant des lettres ou groupes de lettres empruntés aux mots connus.
\end{itemize}

\chapter{Agir, s’exprimer, comprendre à travers l’activité physique}
La pratique d’activités physiques et artistiques contribue au développement moteur, sensoriel, affectif, intellectuel et relationnel des enfants. Ces activités mobilisent, stimulent, enrichissent l’imaginaire et sont l’occasion d’éprouver des émotions, des sensations nouvelles. Elles permettent aux enfants d’explorer leurs possibilités physiques, d’élargir et d’affiner leurs habiletés motrices, de maitriser de nouveaux équilibres. Elles les aident à construire leur latéralité, l’image orientée de leur propre corps et à mieux se situer dans l’espace et dans le temps. 
Ces expériences corporelles visent également à développer la coopération, à établir des rapports constructifs à l’autre, dans le respect des différences, et contribuer ainsi à la socialisation. La participation de tous les enfants à l’ensemble des activités physiques proposées, l’organisation et les démarches mises en œuvre cherchent à lutter contre les stéréotypes et contribuent à la construction de l’égalité entre filles et garçons. Les activités physiques participent d’une éducation à la santé en conduisant tous les enfants, quelles que soient leurs \og performances \fg{}, à éprouver le plaisir du mouvement et de l’effort, à mieux connaitre leur corps pour le respecter. 

\section{Objectifs visés et éléments de progressivité}  
À leur arrivée à l’école maternelle, tous les enfants ne sont pas au même niveau de développement moteur. Ils n’ont pas réalisé les mêmes expériences corporelles et celles-ci ont pris des sens différents en fonction des contextes dans lesquels elles se sont déroulées. Le choix des activités physiques variées, prenant toujours des formes adaptées à l’âge des enfants, relève de l’enseignant, dans le cadre d’une programmation de classe et de cycle pour permettre d’atteindre les quatre objectifs caractéristiques de ce domaine d’apprentissage. Le besoin de mouvement des enfants est réel. Il est donc impératif d'organiser une séance quotidienne (de trente à quarante-cinq minutes environ, selon la nature des activités, l'organisation choisie, l'intensité des actions réalisées, le moment dans l'année, les comportements des enfants\dots). Ces séances doivent être organisées en cycles de durée suffisante pour que les enfants disposent d’un temps qui garantisse une véritable exploration et permette la construction de conquêtes motrices significatives.

\subsection{Agir dans l’espace, dans la durée et sur les objets}
Peu à peu, parce qu’il est sollicité par l’enseignant pour constater les résultats de ses actions, l’enfant prend plaisir à s’investir plus longuement dans les situations d’apprentissage qui lui sont proposées. Il découvre la possibilité d’enchainer des comportements moteurs pour assurer une continuité d’action (prendre une balle, puis courir pour franchir un obstacle, puis viser une cible pour la faire tomber, puis repartir au point de départ pour prendre un nouveau projectile\dots). Il apprend à fournir des efforts dans la durée, à chercher à parcourir plus de distance dans un temps donné (\og matérialisé \fg{} par un sablier, une chanson enregistrée\dots).

En agissant sur et avec des objets de tailles, de formes ou de poids différents (balles, ballons, sacs de graines, anneaux\dots), l’enfant en expérimente les propriétés, découvre des utilisations possibles (lancer, attraper, faire rouler\dots), essaie de reproduire un effet qu’il a obtenu au hasard des tâtonnements. Il progresse dans la perception et l’anticipation de la trajectoire d’un objet dans l’espace qui sont, même après l’âge de cinq ans, encore difficiles. 

\subsection{Adapter ses équilibres et ses déplacements à des environnements ou des contraintes variés}
Certains des plus jeunes enfants ont besoin de temps pour conquérir des espaces nouveaux ou s’engager dans des environnements inconnus. D’autres, au contraire, investissent d’emblée les propositions nouvelles sans appréhension mais également sans conscience des risques potentiels. Dans tous les cas, l’enseignant amène les enfants à découvrir leurs possibilités, en proposant des situations qui leur permettent d’explorer et d’étendre (repousser) leurs limites. Il les invite à mettre en jeu des conduites motrices inhabituelles (escalader, se suspendre, ramper\dots), à développer de nouveaux équilibres (se renverser, rouler, se laisser flotter\dots), à découvrir des espaces inconnus ou caractérisés par leur incertitude (piscine, patinoire, parc, forêt\dots). Pour les enfants autour de quatre ans, l’enseignant enrichit ces expérimentations à l’aide de matériels sollicitant l’équilibre (patins, échasses\dots), permettant de nouveaux modes de déplacement (tricycles, draisiennes, vélos, trottinettes\dots). Il attire l’attention des enfants sur leur propre sécurité et celle des autres, dans des situations pédagogiques dont le niveau de risque objectif est contrôlé par l'adulte. 

\subsection{Communiquer avec les autres au travers d’actions à visée expressive ou artistique}
Les situations proposées à l’enfant lui permettent de découvrir et d’affirmer ses propres possibilités d’improvisation, d’invention et de création en utilisant son corps. L’enseignant utilise des supports sonores variés (musiques, bruitages, paysages sonores\dots) ou, au contraire, développe l’écoute de soi et des autres au travers du silence. Il met à la disposition des enfants des objets initiant ou prolongeant le mouvement (voiles, plumes, feuilles\dots), notamment pour les plus jeunes d’entre eux. Il propose des aménagements d’espace adaptés, réels ou fictifs, incitent à de nouvelles expérimentations. Il amène à s’inscrire dans une réalisation de groupe. L’aller-retour entre les rôles d’acteurs et de spectateurs permet aux plus grands de mieux saisir les différentes dimensions de l’activité, les enjeux visés, le sens du progrès. L’enfant participe ainsi à un projet collectif qui peut être porté au regard d’autres spectateurs, extérieurs au groupe classe.

\subsection{Collaborer, coopérer, s’opposer}
Pour le jeune enfant, l’école est le plus souvent le lieu d’une première découverte des jeux moteurs vécus en collectif. La fonction de ce collectif, l’appropriation de différents modes d’organisation, le partage du matériel et la compréhension des rôles nécessitent des apprentissages. Les règles communes (délimitations de l’espace, but du jeu, droits et interdits\dots) sont une des conditions du plaisir de jouer, dans le respect des autres. 
Pour les plus jeunes, l’atteinte d’un but commun se fait tout d’abord par l’association d’actions réalisées en parallèle, sans réelle coordination. Il s’agit, dans les formes de jeu les plus simples, de comprendre et de s’approprier un seul rôle. L’exercice de rôles différents instaure les premières collaborations (vider une zone des objets qui s’y trouvent, collaborer afin de les échanger, les transporter, les ranger dans un autre camp\dots). Puis, sont proposées des situations dans lesquelles existe un réel antagonisme des intentions (dérober des objets, poursuivre des joueurs pour les attraper, s’échapper pour les éviter\dots) ou une réversibilité des statuts des joueurs (si le chat touche la souris, celle-ci devient chat à sa place\dots). 
D’autres situations ludiques permettent aux plus grands d’entrer au contact du corps de l’autre, d’apprendre à le respecter et d’explorer des actions en relation avec des intentions de coopération ou d’opposition spécifiques (saisir, soulever, pousser, tirer, immobiliser\dots). Que ce soit dans ces jeux à deux ou dans des jeux de groupe, tous peuvent utilement s’approprier des rôles sociaux variés : arbitre, observateur, responsable de la marque ou de la durée du jeu. 

\section{Ce qui est attendu des enfants en fin d’école maternelle}
\begin{itemize}
\item Courir, sauter, lancer de différentes façons, dans des espaces et avec des matériels variés, dans un but précis. 
\item Ajuster et enchainer ses actions et ses déplacements en fonction d’obstacles à franchir ou de la trajectoire d’objets sur lesquels agir.
\item Se déplacer avec aisance dans des environnements variés, naturels ou aménagés. 
\item Construire et conserver une séquence d’actions et de déplacements, en relation avec d’autres partenaires, avec ou sans support musical.
\item Coordonner ses gestes et ses déplacements avec ceux des autres, lors de rondes et jeux chantés.
\item Coopérer, exercer des rôles différents complémentaires, s’opposer, élaborer des stratégies pour viser un but ou un effet commun. 
\end{itemize}

\chapter{Agir, s’exprimer, comprendre à travers les activités artistiques}
Ce domaine d’apprentissage se réfère aux arts du visuel (peinture, sculpture, dessin, photographie, cinéma, bande dessinée, arts graphiques, arts numériques), aux arts du son (chansons, musiques instrumentales et vocales) et aux arts du spectacle vivant (danse, théâtre, arts du cirque, marionnettes, etc.). L’école maternelle joue un rôle décisif pour l’accès de tous les enfants à ces univers artistiques ; elle constitue la première étape du parcours d’éducation artistique et culturelle que chacun accomplit durant ses scolarités primaire et secondaire et qui vise l’acquisition d’une culture artistique personnelle, fondée sur des repères communs. 

\section{Objectifs visés et éléments de progressivité}
\subsubsection{Développer du gout pour les pratiques artistiques}
Les enfants doivent avoir des occasions fréquentes de pratiquer, individuellement et collectivement, dans des situations aux objectifs diversifiés. Ils explorent librement, laissent des traces spontanées avec les outils qu’ils choisissent ou que l’enseignant leur propose, dans des espaces et des moments dédiés à ces activités. Ils font des essais que les enseignants accueillent positivement. Ils découvrent des matériaux qui suscitent l’exploration de possibilités nouvelles, s’adaptent à une contrainte matérielle. Tout au long du cycle, ils s’intéressent aux effets produits, aux résultats d’actions et situent ces effets ou résultats par rapport aux intentions qu’ils avaient.  

\subsubsection{Découvrir différentes formes d’expression artistique}
Des rencontres avec différentes formes d’expression artistique sont organisées régulièrement ; dans la classe, les enfants sont confrontés à des œuvres sous forme de reproductions, d'enregistrements, de films ou de captations vidéo. La familiarisation avec une dizaine d'œuvres de différentes époques dans différents champs artistiques sur l’ensemble du cycle des apprentissages premiers permet aux enfants de commencer à construire des connaissances qui seront stabilisées ensuite pour constituer progressivement une culture artistique de référence. Autant que possible, les enfants sont initiés à la fréquentation d'espaces d'expositions, de salles de cinéma et de spectacles vivants afin qu'ils en comprennent la fonction artistique et sociale et découvrent le plaisir d'être spectateur. 

\subsubsection{Vivre et exprimer des émotions, formuler des choix}
Les enfants apprennent à mettre des mots sur leurs émotions, leurs sentiments, leurs impressions, et peu à peu, à exprimer leurs intentions et évoquer leurs réalisations comme celles des autres. L’enseignant les incite à être précis pour comparer, différencier leurs points de vue et ceux des autres, émettre des questionnements ; il les invite à expliciter leurs choix, à formuler ce à quoi ils pensent et à justifier ce qui présente à leurs yeux un intérêt.

\subsection{Les productions plastiques et visuelles}
\subsubsection{Dessiner}
Les enfants doivent disposer de temps pour dessiner librement, dans un espace aménagé où sont disponibles les outils et supports nécessaires. L’enseignant suscite l’expérimentation de différents outils, du crayon à la palette graphique, et favorise les temps d’échange pour comparer les effets produits. Il permet aux enfants d'identifier les réponses apportées par des plasticiens, des illustrateurs d’albums, à des problèmes qu'ils se sont posés. Il propose des consignes ouvertes qui incitent à la diversité des productions puis à la mutualisation des productions individuelles ; les échanges sur les différentes représentations d’un même objet enrichissent les pratiques et aident à dépasser les stéréotypes.

Les ébauches ou les premiers dessins sont conservés pour favoriser des comparaisons dans la durée et aider chaque enfant à percevoir ses progrès ; ils peuvent faire l’objet de reprises ou de prolongements. 

\subsubsection{S’exercer au graphisme décoratif}
Tout au long du cycle, les enfants rencontrent des graphismes décoratifs issus de traditions culturelles et d’époques variées. Ils constituent des répertoires d’images, de motifs divers où ils puisent pour apprendre à reproduire, assembler, organiser, enchainer à des fins créatives, mais aussi transformer et inventer dans des compositions. L’activité graphique conduite par l’enseignant entraine à l’exécution de tracés volontaires, à une observation fine et à la discrimination des formes, développe la coordination entre l’œil et la main ainsi qu’une habileté gestuelle diversifiée et adaptée. Ces acquisitions faciliteront la maitrise des tracés de l'écriture.

\subsubsection{Réaliser des compositions plastiques, planes et en volume}
Pour réaliser différentes compositions plastiques, seuls ou en petit groupe, les enfants sont conduits à s'intéresser à la couleur, aux formes et aux volumes. 

Le travail de la couleur s’effectue de manière variée avec les mélanges (à partir des couleurs primaires), les nuances et les camaïeux, les superpositions, les juxtapositions, l’utilisation d’images et de moyens différents (craies, encre, peinture, pigments naturels\dots). Ces expériences s'accompagnent de l'acquisition d'un lexique approprié pour décrire les actions (foncer, éclaircir, épaissir\dots) ou les effets produits (épais, opaque, transparent\dots).

Le travail en volume permet aux enfants d’appréhender des matériaux très différents (argile, bois, béton cellulaire, carton, papier, etc.) ; une consigne présentée comme problème à résoudre transforme la représentation habituelle du matériau utilisé. Ce travail favorise la représentation du monde en trois dimensions, la recherche de l’équilibre et de la verticalité. 

\subsubsection{Observer, comprendre et transformer des images}
Les enfants apprennent peu à peu à caractériser les différentes images, fixes ou animées, et leurs fonctions, et à distinguer le réel de sa représentation, afin d’avoir à terme un regard critique sur la multitude d’images auxquelles ils sont confrontés depuis leur plus jeune âge. 

L’observation des œuvres, reproduites ou originales, se mène en relation avec la pratique régulière de productions plastiques et d’échanges. 

\subsection{Univers sonores}
L'objectif de l'école maternelle est d'enrichir les possibilités de création et l'imaginaire musical, personnel et collectif, des enfants, en les confrontant à la diversité des univers musicaux. Les activités d'écoute et de production sont interdépendantes et participent d’une même dynamique. 

\subsubsection{Jouer avec sa voix et acquérir un répertoire de comptines et de chansons}
Par les usages qu'ils font de leur voix, les enfants construisent les bases de leur future voix d'adulte, parlée et chantée. L'école maternelle propose des situations qui leur permettent progressivement d'en découvrir la richesse, les incitent à dépasser les usages courants en les engageant dans une exploration ludique (chuchotements, cris, respirations, bruits, imitations d'animaux ou d'éléments sonores de la vie quotidienne, jeux de hauteur\dots). 

Les enfants apprennent à chanter en chœur avec des pairs ; l’enseignant prend garde à ne pas réunir un trop grand nombre d'enfants afin de pouvoir travailler sur la précision du chant, de la mélodie, du rythme et des effets musicaux. Les enfants acquièrent un répertoire de comptines et de chansons adapté à leur âge, qui s'enrichit au cours de leur scolarité. L'enseignant le choisit en puisant, en fonction de ses objectifs, dans la tradition orale enfantine et dans le répertoire d'auteurs contemporains. Dans un premier temps, il privilégie les comptines et les chants composés de phrases musicales courtes, à structure simple, adaptées aux possibilités vocales des enfants (étendue restreinte, absence de trop grandes difficultés mélodiques et rythmiques). Il peut ensuite faire appel à des chants un peu plus complexes, notamment sur le plan rythmique. 

\subsubsection{Explorer des instruments, utiliser les sonorités du corps}
Les activités mettant en jeu des instruments et les sonorités du corps participent au plaisir de la découverte de sources sonores variées et sont liées à l'évolution des possibilités gestuelles des enfants. Des activités d’exploration mobilisent les percussions corporelles, des objets divers parfois empruntés à la vie quotidienne, des instruments de percussion\dots Elles permettent progressivement aux enfants de maitriser leurs gestes afin d'en contrôler les effets. L'utilisation comparée d'instruments simples conduit les enfants à apprécier les effets produits de manière à regrouper les instruments dans des familles (ceux que l'on frappe, que l'on secoue, que l'on frotte, dans lesquels on souffle\dots). 

\subsubsection{Affiner son écoute}
Les activités d'écoute visent prioritairement à développer la sensibilité, la discrimination et la mémoire auditive. Elles posent aussi les bases de premières références culturelles et favorisent le développement de l'imaginaire. Elles sont constitutives des séances consacrées au chant et aux productions sonores avec des instruments. Les activités d'écoute peuvent faire l'objet de temps spécifiques ritualisés, évolutifs dans leur durée, au cours desquels les enfants découvrent des environnements sonores et des extraits d'œuvres musicales appartenant à différents styles, cultures et époques, choisies par l’enseignant. L'enseignant privilégie dans un premier temps des extraits caractérisés par des contrastes forts (intensité sonore forte ou faible, tempo lent/rapide, sons graves/aigus, timbres de voix ou d'instruments\dots) pour ensuite travailler à partir d'œuvres dont les contrastes sont moins marqués. Les consignes qu’il donne orientent l’attention des enfants de façon à ce qu’ils apprennent à écouter de plus en plus finement.

\subsection{Le spectacle vivant}
\subsubsection{Pratiquer quelques activités des arts du spectacle vivant}
Les activités artistiques relevant des arts du spectacle vivant (danse, cirque, mime, théâtre, marionnettes\dots) sont caractérisées par la mise en jeu du corps et suscitent chez l’enfant de nouvelles sensations et émotions. Elles mobilisent et enrichissent son imaginaire en transformant ses façons usuelles d’agir et de se déplacer, en développant un usage du corps éloigné des modalités quotidiennes et fonctionnelles. Une pratique de ces activités artistiques adaptée aux jeunes enfants leur permet de mettre ainsi en jeu et en scène une expression poétique du mouvement, d’ouvrir leur regard sur les modes d’expression des autres, sur la manière dont ceux-ci traduisent différemment leur ressenti. 

Au fil des séances, l’enseignant leur propose d’imiter, d’inventer, d’assembler des propositions personnelles ou partagées. Il les amène à s’approprier progressivement un espace scénique pour s’inscrire dans une production collective. Il les aide à entrer en relation avec les autres, que ce soit lors de rituels de début ou de fin de séance, lors de compositions instantanées au cours desquelles ils improvisent, ou lors d’un moment de production construit avec l’aide d’un adulte et que les enfants mémorisent. Grâce aux temps d’observation et d’échanges avec les autres, les enfants deviennent progressivement des spectateurs actifs et attentifs. 

\section{Ce qui est attendu des enfants en fin d’école maternelle}
\begin{itemize}
\item Choisir différents outils, médiums, supports en fonction d’un projet ou d’une consigne et les utiliser en adaptant son geste.
\item Pratiquer le dessin pour représenter ou illustrer, en étant fidèle au réel ou à un modèle, ou en inventant. 
\item Réaliser une composition personnelle en reproduisant des graphismes. Créer des graphismes nouveaux.
\item Réaliser des compositions plastiques, seul ou en petit groupe, en choisissant et combinant des matériaux, en réinvestissant des techniques et des procédés.
\item Avoir mémorisé un répertoire varié de comptines et de chansons et les interpréter de manière expressive. 
\item Jouer avec sa voix pour explorer des variantes de timbre, d’intensité, de hauteur, de nuance. 
\item Repérer et reproduire, corporellement ou avec des instruments, des formules rythmiques simples. 
\item Décrire une image, parler d’un extrait musical et exprimer son ressenti ou sa compréhension en utilisant un vocabulaire adapté.
\item Proposer des solutions dans des situations de projet, de création, de résolution de problèmes, avec son corps, sa voix ou des objets sonores.
\end{itemize}

\chapter{Construire les premiers outils pour structurer sa pensée}
\section{Découvrir les nombres et leurs utilisations}
Depuis leur naissance, les enfants ont une intuition des grandeurs qui leur permet de comparer et d’évaluer de manière approximative les longueurs (les tailles), les volumes, mais aussi les collections d’objets divers (\og il y en a beaucoup \fg{}, \og pas beaucoup \fg{}\dots). À leur arrivée à l’école maternelle, ils discriminent les petites quantités, un, deux et trois, notamment lorsqu’elles forment des configurations culturellement connues (dominos, dés). Enfin, s’ils savent énoncer les débuts de la suite numérique, cette récitation ne traduit pas une véritable compréhension des quantités et des nombres. 

L’école maternelle doit conduire progressivement chacun à comprendre que les nombres permettent à la fois d’exprimer des quantités (usage cardinal) et d’exprimer un rang ou un positionnement dans une liste (usage ordinal). Cet apprentissage demande du temps et la confrontation à de nombreuses situations impliquant des activités pré-numériques puis numériques.  

\subsection{Objectifs visés et éléments de progressivité}
La construction du nombre s’appuie sur la notion de quantité, sa codification orale et écrite, l’acquisition de la suite orale des nombres et l’usage du dénombrement. Chez les jeunes enfants, ces apprentissages se développent en parallèle avant de pouvoir se coordonner : l’enfant peut, par exemple, savoir réciter assez loin la comptine numérique sans savoir l’utiliser pour dénombrer une collection. 

Dans l’apprentissage du nombre à l’école maternelle, il convient de faire construire le nombre pour exprimer les quantités, de stabiliser la connaissance des petits nombres et d’utiliser le nombre comme mémoire de la position. L’enseignant favorise le développement très progressif de chacune de ces dimensions pour contribuer à la construction de la notion de nombre. Cette construction ne saurait se confondre avec celle de la numération et des opérations qui relèvent des apprentissages de l'école élémentaire. 

\subsubsection{Construire le nombre pour exprimer les quantités}
Comprendre la notion de quantité implique pour l’enfant de concevoir que la quantité n’est pas la caractéristique d’un objet mais d’une collection d’objets (l’enfant doit également comprendre que le nombre sert à mémoriser la quantité). L’enfant fait d’abord appel à une estimation perceptive et globale (plus, moins, pareil, beaucoup, pas beaucoup). Progressivement, il passe de l’apparence des collections à la prise en compte des quantités. La comparaison des collections et la production d’une collection de même cardinal qu’une autre sont des activités essentielles pour l’apprentissage du nombre. Le nombre en tant qu’outil de mesure de la quantité est stabilisé quand l’enfant peut l’associer à une collection, quelle qu’en soit la nature, la taille des éléments et l’espace occupé : cinq permet indistinctement de désigner cinq fourmis, cinq cubes ou cinq éléphants.

Les trois années de l’école maternelle sont nécessaires et parfois non suffisantes pour stabiliser ces connaissances en veillant à ce que les nombres travaillés soient composés et décomposés. La maitrise de la décomposition des nombres est une condition nécessaire à la construction du nombre. 

\subsubsection{Stabiliser la connaissance des petits nombres}
Au cycle 1, la construction des quantités jusqu’à dix est essentielle. Cela n’exclut pas le travail de comparaison sur de grandes collections. La stabilisation de la notion de quantité, par exemple trois, est la capacité à donner, montrer, évaluer ou prendre un, deux ou trois et à composer et décomposer deux et trois. Entre deux et quatre ans, stabiliser la connaissance des petits nombres (jusqu’à cinq) demande des activités nombreuses et variées portant sur la décomposition et recomposition des petites quantités (trois c’est deux et encore un ; un et encore deux ; quatre c’est deux et encore deux ; trois et encore un ; un et encore trois), la reconnaissance et l’observation des constellations du dé, la reconnaissance et l’expression d’une quantité avec les doigts de la main, la correspondance terme à terme avec une collection de cardinal connu. 

L’itération de l’unité (trois c’est deux et encore un) se construit progressivement, et pour chaque nombre. Après quatre ans, les activités de décomposition et recomposition s’exercent sur des quantités jusqu’à dix. 

\subsubsection{Utiliser le nombre pour désigner un rang, une position}
Le nombre permet également de conserver la mémoire du rang d’un élément dans une collection organisée. Pour garder en mémoire le rang et la position des objets (troisième perle, cinquième cerceau), les enfants doivent définir un sens de lecture, un sens de parcours, c’est-à-dire donner un ordre. Cet usage du nombre s’appuie à l’oral sur la connaissance de la comptine numérique et à l’écrit sur celle de l’écriture chiffrée.

\subsubsection{Construire des premiers savoirs et savoir-faire avec rigueur}
\paragraph{Acquérir la suite orale des mots-nombres}
Pour que la suite orale des mots-nombres soit disponible en tant que ressource pour dénombrer, il faut qu’elle soit stable, ordonnée, segmentée et suffisamment longue. Elle doit être travaillée pour elle-même et constituer un réservoir de mots ordonnés. La connaissance de la suite orale des noms des nombres ne constitue pas l’apprentissage du nombre mais y contribue.

Avant quatre ans, les premiers éléments de la suite numérique peuvent être mis en place jusqu’à cinq ou six puis progressivement étendus jusqu’à trente en fin de grande section. L’apprentissage des comptines numériques favorise notamment la mémorisation de la suite des nombres, la segmentation des mots-nombres en unités linguistiques ; ces acquis permettent de repérer les nombres qui sont avant et après, le suivant et le précédent d’un nombre, de prendre conscience du lien entre l’augmentation ou la diminution d’un élément d’une collection. 

\paragraph{Écrire les nombres avec les chiffres}
Parallèlement, les enfants rencontrent les nombres écrits notamment dans des activités occasionnelles de la vie de la classe, dans des jeux et au travers d’un premier usage du calendrier. Les premières écritures des nombres ne doivent pas être introduites précocement mais progressivement, à partir des besoins de communication dans la résolution de situations concrètes. L’apprentissage du tracé des chiffres se fait avec la même rigueur que celui des lettres. La progression de la capacité de lecture et d’écriture des nombres s’organise sur le cycle, notamment à partir de quatre ans. Le code écrit institutionnel est l’ultime étape de l’apprentissage qui se poursuit au cycle 2.

\paragraph{Dénombrer}
Les activités de dénombrement doivent éviter le comptage-numérotage et faire apparaitre, lors de l’énumération de la collection, que chacun des noms de nombres désigne la quantité qui vient d’être formée (l’enfant doit comprendre que montrer trois doigts, ce n’est pas la même chose que montrer le troisième doigt de la main). Ultérieurement, au-delà de cinq, la même attention doit être portée à l’élaboration progressive des quantités et de leurs relations aux nombres sous les différents codes. Les enfants doivent comprendre que toute quantité s’obtient en ajoutant un à la quantité précédente (ou en enlevant un à la quantité supérieure) et que sa dénomination s’obtient en avançant de un dans la suite des noms de nombres ou de leur écriture avec des chiffres. 

Pour dénombrer une collection d’objets, l’enfant doit être capable de synchroniser la récitation de la suite des mots-nombres avec le pointage des objets à dénombrer. Cette capacité doit être enseignée selon différentes modalités en faisant varier la nature des collections et leur organisation spatiale car les stratégies ne sont pas les mêmes selon que les objets sont déplaçables ou non (mettre dans une boite, poser sur une autre table), et selon leur disposition (collection organisée dans l’espace ou non, collection organisée-alignée sur une feuille ou pas). 

\subsection{Ce qui est attendu des enfants en fin d’école maternelle}
\subsubsection{Utiliser les nombres}
\begin{itemize}
\item Évaluer et comparer des collections d’objets avec des procédures numériques ou non numériques.
\item Réaliser une collection dont le cardinal est donné. Utiliser le dénombrement pour comparer deux quantités, pour constituer une collection d’une taille donnée ou pour réaliser une collection de quantité égale à la collection proposée.
\item Utiliser le nombre pour exprimer la position d’un objet ou d’une personne dans un jeu, dans une situation organisée, sur un rang ou pour comparer des positions.
\item Mobiliser des symboles analogiques, verbaux ou écrits, conventionnels ou non conventionnels pour communiquer des informations orales et écrites sur une quantité. 
\end{itemize}

\subsubsection{Étudier les nombres}
\begin{itemize}
\item Avoir compris que le cardinal ne change pas si on modifie la disposition spatiale ou la nature des éléments.
\item Avoir compris que tout nombre s’obtient en ajoutant un au nombre précédent et que cela correspond à l’ajout d’une unité à la quantité précédente.
\item Quantifier des collections jusqu’à dix au moins ; les composer et les décomposer par manipulations effectives puis mentales. Dire combien il faut ajouter ou enlever pour obtenir des quantités ne dépassant pas dix.
\item Parler des nombres à l’aide de leur décomposition. 
\item Dire la suite des nombres jusqu’à trente. Lire les nombres écrits en chiffres jusqu’à dix.
\end{itemize}

\section{Explorer des formes, des grandeurs, des suites organisées}
Très tôt, les jeunes enfants discernent intuitivement des formes (carré, triangle\dots) et des grandeurs (longueur, contenance, masse, aire\dots). À l’école maternelle, ils construisent des connaissances et des repères sur quelques formes et grandeurs. L’approche des formes planes, des objets de l’espace, des grandeurs, se fait par la manipulation et la coordination d’actions sur des objets. Cette approche est soutenue par le langage : il permet de décrire ces objets et ces actions et favorise l’identification de premières caractéristiques descriptives. Ces connaissances qui resteront limitées constituent une première approche de la géométrie et de la mesure qui seront enseignées aux cycles 2 et 3.

\subsection{Objectifs visés et éléments de progressivité}
Très tôt, les enfants regroupent les objets, soit en fonction de leur aspect, soit en fonction de leur utilisation familière ou de leurs effets. À l’école, ils sont incités à \og mettre ensemble ce qui va ensemble \fg{} pour comprendre que tout objet peut appartenir à plusieurs catégories et que certains objets ne peuvent pas appartenir à celles-ci.
 
Par des observations, des comparaisons, des tris, les enfants sont amenés à mieux distinguer différents types de critères : forme, longueur, masse, contenance essentiellement. Ils apprennent progressivement à reconnaitre, distinguer des solides puis des formes planes. Ils commencent à appréhender la notion d’alignement qu’ils peuvent aussi expérimenter dans les séances d’activités physiques. L’enseignant est attentif au fait que l’appréhension des formes planes est plus abstraite que celle des solides et que certains termes prêtent à confusion (carré/cube). L’enseignant utilise un vocabulaire précis  (cube, boule, pyramide, cylindre, carré, rectangle, triangle, cercle ou disque (à préférer à \og rond \fg{}) que les enfants sont entrainés ainsi à comprendre d’abord puis à utiliser à bon escient, mais la manipulation du vocabulaire mathématique n’est pas un objectif de l’école maternelle. 

Par ailleurs, dès la petite section, les enfants sont invités à organiser des suites d’objets en fonction de critères de formes et de couleurs ; les premiers algorithmes qui leur sont proposés sont simples. Dans les années suivantes, progressivement, ils sont amenés à reconnaitre un rythme dans une suite organisée et à continuer cette suite, à inventer des \og rythmes \fg{} de plus en plus compliqués, à compléter des manques dans une suite organisée. 


\subsection{Ce qui est attendu des enfants en fin d’école maternelle}
\begin{itemize}
\item Classer des objets en fonction de caractéristiques liées à leur forme. Savoir nommer quelques formes planes (carré, triangle, cercle ou disque, rectangle) et reconnaitre quelques solides (cube, pyramide, boule, cylindre).
\item Classer ou ranger des objets selon un critère de longueur ou de masse ou de contenance.
\item Reproduire un assemblage à partir d’un modèle (puzzle, pavage, assemblage de solides).
\item Reproduire, dessiner des formes planes.
\item Identifier le principe d’organisation d’un algorithme et poursuivre son application. 
\end{itemize}

\chapter{Explorer le monde}
\section{Se repérer dans le temps et l’espace}
Dès leur naissance, par leurs activités exploratoires, les enfants perçoivent intuitivement certaines dimensions spatiales et temporelles de leur environnement immédiat. Ces perceptions leur permettent d’acquérir, au sein de leurs milieux de vie, une première série de repères, de développer des attentes et des souvenirs d’un passé récent. Ces connaissances demeurent toutefois implicites et limitées. L’un des objectifs de l’école maternelle est précisément de les amener progressivement à considérer le temps et l’espace comme des dimensions relativement indépendantes des activités en cours, et à commencer à les traiter comme telles. Elle cherche également à les amener à dépasser peu à peu leur propre point de vue et à adopter celui d’autrui.

\subsection{Objectifs visés et éléments de progressivité}
\subsubsection{Le temps}
L’école maternelle vise la construction de repères temporels et la sensibilisation aux durées : temps court (celui d’une activité avec son avant et son après, journée) et temps long (succession des jours dans la semaine et le mois, succession des saisons). L’appréhension du temps très long (temps historique) est plus difficile notamment en ce qui concerne la distinction entre passé proche et passé lointain.

\subsubsection{Stabiliser les premiers repères temporels}
Pour les plus jeunes, les premiers repères temporels sont associés aux activités récurrentes de la vie quotidienne d’où l’importance d’une organisation régulière et de rituels qui marquent les passages d’un moment à un autre. Ces repères permettent à l’enseignant d’\og ancrer \fg{} pour les enfants les premiers éléments stables d’une chronologie sommaire et de leur proposer un premier travail d’évocation et d’anticipation en s’appuyant sur des évènements proches du moment présent.

\subsubsection{Introduire les repères sociaux}
À partir de la moyenne section, les repères sociaux sont introduits et utilisés quotidiennement par les enfants pour déterminer les jours de la semaine, pour préciser les évènements de la vie scolaire. L’enseignant conduit progressivement les enfants à relier entre eux les différents systèmes de repérage, notamment les moments de la journée et les heures pour objectiver les durées et repères utilisés par l’adulte (dans cinq minutes, dans une heure). 

\subsubsection{Consolider la notion de chronologie}
En moyenne section, l’enseignant propose un travail relevant de la construction de la chronologie portant sur des périodes plus larges, notamment la semaine. Il s’appuie pour ce faire sur des évènements vécus, dont le déroulement est perceptible par les enfants et pour lesquels des étapes peuvent être distinguées, ordonnées, reconstituées, complétées. Les activités réalisées en classe favorisent l’acquisition des marques temporelles dans le langage, notamment pour situer un propos par rapport au moment de la parole (hier, aujourd’hui, maintenant, demain, plus tard\dots), ou l’utilisation des formes des verbes correspondantes. L’enseignant crée les conditions pour que les relations temporelles de succession, d’antériorité, de postériorité, de simultanéité puissent être traduites par les formulations verbales adaptées (avant, après, pendant, bien avant, bien après, en même temps, etc.). 

En grande section, des évènements choisis en fonction des projets de classe (la disparition des dinosaures, l’apparition de l’écriture\dots) ou des éléments du patrimoine architectural proche, de la vie des parents et des grands-parents, peuvent être exploités pour mettre en ordre quelques repères communs mais sans souci de prise en compte de la mesure du temps. 

\subsubsection{Sensibiliser à la notion de durée}
La notion de durée commence à se mettre en place vers quatre ans de façon subjective. En recourant à des outils et dispositifs qui fournissent une appréciation plus objective, l’enseignant amène les enfants non pas à mesurer le temps à proprement parler, mais à le matérialiser en visualisant son écoulement. Ainsi, les sabliers, les enregistrements d’une comptine ou d’une chanson peuvent permettre une première appréhension d’une durée stable donnée ou la comparaison avec une autre.

\subsubsection{L’espace}
\paragraph{Faire l’expérience de l’espace}
L’expérience de l’espace porte sur l’acquisition de connaissances liées aux déplacements, aux distances et aux repères spatiaux élaborés par les enfants au cours de leurs activités. L’enseignant crée les conditions d’une accumulation d'expériences assorties de prises de repères sur l’espace en permettant aux enfants de l'explorer, de le parcourir, d’observer les positions d’éléments fixes ou mobiles, les déplacements de leurs pairs, d’anticiper progressivement leurs propres itinéraires au travers d’échanges langagiers. L’enseignant favorise ainsi l’organisation de repères que chacun élabore, par l’action et par le langage, à partir de son propre corps afin d’en construire progressivement une image orientée. 

\paragraph{Représenter l’espace}
Par l’utilisation et la production de représentations diverses (photos, maquettes, dessins, plans\dots) et également par les échanges langagiers avec leurs camarades et les adultes, les enfants apprennent à restituer leurs déplacements et à en effectuer à partir de consignes orales comprises et mémorisées. Ils établissent alors les relations entre leurs déplacements et les représentations de ceux-ci. Le passage aux représentations planes par le biais du dessin les amène à commencer à mettre intuitivement en relation des perceptions en trois dimensions et des codages en deux dimensions faisant appel à certaines formes géométriques (rectangles, carrés, triangles, cercles). Ces mises en relations seront plus précisément étudiées à l’école élémentaire, mais elles peuvent déjà être utilisées pour coder des déplacements ou des représentations spatiales. De plus, les dessins, comme les textes présentés sur des pages ou les productions graphiques, initient les enfants à se repérer et à s’orienter dans un espace à deux dimensions, celui de la page mais aussi celui des cahiers et des livres.

\paragraph{Découvrir différents milieux}
L’enseignant conduit les enfants de l'observation de l'environnement proche (la classe, l'école, le quartier\dots) à la découverte d'espaces moins familiers (campagne, ville, mer, montagne\dots). L'observation des constructions humaines (maisons, commerces, monuments, routes, ponts\dots) relève du même cheminement. Pour les plus grands, une première approche du paysage comme milieu marqué par l'activité humaine devient possible. Ces situations sont autant d'occasions de se questionner, de produire des images (l’appareil photographique numérique est un auxiliaire pertinent), de rechercher des informations, grâce à la médiation du maitre, dans des documentaires, sur des sites Internet. Cette exploration des milieux permet aussi une initiation concrète à une attitude responsable (respect des lieux, de la vie, connaissance de l’impact de certains comportements sur l'environnement\dots). 

À partir des expériences vécues à l’école et en dehors de celle-ci par les enfants de la classe et des occasions qu’il provoque, l’enseignant favorise également une première découverte de pays et de cultures pour les ouvrir à la diversité du monde. Cette découverte peut se faire en lien avec une première sensibilisation à la pluralité des langues. 

\subsection{Ce qui est attendu des enfants en fin d’école maternelle}
\begin{itemize}
\item Situer des évènements vécus les uns par rapport aux autres et en les repérant dans la journée, la semaine, le mois ou une saison. 
\item Ordonner une suite de photographies ou d’images, pour rendre compte d’une situation vécue ou d’un récit fictif entendu, en marquant de manière exacte succession et simultanéité.
\item Utiliser des marqueurs temporels adaptés (puis, pendant, avant, après\dots) dans des récits, descriptions ou explications.
\item Situer des objets par rapport à soi, entre eux, par rapport à des objets repères.
\item Se situer par rapport à d’autres, par rapport à des objets repères.
\item Dans un environnement bien connu, réaliser un trajet, un parcours à partir de sa représentation (dessin ou codage). 
\item Élaborer des premiers essais de représentation plane, communicables (construction d’un code commun).
\item Orienter et utiliser correctement une feuille de papier, un livre ou un autre support d’écrit, en fonction de consignes, d’un but ou d’un projet précis.
\item Utiliser des marqueurs spatiaux adaptés (devant, derrière, droite, gauche, dessus, dessous\dots) dans des récits, descriptions ou explications.
\end{itemize}

\section{Explorer le monde du vivant, des objets et de la matière}
À leur entrée à l’école maternelle, les enfants ont déjà des représentations qui leur permettent de prendre des repères dans leur vie quotidienne. Pour les aider à découvrir, organiser et comprendre le monde qui les entoure, l’enseignant propose des activités qui amènent les enfants à observer, formuler des interrogations plus rationnelles, construire des relations entre les phénomènes observés, prévoir des conséquences, identifier des caractéristiques susceptibles d’être catégorisées. Les enfants commencent à comprendre ce qui distingue le vivant du non-vivant ; ils manipulent, fabriquent pour se familiariser avec les objets et la matière.

\subsection{Objectifs visés et éléments de progressivité}
\subsubsection{Découvrir le monde vivant}
L’enseignant conduit les enfants à observer les différentes manifestations de la vie animale et végétale. Ils découvrent le cycle que constituent la naissance, la croissance, la reproduction, le vieillissement, la mort en assurant les soins nécessaires aux élevages et aux plantations dans la classe. Ils identifient, nomment ou regroupent des animaux en fonction de leurs caractéristiques (poils, plumes, écailles\dots), de leurs modes de déplacements (marche, reptation, vol, nage\dots), de leurs milieux de vie, etc.

À travers les activités physiques vécues à l’école, les enfants apprennent à mieux connaitre et maitriser leur corps. Ils comprennent qu’il leur appartient, qu’ils doivent en prendre soin pour se maintenir en forme et favoriser leur bien-être. Ils apprennent à identifier, désigner et nommer les différentes parties du corps. Cette éducation à la santé vise l’acquisition de premiers savoirs et savoir-faire relatifs à une hygiène de vie saine. Elle intègre une première approche des questions nutritionnelles qui peut être liée à une éducation au gout.

Les enfants enrichissent et développent leurs aptitudes sensorielles, s'en servent pour distinguer des réalités différentes selon leurs caractéristiques olfactives, gustatives, tactiles, auditives et visuelles. Chez les plus grands, il s’agit de comparer, classer ou ordonner ces réalités, les décrire grâce au langage, les catégoriser.

Enfin, les questions de la protection du vivant et de son environnement sont abordées dans le cadre d’une découverte de différents milieux, par une initiation concrète à une attitude responsable.

\subsubsection{Explorer la matière}
Une première appréhension du concept de matière est favorisée par l’action directe sur les matériaux dès la petite section. Les enfants s'exercent régulièrement à des actions variées (transvaser, malaxer, mélanger, transporter, modeler, tailler, couper, morceler, assembler, transformer). Tout au long du cycle, ils découvrent les effets de leurs actions et ils utilisent quelques matières ou matériaux naturels (l’eau, le bois, la terre, le sable, l’air\dots) ou fabriqués par l’homme (le papier, le carton, la semoule, le tissu\dots).

Les activités qui conduisent à des mélanges, des dissolutions, des transformations mécaniques ou sous l’effet de la chaleur ou du froid permettent progressivement d’approcher quelques propriétés de ces matières et matériaux, quelques aspects de leurs transformations possibles. Elles sont l’occasion de discussions entre enfants et avec l’enseignant, et permettent de classer, désigner et définir leurs qualités en acquérant le vocabulaire approprié. 

\subsubsection{Utiliser, fabriquer, manipuler des objets}
L'utilisation d'instruments, d’objets variés, d’outils conduit les enfants à développer une série d’habiletés, à manipuler et à découvrir leurs usages. De la petite à la grande section, les enfants apprennent à relier une action ou le choix d’un outil à l’effet qu’ils veulent obtenir : coller, enfiler, assembler, actionner, boutonner, découper, équilibrer, tenir un outil scripteur, plier, utiliser un gabarit, manipuler une souris d’ordinateur, agir sur une tablette numérique\dots Toutes ces actions se complexifient au long du cycle. Pour atteindre l’objectif qui leur est fixé ou celui qu’ils se donnent, les enfants apprennent à intégrer progressivement la chronologie des tâches requises et à ordonner une suite d’actions ; en grande section, ils sont capables d’utiliser un mode d’emploi ou une fiche de construction illustrés.

Les montages et démontages dans le cadre des jeux de construction et de la réalisation de maquettes, la fabrication d'objets contribuent à une première découverte du monde technique. 

Les utilisations multiples d’instruments et d’objets sont l’occasion de constater des phénomènes physiques, notamment en utilisant des instruments d’optique simples (les loupes notamment) ou en agissant avec des ressorts, des aimants, des poulies, des engrenages, des plans inclinés\dots Les enfants ont besoin d’agir de nombreuses fois pour constater des régularités qui sont les manifestations des phénomènes physiques qu’ils étudieront beaucoup plus tard (la gravité, l’attraction entre deux pôles aimantés, les effets de la lumière, etc.).

Tout au long du cycle, les enfants prennent conscience des risques liés à l’usage des objets, notamment dans le cadre de la prévention des accidents domestiques.

\subsubsection{Utiliser des outils numériques}
Dès leur plus jeune âge, les enfants sont en contact avec les nouvelles technologies. Le rôle de l’école est de leur donner des repères pour en comprendre l’utilité et commencer à les utiliser de manière adaptée (tablette numérique, ordinateur, appareil photo numérique\dots). Des recherches ciblées, via le réseau Internet, sont effectuées et commentées par l’enseignant. 

Des projets de classe ou d’école induisant des relations avec d’autres enfants favorisent des expériences de communication à distance. L’enseignant évoque avec les enfants l’idée d’un monde en réseau qui peut permettre de parler à d’autres personnes parfois très éloignées.

\subsection{Ce qui est attendu des enfants en fin d’école maternelle}
\begin{itemize}
\item Reconnaitre les principales étapes du développement d'un animal ou d'un végétal, dans une situation d’observation du réel ou sur une image.
\item Connaitre les besoins essentiels de quelques animaux et végétaux.
\item Situer et nommer les différentes parties du corps humain, sur soi ou sur une représentation.
\item Connaitre et mettre en œuvre quelques règles d'hygiène corporelle et d’une vie saine.
\item Choisir, utiliser et savoir désigner des outils et des matériaux adaptés à une situation, à des actions techniques spécifiques (plier, couper, coller, assembler, actionner\dots).
\item Réaliser des constructions ; construire des maquettes simples en fonction de plans ou d’instructions de montage. 
\item Utiliser des objets numériques : appareil photo, tablette, ordinateur.
\item Prendre en compte les risques de l'environnement familier proche (objets et comportements dangereux, produits toxiques).
\end{itemize}