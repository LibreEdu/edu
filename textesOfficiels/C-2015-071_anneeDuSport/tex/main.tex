L’année 2015-2016 sera marquée par l’organisation en France de grands évènements sportifs tels que l’Euro 2015 de basket, les championnats d’Europe 2015 de cross-country, les championnats d’Europe de badminton et l’UEFA - Euro 2016 de football.

En prenant appui sur ces différentes manifestations, le ministère de l’éducation nationale, de l’enseignement supérieur et de la recherche souhaite promouvoir la pratique sportive chez les jeunes et mobiliser la communauté éducative autour des valeurs européennes et sportives dans le cadre d’une \textbf{Année du sport de l’école à l’université}.

Cette opération visera à mettre en avant le sport comme vecteur des valeurs éducatives et citoyennes aussi bien à l’école, où il s’inscrit dans le cadre de l’éducation en mobilisant des connaissances et compétences disciplinaires et transversales, que hors de l’école, avec un large réseau associatif qui permet à chacun de découvrir et pratiquer une multitude d’activités. Elle doit permettre de souligner que le sport et l’école s’appuient sur les mêmes valeurs : le gout de l’effort, la persévérance, la volonté de progresser, le respect des autres, de soi et des règles, etc.  

\section{Objectifs de l’opération}
L’opération s’articulera autour de quatre grands axes :
\begin{itemize}
\item Valoriser les pratiques sportives à l’école, et en premier lieu l’éducation physique et sportive (EPS) et le sport scolaire pour :
	\begin{itemize}
	\item améliorer les capacités physiques, motrices et organiques des jeunes dans un but de bienêtre et de santé ;
	\item permettre à ces jeunes d’accéder au domaine de la culture que représente l’ensemble des activités physiques, sportives et artistiques ;
	\item faire acquérir à ce jeune public les compétences nécessaires à l’entretien de leur vie physique et citoyenne à l’âge adulte.
	\end{itemize}
\item Valoriser le sport comme outil pédagogique permettant de contribuer aux différents domaines de formation de l’enseignement scolaire pour :
	\begin{itemize}
	\item s’appuyer sur des actions éducatives existantes et favoriser lors de l’année scolaire 2015-2016 des approches croisées du sport ;
	\item développer et valoriser des ressources pédagogiques qui permettent aux équipes éducatives de se saisir du sport comme objet didactique pour les apprentissages disciplinaires dans le temps et les activités des élèves et des étudiants.
	\end{itemize}
\item Valoriser les pratiques sportives dans l’enseignement supérieur pour :
	\begin{itemize}
	\item améliorer la santé, le développement personnel et le bienêtre des étudiants ;
	\item faciliter l’intégration des étudiants, en particulier celle des primoarrivants et des étudiants étrangers, en créant du lien social ;
	\item contribuer à la réussite des études ;
	\item renforcer la vie de campus pour les étudiants et la communauté universitaire dans son ensemble.
	\end{itemize}
\item Mobiliser le sport comme un outil permettant de renforcer les liens entre les établissements d’enseignement, leur environnement et le milieu associatif pour :
	\begin{itemize}
	\item utiliser le sport scolaire comme un outil permettant de créer ou de renforcer du lien avec les parents d’élèves, notamment ceux qui sont les plus éloignés de l’école ;
	\item développer quantitativement et qualitativement la pratique du sport scolaire, en mettant l’accent sur les établissements de l’éducation prioritaire et la pratique des filles ;
	\item améliorer la qualité de la vie étudiante sur le site grâce à un renforcement des partenariats externes et de la collaboration entre services en interne à l’établissement ;
	\item corriger les inégalités d’accès à la pratique sportive quelles qu’en soient les causes : territoriales, sociales, sexuées, culturelles ou bien liées à un handicap.
	\end{itemize}
\end{itemize}
    
\section{Modalités de l’opération}
\textbf{L’Année du sport de l’école à l’université} doit permettre la construction d’actions partenariales et fédératrices qui bénéficieront au plus grand nombre. L’ensemble des établissements scolaires du premier et second degrés, de métropole et des outre-mer, mais également les établissements français de l’étranger, ainsi que les établissements d’enseignement supérieur, pourront s’inscrire dans le cadre de l’Année du sport en proposant des actions ou projets liés au sport et à sa dimension éducative.

Ces démarches pourront être liées à des opérations d’ampleur nationale existantes (rencontres et compétitions nationales ou internationales organisées par l’Usep, l’UNSS, l’Ugsel et la FFSU) ou à des initiatives strictement locales (projets à l’échelle d’une classe, d’un établissement ou d’un réseau d’établissements, à l’initiative d’un Suaps, d’une grande école ou d’une association étudiante...).

Au-delà des grands évènements sportifs, ces initiatives pourront également s’appuyer sur des temps forts comme la \textbf{Journée du sport scolaire} qui se déroulera le mercredi 16 septembre 2015 ou bien la \textbf{Journée sport campus} d’octobre 2015.

L’ensemble des démarches menées fera l’objet d’un recensement au niveau national afin de valoriser dans une programmation annuelle les actions liées au sport et à sa dimension éducative en milieu scolaire et dans l’enseignement supérieur.

\section{Partenariats nationaux et locaux}
De nombreux partenaires issus du mouvement sportif se sont engagés aux côtés du ministère de l’éducation nationale, de l’enseignement supérieur et de la recherche pour porter ce projet. Au-delà du rôle majeur que seront amenés à jouer les fédérations sportives scolaires et universitaires (Usep, UNSS, Ugsel, FFSU) et le groupement des directeurs de Suaps, le Comité national olympique et sportif français (CNOSF) sera un relais important pour mobiliser les fédérations sportives civiles, dans le cadre de la convention cadre signée le 18 septembre 2013 entre le ministère chargé de l’éducation nationale, le ministère chargé des sports et le CNOSF.

Il conviendra que ce partenariat puisse se décliner au niveau local dans les académies avec les acteurs du mouvement sportif (comités régionaux et départementaux olympiques et sportifs, clubs sportifs...) et plus largement avec l’ensemble de la communauté éducative (parents d’élèves, collectivités locales, associations complémentaires de l’école, associations d’étudiants...) qui peuvent contribuer à enrichir et faire rayonner les actions entreprises par les écoles et établissements.

\section{Pilotage des actions}
Un comité de pilotage national réunissant les services de l’administration centrale, les acteurs du sport scolaire et universitaire et plus largement du mouvement sportif, ainsi que des représentants du ministère chargé des sports et du ministère chargé de l’enseignement agricole, est mis en place pour coordonner cette Année du sport de l’école à l’université. Il travaillera notamment à la mise en valeur des différentes manifestations retenues dans le calendrier événementiel de l’année.

Au sein des académies et des directions départementales des services de l’éducation nationale, les corps d’inspection (IA-IPR EPS, IEN) pourront être missionnés spécifiquement pour accompagner les initiatives locales en lien avec l’Année du sport de l’école à l’université. Des comités de pilotage pourront être mis en place pour associer les partenaires désireux de s’investir dans l’opération.

Afin de mobiliser largement la communauté éducative et de favoriser sa réussite, l’opération pourra être présentée aux conseils académiques et départementaux de l’éducation nationale, ainsi qu’au conseil académique de la vie lycéenne, au cours du troisième trimestre de l’année scolaire 2014-2015. Dans le cadre d’un plan national de formation, les différents acteurs académiques seront réunis dans le courant du mois de mai 2015 pour préparer cette opération.

Les établissements d’enseignement supérieur veilleront à mobiliser l’ensemble des étudiants et des personnels et pourront participer aux comités de pilotage académique.

\section{Labellisation et valorisation des actions}
Dans le cadre de cette opération nationale de promotion du sport pour tous, le label \textbf{Année du sport de l’école à l’université} est attribué par le ministère aux actions qui remplissent les conditions suivantes :
\begin{itemize}
\item se dérouler sur tout ou partie de l’année scolaire et universitaire 2015 - 2016 ;
\item reposer sur un partenariat entre d’une part une école, un établissement scolaire, un établissement d’enseignement supérieur, d’autre part, un acteur du mouvement sportif (fédérations et associations sportives scolaires, universitaires ou civiles, ligues, clubs...) et/ou une association menant des actions en direction du public scolaire ou étudiant (associations complémentaires de l’école, associations étudiantes, fondations...) ;
\item lier la pratique sportive à une ambition éducative, culturelle ou citoyenne.
\end{itemize}

Afin de solliciter ce label, les porteurs de projet doivent inscrire leur action sur la page dédiée \href{http://www.eduscol.education.fr/annee-du-sport}{eduscol.education.fr/annee-du-sport} du site ministériel.

Cette labellisation permet aux porteurs de projet de bénéficier d’un kit de communication numérique déclinable sur tous types de supports imprimés et numériques, élaboré par le ministère de l’éducation nationale, de l’enseignement supérieur et de la recherche, et de faire connaitre largement les actions proposées.

L’évènement labellisé sera référencé dans le calendrier officiel de l’Année du sport de l’école à l’université et pourra faire l’objet d’une valorisation particulière sur les réseaux sociaux.

En outre, certains projets labellisés pourront bénéficier d’un soutien financier dans les conditions prévues par le cahier des charges téléchargeable à l’adresse \href{http://www.eduscol.education.fr/annee-du-sport}{eduscol.education.fr/annee-du-sport}

\section{Création de ressources pédagogiques}
Pour valoriser le sport comme outil pédagogique, le ministère a sollicité le réseau Canopé pour apporter son concours à l’opération par :
\begin{itemize}
\item la création ou la valorisation de ressources pédagogiques. Ces ressources prendront la forme de documents pédagogiques transdisciplinaires, de médias divers (films, expositions, textes ou images) et d’une plateforme dédiée à l’opération en direction des enseignants, des élèves et étudiants, des parents et des animateurs ;
\item la valorisation d’évènements. Canopé réalisera des captations audiovisuelles (conférences, interviews, etc.) pour accompagner les évènements phares de l’année, et animer des ateliers au niveau local.
\end{itemize}

Nous vous remercions du concours que vous apporterez au développement de l’Année du sport de l’école à l’université, au service de la réussite de tous les élèves et tous les étudiants.