Le socle commun de connaissances, de compétences et de culture couvre la période de la scolarité obligatoire, c’est-à-dire dix années fondamentales de la vie et de la formation des enfants, de six à seize ans. Il correspond pour l’essentiel aux enseignements de l’école élémentaire et du collège qui constituent une culture scolaire commune. Précédée pour la plupart des élèves par une scolarisation en maternelle qui a permis de poser de premières bases en matière d’apprentissage et de vivre ensemble, la scolarité obligatoire poursuit un double objectif de formation et de socialisation. Elle donne aux élèves une culture commune, fondée sur les connaissances et compétences indispensables, qui leur permettra de s’épanouir personnellement, de développer leur sociabilité, de réussir la suite de leur parcours de formation, de s’insérer dans la société où ils vivront et de participer, comme citoyens, à son évolution. Le socle commun doit devenir une référence centrale pour le travail des enseignants et des acteurs du système éducatif, en ce qu’il définit les finalités de la scolarité obligatoire et qu’il a pour exigence que l’École tienne sa promesse pour tous les élèves.

Le socle commun doit être équilibré dans ses contenus et ses démarches :
\begin{itemize}
\item il ouvre à la connaissance, forme le jugement et l’esprit critique, à partir d’éléments ordonnés de connaissance rationnelle du monde ;
\item il fournit une éducation générale ouverte et commune à tous et fondée sur des valeurs qui permettent de vivre dans une société tolérante, de liberté ;
\item il favorise un développement de la personne en interaction avec le monde qui l’entoure ;
\item il développe les capacités de compréhension et de création, les capacités d’imagination et d’action ;
\item il accompagne et favorise le développement physique, cognitif et sensible des élèves, en respectant leur intégrité ;
\item il donne aux élèves les moyens de s’engager dans les activités scolaires, d’agir, d’échanger avec autrui, de conquérir leur autonomie et d’exercer ainsi progressivement leur liberté et leur statut de citoyen responsable.\end{itemize}

L’élève engagé dans la scolarité apprend à réfléchir, à mobiliser des connaissances, à choisir des démarches et des procédures adaptées, pour penser, résoudre un problème, réaliser une tâche complexe ou un projet, en particulier dans une situation nouvelle ou inattendue. Les enseignants définissent les modalités les plus pertinentes pour parvenir à ces objectifs en suscitant l’intérêt des élèves, et centrent leurs activités ainsi que les pratiques des enfants et des adolescents sur de véritables enjeux intellectuels, riches de sens et de progrès.
 
Le socle commun identifie les connaissances et compétences qui doivent être acquises à l’issue de la scolarité obligatoire. Une compétence est l’aptitude à mobiliser ses ressources (connaissances, capacités, attitudes) pour accomplir une tâche ou faire face à une situation complexes ou inédites. Compétences et connaissances ne sont ainsi pas en opposition. Leur acquisition suppose de prendre en compte dans le processus d’apprentissage les vécus et les représentations des élèves, pour les mettre en perspective, enrichir et faire évoluer leur expérience du monde.

Par la loi d’orientation et de programmation pour la refondation de l’École de la République du 8 juillet 2013, la République s’engage afin de permettre à tous les élèves d’acquérir le socle commun de connaissances, de compétences et de culture, porteur de la culture commune. Il s’agit de contribuer au succès d’une école de la réussite pour tous, qui refuse exclusions et discriminations et qui permet à chacun de développer tout son potentiel par la meilleure éducation possible.
 
La logique du socle commun implique une acquisition progressive et continue des connaissances et des compétences par l’élève, comme le rappelle l’intitulé des cycles d’enseignement de la scolarité obligatoire que le socle commun oriente : cycle~2 des apprentissages fondamentaux, cycle~3 de consolidation, cycle~4 des approfondissements. Ainsi, la maitrise des acquis du socle commun doit se concevoir dans le cadre du parcours scolaire de l’élève et en référence aux attendus et objectifs de formation présentés par les programmes de chaque cycle. La vérification de cette maitrise progressive est faite tout au long du parcours scolaire et en particulier à la fin de chaque cycle. Cela contribue à un suivi des apprentissages de l’élève. Pour favoriser cette maitrise, des stratégies d’accompagnement sont à mettre en œuvre dans le cadre de la classe, ou, le cas échéant, des groupes à effectifs réduits constitués à cet effet.

\section{Domaine 1. Les langages pour penser et communiquer}
Le domaine des langages pour penser et communiquer recouvre quatre types de langage, qui sont à la fois des objets de savoir et des outils : la langue française ; les langues vivantes étrangères ou régionales ; les langages mathématiques, scientifiques et informatiques ; les langages des arts et du corps. Ce domaine permet l’accès à d’autres savoirs et à une culture rendant possible l’exercice de l’esprit critique ; il implique la maitrise de codes, de règles, de systèmes de signes et de représentations. Il met en jeu des connaissances et des compétences qui sont sollicitées comme outils de pensée, de communication, d’expression et de travail et qui sont utilisées dans tous les champs du savoir et dans la plupart des activités.

\subsection{Objectifs de connaissances et de compétences pour la maitrise du socle commun}
\subsubsection{Comprendre, s’exprimer en utilisant la langue française à l’oral et à l’écrit}
L’élève parle, communique, argumente à l’oral de façon claire et organisée ; il adapte son niveau de langue et son discours à la situation, il écoute et prend en compte ses interlocuteurs.

Il adapte sa lecture et la module en fonction de la nature et de la difficulté du texte. Pour construire ou vérifier le sens de ce qu’il lit, il combine avec pertinence et de façon critique les informations explicites et implicites issues de sa lecture. Il découvre le plaisir de lire.

L’élève s’exprime à l’écrit pour raconter, décrire, expliquer ou argumenter de façon claire et organisée. Lorsque c’est nécessaire, il reprend ses écrits pour rechercher la formulation qui convient le mieux et préciser ses intentions et sa pensée.

Il utilise à bon escient les principales règles grammaticales et orthographiques. Il emploie à l’écrit comme à l’oral un vocabulaire juste et précis.

Dans des situations variées, il recourt, de manière spontanée et avec efficacité, à la lecture comme à l’écriture.

Il apprend que la langue française a des origines diverses et qu’elle est toujours en évolution. Il est sensibilisé à son histoire et à ses origines latines et grecques.

\subsubsection{Comprendre, s’exprimer en utilisant une langue étrangère et, le cas échéant, une langue régionale}
L’élève pratique au moins deux langues vivantes étrangères ou, le cas échéant, une langue étrangère et une langue régionale.

Pour chacune de ces langues, il comprend des messages oraux et écrits, s’exprime et communique à l’oral et à l’écrit de manière simple mais efficace. Il s’engage volontiers dans le dialogue et prend part activement à des conversations. Il adapte son niveau de langue et son discours à la situation, il écoute et prend en compte ses interlocuteurs. Il maitrise suffisamment le code de la langue pratiquée pour s’insérer dans une communication liée à la vie quotidienne : vocabulaire, prononciation, construction des phrases ; il possède aussi des connaissances sur le contexte culturel propre à cette langue (modes de vie, organisations sociales, traditions, expressions artistiques\dots).

L'élève utilise les principes du système de numération décimal et les langages formels (lettres, symboles...) propres aux mathématiques et aux disciplines scientifiques, notamment pour effectuer des calculs et modéliser des situations. Il lit des plans, se repère sur des cartes. Il produit et utilise des représentations d'objets, d'expériences, de phénomènes naturels tels que schémas, croquis, maquettes, patrons ou figures géométriques. Il lit, interprète, commente, produit des tableaux, des graphiques et des diagrammes organisant des données de natures diverses.
Il sait que des langages informatiques sont utilisés pour programmer des outils numériques et réaliser des traitements automatiques de données. Il connaît les principes de base de l'algorithmique et de la conception des programmes informatiques. Il les met en œuvre pour créer des applications simples.

\subsubsection{Comprendre, s’exprimer en utilisant les langages mathématiques, scientifiques et informatiques}
L’élève utilise les principes du système de numération décimal et les langages formels (lettres, symboles\dots) propres aux mathématiques et aux disciplines scientifiques, notamment pour effectuer des calculs et modéliser des situations. Il lit des plans, se repère sur des cartes. Il produit et utilise des représentations d’objets, d’expériences, de phénomènes naturels tels que schémas, croquis, maquettes, patrons ou figures géométriques. Il lit, interprète, commente, produit des tableaux, des graphiques et des diagrammes organisant des données de natures diverses.

Il sait que des langages informatiques sont utilisés pour programmer des outils numériques et réaliser des traitements automatiques de données. Il connait les principes de base de l’algorithmique et de la conception des programmes informatiques. Il les met en œuvre pour créer des applications simples.

\subsubsection{Comprendre, s’exprimer en utilisant les langages des arts et du corps}
Sensibilisé aux démarches artistiques, l’élève apprend à s’exprimer et communiquer par les arts, de manière individuelle et collective, en concevant et réalisant des productions, visuelles, plastiques, sonores ou verbales notamment. Il connait et comprend les particularités des différents langages artistiques qu’il emploie. Il justifie ses intentions et ses choix en s’appuyant sur des notions d’analyse d’œuvres.

Il s’exprime par des activités, physiques, sportives ou artistiques, impliquant le corps. Il apprend ainsi le contrôle et la maitrise de soi.

\section{Domaine 2. Les méthodes et outils pour apprendre}
Ce domaine a pour objectif de permettre à tous les élèves d’apprendre à apprendre, seuls ou collectivement, en classe ou en dehors, afin de réussir dans leurs études et, par la suite, se former tout au long de la vie. Les méthodes et outils pour apprendre doivent faire l’objet d’un apprentissage explicite en situation, dans tous les enseignements et espaces de la vie scolaire.

En classe, l’élève est amené à résoudre un problème, comprendre un document, rédiger un texte, prendre des notes, effectuer une prestation ou produire des objets. Il doit savoir apprendre une leçon, rédiger un devoir, préparer un exposé, prendre la parole, travailler à un projet, s’entrainer en choisissant les démarches adaptées aux objectifs d’apprentissage préalablement explicités. Ces compétences requièrent l’usage de tous les outils théoriques et pratiques à sa disposition, la fréquentation des bibliothèques et centres de documentation, la capacité à utiliser de manière pertinente les technologies numériques pour faire des recherches, accéder à l’information, la hiérarchiser et produire soi-même des contenus.

La maitrise des méthodes et outils pour apprendre développe l’autonomie et les capacités d’initiative ; elle favorise l’implication dans le travail commun, l’entraide et la coopération.
 
\subsection{Objectifs de connaissances et de compétences pour la maitrise du socle commun}
\subsubsection{Organisation du travail personnel}
L’élève se projette dans le temps, anticipe, planifie ses tâches. Il gère les étapes d’une production, écrite ou non, mémorise ce qui doit l’être.

Il comprend le sens des consignes ; il sait qu’un même mot peut avoir des sens différents selon les disciplines.

Pour acquérir des connaissances et des compétences, il met en œuvre les capacités essentielles que sont l’attention, la mémorisation, la mobilisation de ressources, la concentration, l’aptitude à l’échange et au questionnement, le respect des consignes, la gestion de l’effort.

Il sait identifier un problème, s’engager dans une démarche de résolution, mobiliser les connaissances nécessaires, analyser et exploiter les erreurs, mettre à l’essai plusieurs solutions, accorder une importance particulière aux corrections.

L’élève sait se constituer des outils personnels grâce à des écrits de travail, y compris numériques : notamment prise de notes, brouillons, fiches, lexiques, nomenclatures, cartes mentales, plans, croquis, dont il peut se servir pour s’entrainer, réviser, mémoriser.

\subsubsection{Coopération et réalisation de projets}
L’élève travaille en équipe, partage des tâches, s’engage dans un dialogue constructif, accepte la contradiction tout en défendant son point de vue, fait preuve de diplomatie, négocie et recherche un consensus.

Il apprend à gérer un projet, qu’il soit individuel ou collectif. Il en planifie les tâches, en fixe les étapes et évalue l’atteinte des objectifs.

L’élève sait que la classe, l’école, l’établissement sont des lieux de collaboration, d’entraide et de mutualisation des savoirs. Il aide celui qui ne sait pas comme il apprend des autres. L’utilisation des outils numériques contribue à ces modalités d’organisation, d’échange et de collaboration.

\subsubsection{Médias, démarches de recherche et de traitement de l’information}
L’élève connait des éléments d’histoire de l’écrit et de ses différents supports. Il comprend les modes de production et le rôle de l’image.

Il sait utiliser de façon réfléchie des outils de recherche, notamment sur Internet. Il apprend à confronter différentes sources et à évaluer la validité des contenus. Il sait traiter les informations collectées, les organiser, les mémoriser sous des formats appropriés et les mettre en forme. Il les met en relation pour construire ses connaissances.

L’élève apprend à utiliser avec discernement les outils numériques de communication et d’information qu’il côtoie au quotidien, en respectant les règles sociales de leur usage et toutes leurs potentialités pour apprendre et travailler. Il accède à un usage sûr, légal et éthique pour produire, recevoir et diffuser de l’information. Il développe une culture numérique.

Il identifie les différents médias (presse écrite, audiovisuelle et Web) et en connait la nature. Il en comprend les enjeux et le fonctionnement général afin d’acquérir une distance critique et une autonomie suffisantes dans leur usage.

\subsubsection{Outils numériques pour échanger et communiquer}
L’élève sait mobiliser différents outils numériques pour créer des documents intégrant divers médias et les publier ou les transmettre, afin qu’ils soient consultables et utilisables par d’autres. Il sait réutiliser des productions collaboratives pour enrichir ses propres réalisations, dans le respect des règles du droit d’auteur.

L’élève utilise les espaces collaboratifs et apprend à communiquer notamment par le biais des réseaux sociaux dans le respect de soi et des autres. Il comprend la différence entre sphères publique et privée. Il sait ce qu’est une identité numérique et est attentif aux traces qu’il laisse.

\section{Domaine 3. La formation de la personne et du citoyen}
L’École a une responsabilité particulière dans la formation de l’élève en tant que personne et futur citoyen. Dans une démarche de coéducation, elle ne se substitue pas aux familles, mais elle a pour tâche de transmettre aux jeunes les valeurs fondamentales et les principes inscrits dans la Constitution de notre pays. Elle permet à l’élève d’acquérir la capacité à juger par lui-même, en même temps que le sentiment d’appartenance à la société. Ce faisant, elle permet à l’élève de développer dans les situations concrètes de la vie scolaire son aptitude à vivre de manière autonome, à participer activement à l’amélioration de la vie commune et à préparer son engagement en tant que citoyen.

Ce domaine fait appel :
\begin{itemize}
\item à l’apprentissage et à l’expérience des principes qui garantissent la liberté de tous, comme la liberté de conscience et d’expression, la tolérance réciproque, l’égalité, notamment entre les hommes et les femmes, le refus des discriminations, l’affirmation de la capacité à juger et agir par soi-même ;
\item à des connaissances et à la compréhension du sens du droit et de la loi, des règles qui permettent la participation à la vie collective et démocratique et de la notion d’intérêt général ;
\item à la connaissance, la compréhension mais aussi la mise en pratique du principe de laïcité, qui permet le déploiement du civisme et l’implication de chacun dans la vie sociale, dans le respect de la liberté de conscience.
\end{itemize}

Ce domaine est mis en œuvre dans toutes les situations concrètes de la vie scolaire où connaissances et valeurs trouvent, en s’exerçant, les conditions d’un apprentissage permanent, qui procède par l’exemple, par l’appel à la sensibilité et à la conscience, par la mobilisation du vécu et par l’engagement de chacun.
 
\subsection{Objectifs de connaissances et de compétences pour la maitrise du socle commun}
\subsubsection{Expression de la sensibilité et des opinions, respect des autres}
L’élève exprime ses sentiments et ses émotions en utilisant un vocabulaire précis.

Il exploite ses facultés intellectuelles et physiques en ayant confiance en sa capacité à réussir et à progresser.

L’élève apprend à résoudre les conflits sans agressivité, à éviter le recours à la violence grâce à sa maitrise de moyens d’expression, de communication et d’argumentation. Il respecte les opinions et la liberté d’autrui, identifie et rejette toute forme d’intimidation ou d’emprise.  Apprenant à mettre à distance préjugés et stéréotypes, il est capable d’apprécier les personnes qui sont différentes de lui et de vivre avec elles. Il est capable aussi de faire preuve d’empathie et de bienveillance.

\subsubsection{La règle et le droit}
L’élève comprend et respecte les règles communes, notamment les règles de civilité, au sein de la classe, de l’école ou de l’établissement, qui autorisent et contraignent à la fois et qui engagent l’ensemble de la communauté éducative. Il participe à la définition de ces règles dans le cadre adéquat. Il connait le rôle éducatif et la gradation des sanctions ainsi que les grands principes et institutions de la justice.

Il comprend comment, dans une société démocratique, des valeurs communes garantissent les libertés individuelles et collectives, trouvent force d’application dans des règles et dans le système du droit, que les citoyens peuvent faire évoluer selon des procédures organisées.

Il connait les grandes déclarations des droits de l’homme (notamment la Déclaration des droits de l’homme et du citoyen de 1789, la Déclaration universelle des droits de l’homme de 1948), la Convention européenne de sauvegarde des droits de l’homme, la Convention internationale des droits de l’enfant de 1989 et les principes fondateurs de la République française. Il connait le sens du principe de laïcité ; il en mesure la profondeur historique et l’importance pour la démocratie dans notre pays. Il comprend que la laïcité garantit la liberté de conscience, fondée sur l’autonomie du jugement de chacun et institue des règles permettant de vivre ensemble pacifiquement.

Il connait les principales règles du fonctionnement institutionnel de l’Union européenne et les grands objectifs du projet européen.

\subsubsection{Réflexion et discernement}
L’élève est attentif à la portée de ses paroles et à la responsabilité de ses actes.

Il fonde et défend ses jugements en s’appuyant sur sa réflexion et sur sa maitrise de l’argumentation. Il comprend les choix moraux que chacun fait dans sa vie ; il peut discuter de ces choix ainsi que de quelques grands problèmes éthiques liés notamment aux évolutions sociales, scientifiques ou techniques.

L’élève vérifie la validité d’une information et distingue ce qui est objectif et ce qui est subjectif. Il apprend à justifier ses choix et à confronter ses propres jugements avec ceux des autres. Il sait remettre en cause ses jugements initiaux après un débat argumenté, il distingue son intérêt particulier de l’intérêt général. Il met en application et respecte les grands principes républicains.

\subsubsection{Responsabilité, sens de l’engagement et de l’initiative}
L’élève coopère et fait preuve de responsabilité vis-à-vis d’autrui. Il respecte les engagements pris envers lui-même et envers les autres, il comprend l’importance du respect des contrats dans la vie civile. Il comprend en outre l’importance de s’impliquer dans la vie scolaire (actions et projets collectifs, instances), d’avoir recours aux outils de la démocratie (ordre du jour, compte rendu, votes notamment) et de s’engager aux côtés des autres dans les différents aspects de la vie collective et de l’environnement.

L’élève sait prendre des initiatives, entreprendre et mettre en œuvre des projets, après avoir évalué les conséquences de son action ; il prépare ainsi son orientation future et sa vie d’adulte.

\section{Domaine 4. Les systèmes naturels et les systèmes techniques}
Ce domaine a pour objectif de donner à l’élève les fondements de la culture mathématique, scientifique et technologique nécessaire à une découverte de la nature et de ses phénomènes, ainsi que des techniques développées par les femmes et les hommes. Il s’agit d’éveiller sa curiosité, son envie de se poser des questions, de chercher des réponses et d’inventer, tout en l’initiant à de grands défis auxquels l’humanité est confrontée. L’élève découvre alors, par une approche scientifique, la nature environnante. L’objectif est bien de poser les bases lui permettant de pratiquer des démarches scientifiques et techniques.

Fondées sur l’observation, la manipulation et l’expérimentation, utilisant notamment le langage des mathématiques pour leurs représentations, les démarches scientifiques ont notamment pour objectif d’expliquer l’Univers, d’en comprendre les évolutions, selon une approche rationnelle privilégiant les faits et hypothèses vérifiables, en distinguant ce qui est du domaine des opinions et croyances. Elles développent chez l’élève la rigueur intellectuelle, l’habileté manuelle et l’esprit critique, l’aptitude à démontrer, à argumenter.

La familiarisation de l’élève avec le monde technique passe par la connaissance du fonctionnement d’un certain nombre d’objets et de systèmes et par sa capacité à en concevoir et en réaliser lui-même. Ce sont des occasions de prendre conscience que la démarche technologique consiste à rechercher l’efficacité dans un milieu contraint (en particulier par les ressources) pour répondre à des besoins humains, en tenant compte des impacts sociaux et environnementaux.

En s’initiant à ces démarches, concepts et outils, l’élève se familiarise avec les évolutions de la science et de la technologie ainsi que leur histoire, qui modifient en permanence nos visions et nos usages de la planète.

L’élève comprend que les mathématiques permettent de développer une représentation scientifique des phénomènes, qu’elles offrent des outils de modélisation, qu’elles se nourrissent des questions posées par les autres domaines de connaissance et les nourrissent en retour.
 
\subsection{Objectifs de connaissances et de compétences pour la maitrise du socle commun}
\subsubsection{Démarches scientifiques}
L’élève sait mener une démarche d’investigation. Pour cela, il décrit et questionne ses observations ; il prélève, organise et traite l’information utile ; il formule des hypothèses, les teste et les éprouve ; il manipule, explore plusieurs pistes, procède par essais et erreurs ; il modélise pour représenter une situation ; il analyse, argumente, mène différents types de raisonnements (par analogie, déduction logique...) ; il rend compte de sa démarche. Il exploite et communique les résultats de mesures ou de recherches en utilisant les langages scientifiques à bon escient.

L’élève pratique le calcul, mental et écrit, exact et approché, il estime et contrôle les résultats, notamment en utilisant les ordres de grandeur. Il résout des problèmes impliquant des grandeurs variées (géométriques, physiques, économiques...), en particulier des situations de proportionnalité. Il interprète des résultats statistiques et les représente graphiquement.

\subsubsection{Conception, création, réalisation}
L’élève imagine, conçoit et fabrique des objets et des systèmes techniques. Il met en œuvre observation, imagination, créativité, sens de l’esthétique et de la qualité, talent et habileté manuels, sens pratique, et sollicite les savoirs et compétences scientifiques, technologiques et artistiques pertinents.

\subsubsection{Responsabilités individuelles et collectives}
L’élève connait l’importance d’un comportement responsable vis-à-vis de l’environnement et de la santé et comprend ses responsabilités individuelle et collective. Il prend conscience de l’impact de l’activité humaine sur l’environnement, de ses conséquences sanitaires et de la nécessité de préserver les ressources naturelles et la diversité des espèces. Il prend conscience de la nécessité d’un développement plus juste et plus attentif à ce qui est laissé aux générations futures.

Il sait que la santé repose notamment sur des fonctions biologiques coordonnées, susceptibles d’être perturbées par des facteurs physiques, chimiques, biologiques et sociaux de l’environnement et que certains de ces facteurs de risques dépendent de conduites sociales et de choix personnels. Il est conscient des enjeux de bien-être et de santé des pratiques alimentaires et physiques. Il observe les règles élémentaires de sécurité liées aux techniques et produits rencontrés dans la vie quotidienne.
 
Pour atteindre les objectifs de connaissances et de compétences de ce domaine, l’élève mobilise des connaissances sur :
\begin{itemize}
\item les principales fonctions du corps humain, les caractéristiques et l’unité du monde vivant, l’évolution et la diversité des espèces ;
\item la structure de l’Univers et de la matière; les grands caractères de la biosphère et leurs transformations ;
\item l’énergie et ses multiples formes, le mouvement et les forces qui le régissent ;
\item les nombres et les grandeurs, les objets géométriques, la gestion de données, les phénomènes aléatoires ;
\item les grandes caractéristiques des objets et systèmes techniques et des principales solutions technologiques.
\end{itemize}

\section{Domaine 5. Les représentations du monde et l’activité humaine}
Ce domaine est consacré à la compréhension du monde que les êtres humains tout à la fois habitent et façonnent. Il s’agit de développer une conscience de l’espace géographique et du temps historique. Ce domaine conduit aussi à étudier les caractéristiques des organisations et des fonctionnements des sociétés. Il initie à la diversité des expériences humaines et des formes qu’elles prennent : les découvertes scientifiques et techniques, les diverses cultures, les systèmes de pensée et de conviction, l’art et les œuvres, les représentations par lesquelles les femmes et les hommes tentent de comprendre la condition humaine et le monde dans lequel ils vivent.

Ce domaine vise également à développer des capacités d’imagination, de conception, d’action pour produire des objets, des services et des œuvres ainsi que le gout des pratiques artistiques, physiques et sportives. Il permet en outre la formation du jugement et de la sensibilité esthétiques. Il implique enfin une réflexion sur soi et sur les autres, une ouverture à l’altérité, et contribue à la construction de la citoyenneté, en permettant à l’élève d’aborder de façon éclairée de grands débats du monde contemporain.
 
\subsection{Objectifs de connaissances et de compétences pour la maitrise du socle commun}
\subsubsection{L’espace et le temps}
L’élève identifie ainsi les grandes questions et les principaux enjeux du développement humain, il est capable d’appréhender les causes et les conséquences des inégalités, les sources de conflits et les solidarités, ou encore les problématiques mondiales concernant l’environnement, les ressources, les échanges, l’énergie, la démographie et le climat. Il comprend également que les lectures du passé éclairent le présent et permettent de l’interpréter.

L’élève se repère dans l’espace à différentes échelles, il comprend les grands espaces physiques et humains et les principales caractéristiques géographiques de la Terre, du continent européen et du territoire national : organisation et localisations, ensembles régionaux, outre-mer. Il sait situer un lieu ou un ensemble géographique en utilisant des cartes, en les comparant et en produisant lui-même des représentations graphiques.

\subsubsection{Organisations et représentations du monde}
L’élève lit des paysages, identifiant ce qu’ils révèlent des atouts et des contraintes du milieu ainsi que de l’activité humaine, passée et présente. Il établit des liens entre l’espace et l’organisation des sociétés.

Il exprime à l’écrit et à l’oral ce qu’il ressent face à une œuvre littéraire ou artistique ; il étaye ses analyses et les jugements qu’il porte sur l’œuvre ; il formule des hypothèses sur ses significations et en propose une interprétation en s’appuyant notamment sur ses aspects formels et esthétiques. Il justifie ses intentions et ses choix expressifs, en s’appuyant sur quelques notions d’analyse des œuvres. Il s’approprie, de façon directe ou indirecte, notamment dans le cadre de sorties scolaires culturelles, des œuvres littéraires et artistiques appartenant au patrimoine national et mondial comme à la création contemporaine.

\subsubsection{Invention, élaboration, production}
L’élève imagine, conçoit et réalise des productions de natures diverses, y compris littéraires et artistiques. Pour cela, il met en œuvre des principes de conception et de fabrication d’objets ou les démarches et les techniques de création. Il tient compte des contraintes des matériaux et des processus de production en respectant l’environnement. Il mobilise son imagination et sa créativité au service d’un projet personnel ou collectif. Il développe son jugement, son gout, sa sensibilité, ses émotions esthétiques.

Il connait les contraintes et les libertés qui s’exercent dans le cadre des activités physiques et sportives ou artistiques personnelles et collectives. Il sait en tirer parti et gère son activité physique et sa production ou sa performance artistiques pour les améliorer, progresser et se perfectionner. Il cherche et utilise des techniques pertinentes, il construit des stratégies pour réaliser une performance sportive. Dans le cadre d’activités et de projets collectifs, il prend sa place dans le groupe en étant attentif aux autres pour coopérer ou s’affronter dans un cadre règlementé.
 
Pour mieux connaitre le monde qui l’entoure comme pour se préparer à l’exercice futur de sa citoyenneté démocratique, l’élève pose des questions et cherche des réponses en mobilisant des connaissances sur :
\begin{itemize}
\item les principales périodes de l’histoire de l’humanité, situées dans leur chronologie, les grandes ruptures et les évènements fondateurs, la notion de civilisation ;
\item les principaux modes d’organisation des espaces humanisés ;
\item la diversité des modes de vie et des cultures, en lien avec l’apprentissage des langues ;
\item les éléments clés de l’histoire des idées, des faits religieux et des convictions ;
\item les grandes découvertes scientifiques et techniques et les évolutions qu’elles ont engendrées, tant dans les modes de vie que dans les représentations ;
\item les expressions artistiques, les œuvres, les sensibilités esthétiques et les pratiques culturelles de différentes sociétés ;
\item les principaux modes d’organisation politique et sociale, idéaux et principes républicains et démocratiques, leur histoire et leur actualité ;
\item les principales manières de concevoir la production économique, sa répartition, les échanges qu’elles impliquent ;
\item les règles et le droit de l’économie sociale et familiale, du travail, de la santé et de la protection sociale.\end{itemize}