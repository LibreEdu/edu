\section*{Introduction}
Troisième rentrée de la refondation de l’École de la République, celle de septembre~2015 ouvre une étape déterminante.

L’année scolaire~2015-2016 doit d’abord mobiliser tous les acteurs de l’école pour engager ou poursuivre l’évolution des contenus d’enseignement et des pratiques pédagogiques au service de la lutte contre les inégalités et renforcer la transmission des valeurs de la République. C’est tout le sens, notamment, de la priorité au premier degré, mise en œuvre depuis deux ans et qui doit continuer à se déployer pour que chaque enfant puisse disposer, dès son entrée à l’école, des meilleures conditions pour nouer et développer ses apprentissages.

Dans la continuité de la mobilisation pour les valeurs de la République, le dialogue renouvelé avec les familles et les autres partenaires de l’école -- élus locaux, acteurs du monde associatif et du monde professionnel -- constituera un levier tout aussi essentiel pour faire réussir les élèves, transmettre les valeurs républicaines et lutter contre les inégalités.

L’année~2015-2016 sera aussi une année de préparation des personnels aux évolutions majeures de la rentrée~2016. L’opportunité d’une modification simultanée de l’organisation pédagogique du collège et du renouvèlement des contenus d’enseignement dans le cadre de la scolarité obligatoire, tout comme le lancement du grand plan numérique pour l’éducation, imposent la mobilisation et l’engagement de tous, dès cette année, pour créer les conditions du changement. Cette préparation se fera par un effort de formation important et un travail collectif, dans chaque collège, pour définir le projet pédagogique qui sera mis en œuvre à la rentrée~2016, pour construire les enseignements pratiques interdisciplinaires (EPI), l’accompagnement personnalisé et les modalités d’utilisation de la marge d’heures professeurs. Elle pourra s’appuyer sur les innovations pédagogiques développées par les enseignants à l’échelle de la classe et de l’établissement pour répondre aux difficultés scolaires.

Permettre aux équipes enseignantes et éducatives et à chaque professionnel de l’école de s’approprier l’ensemble des évolutions en cours et de faire leurs les ambitions affichées constituera donc un enjeu majeur de l’année scolaire à venir.

\section{Construire une école plus juste pour offrir à chaque élève un parcours de réussite}
Depuis deux ans, toutes les mesures de la refondation de l’École de la République placent la qualité des enseignements et de leur organisation au cœur du système éducatif. Ainsi, de nouveaux cycles, un nouveau socle commun de connaissances, de compétences et de culture (\href{http://www.education.gouv.fr/pid25535/bulletin_officiel.html?cid_bo=87834}{décret \no{}~2015-372 du 31~mars~2015}) et de nouveaux programmes d’enseignement ont été élaborés ; ils ont fait ou feront l’objet d’une large consultation. Les cycles, le socle et les programmes de l’école élémentaire et du collège entreront en vigueur à la rentrée~2016. La rentrée~2015 verra quant à elle la mise en œuvre du programme de l’école maternelle, de l’enseignement moral et civique et des parcours éducatifs.

\subsection{Renforcer l’acquisition du socle commun notamment grâce à la maitrise des langages}
La \textbf{maitrise de la langue et les compétences mathématiques} jouent un rôle crucial dans la réussite scolaire et l’insertion professionnelle et sociale ; leur apprentissage par chaque élève doit être encouragé très tôt et renforcé tout au long de la scolarité, en fonction de ses besoins. L’acquisition de la maitrise de la langue française et des langages scientifiques, est l’objectif premier de l’\textbf{école primaire}, dont la pédagogie doit favoriser l’épanouissement de l’élève, sa motivation et sa pleine implication dans les apprentissages.

Le temps de scolarité déterminant que constitue l’\textbf{école maternelle}, conçue comme un cycle unique et fondamental, centré sur le développement affectif, social, sensoriel, moteur et cognitif de l’enfant, vise à prévenir les difficultés, réduire les inégalités et inscrire chaque enfant dans un parcours de réussite. Pour ce faire, un \textbf{nouveau programme} sera mis en œuvre à partir de septembre~2015 (\href{http://www.education.gouv.fr/pid25535/bulletin_officiel.html?cid_bo=86940}{B.O. spécial \no{}~2 du 26~mars~2015}). Organisé en cinq domaines (\og Mobiliser le langage dans toutes ses dimensions \fg{} ; \og Agir, s’exprimer, comprendre à travers l’activité physique \fg{} ; \og Agir, s’exprimer, comprendre à travers les activités artistiques \fg{} ; \og Construire les premiers outils pour structurer sa pensée \fg{} ; \og Explorer le monde \fg{}), il porte le principe d’une école qui s’adapte aux jeunes enfants, organise des modalités spécifiques d’apprentissage et leur permet d’apprendre ensemble et de vivre ensemble. Des \textbf{ressources d’accompagnement} relatives aux besoins d’un jeune enfant et à la scolarisation des enfants de moins de trois ans, au langage oral et à la découverte de l’écrit, au jeu, au graphisme et à l’écriture, à l’exploration du vivant, des objets et de la matière, et à l’activité physique seront mises en ligne dès la rentrée~2015, pour faciliter le travail des équipes enseignantes (voir partie~III). La mise en œuvre de ce nouveau programme doit favoriser une réflexion des équipes sur l’\textbf{évaluation en maternelle}, qui privilégie l’observation des élèves au cours des activités ordinaires de la classe et permet d’apprécier leurs progrès et d’en rendre compte aux familles.

La \textbf{réforme des rythmes scolaires} désormais généralisée s’appuie sur un pilotage pédagogique renforcé, fondé sur les nouveaux programmes et l’action des inspecteurs de l’éducation nationale (IEN) en matière d’accompagnement pédagogique et de formation des équipes. Pour que les élèves puissent profiter au mieux de l’aménagement des temps d’apprentissage, les emplois du temps à l’école sont réorganisés pour situer ces derniers dans les moments où la capacité d’attention des élèves est la plus grande et instaurer une bonne qualité de vie dans l’école. À cet égard, il conviendra d’être particulièrement attentif à l’organisation de la sieste pour tenir compte de ces besoins et qu’elle ne se substitue pas au temps d’apprentissage de l’après-midi.

Initiée dès l’école maternelle, la \textbf{maitrise de la langue française} fait l’objet d’un \textbf{chantier prioritaire} tout au long de la scolarité, à chaque étape du parcours de l’enfant et du jeune, au service de sa réussite dans ses apprentissages et dans la construction de sa citoyenneté. Ainsi, devra notamment être renforcé l’enseignement du jugement, de l’argumentation et du débat en classe, à l’écrit comme à l’oral, en lien étroit avec l’enseignement moral et civique et le parcours citoyen (cf. II). La valorisation des expériences efficaces et scientifiquement accompagnées, ainsi que la démarche de recherche-action dans des domaines où existent de réels besoins, seront renforcées. Les réussites forgées par les équipes dans le cadre de la \textbf{refondation de l’éducation prioritaire}, qui est principalement une réforme pédagogique, devront être partagées et largement diffusées.

Les \textbf{dispositifs d’inclusion scolaire des élèves allophones nouvellement arrivés (EANA) et des enfants issus de familles itinérantes et de voyageurs (Efiv)} au sein des écoles et des établissements scolaires continueront de faire l’objet d’une attention particulière. Les réseaux de travail et de coopération entre les centres académiques pour la \href{http://eduscol.education.fr/pid28783/scolariser-les-eleves-allophones-et-les-enfants-des-familles-itinerantes.html}{scolarisation des enfants allophones nouvellement arrivés et des enfants issus de familles itinérantes et de voyageurs (Casnav)}, les services académiques et départementaux, les communes et les services sociaux doivent garantir l’accès rapide à l’école, la qualité du parcours scolaire et la continuité éducative pour ces élèves.

La \textbf{maitrise des savoirs et des compétences mathématiques} par tous les élèves et la \textbf{lutte contre l’innumérisme} occupent également une place importante dans la refondation pédagogique du système éducatif. \href{http://www.education.gouv.fr/cid84398/strategie-mathematiques.html}{Les 10 mesures clés de la \textbf{\og stratégie mathématiques \fg{}}} s’articulent autour de trois grands axes : des programmes de mathématiques en phase avec leur temps ; des enseignants mieux formés et mieux accompagnés pour la réussite de leurs élèves ; une image des mathématiques rénovée et dépourvue de préjugés pour favoriser en particulier l’ambition des jeunes filles et leur orientation vers les formations et métiers scientifiques. De nouvelles ressources pédagogiques seront produites ; elles proposeront notamment des situations en lien avec le quotidien, les métiers et les autres disciplines. Un \textbf{portail national dédié aux mathématiques sera créé} ; il constituera un outil de référence pour les enseignants. Dans les académies, les corps d’inspection (IEN et IA-IPR) seront mobilisés, en particulier dans le cadre des conseils école-collège. Les actions éducatives, les projets scolaires et périscolaires seront valorisés.

\textbf{Une évaluation du niveau des élèves en français et en mathématiques, à des fins diagnostiques, est mise en place au début de la classe de CE2} pour permettre aux équipes pédagogiques d’identifier les difficultés et de mettre en place une réponse adaptée aux besoins de chaque enfant. Pour les y aider, une \textbf{banque d’outils d’aide à l’évaluation diagnostique} en ligne sera mise à leur disposition durant le premier trimestre de l’année scolaire~2015-2016. Elle comportera un large choix d’items en français et en mathématiques, testés et se référant explicitement aux domaines du socle. Elle permettra aux enseignants d’évaluer les élèves au moment choisi par eux au cours des premières semaines de l’année et en fonction des objectifs poursuivis au sein de la classe.

La \textbf{continuité pédagogique} entre le collège et le lycée doit encore être renforcée pour consolider la maitrise des compétences en langue française et en mathématiques, indispensables à la poursuite des apprentissages. À cet effet, dès l’entrée en seconde générale et technologique, en seconde professionnelle et en première année de CAP, les équipes pédagogiques seront attentives aux acquis des élèves issus de troisième et organiseront, notamment dans le cadre de l’accompagnement personnalisé, le soutien adapté à ce premier diagnostic.

Enfin, la refondation entend développer les compétences des élèves en langues vivantes et favoriser l’enrichissement culturel et l’ouverture au monde. Deux dispositions entreront en vigueur à la rentrée~2016 : l’apprentissage d’une langue vivante dès le CP, prévu par la loi d’orientation et de programmation du 8~juillet~2013, et un enseignement de langue vivante 2 dès la classe de cinquième dans le cadre de la réforme du collège. Ainsi, un véritable continuum de l’école au lycée permettra de construire un parcours linguistique progressif et cohérent. Pour préparer ces évolutions, \href{http://eduscol.education.fr/pid31432/enseigner-les-langues-vivantes.html}{\textbf{de nouvelles ressources pédagogiques d’accompagnement} pour enseigner les langues} dans le premier et le second degrés ont d’ores et déjà été mises en ligne sur le site Éduscol. De plus, un \href{http://eduscol.education.fr/langues-vivantes/}{\textbf{portail national} dédié aux langues vivantes} a été créé ; il constitue désormais pour les enseignants un outil de référence pour enseigner, s’informer et se former. Dans chaque académie, une \textbf{nouvelle carte des langues vivantes} sera mise en place. Présentée en décembre~2015, elle indiquera, pour chaque école et chaque collège, les langues offertes aux élèves et s’assurera de la continuité de l’offre du cours préparatoire jusqu’à la terminale, dans toutes les voies d’enseignement et séries. Les recteurs seront chargés de son élaboration et le travail devra être finalisé à la fin du premier trimestre de l’année scolaire~2015-2016.

\subsection{Tenir compte des spécificités de chaque élève pour permettre la réussite de tous} 
Tous les enfants, sans aucune distinction, sont capables d’apprendre et de progresser : ce principe d’une école inclusive qui ne stigmatise pas les difficultés mais accompagne tous les élèves dans leur parcours scolaire constitue le cœur du \textbf{\href{http://www.education.gouv.fr/pid25535/bulletin_officiel.html?cid_bo=84055}{décret \no{}~2014-1377 du 18~novembre~2014} relatif au suivi et à l’accompagnement pédagogique des élèves} et doit concerner l’ensemble des pratiques pédagogiques. Dans son travail quotidien en classe, l’enseignant fait en sorte que chaque élève progresse au mieux dans ses apprentissages. Il ne s’agit plus seulement de répondre aux difficultés de certains élèves mais de donner à tous les moyens de progresser, en mobilisant des \textbf{pratiques pédagogiques diversifiées et différenciées}, grâce notamment aux outils et ressources numériques. Les enseignants organisent leurs enseignements en équipe afin d’assurer une \textbf{continuité des apprentissages} des élèves au sein de chaque cycle ; le \textbf{conseil école-collège} est en cela un outil important pour le cycle 3.

Le \textbf{programme personnalisé de réussite éducative} (PPRE) est désormais défini comme \og un ensemble coordonné d’actions conçu pour répondre aux besoins d’un élève lorsqu’il apparait qu’il risque de ne pas maitriser à un niveau suffisant les connaissances et compétences attendues à la fin d’un cycle \fg{}. Le \textbf{redoublement} ne peut être proposé qu’à titre \textbf{exceptionnel}, à l’issue d’un dialogue avec l’élève et sa famille, et il est proscrit à l’école maternelle.

Dans le premier degré, le travail spécifique des personnels des \textbf{réseaux d’aides spécialisées aux élèves en difficulté} (Rased) (\href{http://www.education.gouv.fr/pid25535/bulletin_officiel.html?cid_bo=81597}{circulaire \no{}~2014-107 du 18~aout~2014}), complémentaire de celui des enseignants des classes, permet de mieux répondre en équipe aux difficultés d’apprentissage et d’adaptation aux exigences scolaires que rencontrent certains élèves. Le Rased est l’une des composantes du \textbf{pôle ressource} qui, dans chaque circonscription, fédère tous les personnels que l’inspecteur de l’éducation nationale (IEN) peut solliciter pour répondre aux demandes émanant d’un enseignant ou d’une école.

Pour répondre à l’exigence d’une \textbf{école inclusive}, les élèves dont les difficultés scolaires relèvent d’un trouble des apprentissages peuvent désormais bénéficier d’un \textbf{plan d’accompagnement personnalisé} (Pap). Un \href{http://cache.media.education.gouv.fr/file/5/50/4/ensel1296_annexe_plan_daccompagnement_personnalise_386504.pdf}{document type national} est mis à disposition des équipes afin de les accompagner dans la prise en compte des besoins de l’élève. Des évolutions règlementaires permettent une meilleure prise en compte des \textbf{élèves en situation de handicap} tout au long de leur scolarité, tandis que le soutien de la Caisse nationale d’allocations familiales aide à leur accès aux activités périscolaires. Le \textbf{\href{http://www.education.gouv.fr/pid25535/bulletin_officiel.html&cid_bo=86108}{projet personnalisé de scolarisation}} (PPS) et le \href{http://www.education.gouv.fr/pid25535/bulletin_officiel.html&cid_bo=86110}{\textbf{guide d’évaluation des besoins de compensation en matière de scolarisation}} (Geva-Sco) favorisent un dialogue accru entre les familles, les équipes éducatives de suivi de la scolarisation et les maisons départementales des personnes handicapées (MDPH) ; ils garantissent aussi l’harmonisation des procédures et des décisions au plan national. S’agissant de la scolarisation des élèves en situation de handicap, pour favoriser la continuité des parcours et harmoniser les pratiques entre le premier et le second degrés, la nouvelle circulaire sur les unités localisées pour l’inclusion scolaire (Ulis) redéfinit les classes pour l’inclusion scolaire (Clis) qui deviennent des \textbf{\og Ulis école \fg{}}. Désormais appelés \og Ulis école \fg{}, \og Ulis collège \fg{} et \og Ulis lycée \fg{}, ces dispositifs ont vocation à accompagner les élèves en situation de handicap vers une meilleure insertion professionnelle. Enfin, les élèves ayant une notification d’aide humaine individuelle ou mutualisée bénéficient d’un accompagnement par des personnels recrutés à cet effet et formés. 5 000 accompagnants des élèves en situation de handicap (AESH) se sont vus proposer un CDI en~2014-2015, ce mouvement se poursuit cette année et permettra à terme aux 28~000 AESH de bénéficier d’un CDI. Par ailleurs, à la rentrée~2015, \textbf{100 unités d’enseignement (UE) supplémentaires}, actuellement situées dans les établissements médico-sociaux, seront \textbf{relocalisées} au sein même des établissements scolaires, ce qui portera leur nombre total à près de 300.

Pour accompagner et faciliter la scolarisation des élèves en situation de handicap, une politique de soutien à la production et au développement de ressources pédagogiques numériques adaptées a été mise en place.

Prévenir les ruptures et favoriser la continuité de la trajectoire de chaque jeune, c’est aussi faire en sorte que celle-ci puisse s’enrichir de nouveaux apports tout au long de la scolarité. C’est tout le sens de la mise en œuvre, à la rentrée~2015, des deux \textbf{parcours éducatifs} qui visent à garantir l’accès de tous aux conditions de la réussite : le parcours d’éducation artistique et culturelle (PEAC) et le parcours individuel d’information, d’orientation et de découverte du monde économique et professionnel. Pour fournir un support dynamique à ces parcours, l’accès à l’\textbf{application Folios}, qui a pour objectif de suivre tout au long de leur parcours les élèves de la sixième à la terminale et de conforter les compétences acquises à l’école ou en dehors de l’école, et notamment les expériences et les engagements des élèves, durant cette période, sera généralisé.

Avec l’ambition de mettre en cohérence enseignements et actions éducatives, de les relier aux expériences personnelles, de les enrichir et de les diversifier, le \textbf{parcours d’éducation artistique et culturelle} entend favoriser un égal accès de tous les jeunes à l’art et à la culture. Sa mise en œuvre résulte de la concertation entre les différents acteurs d’un territoire afin de construire une offre éducative cohérente à destination des jeunes, à l’échelon académique et à l’échelon local (\href{http://www.education.gouv.fr/pid25535/bulletin_officiel.html?cid_bo=71673}{circulaire \no{}~2013-073 du 3~mai~2013} et \href{http://cache.media.education.gouv.fr/file/CSP/16/2/Projet_de_referentiel_pour_le_parcours_d_education_artistique_et_culturelle_379162.pdf}{référentiel}).

Pour permettre aux élèves de construire progressivement, tout au long de leurs études secondaires, une véritable compétence à s’orienter, notamment en connaissant mieux le monde professionnel, le \textbf{parcours individuel d’information, d’orientation et de découverte du monde économique et professionnel} est généralisé de la sixième à la terminale. Ouvrant un accès pour tous à une culture économique et professionnelle, il vise à développer l’esprit d’entreprendre et l’ambition sociale, à mieux faire connaitre les différentes voies d’accès à la qualification (sous statut scolaire, d’étudiant ou d’apprenti), à encourager la diversification des parcours d’orientation des élèves et toutes les formes de mixité des filières de formation et des métiers, en veillant à favoriser l’égalité, en particulier entre les filles et les garçons. Il se distingue de l’ancien parcours de découverte des métiers et des formations (PDMF) en ceci qu’il s’inscrit dans une progression disciplinaire, voire interdisciplinaire, et qu’il suscite les initiatives permettant de développer, à l’échelle d’un territoire, des projets partagés avec des partenaires extérieurs.

Pour ce faire, la \textbf{réorganisation territoriale de l’implantation des centres d’information et d’orientation} (CIO) est engagée. Le ministère se préoccupe de maintenir un service public d’orientation scolaire de proximité à même de garantir le bon exercice des missions des personnels d’orientation au bénéfice des élèves et des familles. Dans cette perspective et face au désengagement de certains conseils départementaux de leur prise en charge, chaque académie doit élaborer une carte cible des CIO dans un dialogue permanent avec toutes les parties prenantes. Il en est de même s’agissant de la contribution des CIO à la mise en place du service public régional de l’orientation (SPRO).

En outre, le \textbf{parcours de santé} s’inscrit dans une politique éducative globale et est adossé à la nouvelle gouvernance académique. L’objectif de ce parcours vise la réussite scolaire de tous les élèves et la réduction des inégalités sociales. Ce dispositif est structuré autour de trois axes : l’éducation à la santé, la prévention et la protection de la santé.

Pour favoriser la \textbf{fluidité des parcours}, enfin, toutes les transitions doivent être mieux accompagnées : celles entre chaque cycle, à l’école comme au collège ; celle entre le collège et les trois voies du lycée ; celle, enfin, entre le lycée et l’enseignement supérieur, en étant attentif aux acquis des élèves, aux méthodes de travail et aux pratiques pédagogiques, à la continuité de l’orientation et au repérage des signes précurseurs du décrochage. Dans ce domaine, le \textbf{renforcement du continuum de formation de l’enseignement scolaire au supérieur} se poursuivra au cours de l’année scolaire~2015-2016 avec l’application de conventions entre les établissements scolaires et supérieurs et l’attention croissante portée à l’orientation des nouveaux bacheliers, à leur préparation à la poursuite d’études et à leur accompagnement dans l’enseignement supérieur. À ce titre, les initiatives locales associant les acteurs de l’enseignement secondaire et supérieur seront encouragées et valorisées.

\subsection{Favoriser l’insertion professionnelle et sociale}
Les acteurs du système éducatif doivent se mobiliser pour \textbf{mieux articuler formation et emploi}. Dans ce cadre, le \textbf{lycée} doit faire l’objet d’une vigilance toute particulière, puisqu’il est souvent le lieu où l’orientation se concrétise progressivement par des choix successifs. Le parcours individuel d’information, d’orientation et de découverte du monde économique et professionnel précédemment évoqué, les enseignements d’exploration au lycée général et technologique, les périodes de détermination en seconde professionnelle, les passerelles, les stages passerelle ou de mise à niveau, les possibilités offertes d’accéder à l’apprentissage participent ainsi à la construction du parcours de formation de chaque jeune.

Dans cette perspective, les actions partenariales conduites avec les acteurs économiques et sociaux visent à mieux faire connaitre le monde économique, le monde de l’entreprise et les métiers, ainsi qu’à développer le \textbf{gout d’entreprendre} et l’\textbf{esprit d’initiative}. Elles seront renforcées, structurées, coordonnées et largement diffusées afin que l’ensemble des élèves en bénéficient. À cet égard, les \textbf{pôles de stages} (\href{http://www.education.gouv.fr/pid25535/bulletin_officiel.html?cid_bo=86521}{circulaire \no{}~2015-035 du 25~février~2015}) constituent une traduction concrète de la relation entre école et entreprise en faveur de l’orientation et de la formation. Chaque pôle de stages devra être opérationnel dès la rentrée~2015, avec un objectif : faciliter l’accès des jeunes aux stages et aux périodes de formations en milieu professionnel (PFMP). Par ailleurs, de nombreux établissements ont développé des \textbf{actions d’accompagnement} de leurs élèves \textbf{vers l’insertion professionnelle}, en partenariat avec le monde professionnel et d’autres acteurs de la formation professionnelle et de l’emploi. Dans chaque académie, ces actions seront valorisées en vue de leur diffusion.

Les \textbf{Campus des métiers et des qualifications} feront l’objet d’une attention particulière au sein de chaque académie. La dynamique de développement de ces Campus démontre l’intérêt des partenariats locaux entre recteurs, présidents de région, enseignement supérieur et tissu économique, afin de concevoir des parcours de formation diversifiés et ouverts autour de champs d’activité répondant à des besoins économiques et sociaux clairement identifiés au sein d’une filière. Cette dynamique sera poursuivie en 2015, en veillant à mettre en place un pilotage académique renforcé et à renforcer la visibilité de ces Campus pour les élèves et leurs familles.

La politique générale de \textbf{valorisation de l’enseignement professionnel} doit se poursuivre, en cohérence avec les orientations définies dans le cadre de la grande conférence sociale pour l’emploi de juillet~2014. Aussi, dans chaque académie, les initiatives de valorisation de cet enseignement seront encouragées.

Parce qu’il contribue aussi à former aux métiers dont notre pays a besoin, selon des modalités différentes, l’\textbf{apprentissage} sous statut scolaire sera développé dans les EPLE pour atteindre l’objectif de 60~000~apprentis fixé au ministère chargé de l’éducation nationale. Ce développement visera principalement les niveaux~V et IV de formation et il s’appuiera à la fois sur une meilleure information des familles et sur le déploiement des parcours mixtes de formation que peut offrir le lycée professionnel. Dans chaque académie, l’apprentissage sera présenté dans le cadre des journées de découverte des métiers et du monde professionnel.

La \textbf{formation continue} assurée par le réseau des Greta peut inspirer en formation initiale des pratiques pédagogiques individualisées et facilitant l’insertion professionnelle.

Faire en sorte que chaque jeune puisse construire son avenir professionnel, combattre les stéréotypes notamment sociaux ou sexués qui entravent le libre choix de son orientation et s’intégrer pleinement dans la société sont des missions de l’école. Le ministère s’est fixé deux objectifs clairs : \textbf{prévenir plus efficacement le décrochage} et \textbf{faciliter le retour vers l’école des jeunes ayant déjà décroché}.

Toutes les mesures du \href{http://www.education.gouv.fr/cid84031/tous-mobilises-pour-vaincre-le-decrochage-scolaire.html#Le_d\%C3\%A9tail\%20du\%20plan\%20de\%20lutte\%20contre\%20le\%20d\%C3\%A9crochage}{plan d’action} doivent être progressivement mises en œuvre, en mettant l’accent sur la \textbf{persévérance scolaire}, à laquelle une semaine sera consacrée. Des parcours de formation spécifiquement dédiés à une meilleure prise en compte, dans l’action, de la lutte contre le décrochage seront mis à disposition des équipes. Le tutorat adulte/élève sera encouragé au collège et au lycée ainsi que l’entraide et le travail collaboratif entre élèves. La construction d’alliances éducatives, avec les parents au sein des écoles, et avec les différents partenaires au sein des établissements, sera développée. Enfin, une plus grande diversité et souplesse des parcours seront encouragées, notamment avec la validation modulaire et progressive, dans un cadre règlementaire adapté à titre expérimental, de certaines spécialités de diplômes professionnels, la prise en compte des acquis et la conservation des notes au-dessus de 10 pour tous les candidats au baccalauréat ayant échoué à l’examen. Les académies poursuivront les expérimentations visant à organiser une période de détermination de quelques semaines en début de seconde professionnelle ou en première année de CAP, pour permettre aux jeunes de choisir au mieux leur spécialité. À compter de la rentrée~2015, les académies pourront proposer un parcours aménagé de \og stagiaire de la formation initiale \fg{} pour prévenir l’abandon scolaire précoce. Il concernera les jeunes à partir de 15 ans scolarisés dans un établissement du second degré ; les jeunes conserveront le statut scolaire et bénéficieront d’un parcours de formation \og sur mesure \fg{} et d’un accompagnement personnalisé formalisés sous forme de contrat. Le ministère maintiendra sa contribution au réseau des plates-formes (Foquale et MLDS) et accompagnera leur évolution dans le cadre de la \href{http://www.legifrance.gouv.fr/affichTexte.do?cidTexte=JORFTEXT000028683576}{loi du 5~mars~2014} relative à la formation professionnelle, à l’emploi et à la démocratie sociale.

\textbf{Les jeunes sortis prématurément du système scolaire doivent pouvoir revenir en formation afin de se qualifier.} Les articles \href{http://www.legifrance.gouv.fr/affichCodeArticle.do?cidTexte=LEGITEXT000006071191&idArticle=LEGIARTI000006524397}{L. 122-2} et \href{http://www.legifrance.gouv.fr/affichCodeArticle.do?cidTexte=LEGITEXT000006071191&idArticle=LEGIARTI000029852620}{D. 122-3-1} à \href{http://www.legifrance.gouv.fr/affichCodeArticle.do?cidTexte=LEGITEXT000006071191&idArticle=LEGIARTI000029852768}{D. 122-3-8} du code de l’éducation accordent des droits nouveaux à ces jeunes qui pourront être accueillis dans les lycées d’enseignement général et technologique comme dans les lycées professionnels pour tout ou partie de la formation leur permettant d’acquérir la qualification qui leur fait défaut. Tous les leviers disponibles doivent être mobilisés pour que ce droit soit connu des jeunes et de leurs familles, et pour qu’un jeune qui exprime la volonté de reprendre une formation soit rapidement pris en charge, que ce soit sous statut scolaire ou, en liaison avec les régions, dans le cadre d’un contrat en alternance ou de la formation professionnelle continue (\href{http://www.education.gouv.fr/pid25535/bulletin_officiel.html?cid_bo=86719}{circulaire~2015-041 du 20~mars~2015}).

\subsection{Développer les compétences des élèves avec le numérique}
L’usage des outils numériques doit contribuer au renforcement des apprentissages fondamentaux et à la lutte contre le décrochage, faciliter la différenciation des démarches et l’individualisation des parcours pour répondre aux besoins de chaque élève. Le déploiement généralisé des technologies numériques dans la société implique aussi l’acquisition par les élèves, dès l’école primaire, de nouvelles compétences. La connaissance des principes fondamentaux de l’informatique doit permettre à tous les élèves de mieux comprendre les enjeux d’un monde toujours plus connecté et d’en être des acteurs demain. Le développement d’une véritable culture numérique doit devenir un objectif de formation, afin de forger l’esprit critique des élèves. L’éducation aux médias et à l’information, qui contribue au parcours citoyen, doit en particulier être renforcée à l’heure du numérique et des réseaux sociaux.

Conformément au cap fixé par le Président de la République, la mise en place d’un programme de préfiguration du plan numérique dans~200 collèges et 300~écoles des réseaux d’éducation prioritaire à la rentrée scolaire~2015 doit permettre de tracer les grandes orientations d’une politique coordonnée de déploiement massif des usages, des ressources et des équipements mobiles au service de la réussite des élèves. Les services académiques s’attacheront, en relation avec les collectivités territoriales, à accompagner la mise en œuvre de cette préfiguration et faciliteront le développement et le suivi des projets numériques d’établissement dans les autres collèges. Les résultats de cette expérimentation contribueront à déterminer le cadre d’un déploiement plus important des équipements individuels mobiles dans les écoles et les établissements scolaires.

Le numérique doit enfin faciliter le développement de nouvelles pratiques pédagogiques, tant pour les enseignements disciplinaires qu’interdisciplinaires, en offrant des outils et des services favorisant la mise en place de projets et de parcours éducatifs. La mise en avant de l’autonomie, de la créativité et de la responsabilité des élèves mais aussi le développement de l’entraide et de la coopération sont autant de situations d’apprentissage favorisées par le développement du numérique.

\section{Garantir l’égalité et développer la citoyenneté}
Pour mieux assurer les missions que la République lui a confiées, l’école doit réduire les inégalités de réussite scolaire qu’elle peut produire et parvenir à être le creuset de la citoyenneté. Cette action sera d’autant plus efficace qu’elle prendra appui sur la mobilisation des partenaires de l’école.

\subsection{Agir contre les déterminismes sociaux et territoriaux}
En France, aujourd’hui, un élève sur dix vit dans une famille pauvre : c’est une réalité que l’école ne peut pas ignorer. Aussi, aucun élève ne saurait être mis en difficulté dans le cadre d’une demande de fournitures scolaires ou empêché de participer à une sortie ou un voyage scolaire pour des raisons financières.

Précisément pour lutter contre ces inégalités, la \textbf{refondation de l’éducation prioritaire}, après une phase de préfiguration, entre pleinement en œuvre à la rentrée~2015. Des moyens importants sont mobilisés pour des écoles et collèges travaillant en réseau dans des secteurs où la mixité sociale est absente et difficile à réaliser rapidement. Une nouvelle carte de 350 Rep+ et de 739 Rep prend en compte, pour la rentrée~2015, la nouvelle réalité économique et sociale du pays, tant en métropole que dans les outre-mer. Cette politique entend renforcer l’action pédagogique et éducative, développer le travail collectif et la formation des personnels et reconnaitre l’engagement des personnels (\href{http://www.education.gouv.fr/pid25535/bulletin_officiel.html?cid_bo=80035}{circulaire \no{}~2014-077 du 4~juin~2014} et \href{http://cache.media.eduscol.education.fr/file/education_prioritaire_et_accompagnement/53/5/referentiel_education_prioritaire_294535.pdf}{référentiel de l’éducation prioritaire}).

Dans ce contexte en faveur de l’égalité des territoires, les deux dispositifs de \textbf{priorité au premier degré} seront développés en priorité dans les Rep+ puis dans les Rep. La \textbf{scolarisation des enfants de moins de trois ans} requiert une réelle concertation avec les partenaires territoriaux et les professionnels de la petite enfance pour s’adresser aux élèves qui en ont le plus besoin, en veillant à la qualité de l’accueil à l’école, déterminante pour que s’installe le sentiment de sécurité et de confiance nécessaire à l’investissement du jeune enfant dans un univers nouveau. Un objectif de 50 \% de scolarisation a été fixé en Rep+ lors du comité interministériel Égalité et citoyenneté du 6~mars~2015 ; il nécessite un suivi et une mobilisation renforcés. Le \textbf{dispositif Plus de maitres que de classes} doit, quant à lui, permettre des modalités d’intervention efficaces en fonction des objectifs d’apprentissage poursuivis. Le maitre supplémentaire ne se substitue pas aux aides spécialisées. Il s’agit, dans les zones les plus fragiles, dans un contexte d’enseignement ordinaire, de diversifier les modalités d’enseignement au service d’une plus grande maitrise des compétences essentielles par tous les élèves. À cet effet, il conviendra de renforcer le pilotage de ce dispositif en se concentrant plus particulièrement sur le cycle 2 et en veillant à ce que l’action de l’enseignant supplémentaire ne se trouve pas diluée dans un trop grand nombre de classes.

Puissant vecteur d’égalité en matière de réussite scolaire et éducative, les \textbf{internats de la réussite} doivent être développés, d’abord au bénéfice des élèves des quartiers prioritaires de la politique de la ville, de l’éducation prioritaire et des territoires ultramarins. Il convient de mieux identifier les besoins et d’y répondre par des projets pédagogiques et éducatifs (\href{http://cache.media.eduscol.education.fr/file/internat/01/5/REFERENTIEL_INTERNAT_vf_415015.pdf}{référentiel national des internats : \og L’internat de la réussite pour tous \fg{}}) construits en lien avec les conseils régionaux et départementaux. On veillera à inscrire autant que possible ces projets dans le cadre du programme d’investissements d’avenir.

Afin de \textbf{lutter contre les inégalités sociales et territoriales} au sein du système éducatif, les autorités académiques se rapprocheront des collectivités territoriales compétentes pour fixer des objectifs partagés en matière de \textbf{mixité sociale} des établissements d’enseignement. Pour les collèges, la coopération entre l’État et le département peut, dorénavant, être formalisée par une convention passée entre l’IA-Dasen et le président du conseil départemental lorsque le département décide d’instaurer les secteurs communs à plusieurs collèges. Cette démarche sera promue et suivie au niveau national de manière à favoriser des approches communes dans plusieurs départements pilotes en accord étroit avec les conseils départementaux concernés.

Pour favoriser la mutualisation de l’offre de formation en langues vivantes et ainsi l’attractivité d’un plus grand nombre d’établissements dans un souci de mixité sociale, le Cned proposera à titre expérimental, dans quelques académies, un dispositif de formation hybride (en présence et à distance), sur des langues rares ou peu enseignées, à la rentrée scolaire~2015. Une palette de langues sera proposée au fur et à mesure du déploiement de ce dispositif, qui a pour premier objectif d’offrir à tous les élèves un égal accès aux langues vivantes et d’éviter des stratégies qui entravent la mixité scolaire.

\subsection{Renforcer la transmission des valeurs de la République}
Le rôle et la place de l’école dans la République sont inséparables de sa capacité à en faire vivre et à en transmettre les valeurs. L’école entend répondre avec pédagogie et fermeté à une double mission : transmettre des connaissances, des compétences et une culture commune d’une part ; être, d’autre part, un creuset de la citoyenneté.

Un \textbf{parcours citoyen}, appuyé notamment sur la mise en place à tous les niveaux d’enseignement à la rentrée~2015 de l’\textbf{enseignement moral et civique}, devra être organisé de l’école élémentaire à la terminale. Il doit permettre aux élèves de comprendre le \textbf{principe de laïcité}, en s’appuyant notamment sur la \href{http://www.education.gouv.fr/pid25535/bulletin_officiel.html?cid_bo=73659}{Charte de la laïcité à l’École}, qui sera présentée aux élèves et à leurs parents à la rentrée scolaire et signée par eux pour attester la reconnaissance par chacun de ses principes. Pour mettre en œuvre le principe de laïcité et promouvoir une pédagogie de la laïcité dans l’ensemble des temps de la vie scolaire, un livret dédié sera disponible dans toutes les écoles et les établissements du second degré. Le parcours citoyen vise aussi à expliciter le bien-fondé des valeurs et des règles qui régissent les comportements individuels et collectifs, à reconnaitre le pluralisme des opinions (le travail sur la maitrise de la langue pourra être ici pleinement mobilisé ; voir partie I) et à construire du lien social et politique. Il devra intégrer pleinement la \textbf{participation} de l’élève à la vie de l’école et de l’établissement et les expériences et engagements qu’il connaitra en dehors de l’école, notamment avec les partenaires associatifs. Il visera également à développer l’\textbf{éducation aux médias et à l’information}. Il pourra prendre appui sur des actions éducatives et favoriser l’implication active de chaque élève dans les journées (notamment la Journée nationale du 9 décembre dédiée à la laïcité) ou semaines spécifiques (notamment les Semaines de l’engagement lycéen), les campagnes nationales de solidarité, les concours et olympiades, et les commémorations patriotiques. Comme le prévoit la grande mobilisation de l’École pour les valeurs de la République, les écoles, collèges, lycées et lycées professionnels devront d’ailleurs intégrer à leurs \textbf{projets d’école et d’établissement} les modalités de la participation des élèves à ces différents temps, en lien avec les conseils à la vie collégienne et les conseils de vie lycéenne.

Le \textbf{respect de la liberté et de la dignité d’autrui}, le \textbf{rejet de toutes les discriminations}, l’engagement au service de la communauté et la \textbf{prévention du racisme et de l’antisémitisme} doivent fonder les projets éducatifs et s’inscrire au cœur de la vie scolaire. Autour de la Journée internationale du 21 mars, la Semaine d’éducation contre le racisme et l’antisémitisme sera un évènement d’ampleur fédérant l’école et l’ensemble de ses partenaires, institutions républicaines, associations qualifiées, réservistes de l’éducation nationale.

Par ailleurs, les \textbf{projets d’ouverture sur l’Europe et le monde} seront encouragés. Rencontrer des cultures différentes, apprendre de l’autre et expérimenter avec lui, s’inspirer des expériences menées ailleurs, utiliser ses savoirs pour se faire mieux connaitre et/ou partager des pratiques : autant d’aspects qui pourront être explorés pour bâtir ces projets, par exemple dans le cadre du programme Erasmus +, mais aussi de manière plus large.

La réussite de tous les élèves est subordonnée à l’installation durable d’une \textbf{culture de l’égalité entre les sexes et du respect mutuel} qui garantit à chaque élève, fille ou garçon, un traitement égal et une même attention portée à ses compétences, son parcours scolaire, sa réussite et son bien-être. Les enjeux de mixité des filières et des métiers, d’insertion professionnelle et de prévention des comportements à caractère sexiste imposent de poursuivre la structuration du réseau des chargés de mission à l’égalité en académie et l’effort engagé en matière de formation de l’ensemble des personnels ainsi que de prendre en compte l’égalité dans toutes les dimensions, dans tous les enseignements, dans les processus d’orientation et à tous les niveaux de la politique éducative. Ces priorités pourront s’appuyer sur l’enrichissement régulier des outils pour l’égalité entre les filles et les garçons.

Pour lutter contre toutes les formes de discriminations et de violences et pour favoriser une culture du respect et de l’égalité, l’approche globale par le climat scolaire est reconnue. Les groupes \og climat scolaire \fg{}, en articulation avec les comités départementaux d’éducation à la santé et à la citoyenneté (CESC), mis en place sur le sujet dans la plupart des académies, doivent poursuivre leur action d’accompagnement des écoles et établissements soucieux d’adhérer à cette approche systématique des questions éducatives et pédagogiques. C’est aussi dans cette perspective que la \textbf{lutte contre toutes les formes de harcèlement en milieu scolaire} est résolument menée par le ministère en lien avec la lutte contre les discriminations. Le prix \og Mobilisons-nous contre le harcèlement \fg{} sera reconduit en 2015 et une campagne de communication renouvelée en matière de lutte contre l’homophobie (\og L’homophobie n’a pas sa place à l’école \fg{}) sera lancée dès la rentrée scolaire, en concertation avec les associations et les fédérations de parents d’élèves. Pour autant, une approche permettant de mieux aborder la \textbf{gestion de crise} doit aussi se développer, en s’appuyant sur les équipes mobiles de sécurité (EMS) et les assistants chargés de prévention et de sécurité (APS).

L’\textbf{éducation au développement durable}, par la prise en compte des interdépendances entre l’environnement, dont le climat et la biosphère, la société, l’économie et la culture, est généralisée dans les programmes d’enseignement et les formations, dans les projets des écoles et des établissements scolaires, en s’appuyant sur les partenariats, en particulier territoriaux. C’est dans le cadre de cette éducation transversale que notre ministère se mobilise pour l’accueil, en décembre~2015, de la \textbf{conférence des Nations unies sur le changement climatique \og Paris Climat~2015-COP 21 \fg{}}. Dans les établissements, des débats sur le changement climatique seront organisés, notamment pendant la Semaine du climat, à partir du 5 octobre. Les établissements scolaires s’inscriront à titre individuel ou en lien avec d’autres établissements dans l’organisation d’un projet pédagogique, de simulations de négociations internationales sur le changement climatique, afin de permettre à la communauté éducative de s’approprier ces enjeux et de participer à la mobilisation citoyenne de l’école contre le changement climatique.

\subsection{Développer les partenariats et la culture de l’engagement avec tous les acteurs de l’école}
La convergence des nouveaux quartiers de la politique de la ville et des nouveaux réseaux d’éducation prioritaire permet que le \textbf{volet éducatif des contrats de ville} soit porteur d’orientations partagées par les différents ministères, les collectivités territoriales et les associations. Celles-ci doivent être complémentaires et cohérentes pour les enfants et les jeunes qui en ont le plus besoin. Il s’agit en particulier de travailler ensemble à réduire les écarts de réussite scolaire et le nombre de décrocheurs, à améliorer le bien-être des enfants et des jeunes dans le quartier et à assurer la participation des parents. Le pacte pour la réussite éducative du 6~novembre~2013 permet d’assurer la mise en cohérence des actions des différents partenaires.

Mis en place dans le cadre de la réforme des rythmes scolaires à l’école primaire, les \textbf{projets éducatifs territoriaux} (PEDT) sont généralisés en 2015. Ils permettent aux collectivités territoriales de proposer à chaque enfant un parcours éducatif cohérent et de qualité avant, pendant et après l’école, organisant ainsi la complémentarité des temps éducatifs. Cette action éducative partenariale doit contribuer à une politique de réussite pour tous et de lutte contre les inégalités d’accès aux loisirs éducatifs. L’inclusion d’un volet \og laïcité et citoyenneté \fg{} dans chaque PEDT doit être encouragée et s’appuie sur les ressources mises à disposition sur le site ministériel : \href{http://pedt.education.gouv.fr/}{pedt.education.gouv.fr}. L’attribution de l’aide du \og fonds de soutien \fg{} aux rythmes scolaires étant subordonnée à la conclusion d’un PEDT, les communes ou établissements publics de coopération intercommunale concernés seront accompagnés par les services de l’État (groupes d’appui départementaux) jusqu’à la signature du PEDT.

La grande mobilisation de l’École pour les valeurs de la République doit être l’occasion de renforcer le \textbf{pilotage académique des partenariats} avec les associations éducatives complémentaires de l’école, notamment dans les domaines de la promotion de l’engagement, de la lutte contre le racisme et l’antisémitisme ou de l’éducation aux médias et à l’information. Les \textbf{conventions pluriannuelles d’objectifs} conclues entre le ministère et les principaux mouvements d’éducation populaire et de jeunesse constituent en particulier des points d’appui pour favoriser des interventions dans le cadre scolaire. La délivrance de l’\textbf{agrément}, national ou académique, permet de certifier la qualité de l’action de ces associations.

L’\textbf{association sportive} permet au sein de chaque établissement d’engager les élèves et leurs familles dans la prise de responsabilités et la participation à la vie de l’établissement ; elle doit être encouragée et renforcée. Par ailleurs, l’année scolaire~2015-2016, marquée par de grands évènements sportifs, sera celle du \textbf{sport scolaire de l’école à l’université}. Il s’agit de promouvoir la pratique sportive des jeunes et de mobiliser la communauté éducative autour des valeurs éducatives et citoyennes transmises aussi par le sport. 

L’école se construit aussi grâce à la participation de \textbf{tous les parents}, dans le cadre de la \textbf{coéducation} ; le dialogue avec ceux-ci, notamment les plus éloignés de l’institution scolaire, devra être redynamisé. Pour cela, on pourra s’appuyer sur l’aménagement des \textbf{espaces parents} au sein des écoles et des établissements, la généralisation du dispositif de la Mallette des parents, le renforcement du dispositif Ouvrir l’École aux parents, la généralisation des environnements numériques de travail et les différentes actions de soutien à la parentalité (actions éducatives familiales, notamment). À la rentrée~2015, un \textbf{comité départemental d’éducation à la santé et à la citoyenneté} (CESC) sera mis en place là où il n’existe pas encore ; il conviera à ses travaux l’ensemble des partenaires soucieux et susceptibles d’apporter leur concours aux projets départementaux, notamment en matière d’éducation à la citoyenneté et de définition des actions du parcours citoyen. La semaine de la démocratie devra être un temps fort des écoles et des établissements pour valoriser les élections des représentants de parents d’élèves.

Complémentaire d’un engagement associatif et du service civique universel, qui se déploiera fortement dans les écoles et collèges à partir de la rentrée~2015, la \textbf{réserve citoyenne de l’éducation nationale} permet de répondre aux demandes des citoyens désireux de faire partager leurs expériences professionnelles et personnelles et d’apporter leur concours à la mobilisation de l’École pour les valeurs de la République, voire aux actions en ce sens conduites dans le cadre d’activités périscolaires mises en place par les collectivités territoriales. La réserve citoyenne constitue ainsi, pour l’institution scolaire, l’occasion de mobiliser, au-delà des différentes composantes de la communauté éducative et des acteurs qui interviennent déjà aujourd’hui, les forces vives de la société civile. L’animation de la réserve est assurée au niveau académique, en lien avec l’échelon départemental et en relation avec le secteur associatif.

\section{Former et accompagner les équipes éducatives et enseignantes pour la réussite des élèves}
Les réformes engagées doivent mobiliser des pratiques pédagogiques diversifiées, innovantes, capables de répondre aux besoins pluriels des élèves. Ces évolutions exigent que les équipes puissent s’appuyer sur une formation renouvelée et ambitieuse ainsi que sur des ressources de référence, opératoires, efficaces, actualisées.

\subsection{Une politique globale de formation}
Permettre aux équipes enseignantes et éducatives et, plus largement, à chaque professionnel de l’école, de s’approprier l’ensemble des évolutions en cours suppose de bien articuler les actions mises en œuvre au niveau national et celles développées au niveau local. L’apport spécifique du premier réside dans la formation des personnels d’encadrement (inspecteurs, personnels de direction) et des formateurs, auxquels il revient ensuite d’assurer le déploiement en académie. Aussi, le \textbf{plan national de formation} (PNF) pour l’année~2015-2016 s’attachera-t-il à privilégier les actions portant les priorités relatives à l’école maternelle, à la scolarité obligatoire, dont la réforme du collège, et à l’éducation aux valeurs de la République, dans un contexte marqué par le nécessaire développement des usages du numérique. Pour faciliter la formation des équipes de terrain, l’accent sera mis, dans chaque formation du PNF, sur l’accompagnement du transfert en académie. Des ressources de formation diversifiées, axées à la fois sur l’appropriation de la nouveauté et sur le développement des compétences professionnelles seront proposées.

Afin que chaque équipe soit soutenue et chaque enseignant accompagné, la \textbf{formation continue} doit privilégier plusieurs modalités. D’abord, les actions doivent s’inscrire dans la proximité : formations en circonscription, en bassin, en réseau ou directement au sein de l’école ou de l’établissement. Ensuite, si les formations individuelles restent nécessaires, l’enjeu réside bien, aussi, dans le développement d’actions impliquant les équipes pour favoriser une culture partagée et, ainsi, faciliter la mise en œuvre des enseignements, disciplinaires comme interdisciplinaires, inscrire l’accompagnement des élèves dans une logique commune et créer une dynamique au sein de l’école et/ou de l’établissement. Le projet d’école ou projet d’établissement doit consacrer un volet important à la formation, levier déterminant de sa mise en œuvre. Enfin, la formation continue doit être pensée en lien étroit avec la recherche et l’innovation, tant en termes de contenus disciplinaires qu’en didactique, en faisant mieux connaitre les avancées réalisées par les sciences cognitives et la sociologie de l’éducation et en valorisant les expériences réussies. Dans ce cadre, l’implication des universitaires, et plus particulièrement des enseignants des ESPE, doit être largement sollicitée. Les formateurs académiques du second degré, ainsi que les IEN et professeurs des écoles maitres formateurs dans le premier degré, constituent des relais premiers pour faire vivre cette liaison avec pertinence. Au-delà, l’inscription des ESPE dans la formation continue, via un conventionnement avec le rectorat, doit être une occasion privilégiée pour encourager l’accès des enseignants ou personnels d’éducation en poste à certaines UE de master (en particulier celles du master MEEF mention 4 \og Pratiques et ingénierie de formation \fg{}) en permettant la délivrance d’ECTS. Pour garantir cette formation de qualité, une université d’automne sera organisée, permettant de réunir tous les pilotes et opérateurs en charge de la formation.

Le numérique doit être pris en compte comme une modalité de formation à part entière. Les \textbf{parcours M@gistère} permettent d’accroitre et de diversifier l’offre de formation. Mobilisant les apports de la recherche, adaptables au contexte académique et fondés sur la responsabilisation des enseignants, ces parcours ne se substituent pas aux autres modes de formation mais les complètent utilement. L’effort de production sera ainsi poursuivi, notamment à destination du second degré. Pilotée par les corps d’inspection, la mise en œuvre des nouveaux parcours appuiera en particulier la formation des enseignants aux priorités nationales définies par le PNF. Les académies pourront proposer un accompagnement de proximité pour les nouveaux utilisateurs.

La nouvelle \textbf{politique de ressources d’accompagnement} vise à répondre aux besoins diversifiés des acteurs. Il s’agit de fournir aux enseignants et personnels d’éducation un ensemble cohérent de supports de nature variée, adapté aussi bien aux personnels débutants qu’à ceux qui sont plus confirmés, dans un double objectif : faciliter une première appropriation des programmes et dispositifs nouveaux et compléter les formations mises en place. Les premiers ensembles de ressources, destinés à la maternelle, seront disponibles avant l’été ; suivront les ressources pour l’enseignement moral et civique puis l’ensemble des supports d’appui à la réforme de la scolarité obligatoire. Par ailleurs, de nouvelles ressources seront produites à l’attention des formateurs pour permettre le déploiement des actions de formation en académie.

Deuxième volet majeur de la politique de formation, la \textbf{formation initiale en alternance} doit mettre en œuvre un lien effectif entre temps de formation en ESPE et temps de formation en situation professionnelle. Pour ce faire, les enseignants des ESPE et les professionnels de terrain doivent travailler ensemble ; ils pourront s’appuyer sur l’outil conçu pour faciliter le suivi conjoint des jeunes professeurs par les tuteurs et formateurs (\href{http://www.education.gouv.fr/pid25535/bulletin_officiel.html?cid_bo=87000}{note de service \no{}~2015-055 du 17~mars~2015}). Des journées à l’intention des équipes pluri-catégorielles et/ou des tuteurs pourront être inscrites dans les plans académiques de formation ; par ailleurs, lors des réunions de rentrée, les actions menées par l’académie pour accompagner l’entrée des jeunes professeurs dans le métier pourront être présentées.

Au cœur de cette logique intégrative, se trouvent les quatre domaines du tronc commun (\href{http://www.legifrance.gouv.fr/affichTexte.do?cidTexte=JORFTEXT000027905257&categorieLien=id}{arrêté du 27~aout~2013}) : les gestes professionnels liés aux situations d’apprentissage ; les connaissances liées au parcours des élèves ; les enseignements associés aux principes et à l’éthique du métier ; les thèmes d’éducation transversaux et des grands sujets sociétaux. Il importe, d’une part, de s’assurer que les deux temps de la formation participent effectivement à professionnaliser les nouveaux entrants dans le métier dans ces quatre domaines ; d’autre part, de penser des modalités de mise en œuvre qui permettent de confronter les apports théoriques aux situations réelles et prévoient des temps pour une analyse réflexive. Une note de cadrage sur le tronc commun sera prochainement publiée.

Au regard de la diversité des situations rencontrées par les enseignants et personnels d’éducation en école ou établissement, le degré d’expertise atteint au moment de la titularisation ne saurait suffire à une réelle professionnalisation. Penser une formation continuée et continue qui prolonge et installe ces acquisitions est donc essentiel.

Pour la rentrée~2015, il conviendra d’être particulièrement attentif à la mise en place des parcours adaptés, en veillant dans la mesure du possible à anticiper les modalités à même d’être offertes. La diversité des situations des stagiaires ainsi que les conditions de leur accompagnement et de titularisation sont précisées dans la \href{http://www.education.gouv.fr/pid25535/bulletin_officiel.html?cid_bo=87000}{note de service \no{}~2015-055 du 17~mars~2015}.

\subsection{Former les enseignants et le personnel d’encadrement au numérique pour mieux accompagner les élèves}
La \textbf{formation du corps enseignant et du personnel d’encadrement au numérique} est indispensable pour répondre aux nouveaux contextes d’éducation liés à l’évolution des technologies et à l’apparition de nouvelles pratiques culturelles et sociales. Tous les futurs enseignants ou conseillers principaux d’éducation doivent être conscients des enjeux du numérique et doivent pouvoir porter un regard critique et réfléchi sur les évolutions induites par le développement de ses techniques et de ses usages. Cela recouvre non seulement les nouvelles modalités de diffusion de la connaissance et les stratégies d’apprentissage, mais aussi le fait que les élèves sont désormais eux-mêmes producteurs de contenus et d’informations qui se diffusent en ligne, notamment sur les réseaux sociaux. Le travail mené avec le centre de liaison de l’enseignement et des médias d’information (Clemi) au sein des écoles et des établissements doit être mieux connu.

Former à l’enseignement \og au numérique et par le numérique \fg{} constitue une priorité nationale, traduite par une politique volontariste de production de ressources mobilisant le numérique, mais également de formation. Elle permettra en particulier l’intégration de nouveaux éléments de connaissance d’informatique dans les parcours des élèves, du primaire au lycée, dès la rentrée~2016. Un effort exceptionnel sera mis en œuvre pour accompagner le Plan numérique, la formation de l’encadrement, des formateurs et des personnes ressources pour le numérique précédant les formations des enseignants au plus près de leurs activités. Pour que les usages du numérique irriguent largement le système et deviennent une réalité au sein des classes, des supports adaptés doivent être proposés aux enseignants. Les équipes de correspondants académiques Tice renforceront le travail engagé au niveau national de production de séquences pédagogiques destinées à accompagner la réforme de la scolarité obligatoire.

\subsection{Mieux accompagner les professionnels dans l’exercice de leurs missions}
Pour que soient mises en œuvre, au service de la réussite des élèves, les nouvelles orientations pédagogiques et éducatives de la refondation de l’École, les missions des personnels enseignants de l’éducation nationale, dont le contenu a évolué et s’est progressivement enrichi, doivent être redéfinies.

Les \href{http://legifrance.gouv.fr/affichTexte.do?cidTexte=JORFTEXT000029390906&categorieLien=id}{décrets \no{}~2014-940} et \href{http://www.legifrance.gouv.fr/affichTexte.do?cidTexte=JORFTEXT000029390951&categorieLien=id}{941} du 20~aout~2014 traduisent et consolident, à partir de la rentrée~2015, dans un cadre rénové et clarifié, l’ensemble de ces évolutions pour les \textbf{enseignants qui exercent dans le second degré}, en reconnaissant l’éventail de leurs missions. Alors que seule la mission d’enseignement était identifiée dans les décrets du 25~mai~1950, ces nouveaux textes, tout en réaffirmant le caractère primordial de celle-ci, reconnaissent, dans le cadre général défini par l’\href{http://www.legifrance.gouv.fr/affichCodeArticle.do?cidTexte=LEGITEXT000006071191&idArticle=LEGIARTI000006525568}{article L.~912-1 du code de l’éducation}, l’ensemble des missions inhérentes au métier enseignant dans le second degré, y compris celles qui sont le complément et le prolongement indispensables de l’activité d’enseignement au sens strict. Désormais, sont prises en compte :
\begin{itemize}
\item tout d’abord, la mission d’enseignement, qui continue à s’accomplir dans le cadre des maxima hebdomadaires de service actuels ;
\item corrélativement, l’ensemble des missions liées directement au service d’enseignement ; sont ainsi reconnus les temps de préparation et de recherche nécessaires à la réalisation des heures d’enseignement, les activités de suivi, d’évaluation et d’aide à l’orientation des élèves inhérentes à la mission d’enseignement, le travail en équipe pédagogique ou pluri-professionnelle ainsi que les relations avec les parents d’élèves ;
\item des missions complémentaires exercées par certains enseignants, qui se verront attribuer des responsabilités particulières afin de mener des actions pédagogiques dans l’intérêt des élèves. Ces missions pourront être exercées au niveau d’un établissement ou au niveau académique.
\end{itemize}

Dans le même esprit, des projets sont en préparation afin de mieux identifier et reconnaitre l’ensemble des missions des \textbf{personnels enseignants du premier degré}.

D’ores et déjà, pour faciliter l’exercice de leurs responsabilités par les \textbf{directeurs d’école}, qui jouent un rôle majeur dans la réussite des réformes engagées dans le premier degré, les démarches académiques et départementales visant à simplifier leurs tâches administratives doivent être poursuivies et se concrétiser de manière significative (\href{http://www.education.gouv.fr/pid25535/bulletin_officiel.html?cid_bo=83288}{circulaire \no{}~2014-138 du 23~octobre~2014}). Par ailleurs, les plans académiques et départementaux de formation doivent s’inscrire dans l’organisation de la formation des directeurs d’école (\href{http://www.education.gouv.fr/pid25535/bulletin_officiel.html?cid_bo=84361}{arrêté du 28~novembre~2014} et \href{http://www.education.gouv.fr/pid25535/bulletin_officiel.html?cid_bo=84367}{circulaire \no{}~2014-164 du 1er~décembre~2014}). Ces derniers bénéficient en outre d’un nouveau régime de décharge.

\section*{Conclusion}
Fédérer les efforts de tous, au sein de l’école et au-delà de l’école, vers un objectif partagé, celui de la réussite des élèves, en s’appuyant sur tous les leviers créés par la refondation : tel est bien l’enjeu de la rentrée~2015-2016.

Pour que cette ambition puisse s’incarner, un effort important sera accordé à la formation et à l’accompagnement des équipes. Chacun doit se sentir pleinement engagé et responsable dans cette mission au service de notre jeunesse.