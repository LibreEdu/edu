La Charte de la laïcité à l’École, dont le texte est annexé à cette circulaire, a été élaborée à l’intention des personnels, des élèves et de l’ensemble des membres de la communauté éducative. Dans un langage accessible à tous, cette Charte explicite les sens et enjeux du principe de laïcité à l’École, dans son rapport avec les autres valeurs et principes de la République. Elle offre ainsi un support privilégié pour enseigner, faire partager et faire respecter ces principes et ces valeurs, mission confiée à l’École par la Nation et réaffirmée dans la loi d’orientation et de programmation pour la refondation de l’École de la République du 8~juillet~2013.

Adaptée aux spécificités de la mission éducative de l’École, la Charte de la laïcité à l’École vise à réaffirmer l’importance de ce principe indissociable des valeurs de liberté, d’égalité et de fraternité exprimées par la devise de la République française. La laïcité souffre trop souvent de méconnaissance ou d’incompréhension. Ce texte permet d’en comprendre l’importance, comme garante à la fois des libertés individuelles et des valeurs communes d’une société qui dépasse et intègre ses différences pour construire ensemble son avenir. La laïcité doit être comprise comme une valeur positive d’émancipation et non pas comme une contrainte qui viendrait limiter les libertés individuelles. Elle n’est jamais dirigée contre des individus ou des religions, mais elle garantit l’égal traitement de tous les élèves et l’égale dignité de tous les citoyens. Elle est l’une des conditions essentielles du respect mutuel et de la fraternité. Ce texte s’attache aussi à montrer le rôle de l’École dans la transmission du sens et des enjeux de la laïcité. La transmission de ce principe par l’École est indispensable pour permettre l’exercice de la citoyenneté et l’épanouissement de la personnalité de chacun, dans le respect de l’égalité des droits et des convictions, et dans la conscience commune d’une fraternité partagée autour des principes fondateurs de notre République.

Dans les écoles et les établissements d’enseignement du second degré publics, il est demandé de procéder à un affichage visible de la Charte de la laïcité à l’École. La transmission des valeurs et principes de la République requiert en outre, dans l’ensemble des établissements d’enseignement, un affichage visible de ses symboles - drapeau et devise notamment - ainsi que de la Déclaration des droits de l’homme et du citoyen du 26~aout~1789. La Charte de la laïcité à l’École prendra ainsi tout son sens, en cohérence avec l’article~3 de la loi du 8~juillet~2013, codifié à l’article L.~111-1-1 du code de l’éducation, qu’il convient de mettre en œuvre en lien avec les collectivités territoriales.

Dans toutes les écoles et tous les établissements scolaires, ces dispositions doivent être accompagnées par une pédagogie de la laïcité et des autres principes et valeurs de la République, qui s’appuie notamment sur la Charte de la laïcité à l’École et qui permette à la communauté éducative de se les approprier.

\section{Charte de la laïcité à l’École : diffusion, pistes pour une appropriation par l’ensemble de la communauté éducative et propositions d’exploitation pédagogique}
\subsection{Affichage de la Charte}
Dans les écoles et établissements d’enseignement du second degré publics, la Charte de la laïcité à l’École \textbf{est affichée de manière à être visible de tous. Les lieux d’accueil et de passage sont à privilégier}. À cette fin, des affiches de grand format vous ont été envoyées pour mise à disposition de l’ensemble des écoles et des établissements. Ces affiches sont également disponibles, en tant que de besoin, auprès des centres régionaux de documentation pédagogique (CRDP). Sur le site Éduscol, la Charte de la laïcité à l’École est téléchargeable dans un format maniable et propice à sa diffusion la plus large possible et à son appropriation par l’ensemble de la communauté éducative.

Il est souhaitable que l’affichage de la Charte de la laïcité à l’École dans les locaux scolaires revête un caractère solennel et constitue un moment fort dans la vie des écoles et des établissements. Le texte de la Charte de la laïcité à l’École devra, à tout le moins, être porté dans les meilleurs délais à la connaissance de l’ensemble des membres de la communauté éducative.

Il revient en priorité aux chefs d’établissement et directeurs d’école d’assurer non seulement l’affichage, mais la diffusion de cette Charte, en direction de l’ensemble de la communauté éducative, ainsi que des partenaires locaux de l’École, acteurs éducatifs et représentants associatifs notamment. Il leur revient aussi de réfléchir avec l’ensemble des équipes pédagogiques aux moyens de faire vivre la réflexion sur la laïcité dans leur établissement, à partir de cette Charte. Les membres des corps d’inspection l’incluront dans leur accompagnement pédagogique. Les conseillers principaux d’éducation, les référents vie lycéenne et les professeurs principaux donneront aux élèves, notamment aux élèves élus, les moyens de se saisir de ce texte et, par là, de participer activement à la connaissance du principe de laïcité au sein de leur établissement scolaire.

\subsection{Diffusion et appropriation}
Les moyens d’une diffusion et d’une appropriation de la Charte peuvent être les suivants.

Il est recommandé de joindre la Charte de la laïcité à l’École \textbf{au règlement intérieur}. Sa présentation aux parents lors des réunions annuelles de rentrée sera l’occasion, pour les directeurs et directrices d’école et pour les chefs d’établissement, de faire connaitre la Charte, d’en éclairer le sens et d’en assurer le respect.

Dans les différentes instances des établissements scolaires (conseil d’école et conseil d’administration), la Charte de la laïcité nourrira les réflexions et les échanges propres à inspirer un axe du projet d’école ou d’établissement. Les modalités de son utilisation à des usages et fins pédagogiques feront l’objet de propositions des conseils pédagogiques. Enfin, les conseils de la vie lycéenne et les conseils de la vie collégienne seront des lieux d’impulsion d’activités visant à faire vivre la laïcité au sein des établissements par l’initiative des élèves eux-mêmes, ainsi que l’article 15 de la Charte les y invite.

\subsection{Ressources pour une pédagogie de la Charte de la laïcité à l’École}
La nécessaire maitrise par les élèves du principe de laïcité et des valeurs qui fondent notre République requiert une pédagogie qui les fasse connaitre, comprendre et partager.

La Charte de la laïcité en est un vecteur privilégié, qu’elle soit étudiée dans le cadre des enseignements, notamment en instruction civique et morale, en éducation civique, en éducation civique, juridique et sociale puis, à partir de la rentrée 2015, dans celui de l’enseignement moral et civique, de l’heure de vie de classe ou encore qu’elle soit mise en valeur dans le cadre d’actions éducatives.

La Charte de la laïcité à l’École a été élaborée avec l’intention d’en permettre la pédagogie. Elle énonce dans un langage simple les significations du principe de laïcité, des règles qui en découlent et de leur bienfondé tout en clarifiant, pour la compréhension de tous, les garanties que ce principe apporte à l’exercice de la liberté, au respect de l’égalité, à la poursuite de l’intérêt général et à la fraternité.

Les cinq premiers articles rappellent les principes fondamentaux de la République indivisible, laïque, démocratique et sociale et le fondement solide que la laïcité offre à l’épanouissement de ces valeurs. Les dix articles suivants expliquent ce que doit être la laïcité de l’École, qui assure aux élèves l’accès à une culture commune et partagée. La neutralité des personnels et la laïcité des enseignements y sont rappelées, de même que les règles de vie, respectueuses de la laïcité, dans les différents espaces des établissements scolaires publics.

En complément du texte de la Charte, des documents offrent des pistes d’approfondissement précises et développées pour faciliter sa lecture et son étude. Ces documents constituent des ressources qui permettent d’accompagner sa mise en œuvre par les personnels. Ils sont téléchargeables sur le site Éduscol et se composent notamment d’un commentaire du préambule et de chacun des quinze articles de la Charte, d’un fascicule rassemblant les textes de référence et des pistes bibliographiques, enfin d’un guide d’entrée par les programmes à partir de mots-clés.

Tout au long de l’année scolaire 2013-2014, le site Éduscol sera alimenté de ressources pédagogiques autour des questions de laïcité.

\section{Visibilité des symboles de la République à l’École}
\subsection{Mise en œuvre de l’article L.~111-1-1 du code de l’éducation}
Afin de mettre en œuvre l’article L.~111-1-1 du code de l’éducation, dont les dispositions s’appliquent aux écoles et aux établissements d’enseignement du second degré publics et privés sous contrat, il revient aux chefs d’établissement, aux directrices et directeurs d’école et aux directrices et directeurs d’établissement d’enseignement privé sous contrat :

- d’une part, en lien avec les collectivités territoriales, de prendre les dispositions nécessaires pour que la devise de la République et les drapeaux tricolore et européen soient apposés sur la façade ;

- d’autre part, d’afficher à l’intérieur des locaux la Déclaration des droits de l’homme et du citoyen, de manière visible et dans des endroits accessibles à l’ensemble de la communauté éducative. Les lieux de passage et d’accueil sont à privilégier. À cette fin, des affiches de grand format seront disponibles auprès des centres régionaux de documentation pédagogique (CRDP). La Déclaration du 26~aout~1789, ainsi qu’un livret pédagogique élaboré par le CNDP, sont également téléchargeables sur le site Éduscol.

Des indications relatives aux normes en matière de pavoisement et d’inscription de la devise feront l’objet d’une instruction ministérielle séparée et seront les seules faisant foi. Toute sollicitation commerciale relative à la mise en œuvre de l’article L.~111-1-1 du code de l’éducation doit être considérée avec prudence.

\appendix
\section{Charte de la laïcité à l'École. La Nation confie à l'École la mission de faire partager aux élèves les valeurs de la République}
\subsection{La République est laïque}
\begin{enumerate}
\item La France est \textbf{une République indivisible, laïque, démocratique et sociale}. Elle assure l'égalité devant la loi, sur l'ensemble de son territoire, de tous les citoyens. Elle respecte toutes les croyances.
\item La République laïque organise \textbf{la séparation des religions et de l'État}. L'État est neutre à l'égard des convictions religieuses ou spirituelles. Il n'y a pas de religion d'État.
\item La laïcité garantit \textbf{la liberté de conscience} à tous. \textbf{Chacun est libre de croire ou de ne pas croire}. Elle permet la libre expression de ses convictions, dans le respect de celles d'autrui et dans les limites de l'ordre public.
\item La laïcité permet l'exercice de la citoyenneté, en conciliant \textbf{la liberté de chacun} avec \textbf{l'égalité et la fraternité de tous} dans le souci de l'intérêt général.
\item La République assure dans les établissements scolaires le respect de chacun de ces principes.
\subsection{L'École est laïque}
\item La laïcité de l'École offre aux élèves les conditions pour forger leur personnalité, exercer leur libre arbitre et faire l'apprentissage de la citoyenneté. \textbf{Elle les protège de tout prosélytisme et de toute pression} qui les empêcheraient de faire leurs propres choix.
\item La laïcité assure aux élèves l'accès à \textbf{une culture commune et partagée}.
\item La laïcité permet l'exercice de la \textbf{liberté d'expression} des élèves dans la limite du bon fonctionnement de l'École comme du respect des valeurs républicaines et du \textbf{pluralisme des convictions}.
\item La laïcité implique \textbf{le rejet de toutes les violences et de toutes les discriminations}, garantit \textbf{l'égalité entre les filles et les garçons} et repose sur une culture du \textbf{respect} et de la compréhension de l'autre.
\item \textbf{Il appartient à tous les personnels de transmettre aux élèves le sens et la valeur de la laïcité}, ainsi que des autres principes fondamentaux de la République. Ils veillent à leur application dans le cadre scolaire. Il leur revient de porter la présente charte à la connaissance des parents d'élèves.
\item \textbf{Les personnels ont un devoir de stricte neutralité} : ils ne doivent pas manifester leurs convictions politiques ou religieuses dans l'exercice de leurs fonctions.
\item \textbf{Les enseignements sont laïques}. Afin de garantir aux élèves l'ouverture la plus objective possible à la diversité des visions du monde ainsi qu'à l'étendue et à la précision des savoirs, \textbf{aucun sujet n'est à priori exclu du questionnement scientifique et pédagogique}. Aucun élève ne peut invoquer une conviction religieuse ou politique pour contester à un enseignant le droit de traiter une question au programme.
\item Nul ne peut se prévaloir de son appartenance religieuse pour refuser de se conformer aux règles applicables dans l'École de la République.
\item Dans les établissements scolaires publics, les règles de vie des différents espaces, précisées dans le règlement intérieur, sont respectueuses de la laïcité. \textbf{Le port de signes ou tenues par lesquels les élèves manifestent ostensiblement une appartenance religieuse est interdit}.
\item Par leurs réflexions et leurs activités, \textbf{les élèves contribuent à faire vivre la laïcité} au sein de leur établissement.
\end{enumerate}
