\documentclass{article}
\usepackage[utf8]{inputenc}
\usepackage[T1]{fontenc}
%\usepackage{lmodern}
%\usepackage{fourier}
\usepackage{helvet}
\renewcommand{\familydefault}{\sfdefault}
\usepackage{microtype}
\usepackage[svgnames]{xcolor}
\usepackage{fontawesome}
\usepackage[french]{babel}

%\usepackage{layout}
%\usepackage{showframe}
%\usepackage{lipsum}

\setlength\parindent{0pt}

\newcommand{\lun}{Lundi }
\newcommand{\mar}{Mardi }
\newcommand{\mer}{Mercredi }
\newcommand{\jeu}{Jeudi }
\newcommand{\ven}{Vendredi }

\newcommand{\sep}{septembre 2016}

\newcommand{\bcb}[2]{\colorbox{#1}{\textbf{#2}} }
\newcommand{\sdl}[1]{
	
	#1}

\newcommand{\mrg}[2]{\marginpar{\begin{flushright}\colorbox{#1}{#2}\end{flushright}}}
\newcommand{\msb}[1]{\marginpar{\begin{flushright}#1\end{flushright}}}
\newcommand{\hre}[2]{\mrg{white}{#1 h #2}}
\newcommand{\hrb}[2]{\marginpar{\begin{flushright}\bcb{\cge}{#1 h #2}\end{flushright}}}

\newcommand{\dev}{\textbf{Devoirs : }\msb\faHome}
\newcommand{\nte}{\sdl\textbf{Note : }\msb\faPencil}
\newcommand{\obj}{\sdl\textbf{Objectif : }\msb\faBullseye}
\newcommand{\bil}{\sdl\textbf{Bilan : }\msb\faStethoscope}
\newcommand{\cdj}[2]{\sdl\textbf{Cahier du jour : }\msb\faLeanpub Ex. #1 p. #2.}
\newcommand{\Cdj}{\sdl\textbf{Cahier du jour : }\msb\faLeanpub}
\newcommand{\cde}[2]{\sdl\textbf{Cahier d’essai : }\msb\faBook Ex. #1 p.~#2.}
\newcommand{\Ces}{\sdl\textbf{Cahier d’essai : }\msb\faBook }
\newcommand{\ard}[2]{\sdl\textbf{Ardoise : }\msb\faTablet Ex. #1 p.~#2.}
\newcommand{\Ard}{\sdl\textbf{Ardoise : }\msb\faTablet}
\newcommand{\ptc}{\sdl\textbf{Photocopie : }\msb\faClone}


\newcommand{\dis}[6]{\sdl{\bcb{#1}{#2}}\hre{#3}{#4} \textit{#5}, p.~#6.}
\newcommand{\Dis}[5]{\sdl{\bcb{#1}{#2}}\hre{#3}{#4} \textit{#5}.}
\newcommand{\dIs}[4]{\sdl{\bcb{#1}{#2}}\hre{#3}{#4}}

\newcommand{\cge}{Gainsboro}
\newcommand{\gdc}{Gestion de classe}
\newcommand{\adm}{Administratif}
\newcommand{\clc}{Classcraft}
\newcommand{\gec}{Gestion de conflit}

\newcommand{\cfr}{cyan}
\newcommand{\lec}{Lecture et compréhension de l’écrit}
\newcommand{\voc}{Vocabulaire}
\newcommand{\ecr}{Écriture}
\newcommand{\gra}{Grammaire}
\newcommand{\ort}{Orthographe}
\newcommand{\jdl}{Jeux de langage}

\newcommand{\cma}{GreenYellow}
\newcommand{\nec}{Nombres et calculs}
\newcommand{\gem}{Grandeurs et mesures}
\newcommand{\eeg}{Espace et géométrie}
\newcommand{\jma}{Jeux maths}
\newcommand{\rdp}{Résolution de problèmes}

\newcommand{\chg}{Orange}
\newcommand{\his}{Histoire}
\newcommand{\geo}{Géographie}
\newcommand{\emc}{Enseignement moral et civique}

\newcommand{\cen}{Red}
\newcommand{\eng}{Anglais}

\newcommand{\csp}{Magenta}
\newcommand{\eps}{Éducation physique et sportive}

\newcommand{\csi}{Yellow}
\newcommand{\sci}{Sciences et technologie}

\newcommand{\car}{Violet}
\newcommand{\art}{Enseignements artistiques}

\newcommand{\cpr}{Salmon}
\newcommand{\ppd}{Projet pluridisciplinaire}


\begin{document}
%\layout
\reversemarginpar
\maketitle

\shorttableofcontents{Sommaire}{1}
\setlength{\parskip}{1ex}

\section{Période 1}
\subsection{\jeu 1\ier{} \sep}
Présentation des élèves : se caractériser par un trait qui les différencie des autres.

Distribution du cahier de liaison, création de la page de garde et du sommaire et pagination du cahier. Distribution d’un mot de présentation collé dans le cahier ainsi que de documents à remplir, à signer ou à lire et à remettre au plus tard lundi 5~septembre :
\begin{itemize}
	\item la \textit{Note de rentrée pour l’année scolaire 2016/2017} à lire et à signer ;
	\item une \textit{Fiche individuelle de renseignements} à remplir en deux exemplaires et à signer ;	
	\item une \textit{Fiche d’urgence} à remplir ;
	\item le \textit{Règlement intérieur} à lire et à signer ;
	\item une feuille d’autorisations à remplir et à signer ;
	\item une note aux parents concernant la photo collective à remplir et à signer ;
	\item une version remise en page de la \textit{Sécurité des écoles : Le guide des
		parents d’élèves} du ministère de l’Éducation nationale sur \og Comment développer une culture commune de la sécurité ? \fg{} à lire et à garder.
\end{itemize}\vspace{1ex}

Attribution d’un numéro d’élève selon l’ordre alphabétique des noms de famille et des différentes responsabilités.
\obj Apprendre à connaitre les élèves et leur présenter une méthodologie réutilisable pour les différents cahiers.


\subsection{\ven 2 \sep}
Évaluation diagnostique en français et en mathématiques.

\emc Lecture du début du règlement intérieur de l’école.
\obj Connaitre le niveau scolaire des élèves et prendre connaissance du règlement intérieur, document que les élèves devront signer.


\subsection{\lun 5 \sep}
Modification de la place des élèves.

Mise en place du système de gestion de comportement Classcraft:
\begin{itemize}
	\item présentation du jeu ;
	\item distribution du règlement;
	\item lecture du règlement ;
	\item signature du pacte ;
	\item création des groupes ;
	\item choix des rôles (guerrier, mage ou guérisseur).
\end{itemize}\vspace{1ex}

\ecr Raconter ses vacances, réelles ou imaginaires.
\obj Avoir un aperçu du niveau en écriture.
\bil Difficulté pour certains élèves de raconter leurs vacances réelles.


\subsection{\mar 6 \sep}
\gem\ham Convertir des longueurs
\obj Savoir utiliser un tableau de conversion afin de convertir une longueur selon les différentes unités de longueur : le millimètre (mm), le centimètre (cm), le décimètre (dm), le mètre (m), le décamètre (dam), l’hectomètre (hm) et le kilomètre (km).
\bil Grande difficulté pour convertir d’une unité à l’autre pour la moitié de la classe.

\gra\hpm Le verbe : les verbes d’action et les verbes d’état, les verbes du 1\ier{}, du 2\ieme{} et du 3\ieme{}~groupe.
\obj Savoir identifier le verbe dans une phrase.
\bil La notion d’infinitif n’est pas forcément maitrisée.


\subsection{\mer 7 \sep}
\ort\ham Les mots en –ail, -eil, -euil et -ouil.
\obj Savoir écrire des phonèmes complexes
\bil Compétence globalement maitrisée.

\nec\hpm Les nombres de plus de quatre chiffres
\obj Savoir comment écrire un grand nombre et savoir la différence entre chiffre et nombre.
\bil Grande difficulté à savoir quel est le nombre d’unités, de dizaines\dots


\subsection{\jeu 8 \sep}
\gdc Distribution des emplois du temps.

\eeg\ham Angles : aigus, obtus ou droit
\obj Savoir reconnaitre un angle aigu d’un angle obtus.
\bil Difficulté pour certains de voir sur la feuille d’exercice la différence entre les deux types d’angles.

\lec\hpm Lecture du texte \emph{Le cuisinier}.
\obj Travailler la compréhension de texte.


\subsection{\ven 9 \sep}
\adm\ham Trie  des différentes fiches : \textit{Fiche individuelle de renseignements}, autorisation photo collective, \textit{Fiche d’urgence} et assurances.

\lec\amr Correction de la première partie du texte \emph{Le cuisinier}.
\obj Travailler la compréhension de texte.

\emc\hpm Lecture de la fin du règlement intérieur de l’école et signature.
\obj Signer le règlement intérieur en connaissance de cause.

\adm\heure{16 h 00} Distribution des invitations pour la réunion de parents.


\setlength{\parskip}{0ex}
\tableofcontents
\addcontentsline{toc}{section}{Table des matières}
\end{document}


