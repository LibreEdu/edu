\documentclass{article}
%\usepackage[a4paper]{geometry}
%\usepackage[a4paper,bmargin=1in,rmargin=1in]{geometry}
\usepackage[a4paper,left=1.3in,right=.7in,top=1in,bottom=1in]{geometry}
\usepackage{titlesec}
\usepackage{fancyhdr}
%\usepackage{shorttoc}
\usepackage[colorlinks]{hyperref}
\usepackage{bbding}
\usepackage{datetime2}

\title{Cahier journal}
\author{Alexandre Pachot}
\pagestyle{fancy}
\lhead{Alexandre Pachot}
\chead{}
\rhead{Cahier journal}
\lfoot{\DTMnow}
\cfoot{}
\rfoot{\thepage}
\titleformat*{\subsection}{\large\bfseries\color{DarkBlue}}

\newcommand{\sep}{septembre 2016}
%\newcommand{\note}{\heure{\PencilLeftDown}}
\newcommand{\note}{
	
	 \PencilLeftDown{} }

\usepackage[utf8]{inputenc}
\usepackage[french]{babel}
\usepackage[T1]{fontenc}
\usepackage{graphicx}
\usepackage{fourier}
%\usepackage{lmodern}
\usepackage{helvet}
%\usepackage[scaled]{helvet}
\renewcommand{\familydefault}{\sfdefault}
\usepackage[a4paper, hmargin=113pt, vmargin=111pt]{geometry}
%\usepackage[bindingoffset=5mm]{geometry}
\usepackage{microtype}
\usepackage[hidelinks]{hyperref}
%\usepackage{microtype}%meilleur réglage espaces typo, pb avec les -- dans les tableaux
\usepackage{pdflscape}
\usepackage{longtablex}
\usepackage{tabularx}
\usepackage{multirow}
\usepackage{calc}
\usepackage{wrapfig}
\usepackage{xcolor}
\usepackage{pdfpages}
\usepackage{epigraph}
%\usepackage{scrtime}%heure système
\usepackage{fancyhdr}
\usepackage{titling}


%\usepackage{showframe}
\usepackage{layout}
%\usepackage{lipsum}

\graphicspath{{images/}}
\newlength{\HauteurTexte}
\setlength{\HauteurTexte}{\textheight}

\hypersetup{
    pdftitle={La germination},
    pdfauthor={Pachot, Alexandre},
    pdfsubject={Dossier CRPE 2015},
    pdfkeywords={Germination, PS, MS, maternelle, éducation}
}

\pagestyle{fancy}
%\lhead{}
%\chead{\thetitle}
%\rhead{} 
%\cfoot{\thepage}
\renewcommand{\headrulewidth}{0pt}

%\KOMAoptions{%
%	paper=a4,
%	fontsize=12pt,
%	DIV=calc,
%%	BCOR=5mm,
%%	twoside=on,
%}

\begin{document}
	%\layout
	\maketitle
	\reversemarginpar
	\setlength{\parskip}{2ex}
	
	%\begin{abstract}\end{abstract}
	%\shorttableofcontents{Sommaire}{1}
	
	
	\section{Période 1}
	\subsection{\jeu 1\ier{} \sep}
	\textbullet{} Présentation des élèves : se caractériser par un trait qui les différencie des autres.
	
	\textbullet{} Distribution du cahier de liaison, création de la page de garde et du sommaire et pagination du cahier. Distribution d’un mot de présentation collé dans le cahier ainsi que de documents à remplir, à signer ou à lire et à remettre au plus tard lundi 5~septembre :
	\begin{itemize}
		\item la \textit{Note de rentrée pour l’année scolaire 2016/2017} à lire et à signer ;
		\item une \textit{Fiche individuelle de renseignements} à remplir en deux exemplaires et à signer ;	
		\item une \textit{Fiche d’urgence} à remplir ;
		\item le \textit{Règlement intérieur} à lire et à signer ;
		\item une feuille d’autorisations à remplir et à signer ;
		\item une note aux parents concernant la photo collective à remplir et à signer ;
		\item une version remise en page de la \textit{Sécurité des écoles : Le guide des
			parents d’élèves} du ministère de l’Éducation nationale sur \og Comment développer une culture commune de la sécurité ? \fg{} à lire et à garder.
	\end{itemize}\vspace{1ex}
	
	\textbullet{} Attribution d’un numéro d’élève selon l’ordre alphabétique des noms de famille et des différentes responsabilités.
	\obj Apprendre à connaitre les élèves et leur présenter une méthodologie réutilisable pour les différents cahiers.
	
	
	\subsection{\ven 2 \sep}
	\textbullet{} Évaluation diagnostique en français et en mathématiques.
	
	\emc{8}{45} Lecture du début du règlement intérieur de l’école.
	\obj Connaitre le niveau scolaire des élèves et prendre connaissance du règlement intérieur, document que les élèves devront signer.
	
	
	\subsection{\lun 5 \sep}
	\textbullet{} Modification de la place des élèves.
	
	\textbullet{} Mise en place du système de gestion de comportement Classcraft:
	\begin{itemize}
		\item présentation du jeu ;
		\item distribution du règlement;
		\item lecture du règlement ;
		\item signature du pacte ;
		\item création des groupes ;
		\item choix des rôles (guerrier, mage ou guérisseur).
	\end{itemize}\vspace{1ex}
	
	\ecr{14}{00} Raconter ses vacances, réelles ou imaginaires.
	\obj Avoir un aperçu du niveau en écriture.
	\bil Difficulté pour certains élèves de raconter leurs vacances réelles.
	
	
	\subsection{\mar 6 \sep}
	\dIs\cma\gem{8}{45} Convertir des longueurs
	\obj Savoir utiliser un tableau de conversion afin de convertir une longueur selon les différentes unités de longueur : le millimètre (mm), le centimètre (cm), le décimètre (dm), le mètre (m), le décamètre (dam), l’hectomètre (hm) et le kilomètre (km).
	\bil Grande difficulté pour convertir d’une unité à l’autre pour la moitié de la classe.
	
	\Dis\cfr\gra{14}{00} Le verbe : les verbes d’action et les verbes d’état, les verbes du 1\ier{}, du 2\ieme{} et du 3\ieme{}~groupe.
	\obj Savoir identifier le verbe dans une phrase.
	\bil La notion d’infinitif n’est pas forcément maitrisée.
	
	
	\subsection{\mer 7 \sep}
	\dIs\cfr\ort{8}{45} Les mots en –ail, -eil, -euil et -ouil.
	\obj Savoir écrire des phonèmes complexes
	\bil Compétence globalement maitrisée.
	
	\dIs\cma\nec{14}{00} Les nombres de plus de quatre chiffres
	\obj Savoir comment écrire un grand nombre et savoir la différence entre chiffre et nombre.
	\bil Grande difficulté à savoir quel est le nombre d’unités, de dizaines\dots
	
	
	\subsection{\jeu 8 \sep}
	\dIs\cge\gdc{8}{45} Distribution des emplois du temps.
	
	\dIs\cma\eeg{8}{45} Angles : aigus, obtus ou droit
	\obj Savoir reconnaitre un angle aigu d’un angle obtus.
	\bil Difficulté pour certains de voir sur la feuille d’exercice la différence entre les deux types d’angles.
	
	\dIs\cge\adm{12}{00} L’effectif de la classe passe de vingt-huit élèves à vingt-sept.
	
	\dIs\cfr\lec{14}{00} Lecture du texte \emph{Le cuisinier} et réponses aux questions de compréhension.
	\obj Travailler la compréhension de texte.
	
	
	\subsection{\ven 9 \sep}
	\dIs\cge\adm{8}{45} Trie  des différentes fiches : \textit{Fiche individuelle de renseignements}, autorisation photo collective, \textit{Fiche d’urgence} et assurances.
	
	\dIs\cfr\lec{10}{20} Correction de la première partie du texte \emph{Le cuisinier}.
	\obj Travailler la compréhension de texte.
	
	\dIs\chg\emc{14}{00} Lecture de la fin du règlement intérieur de l’école et signature.
	\obj signer le règlement intérieur en connaissance de cause.
	
	\dIs\cge\adm{16}{00} Distribution des invitations pour la réunion de parents.
	\bil gestion de la classe moins éreintante que les jours précédents. 
	
	
	\subsection{\lun 12 \sep}
	\dIs\cge\clc{8}{45} Retour sur le nombre de points d’expérience et de vie de la semaine précédente, ainsi que sur la conséquence de la perte de la totalité du nombre de points de vie sur le reste du groupe.
	\obj Se rendre compte de la conséquence de la perte de la totalité du nombre de points de vie sur le reste du groupe.
	
	\dIs\cfr\lec{10}{00} Mise en place du défi lecture : lire vingt minutes quotidiennement, sept jours par semaine.
	\obj lire quotidiennement
	
	\dIs\cge\gec{10}{30}
	
	\dis\cma\nec{11}{00}{Nombres entiers (1)}{8} Différence entre chiffre et nombre, et existence de nombres à un chiffre.             
	
	\dIs\csp\eps{14}{00} Course d’endurance
	\obj apprendre à ralentir afin de courir à allure constante pendant 8 à 15 minutes.
	\nte sept élèves ont réussi à courir six minutes.
	
	\dIs\cge\adm{15}{20} Distribution de documents : photographe, OCCE et \emph{Petit quotidien}.
	
	
	\subsection{\mar 13 \sep}
	\nte formation natation 
	
	\Dis\cma\gem{8}{45}{Connaitre et utiliser les unités de longueur} Correction de la fiche : ex.~1, 2 et 3.
	\obj Savoir convertir des longueurs.
	\bil Reste à faire : exercices 4 et 5
	
	\dIs\cfr\gra{9}{30} Correction de la fiche \emph{Le verbe} : exercices 2 et 3.
	\obj savoir reconnaitre le verbe.
	
	\dIs\cfr\lec{10}{45}{Le cuisinier} : correction du texte.
	\obj travailler la compréhension de texte
	\bil la feuille est à finir.
	
	\dIs\cfr\jdl{14}{45} trouver les abréviations dans un petit texte.
	\obj savoir reconnaitre des abréviations courantes.
	
	\dIs\car\art{15}{20} se dessiner et colorier des espaces.
	\nte activité artistique laissée au libre choix de la remplaçante.
	
	\dev
	\sdl\bcb\cfr\lec Défi lecture, continuer le texte \textit{Le cuisinier} $\rightarrow$ jeudi
	\sdl\bcb\cma\gem Exercices 4 et 5 de \emph{Connaitre et utiliser les unités de longueur} $\rightarrow$ mardi
	
	
	\subsection{\mer 14 \sep}
	\dIs\cge\clc{8}{45} Réajustement pour les points d’expériences, discussion sur l’intérêt de l’école, explication des devoirs et de l’emploi du temps.
	\obj Avoir une meilleure gestion de classe, une responsabilisation et une plus grande autonomie.
	
	\dIs\cpr\ppd{10}{20} Explication du projet d’école : écrire un journal de classe. Premier article : Classcraft. Attribution d’un sujet à chacun des groupes.
	\obj Prendre connaissance du projet
	
	\dev
	\sdl\bcb\cma\nec Lire \textit{Je retiens} et faire des exercices de la page 9 $\rightarrow$ vendredi
	
	
	\subsection{\jeu 15 \sep}
	\dIs\cfr\lec{8}{45} Lecture silencieuse (défi lecture) ou évaluation (deux élèves dont l’évaluation n’était pas finie).
	\dIs\cma\eeg{10}{20} Correction de la feuille sur la mesure des angles.
	\dIs\cen\eng{14}{00} Se présenter :
	\begin{itemize}
		\item Donner son nom ;
		\item Donner son âge ;
	\end{itemize}
	\dIs\cma\rdp{14}{20} Problèmes de logique : \textit{La belle peinture !} et \textit{Les copains}
	
	
	\subsection{\ven 16 \sep}
	\dIs\cfr\voc{8}{45} Le dictionnaire : ordre alphabétique et écriture phonétique.
	\dIs\cma\nec{14}{00} Chiffres et nombres.
	
	\dev devoirs pour la semaine.
	
	
	\subsection{\lun 19 \sep}
	\dIs\cma\nec{8}{45} \textit{Nombres entiers (1)}, page 9, \textit{Je m’entraine} exercices 1 à 3 à l’oral, exercice 4 dans le cahier du jour et exercice 8 pour les plus forts.
	
	\obj écrire correctement un nombre en toutes lettres.
	
	\dIs\cen\eng{10}{20} Introduce yourself : I am Georges. I am eight and a half. I have two brothers and two sisters
	
	\dIs\csp\eps{14}{00} La balle au capitaine avec joueurs relais sur les côtés et passe à dix.
	\obj évaluer les élèves en sport collectif
	\bil être clair dans les consignes. Prévoir le matériel\dots{} et le temps.
	
	
	\subsection{\mar 20 \sep}
	\nte exercice incendie.
	\bil fermer porte et fenêtres. Éteindre lumières.
	
	\dIs\cma\gem{8}{45} Feuille \emph{Connaitre et utiliser les unités de longueur}, correction des exercices~4 et 5.
	\cdj{5}{15}
	\obj savoir convertir les unités de mesure.
	\bil bien vérifier ce que la remplaçante a fait et faire plus d’exercices d’entrainement avant de passer au cahier du jour.
	
	\dIs\cfr\gra{10}{20} \textit{Le verbe, mot essentiel de la phrase}
	\ces{1 et 2}{11}
	\cdj{3}{11}
	\obj savoir identifier le verbe.
	\bil expliquer l’objectif de chaque exercice ainsi que celui de la leçon. Pour ceux qui sont lents à écrire, photocopier la consigne pour le cahier du jour.
	
	\dIs\cma\jma{14}{00}
	\Ard écriture en toutes lettres de nombres contenant 20 ou 100.
	\obj savoir quand il faut mettre un \og s \fg{} ou pas.
	\bil attention au niveau sonore par rapport aux classes mitoyennes.
	
	\dIs\car\art{14}{30} Finir le dessin sur les espaces. Faire un dessin de sa main et préciser sur chacun des cinq doigts les cinq \og choses \fg{} que l’on préfère.
	\bil Faire attention aux élèves qui ont tendance à bâcler leur travail.
	
	
	\subsection{\mer 21 \sep}
	\dis\cfr\ort{8}{45}{Les mots de la même famille}{114}
	\ard dictée de mots d’une même famille.
	\obj déduire l’écriture de mots à partir du radical.
	\bil avoir des mots étiquettes pour la correction.
	
	\dis\cma\nec{9}{30}{Nombres entiers (1)}{8}
	\ard{8.1}{9} Écrire le plus grand nombre pair constitué de quatre chiffres.
	\obj savoir ce qu’est un nombre pair.
	
	\dIs\cge\gdc{10}{20} Sortir de la classe en silence, se déplacer dans les couloirs en silence.
	
	\dIs\cpr\ppd{11}{15} Récapitulation des sujets par groupe :
	\begin{enumerate}
		\item les sanctions ;
		\item les points ;
		\item les personnages ;
		\item les récompenses ;
		\item les raisons pour lesquelles on ne participe pas.
	\end{enumerate}
	
	\dev
	\sdl\bcb\cpr\ppd Écrire un texte de 10 lignes par rapport à la thématique de son groupe.
	
	
	\subsection{\jeu 22 \sep}
	\Dis\cma\eeg{8}{45}{Mesure des angles} Fin de la correction de la feuille.
	\obj savoir reconnaitre un angle aigu d’un angle obtus.
	
	\dIs\cfr\lec{9}{30} Numérotation des manuels de français et de mathématiques et distribution de romans.
	
	\dIs\cma\jma{10}{20}
	\Ard écriture en lettre de nombres contenant \og vingt \fg{}, \og cent \fg{} et \og mille\fg{}.
	\obj connaitre les règles d’accord pour \og vingt \fg{}, \og cent \fg{} et \og mille\fg{}.
	
	\dIs\cma\rdp{14}{00} Le plus petit nombre impair de quatre chiffres.
	\obj faire évoluer son raisonnement
	
	\dIs\csp\eps{15}{45} Courir six minutes sans s’arrêter.
	\obj gérer son effort.
	
	
	
	
	
	
	\setlength{\parskip}{0ex}
	\tableofcontents
	%\addcontentsline{toc}{section}{Table des matières}
\end{document}


