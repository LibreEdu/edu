\section{Synthèse des fondements scientifiques}
\marginpar{\includegraphics[width=35pt]{vignette.png}}
\subsection{La graine}
La graine est la structure qui contient et protège l’embryon végétal issu de la reproduction des plantes à fleurs. Elle est constituée d’un \textbf{embryon}, d’une enveloppe protectrice, le \textbf{tégument}, et d’un ou plusieurs cotylédons. Le \textbf{cotylédon} est la feuille primordiale constitutive de la graine qui contient les réserves nutritives nécessaires au premier développement de la plante. Ces réserves peuvent être riches en glucides (ex. : riz), en lipides (ex. : noix) ou en protides (ex. : arachide). Le nombre de cotylédons varie selon les espèces : un pour le blé et le maïs, deux pour les graines de haricot, le pois et le marronnier, et de dix à douze pour les conifères. Pour certaines graines comme la graine de ricin, les réserves ne sont pas au niveau des cotylédons, mais de l’albumen. La graine est un organe fortement déshydraté, ce qui entraine une vie ralentie et permet de survivre à des conditions extrêmes de température, de longévité et de sècheresse. L’organe transporté se nomme la \textbf{semence}, elle se détache de la plante mère ; cela peut être soit la graine soit le fruit et sa dissémination peut se faire par projection de la plante, par le vent, par l’eau ou par les animaux.

\subsection{La germination}
La germination se produit que si des conditions extérieures (\textbf{humidité}, \textbf{température}, \textbf{oxygène}) sont conjointement présentes. La lumière\footnote{Chez les graines à photosensibilité positive, le tégument contient des phytochromes. Ce sont des photorécepteurs d’une des trois familles de photorécepteurs du monde végétal. Sensibles au rouge (660 nm) et au rouge lointain (730 nm), ils jouent un rôle en stimulant ou inhibant la germination.} influence également la germination des graines~\cite[p.~374]{Hopkins2003}. %http://www.botanic06.com/site/EvolVie/stade2.htm
%http://forums.futura-sciences.com/biologie/270616-phytochrome-germination.html
%http://www.ressources-pedagogiques.ups-tlse.fr/physiologie-vegetale/M8G08/CHAPITRE%20IV.pdf
%http://www-ijpb.versailles.inra.fr/fr/bs/equipes/physio-germ/
Tant que les facteurs favorables à la germination ne sont pas réunis, la graine est en état de \textbf{dormance}. Certaines graines ont besoin de passer par une période froide pour pouvoir germer. La germination commence avec l’hydratation de la semence et se finit avec la croissance de la radicule. La jeune plante issue de la germination de la graine se nomme la \textbf{plantule}. Elle est constituée d’une \textbf{radicule} qui deviendra la première racine, d’une \textbf{tigelle} qui deviendra la tige, et d’un ou plusieurs cotylédons.

\subsection{Le développement de l’enfant}
Dans son livre \emph{Jean Piaget simplement expliqué aux étudiants}, Jean Bernard Makanga, psychologue du développement de l’enfant, considère que la période qui va deux à six ans est une des périodes les plus importantes au point de vue intellectuel. Lorsqu’il s’agit de dessiner à partir d’un modèle, l’enfant de trois à six ans reproduit ce qu’il sait déjà faire, avec la signification qu’il attribue au modèle. Il a des difficultés à coordonner les éléments qui composent le dessin, il est au stade du \og \textbf{réalisme manqué} \fg{}. En ce qui concerne la logique, l’enfant ne démontre pas encore ce qu’il dit, c’est le stade de la \og \textbf{pensée intuitive} \fg{}~\cite{Makanga2015}.