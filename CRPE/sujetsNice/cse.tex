\cse{01}{Place et rôle de l’ELVE dans la construction globale des savoirs des élèves}
\cse{02}{Place et rôle des cycles à l’école primaire}
\cse{03}{Le numérique à l’école primaire}
\txt{03}{3}{Philippe}{Testard-Vaillant}{2009}{Internet, un outil au service de la démocratie ?}
\cse{04}{Enseignement du fait religieux et de la morale}
\txt{04}{1}{Philippe}{Claus}{2011}{Enseigner les faits religieux dans le premier degré}
\txt{04}{2}{Vincent}{Peillon}{2013}{Pour un enseignement laïque de la morale}
\cse{05}{Hétérogénéité et différenciation pédagogique}
\txt{05}{1}{Bruno}{Robbes}{2009}{La pédagogie différenciée : historique, problématique, cadre conceptuel [\dots]}
\txt{05}{2}{Bruno}{Suchaut}{2007}{L’hétérogénéité des élèves : un éclairage par la recherche en éducation}
\cse{06}{Le climat scolaire}
\txto{06}{1}{Centre d’analyse stratégique}{2013}{Favoriser le bien-être des élèves, condition de la réussite éducative}
\cse{07}{Prévention et lutte contre le harcélement à l’école}
\cse{08}{La violence à l’école}
\txto{08}{2}{Figaro.fr}{2013}{Une campagne contre le harcèlement scolaire}
\cse{09}{Le travail personnel à la maison}
\txto{09}{1}{Circulaire \no 94-226}{1994}{Suppression des devoirs à la maison}
\txto{09}{2}{Circulaire \no 64-496}{1964}{Interdiction des devoirs à la maison pour les élèves des classes primaires}
\txto{09}{3}{DILA (Premier ministre)}{2013}{Peut-on donner des devoirs à la maison à un élève de l’école primaire ?}
\txt{09}{4}{Marie-Estelle}{Pech}{2013}{Les parents plébiscitent les devoirs à la maison}
\cse{10}{Les rythmes scolaires à l’école maternelle}
\txtnd{10}{2}{François}{Testu}{Dans le futur débat sur l’école, le problèmedes rythmes scolaires}
\cse{11}{Évaluer les élèves}
\txto{11}{2}{Dictionnaire de l’éducation}{2008}{Évaluation (Théories de l’)}
\txt{11}{3}{Pierre}{Merle}{2013}{Entretien}
\cse{12}{Inclusion scolaire des élèves handicapés}
\txto{12}{1}{Circonscription Toulon Var ASH}{2012}{Guide pratique relatif aux missions d’AVS-i}
\txt{12}{2}{Véronique}{Soulé}{2013}{\og À l’école, un statut et des CDI \fg}
\txt{12}{3}{Vincent}{Martinez}{2013}{\og S’interroger sur les réponses pédagogiques et extra pédagogiques\fg}
\cse{13}{La scolarisation des enfants de moins de trois ans}
\txt{13}{3}{Thierry}{Vasse}{2011}{Le guide de l’ATSEM}
\cse{14}{Le harcèlement à l’école}
\txto{14}{1}{Ministère de l'Éducation nationale}{2013}{La lutte contre le harcèlement est l’affaire de tous}
\txt{14}{2}{Nicole}{Cathelin}{2008}{Le harcèlement à l’école}
\cse{15}{La scolarisation des enfants de moins de trois ans à l’école maternelle}
\txto{15}{2}{Mission maternelle Val de Marne}{2013}{Scolarisation des moins de trois ans}
\cse{16}{L’argent à l’école}
\txtond{16}{2}{Éduscol}{Financement des écoles}
\txto{16}{3}{Circulaire \no 99-136}{1999}{Sorties scolaires}
\cse{17}{Violence et incivilités}
\txto{17}{1}{Autonomes de solidarité laïques}{2012}{L’incivilité scolaire}
\txt{17}{2}{Alain}{Bauer}{2010}{Mission sur les violences en milieu scolaire et la place de la famille}
\cse{18}{La liaison école-collège}
\txto{18}{3}{IA Tours}{2009}{Pour un PPRE passerelle efficace}
\cse{19}{La motivation scolaire}
\txt{19}{1}{Fabien}{Fenouillet}{1999}{Apprendre autrement aujourd’hui : la motivation à l’école}
\txt{19}{2}{André}{Giordan}{2010}{Le désir d’apprendre : un oublié de l’école}
\txt{19}{3}{Claude}{Seibel}{2013}{Vers la réussite de tous les enfants à l'école}
\cse{20}{La lutte contre le harcèlement à l’école}
\txto{20}{2}{MEN}{2011}{Le harcèlement entre élèves : le reconnaitre, le prévenir, le traiter}
\txt{20}{3}{Éric}{Debarbieux}{2011}{Refuser l’oppression quotidienne : la prévention du harcèlement à l’École}
\cse{21}{Séjours scolaires dans le premier degré}
\txto{21}{1}{Circulaire \no 2005-001}{2005}{Séjours scolaires courts et classes de découvertes dans le premier degré}
\txta{21}{2}{Véronique}{Conche}{Laurence}{Leparoux}{2002}{En classe de découverte}
\cse{22}{La violence en milieu scolaire}
\txto{22}{1}{Circulaire \no 2006-125}{2006}{Prévention et lutte contre la violence en milieu scolaire}
\txt{22}{2}{Emmanuel}{Davidenkoff}{2013}{En France, le climat de discipline dans les écoles s’est dégradé [\dots]}
\txta{22}{3}{Laurence}{Ukropina}{Jean}{Rostand}{2011}{Les enseignants en cause}
\cse{23}{Agir contre l’illettrisme}
\txt{23}{1}{Jean-Paul}{Brigheli}{2013}{Comment l’école fabrique l’échec scolaire}
\txto{23}{2}{Circulaire \no 2013-179}{2013}{Prévenir l’illettrisme}
\txtnd{23}{3}{George}{Pau-Langevin}{Agir contre l’illettrisme : l’école se mobilise}
\cse{24}{L’accueil et la scolarisation d’enfants jeunes}
\txta{24}{1}{Marie-Pierre}{Hamel}{Sylvain}{Lemoine}{2012}{Quel avenir pour l'accueil des jeunes enfants ?}
\txt{24}{3}{Daniel}{Calin}{1999}{L’accueil des jeunes enfants à l’école maternelle}
\cse{25}{Vie scolaire}
\cse{26}{Le principe de gratuité et fournitures scolaires}
\txto{26}{1}{Bulletin officiel hors série \no 7}{1999}{Principe de gratuité}
\cse{27}{Le socle commun des connaissances et des compétences}
\txto{27}{1}{IGEN}{2007}{Les livrets de compétences : nouveaux outils pour l’évaluation des acquis}
\txto{27}{3}{Académie de Grenoble}{2012}{Mise en œuvre du socle commun des connaissances et des compétences}
\cse{28}{L’éducation artistique et culturelle à l’école}
\txtlond{28}{3}{DGESCO}{Palier 2 du LPC}
\cse{29}{Enseignements primaire et secondaire / socle commun}
\cse{30}{L’enseignement de la laïcité à l’école}
\txt{30}{1}{Bernard}{Stasi}{2004}{Laïcité}
\cse{31}{La formation du citoyen au cœur des programmes de l’École}
\txt{31}{1}{Dominique}{Schnapper}{2001}{La citoyenneté}
\txt{31}{2}{Mona}{Ozouf}{1984}{Histoire et instruction civique}
\txto{31}{3}{Rapport \og Refondons l’École\fg}{2012}{L’École comme lieu de formation civique et éthique}
\txta{31}{4}{Alain}{Etchegoyen}{André}{Comte-Sponville}{2004}{Égalité / Égalité des chances}
\cse{32}{Agir sur le climat scolaire}
\cse{33}{Le suivi de la scolarité par les parents}
\cse{34}{Rythmes scolaires en maternelle}
\txtna{34}{1}{2013}{Rythmes scolaires, des recommandations pour les maternelles}
\txto{34}{2}{AGEEM}{2013}{Refondation de l’école de la République}
\cse{35}{Continuité des apprentissages et personnalisation des parcours scolaires}
\cse{36}{Parcours de scolarisation de l’enfant handicapé}
\txt{36}{1}{Éric}{Plaisance}{2007}{Intégration ou inclusion ?}
\txto{36}{2}{Rapport \no 2012-162}{2012}{L’accompagnement des élèves en situation de handicap}
\cse{37}{Les sorties scolaires}
\cse{38}{Les sanctions à l’école}
\txt{38}{2}{Jean-François}{Vincent}{1999}{Associer les élèves à l’élaboration des règles}
\cse{39}{Les TICE à l’école}
\txto{39}{1}{MEN}{2012}{Faire entrer l’École dans l’ère du numérique : un impératif pédagogique [\dots]}
\txto{39}{2}{Éduscol}{2013}{Études européennes sur les TICE à l’école}
\txt{39}{3}{André}{Tricot}{2010}{Grâce aux TICE, une école plus efficace ? À voir\dots}
\cse{40}{Sujet40}
\txtop{40}{2}{Circulaire \no 2013-142}{2013}{Renforcer la coopération entre les parents et l’école dans les territoires}{1}
\txtp{40}{2}{Jean-Louis}{Auduc}{2012}{Dix conseils pour bien gérer les relations parents-enseignants}{2-5}{2}
\cse{41}{Les devoirs à la maison}
\txt{41}{1}{Louise}{Cuneo}{2012}{Les devoirs à la maison}
\txto{41}{2}{HCEE}{2005}{Le travail des élèves pour l’école dehors de l’école}
\txtond{41}{3}{Accompagnement à la scolarité: guide pratique}{Comment aider mon enfant dans son travail personnel ?}
\txt{41}{4}{Philippe}{Meirieu}{2012}{Le travail à la maison: question pédagogique, question sociale [\dots]}
\cse{42}{La gestion de la difficulté scolaire}
\txto{42}{1}{Extrait du rapport \og Vaincre l’échec à l’école primaire \fg}{2010}{Le redoublement, inutile et dangereux}
\txtond{42}{2}{Académie de Montpellier}{Pistes à valoriser pour des stratégies alternatives au redoublement}
\txt{42}{3}{Bruno}{Suchaut}{2009}{L’aide aux élèves : diversité des formes et des effets des dispositifs}
\cse{43}{Les devoirs à la maison, le travail hors de la classe}
\txto{43}{1}{Entretien de Séverine \textsc{Kapko}}{2012}{Le \og bon devoir \fg , c'est celui que l'élève peut faire seul}
\txto{43}{2}{IGEN}{2008}{Le travail des élèves en dehors de la classe, état des lieux et conditions d’efficacité}
\cse{44}{Continuité des apprentissages}
\txto{44}{2}{Collège Rosa Parks -- Chateauroux (36)}{2013}{Aménagement harmonisé CM2 – 6ème}
\cse{45}{Réussir à l’école}
\txt{45}{1}{George}{Pau-Langevin}{2013}{Pacte pour la réussite éducative}
\txt{45}{2}{Philippe}{Meirieu}{2011}{Éducation formelle et non formelle}
\cse{46}{Le harcèlement à l’école}
\cse{47}{Punitions et sanctions}
\txt{47}{2}{Bruno}{Robbes}{2005}{Quelles sanctions possibles à l’école maternelle et élémentaire ?}
\txt{47}{3}{Nathalie}{Anton}{2012}{Les punitions à l’école sont-elles vraiment efficaces ?}
\cse{48}{Scolarisation des enfants handicapés}
\cse{49}{La mission de l’école}
\cse{50}{Scolarisation des élèves handicapés}
\cse{51}{La violence à l’école}
\txt{51}{1}{Philippe}{Meirieu}{2006}{Violences scolaires\dots}
\txtna{51}{2}{2013}{Convention pour la protection de l’enfance et la prévention de la violence en milieu scolaire}
\txt{51}{3}{Éric}{Debarbieux}{2008}{Les dix commandements contre la violence à l’école}
\cse{52}{Sanction et Éducation}
\txto{52}{2}{Direction des services départementaux des Alpes-Maritimes}{2013}{Règlement scolaire départemental}
\txt{52}{3}{Philippe}{Meirieu}{2006}{La question des sanctions et la problématique \og comprendre ou juger \fg}
\cse{53}{Gestion des intervenants extérieurs}
\txto{53}{2}{Direction académique du Jura}{2009}{Les intervenants extérieurs à l’école primaire}
\cse{54}{Le partenariat dans l’Éducation au développement durable}
\txtt{54}{2}{Éducation au développement durable : Seconde phase de généralisation de l’EDD}
\txto{54}{3}{Circulaire \no 2010-083}{2010}{Mise en place de l’ASTEP}
\txt{54}{4}{Bruno}{Magnes}{2013}{500 élèves sensibilisés au développement durable avec Leclerc}
\txtond{54}{5}{Présentation du projet de la Main à la pâte}{\og Quand la Terre gronde \fg , un projet d’éducation au risque}
\cse{55}{Le numérique à l’école}
\txto{55}{2}{MEN}{2013}{Les nouveautés de la rentrée 2013}
\txt{55}{3}{Jean-Michel}{Le Baut}{2014}{Les chemins numériques du savoir}
\cse{56}{Le cyber harcèlement}
\txto{56}{1}{MAIF}{2013}{Réseaux sociaux, quels risques pour vos enfants ?}
\txto{56}{2}{MEN et Non au harcèlement}{2013}{Agir contre le harcèlement à l’école}
\txto{56}{4}{MEN}{2013}{Guide de prévention de la cyberviolence entre élèves}
\cse{57}{La place de la laïcité à l’école primaire}
\txt{57}{1}{Jacques}{Valade}{2004}{L’école de la République, pilier de la laïcité}
\cse{58}{La lutte contre le harcèlement et les violences scolaires à l’école}
\txtond{58}{3}{MEN}{Charte d’engagement contre le harcèlement}
\cse{59}{Enseigner par et au numérique à l’école}
\txt{59}{1}{Vincent}{Peillon}{2012}{Stratégie pour le numérique à l’École}
\txt{59}{2}{Périne}{Brotcorne}{2012}{Génies contre incompétents}%\cse{60}{Sujet60}
\cse{60}{Place de l’élève dans la scolarité réfléchie sur la base des textes de lois}
\txt{60}{1}{Michèle}{Sellier}{2008}{Histoire et changement}
\txt{60}{2}{Vincent}{Peillon}{2013}{Loi d’orientation et de programmation pour la refondation de l’École de la République}
\cse{61}{Travailler en équipe de cycle}
\txt{61}{1}{Philippe}{Perrenoud}{2008}{Travailler en équipe est un choix stratégique, pas un dogme}
\cse{62}{S’approprier le langage et l’écrit à l’école maternelle}
\txtond{62}{2}{Éduscol}{Pour une première culture littéraire à l’école maternelle}
\txtond{62}{3}{Éduscol}{Le développement de l’enfant}
\cse{63}{La place des parents à l’école}
\cse{64}{L’innovation}
\txtnd{64}{1}{Jacques}{Georges}{Clarifier la notion d’innovation}
\txt{64}{2}{Vincent}{Peillon}{2013}{La refondation de l’école faite sa rentrée}
\txto{64}{3}{MEN}{2013}{La priorité à l’école primaire}
\cse{65}{La personnalisation des parcours}
\txt{65}{2}{Sylvain}{grandserre}{2013}{Pédagogie différenciée : 10 conseils + 1}
\cse{66}{Enseignement de l’histoire}
\txtna{66}{1}{2001}{Annexe à la Recommandation relative à l’enseignement de l’histoire en Europe au XXI\ieme{} siècle}
\txt{66}{2}{Dominique}{Desvignes}{2008}{La culture historique en questions: les programmes de l’école primaire}
\cse{67}{La laïcité à l’école}\txt{67}{2}{Abdennour}{Bidar}{2012}{Pour une pédagogie de la laïcité à l’école}
\cse{68}{Le climat scolaire}
\cse{69}{Lutte contre le décrochage scolaire}
\txto{69}{3}{Inspection générale}{2014}{Évaluation partenariale de la politique de lutte contre le décrochage scolaire}
\txto{69}{4}{Inspection générale}{2014}{Pistes d’évolution pour une politique de lutte contre le décrochage scolaire}
\cse{70}{Le projet d’école}
\txt{70}{2}{Philippe}{Perrenoud}{2008}{Travailler en équipe est un choix stratégique, pas un dogme}
\cse{71}{Le redoublement}
\cse{72}{La place des parents à l’école}
\txt{72}{1}{Georges}{Fotinos}{2014}{L’état des relations école-parents: Entre méfiance, défiance et bienveillance}
\txt{72}{2}{Philippe}{Meirieu}{2014}{Pour un nouveau contrat entre l’école et les parents\dots}
\txtnd{72}{3}{Denis}{Meuret}{Les parents et l’école au Québec et en France}
\cse{73}{Former des citoyens réflexifs}
\txto{73}{1}{CSP}{2014}{Projet de socle commun de connaissance, de compétence et de culture}
\cse{74}{Dispositif plus de maitres que de classes}
\txtond{74}{3}{Académie de Paris, site de la circonscription de la Goutte d’Or}{La co-intervention}
\txto{74}{4}{Inspection générale}{2014}{Rapport sur le dispositif \og plus de maitres que de classes \fg}
\cse{75}{Le conseil école-collège}
\cse{76}{Le conseil école-collège}
\txto{76}{1}{Inspection générale}{2014}{La mise en place des conseils école-collège}
\cse{77}{L’évaluation des élèves}
\cse{78}{L’évaluation des élèves}
\cse{79}{L’évaluation}