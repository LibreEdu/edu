\begin{landscape}
\subsection{Séance détaillée}
\fontsize{10}{12}\selectfont
\newcounter{CompteurDuree}
\newcounter{noActivite}

\newcommand{\duree}[1]{%
	\addtocounter{CompteurDuree}{#1}%
	#1~min (\theCompteurDuree{}~min)%
	}
\newcommand{\activite}{%
    \stepcounter{noActivite}%
    {\bf\No \thenoActivite{} : }}
    
\setcounter{noActivite}{0}

%\makeatletter
%\ifx\DureeTotale\@undefined\relax\else\setcounter{CompteurDuree}{\DureeTotale}\fi
%\makeatother

\setlength\parindent{0pt}

\begin{tabularx}{\HauteurTexte}{X}
\textbf{Séance \no 4 :}  Que va-t-on planter ?\hfill
%\textbf{Durée}: \theCompteurDuree{} min\hfill
\textbf{Matériel :} \emph{Toujours rien ?}, haricots, pâtes\tabularnewline

\textbf{Compétence (B.O. 2015):} Prévoir des conséquences.\hfill
\textbf{Objectif:} Formuler des hypothèses\tabularnewline

\textbf{Vocabulaire :} terre, planter, plante, graine, gland, marron, feuille, tige, bois, caillou\hfill
\textbf{Organisation}: Groupe classe\tabularnewline
\end{tabularx}

\setcounter{CompteurDuree}{0}

\vspace{-12pt}

\begin{longtablex}{\HauteurTexte}{%
|>{\raggedright}p{2cm}
|>{\raggedright}p{6.1cm}
|>{\raggedright}X
|>{\raggedright}p{2.8cm}
|>{\raggedright}p{2cm}
|>{\centering\arraybackslash}p{1.3cm}
|}

\hline
\bf \centering\arraybackslash Activité&
\bf \centering\arraybackslash Enseignant&
\bf \centering\arraybackslash Consignes&
\bf \centering\arraybackslash Élèves&
\bf \centering\arraybackslash Vocabulaire&
\bf \centering\arraybackslash Durée (Total)
\tabularnewline
\hline\endhead

%\hline
%\endfoot

\hline \hline
\endlastfoot


%%%%%%%%%%%%%%%
% Activité #1 %
%%%%%%%%%%%%%%%
\activite{Retour sur l’album}
&
% Enseignant
%%%%%%%%%%%%%%%%%%%
-- Montrer la première de couverture de l’album.
&
% Consignes
%%%%%%%%%%%%%%%%%%%
-- Quel est le titre ?\newline
-- Que se passe-t-il ?\newline
-- Avez-vous envie de faire comme M. Louis ?
&
% Élèves
%%%%%%%%%%%%%%%%%%%
-- Se rappeler du titre de l’album, raconter l’histoire
&
% Vocabulaire
%%%%%%%%%%%%%%%%%%%
-- trou, terre, graine, tasser, oiseau, fleur
&
% Durée
%%%%%%%%%%%%%%%%%%%
\duree{3}
\tabularnewline
\hline


%%%%%%%%%%%%%%%
% Activité #2 %
%%%%%%%%%%%%%%%
\activite{Je me demande}
&
% Enseignant
%%%%%%%%%%%%%%%%%%%
-- Présenter la trace écrite\newline
-- Compléter la première colonne (Je me demande) avec : \og Qu’est-ce qui pousse lorsque je l’enterre ? \fg{} \newline
&
% Consignes
%%%%%%%%%%%%%%%%%%%
-- Peut-on planter n’importe quoi ?
&
% Élèves
%%%%%%%%%%%%%%%%%%%
-- Faire des propositions de ce qu’on va planter.
&
% Vocabulaire
%%%%%%%%%%%%%%%%%%%
-- graine
&
% Durée
%%%%%%%%%%%%%%%%%%%
\duree{3}
\tabularnewline
\hline


%%%%%%%%%%%%%%%
% Activité #2 %
%%%%%%%%%%%%%%%
\activite{Je pense}
&
% Enseignant
%%%%%%%%%%%%%%%%%%%
-- Demander ce que les élèves veulent planter\newline
-- Noter les propositions au tableau (Je pense).
&
% Consignes
%%%%%%%%%%%%%%%%%%%
-- Qu’allons-nous planter ?
&
% Élèves
%%%%%%%%%%%%%%%%%%%
-- Réfléchir sur ce qu’on peut planter.
&
% Vocabulaire
%%%%%%%%%%%%%%%%%%%
-- planter
&
% Durée
%%%%%%%%%%%%%%%%%%%
\duree{6}
\tabularnewline
\hline


%%%%%%%%%%%%%%%
% Activité #4 %
%%%%%%%%%%%%%%%
\activite{La collecte du parc}
&
% Enseignant
%%%%%%%%%%%%%%%%%%%
-- Présenter ce qui a été récolté au parc et trié par les élèves\newline
-- Noter les propositions au tableau (Je pense).
&
% Consignes
%%%%%%%%%%%%%%%%%%%
-- Parmi tout ce que nous avons récolté au parc, qu’allons-nous planter ?
&
% Élèves
%%%%%%%%%%%%%%%%%%%
-- Faire des propositions de ce qu’on va planter.
&
% Vocabulaire
%%%%%%%%%%%%%%%%%%%
-- gland, marron, feuille, tige, bois, caillou
&
% Durée
%%%%%%%%%%%%%%%%%%%
\duree{4}
\tabularnewline
\hline


%%%%%%%%%%%%%%%
% Activité #5 %
%%%%%%%%%%%%%%%
\activite{Pâtes et haricots}
&
% Enseignant
%%%%%%%%%%%%%%%%%%%
-- Présenter d’autres objets à planter\newline
-- Noter les propositions au tableau (Je pense).
&
% Consignes
%%%%%%%%%%%%%%%%%%%
-- Regarder ce que j’ai trouvé à la maison, il y a des haricots, des pâtes et une bille. Parmi ces objets, que peut-on planter ?
&
% Élèves
%%%%%%%%%%%%%%%%%%%
-- Faire des propositions de ce qu’on va planter.
&
% Vocabulaire
%%%%%%%%%%%%%%%%%%%
-- haricots, pâtes, bille
&
% Durée
%%%%%%%%%%%%%%%%%%%
\duree{4}
\tabularnewline
\end{longtablex}



%\makeatletter
%\immediate\write\@auxout{\gdef\string\DureeTotale{\theCompteurDuree}}
%\makeatother
%\fontsize{12}{14.5}\selectfont
%
%\subsection{Évaluation}
%
%L’évaluation est constituée de vignettes à découper (fig. \ref{bande}) et à coller sur la feuille d’évaluation (fig. \ref{eval}).


\end{landscape}




