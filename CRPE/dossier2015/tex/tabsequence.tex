\newcommand{\seancei}{Le parc}
\newcommand{\seanceii}{Le tri}
\newcommand{\seanceiii}{Toujours rien ?}
\newcommand{\seanceiv}{Je pense}
\newcommand{\seancev}{Je plante}
\newcommand{\seancevi}{J’observe}
\newcommand{\seancevii}{Je conclus}
\newcommand{\seanceviii}{Dessin du gland}
\newcommand{\seanceix}{Évaluation}
\newcommand{\seancex}{Les bulbes}

\begin{landscape}
%\makeatletter
%\newcommand{\showfontsize}{\f@size{} pt}
%\makeatother
%\showfontsize
%\the\baselineskip
\fontsize{10}{12}\selectfont
\setcounter{NbSeance}{0}
\setcounter{DureeTotale}{0}
\newcommand{\seance}{%
	\stepcounter{NbSeance}%
	{\bf \No \theNbSeance{} : }}
\newcommand{\obj}{%
	{\bf Obj. : }}
\newcommand{\duree}[1]{%
	\addtocounter{DureeTotale}{#1}%
	{#1 min}}

\begin{longtablex}{\HauteurTexte}{%
|>{\raggedright}p{3cm}
|>{\raggedright}X
|>{\raggedright}p{3.7cm}
|>{\raggedright}p{2.4cm}
|>{\raggedright}p{2.8cm}
|>{\centering\arraybackslash}p{1cm}
|}

\hline
\bf \centering\arraybackslash Séance \quad Objectif&
\bf \centering\arraybackslash Déroulement&
\bf \centering\arraybackslash Compétences (B.O.~2015)&
\bf \centering\arraybackslash Savoirs&
\bf \centering\arraybackslash Organisation et matériel&
\bf Durée\tabularnewline
\hline\endhead

%\multicolumn{6}{r}{\dots}
%\endfoot

\hline \hline
\endlastfoot


%%%%%%%%%%%%%
% Séance #1 %
%%%%%%%%%%%%%
% Séance / Objectif
%%%%%%%%%%%%%%%%%%%
\seance \seancei \newline
\obj Ramasser des graines.
&
% Déroulement
%%%%%%%%%%%%%
-- Aller au parc.\newline
-- Ramasser des glands, des marrons, des bouts de branche, des feuilles, des cailloux, \dots\newline
-- Retour à l’école.	
&
% Compétences
%%%%%%%%%%%%%
-- Découvrir un nouveau milieu.\newline
-- Produire des images.
&
% Savoirs
%%%%%%%%%
-- La forme du gland et du marron
&
% Organisation et matériel
%%%%%%%%%%%%%%%%%%%%%%%%%
-- 4~groupes\newline
-- Atsem + 3~parents\newline
-- 1~sac / groupe\newline
-- Appareil photo
&
%Durée
%%%%%%
\duree{40}
\tabularnewline
\hline


%%%%%%%%%%%%%
% Séance #2 %
%%%%%%%%%%%%%
% Séance / Objectif
%%%%%%%%%%%%%%%%%%%
\seance \seanceii \newline
\obj Trier ce qui a été récupéré au parc.
&
% Déroulement
%%%%%%%%%%%%%
-- Mettre ensemble ce qui se ressemble.
&
% Compétences
%%%%%%%%%%%%%
-- Classer des objets en fonction de caractéristiques liées à leur forme.\newline
-- Produire des images.
&
% Savoirs
-- Voc. : pareil
%%%%%%%%%
&
% Organisation et matériel
%%%%%%%%%%%%%%%%%%%%%%%%%
-- Ateliers\newline
-- Ce qui a été récupéré au parc\newline
-- Barquettes\newline
-- Appareil photo
&
%Durée
%%%%%%
\duree{20}
\tabularnewline
\hline


%%%%%%%%%%%%%
% Séance #3 %
%%%%%%%%%%%%%
% Séance / Objectif
%%%%%%%%%%%%%%%%%%%
\seance \seanceiii \newline
\obj Donner l’envie de planter des graines.
&
% Déroulement
%%%%%%%%%%%%%
-- Lecture de l’album \emph{Toujours rien ?} de Christian Voltz.\newline
-- \og Avez-vous envie de planter des graines, comme M. Louis ? \fg{} 
&
% Compétences
%%%%%%%%%%%%%
-- S’exprimer dans un langage syntaxiquement correct et précis. Reformuler pour se faire mieux comprendre.
&
% Savoirs
%%%%%%%%%
-- Lorsqu’on plante une graine, il y a une fleur qui pousse.
&
% Organisation et matériel
%%%%%%%%%%%%%%%%%%%%%%%%%
-- Gr. classe\newline
-- \emph{Toujours rien ?}
&
%Durée
%%%%%%
\duree{20}
\tabularnewline
\hline


%%%%%%%%%%%%%
% Séance #4 %
%%%%%%%%%%%%%
% Séance / Objectif
%%%%%%%%%%%%%%%%%%%
\seance \seanceiv \newline
\obj Formuler des hypothèses.
&
% Déroulement
%%%%%%%%%%%%%
-- Retour sur l’album et sur leur désir de planter des graines.\newline
-- Qu’est-ce qui pousse lorsque je l’enterre ? (\textbf{Je me demande})\newline
-- Dictée à l’enseignant (\textbf{Je pense}).\newline
-- Présenter ce qui a été récupérer au parc ainsi que d’autres objets : haricots, pâtes, billes\dots
&
% Compétences
%%%%%%%%%%%%%
-- Prévoir des conséquences.
&
% Savoirs
%%%%%%%%%
-- Voc. : terre, planter, plante, graine, gland, marron, feuille, tige, bois, caillou, pâtes, haricots, billes
&
% Organisation et matériel
%%%%%%%%%%%%%%%%%%%%%%%%%
-- Gr. classe\newline
-- \emph{Toujours rien ?}\newline
-- Haricots, pâtes
&
%Durée
%%%%%%
\duree{20}
\tabularnewline
\hline


%%%%%%%%%%%%%
% Séance #5 %
%%%%%%%%%%%%%
% Séance / Objectif
%%%%%%%%%%%%%%%%%%%
\seance \seancev \newline
\obj Planter ce qui a été suggéré à la séance précédente.
&
% Déroulement
%%%%%%%%%%%%%
-- Choisir ce qu’on va planter.\newline
-- Coller sur le pot l’étiquette de son prénom et de ce qu’on a choisi de planter.\newline
-- Pot : mettre de la terre, ce qui a été choisi et recouvrir de terre.
&
% Compétences
%%%%%%%%%%%%%
-- Intégrer la chronologie des tâches requises
&
% Savoirs
%%%%%%%%%
-- Voc. : séance précédente + pot, étiquette, cuillère
&
% Organisation et matériel
%%%%%%%%%%%%%%%%%%%%%%%%%
-- Atelier\newline
-- Terre, 1 pot / élève, \og graines \fg{}, étiquette, cuillère
&
%Durée
%%%%%%
\duree{20}
\tabularnewline
\hline


%%%%%%%%%%%%%
% Séance #6 %
%%%%%%%%%%%%%
% Séance / Objectif
%%%%%%%%%%%%%%%%%%%
\seance \seancevi \newline
\obj Observer ce qui a germé ou pas.
&
% Déroulement
%%%%%%%%%%%%%
-- Déterrer les \og graines \fg{}.\newline
-- Définir ce qui a germé ou pas.\newline
-- Dictée à l’enseignant (\textbf{J’observe}).\newline
-- Prendre des photos\newline
-- Replanter les graines
&
% Compétences
%%%%%%%%%%%%%
-- Observer une manifestation de la vie végétale.\newline
-- Produire des images.
&
% Savoirs
%%%%%%%%%
-- Voc. : germer, graine, gland, marron, feuille, tige, bois, caillou
&
% Organisation et matériel
%%%%%%%%%%%%%%%%%%%%%%%%% 
-- Atelier\newline
-- Terre, 1 pot / élève, \og graines \fg{}, étiquettes, cuillères\newline
-- Appareil photo
&
%Durée
%%%%%%
\duree{30}
\tabularnewline
\hline


%%%%%%%%%%%%%
% Séance #7 %
%%%%%%%%%%%%%
% Séance / Objectif
%%%%%%%%%%%%%%%%%%%
\seance \seancevii \newline
\obj Distinguer ce qui germe de ce qui ne germe pas.
&
% Déroulement
%%%%%%%%%%%%%
-- Dire ce qui a germé.\newline
-- Dictée à l’enseignant (\textbf{Je conclus}).\newline
-- Comprendre la notion de ce qui germé de ce qui ne germe pas.\newline
-- Relier la notion du vivant à la germination.
&
% Compétences
%%%%%%%%%%%%%
-- Comprendre ce qui distingue le vivant du non-vivant.
&
% Savoirs
%%%%%%%%%
-- Voc. : germer, gland, tige, cotylédon, racines
&
% Organisation et matériel
%%%%%%%%%%%%%%%%%%%%%%%%%
-- Gr. classe\newline
&
%Durée
%%%%%%
\duree{20}
\tabularnewline
\hline


%%%%%%%%%%%%%
% Séance #8 %
%%%%%%%%%%%%%
% Séance / Objectif
%%%%%%%%%%%%%%%%%%%
\seance \seanceviii \newline
\obj Dessiner un gland germé
&
% Déroulement
%%%%%%%%%%%%%
-- Dessiner un gland germé.
&
% Compétences
%%%%%%%%%%%%%
-- Pratiquer le dessin pour représenter, en étant fidèle au réel.
&
% Savoirs
%%%%%%%%%
-- Voc. :  gland, tige, cotylédon, racines
&
% Organisation et matériel
%%%%%%%%%%%%%%%%%%%%%%%%%
-- Atelier\newline
-- Cahier de sciences
&
%Durée
%%%%%%
\duree{20}
\tabularnewline
\hline


%%%%%%%%%%%%%
% Séance #9 %
%%%%%%%%%%%%%
% Séance / Objectif
%%%%%%%%%%%%%%%%%%%
\seance \seanceix \newline
\obj Distinguer ce qui germe de ce qui ne germe pas.
&
% Déroulement
%%%%%%%%%%%%%
-- Associer des vignettes représentant ce qu’on a planté avec des vignettes représentant la germination et la non-germination.
&
% Compétences
%%%%%%%%%%%%%
-- Reconnaitre une étape du développement d’un végétal.
&
% Savoirs
%%%%%%%%%
-- Ce qu’est une graine.
&
% Organisation et matériel
%%%%%%%%%%%%%%%%%%%%%%%%%
-- Atelier
&
%Durée
%%%%%%
\duree{20}
\tabularnewline
\hline


%%%%%%%%%%%%%%
% Séance #10 %
%%%%%%%%%%%%%%
% Séance / Objectif
%%%%%%%%%%%%%%%%%%%
\seance \seancex \newline
\obj Planter des bulbes dans la cour.
&
% Déroulement
%%%%%%%%%%%%%
-- Présentation du matériel de jardinage.\newline
-- Planter le bulbe
&
% Compétences
%%%%%%%%%%%%%
-- Assurer les soins nécessaires aux plantations\newline
-- Produire des images.
&
% Savoirs
%%%%%%%%%
--  Voc. : bulbe, pelle, râteau, plantoir
&
% Organisation et matériel
%%%%%%%%%%%%%%%%%%%%%%%%%
-- Gr. classe\newline
-- Appareil photo
&
%Durée
%%%%%%
\duree{20}
\tabularnewline
\end{longtablex}

\makeatletter
\immediate\write\@auxout{\gdef\string\NbSeancestocke{\theNbSeance}}
\immediate\write\@auxout{\gdef\string\DureeTotalestockee{\theDureeTotale}}
\makeatother

\end{landscape}