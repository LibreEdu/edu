%%!TEX TS-program = latex
\documentclass[a4paper,11pt]{article}
\usepackage[utf8]{inputenc} % UTF-8
\usepackage[T1]{fontenc}
\usepackage{lmodern} % Prévient un bug d'affichage evince lié à [T1]{fontenc}
\usepackage[frenchb]{babel} % francisation
\usepackage{adjustbox}
\usepackage[fleqn]{amsmath} % aligne le mode maths à gauche
\usepackage{amssymb} % the amsfont symbols
\usepackage[table, usenames, svgnames]{xcolor} % Couleurs
\usepackage{multicol} % Multi-colonnes
\usepackage{fancyhdr} % Mise en page, en-tête et pied de page
\usepackage{calc} % Opérations
\usepackage{marvosym} % Martin Vogels Symbole (\EUR)
\usepackage{cancel} % draw diagonal lines
\usepackage{units} % typesetting units and nice fractions
\usepackage[autolanguage]{numprint} % écrituredes virgules
\usepackage{tabularx} % creates a paragraph-like column whose width
% automatically expands
\usepackage{wrapfig} % allows figures or tables to have text wrapped around
\usepackage{pst-eucl, pst-plot} % figures géométriques
\usepackage{enumitem}
\usepackage{wasysym} % Symbole Euro
\usepackage{mathtools} % Encadrement dans align*
\usepackage[inline]{asymptote}
\usepackage{tkz-tab}
\usetikzlibrary{external} % set up externalization
\tikzexternalize[shell escape=-enable-write18] % activate externalisation
\tikzset{%
  external/system call={%
    latex \tikzexternalcheckshellescape -halt-on-error
    -interaction=batchmode -jobname "\image" "\texsource" &&
    dvips -o "\image".ps "\image".dvi &&
    ps2eps -f "\image.ps"
    }
  }

%\usepackage{textcomp}

\usepackage[a4paper, dvips, left=1.5cm, right=1.5cm, top=2cm,%
bottom=2cm, marginpar=5mm, marginparsep=5pt]{geometry}
\newcounter{exo}
\frenchbsetup{StandardItemLabels} % remet \textbullet pour les listes
\setlength{\headheight}{18pt}
\setlength{\fboxsep}{1em}
\setlength\parindent{0em}
\setlength\mathindent{0em}
\setlength{\columnsep}{30pt}
\usepackage[bookmarks=true, bookmarksnumbered=true, ps2pdf, pagebackref=true,%
colorlinks=true,linkcolor=blue,plainpages=true,unicode]{hyperref}
\hypersetup{pdfauthor={Jérôme Ortais},pdfsubject={Exercices de
    mathématiques},pdftitle={Exercices créés par Pyromaths, un logiciel libre
    en Python sous licence GPL}}

\def\pshlabel#1{\psframebox*[fillcolor=White,framearc=.2]{\footnotesize $#1$}}
\def\psvlabel#1{\psframebox*[fillcolor=White,framearc=.2]{\footnotesize $#1$}}

\makeatletter
\newcommand\styleexo[1][]{
  \renewcommand{\theenumi}{\arabic{enumi}}
  \renewcommand{\labelenumi}{$\blacktriangleright$\textbf{\theenumi.}}
  \renewcommand{\theenumii}{\alph{enumii}}
  \renewcommand{\labelenumii}{\textbf{\theenumii)}}
  {\fontfamily{pag}\fontseries{b}\selectfont \underline{#1 \theexo}}
  \par\@afterheading\vspace{0.5\baselineskip minus 0.2\baselineskip}}
\newcommand*\exercice{%
  \psset{unit=1cm, dash=4pt 4pt, PointName=default,linecolor=Maroon,
    dotstyle=x, linestyle=solid, hatchcolor=Peru, gridcolor=Olive,
    subgridcolor=Olive, fillcolor=Peru}
  %\ifthenelse{\equal{\theexo}{0}}{}{\filbreak}
  \refstepcounter{exo}%
  \stepcounter{nocalcul}%
  \par\addvspace{1.5\baselineskip minus 1\baselineskip}%
  \@ifstar%
  {\penalty-130\styleexo[Corrigé de l'exercice]}%
  {\penalty-130\styleexo[Exercice]}%
  }
\makeatother
\newlength{\ltxt}
\newcounter{fig}
\newcommand{\figureadroite}[2]{
  \setlength{\ltxt}{\linewidth}
  \setbox\thefig=\hbox{#1}
  \addtolength{\ltxt}{-\wd\thefig}
  \addtolength{\ltxt}{-10pt}
  \begin{minipage}{\ltxt}
    #2
  \end{minipage}
  \hfill
  \begin{minipage}{\wd\thefig}
    #1
  \end{minipage}
  \refstepcounter{fig}
  }
\count1=\year \count2=\year
\ifnum\month<8\advance\count1by-1\else\advance\count2by1\fi
\pagestyle{fancy}
\cfoot{\textsl{\footnotesize{Année \number\count1/\number\count2}}}
\rfoot{\textsl{\tiny{http://www.pyromaths.org}}}
\lhead{\textsl{\footnotesize{Page \thepage/ \pageref{LastPage}}}}
\chead{\Large{\textsc{Révisions}}}
\rhead{\textsl{\footnotesize{Classe de 4\ieme}}}

\begin{document}
  \currentpdfbookmark{Les énoncés des exercices}{Énoncés}
  \newcounter{nocalcul}[exo]
  \renewcommand{\thenocalcul}{\Alph{nocalcul}}
  \raggedcolumns
  \setlength{\columnseprule}{0.5pt}

  \exercice
  Effectuer sans calculatrice :
  \begin{multicols}{3}\noindent
    \begin{enumerate}
    \item $8 \times \left( -4\right) = \ldots\ldots$
    \item $\ldots\ldots + 9 = 11$
    \item $\ldots\ldots + \left( -7\right) = -9$
    \item $\ldots\ldots \div \left( -6\right) = 8$
    \item $27 \div 3 = \ldots\ldots$
    \item $-9 + \ldots\ldots = 1$
    \item $24 \div 6 = \ldots\ldots$
    \item $-1 \times \left( -4\right) = \ldots\ldots$
    \item $-12 - \left( -7\right) = \ldots\ldots$
    \item $\ldots\ldots \times \left( -7\right) = -56$
    \item $30 \div 10 = \ldots\ldots$
    \item $\ldots\ldots \div \left( -7\right) = 4$
    \item $1 - \ldots\ldots = 8$
    \item $\ldots\ldots \times 9 = -90$
    \item $\ldots\ldots \times 9 = 27$
    \item $-5 + \left( -1\right) = \ldots\ldots$
    \item $0 - \left( -8\right) = \ldots\ldots$
    \item $1 + 6 = \ldots\ldots$
    \item $\ldots\ldots - 8 = 7$
    \item $-9 - \left( -6\right) = \ldots\ldots$
    \end{enumerate}
  \end{multicols}

  \exercice
  Calculer en détaillant les étapes. Donner le résultat sous la forme d'une
  fraction la plus simple possible (ou d'un entier lorsque c'est possible).
  \begin{multicols}{4}
    \begin{enumerate}
    \item $\thenocalcul = \dfrac{7}{9}+\dfrac{10}{8}$
      \stepcounter{nocalcul}
    \item $\thenocalcul = 5,8-\dfrac{9}{2}$
      \stepcounter{nocalcul}
    \item $\thenocalcul = \dfrac{8}{3}-\dfrac{8}{24}$
      \stepcounter{nocalcul}
    \item $\thenocalcul = \dfrac{4}{10}-\dfrac{4}{10}$
      \stepcounter{nocalcul}
    \item $\thenocalcul = \dfrac{5}{5}+\dfrac{8}{2}$
      \stepcounter{nocalcul}
    \item $\thenocalcul = \dfrac{1}{7}-\dfrac{7}{8}$
      \stepcounter{nocalcul}
    \item $\thenocalcul = 5-\dfrac{1}{7}$
      \stepcounter{nocalcul}
    \item $\thenocalcul = \dfrac{10}{6}-1$
      \stepcounter{nocalcul}
    \end{enumerate}
  \end{multicols}

  \exercice
  \begin{multicols}{2}
    \begin{enumerate}
    \item Soit $YGC$ un triangle rectangle en $Y$ tel que :\par
      $GY=\unit[2{,}1]{cm}$ et $GC=\unit[2{,}9]{cm}$.\par
      Calculer la longueur $CY$.
      \columnbreak
    \item Soit $IVT$ un triangle rectangle en $T$ tel que :\par
      $IT=\unit[4{,}4]{cm}$ et $VT=\unit[11{,}7]{cm}$.\par
      Calculer la longueur $VI$.
    \end{enumerate}
  \end{multicols}


  \exercice
  Soit $XEQ$ un triangle tel que : $\quad QX=\unit[2{,}4]{cm}\quad$, $\quad
  EX=\unit[1]{cm}\quad$ et $\quad QE=\unit[2{,}6]{cm}$.\par
  Quelle est la nature du triangle $XEQ$?


  \exercice
  \figureadroite{
    \psset{PointSymbol=none,unit=0.48}
    \begin{pspicture}(-1.5,-1.5)(6.82,6.34)
      \SpecialCoor
      \pstTriangle[PosAngleA=225,PosAngleB=-45,PosAngleC=97.02,PointNameA=I,
      PointNameB=K,PointNameC=R](0,0){a}(5.32,0){b}(7.42;52.02){c}
      \pstTriangle[PosAngleB=-45,PosAngleC=142.02,PointSymbolA=none,
      PointName=none,PointNameB=Z,PointNameC=T](0,0){a}(2.8,0){b}(3.9;52.02){c}
    \end{pspicture}
    }
  {Sur la figure ci-contre, les droites $(KR)$ et $(ZT)$ sont parallèles.\par
    On donne $KR=\unit[5{,}9]{cm}$,$\quad IZ=\unit[2{,}8]{cm}$, $\quad
    IT=\unit[3{,}9]{cm}\quad$ et $\quad ZT~=~\unit[3{,}1]{cm}$.\par
    Calculer $IK$ et $IR$, arrondies au millième}


  \exercice
  \begin{minipage}{4cm}
    \begin{pspicture}(-2,-2)(2,2)
      \SpecialCoor\psset{PointSymbol=none}
      \pstGeonode[PointName=V,PosAngle=138](1.5;138){a}
      \pstGeonode[PointName=Z,PosAngle=318](1.5;318){b}
      \pstGeonode[PointName=W,PosAngle=244.26](1.5;244.26){c}
      \pspolygon(a)(b)(c)\pscircle(0,0){1.5}
      \rput(1.8;48){$\big(\mathcal{C}\big)$}
    \end{pspicture}
  \end{minipage}\hfill
  \begin{minipage}{13cm}
    $\big(\mathcal{C}\big)$ est un cercle de diamètre $[VZ]$ et $W$ est un
    point de $\big(\mathcal{C}\big)$.\par
    On donne $ZW=\unit[3{,}6]{cm}$ et $VZ=\unit[6]{cm}$.\par
    Calculer la longueur $VW$.
  \end{minipage}


  \exercice
  \begin{multicols}{2}
    \begin{enumerate}
    \item $DCB$ est un triangle rectangle en $D$ tel que :\par
      $DB=\unit[3{,}5]{cm}$ et $BC=\unit[11{,}5]{cm}$.\par
      Calculer la mesure de l'angle $\widehat{DBC}$, arrondie au millième.\par
      \columnbreak
    \item $JMZ$ est un triangle rectangle en $J$ tel que :\par
      $JZ=\unit[2]{cm}$ et $\widehat{JZM}=59\degres$.\par
      Calculer la longueur $ZM$, arrondie au millième.\par
    \end{enumerate}
  \end{multicols}

  \exercice
  Développer et réduire chacune des expressions littérales suivantes :
  \begin{multicols}{2}
    $A=4\times 5\,x$\\
    $B=4\,x\times 2$\\
    $C=-8\,x+4+7\times \left( 3\,x+9\right) $\\
    $D=8\times \left( -7\,x+5\right) -10\,x$\\
    $E=9+2\times \left( -4\,x-8\right) $
  \end{multicols}

  \exercice
  Développer et réduire chacune des expressions littérales suivantes :
  \begin{multicols}{2}
    $A=7\,x\times x$\\
    $B=4\,x\times 8\,x$\\
    $C=\left( 10\,x-4\right) \times \left( 5\,x+3\right) +8$\\
    $D=2\,x^{2}+\left( -2\,x-9\right) \times \left( -8\,x-6\right) $\\
    $E=\left( 4\,x+8\right) \times \left( -4\,x-1\right) -10\,x+5$
  \end{multicols}

  \exercice
  Le principe est le suivant : l'extrémité de chaque flèche indique la somme de
  la ligne ou de la colonne correspondante. Compléter, sachant que $x$
  représente un nombre quelconque et que le contenu des deux cases grises doit
  être le même.\par
  \begin{center}
    \begin{pspicture}(16,6)
      \psframe[fillstyle=solid,fillcolor=Gray](0,0)(2,1)
      \psframe[fillstyle=solid,fillcolor=Gray](14,5)(16,6)
      \psframe(14,2)(16,3)
      \psframe(14,3)(16,4)
      \multido{\i=4+2}{4}{
        \rput(\i,0){\psframe(0,0)(2,1)\psline[linewidth=2pt]{<-}(1,1.2)(1,1.8)}
        \rput(\i,2){\psframe(0,0)(2,1)}
        \rput(\i,3){\psframe(0,0)(2,1)}}
      \psline[linewidth=2pt]{<-}(2.2,.5)(3.8,.5)
      \psline[linewidth=2pt]{->}(12.2,2.5)(13.8,2.5)
      \psline[linewidth=2pt]{->}(12.2,3.5)(13.8,3.5)
      \psline[linewidth=2pt]{<-}(15,4.8)(15,4.2)
      \rput(5,2.5){$-7\,x-7$}
      \rput(7,2.5){$-2\,x-10$}
      \rput(9,2.5){$9\,x-9$}
      \rput(11,2.5){$-5\,x+5$}
      \rput(5,3.5){$9\,x-4$}
      \rput(7,3.5){$5\,x+2$}
      \rput(9,3.5){$-3\,x-2$}
      \rput(11,3.5){$7\,x-7$}
    \end{pspicture}
  \end{center}

  \exercice
  Réduire, si possible, les expressions suivantes :
  \begin{multicols}{3}\noindent
    \begin{enumerate}
    \item $\thenocalcul = 2\,a^{2}+a^{2}$
      \stepcounter{nocalcul}
    \item $\thenocalcul = 10\,a\times 4$
      \stepcounter{nocalcul}
    \item $\thenocalcul = -8\,y^{2}\times 4$
      \stepcounter{nocalcul}
    \item $\thenocalcul = 3\times 10\,x^{2}$
      \stepcounter{nocalcul}
    \item $\thenocalcul = -10\,a^{2}+4\,a^{2}$
      \stepcounter{nocalcul}
    \item $\thenocalcul = 8\,y^{2}+9\,y^{2}$
      \stepcounter{nocalcul}
    \item $\thenocalcul = 3\,y^{2}-3\,y$
      \stepcounter{nocalcul}
    \item $\thenocalcul = y^{2}-10\,y^{2}$
      \stepcounter{nocalcul}
    \item $\thenocalcul = 7\,a^{2}\times \left( -3\right) $
      \stepcounter{nocalcul}
    \end{enumerate}
  \end{multicols}

  \exercice
  Réduire chacune des expressions littérales suivantes :
  \begin{multicols}{2}
    $A=-4\,x-\left( -2\,x-4\right) +2$\\
    $B=-\left( 7\,x-1\right) +2\,x-8$\\
    $C=-\left( 5\,x+3\right) +4-7\,x$\\
    $D=\left( -4\,x-10\right) +10-6\,x$\\
    $E=8\,x+3-\left( x-8\right) $\\
    $F=10+\left( 7\,x-10\right) +3\,x$
  \end{multicols}

  \exercice
  Compléter par le nombre qui convient :
  \begin{multicols}{3}
    \noindent%
    \begin{enumerate}
    \item $0{,}000\,308\,1=3{,}081\times\dotfill$
    \item $4{,}083\times\dotfill=0{,}000\,004\,083$
    \item $9{,}609\times\dotfill=96\,090$
    \item $8{,}082\times\dotfill=8\,082\,000$
    \item $0{,}039=3{,}9\times\dotfill$
    \item $3{,}098\times\dotfill=309{,}8$
    \end{enumerate}
  \end{multicols}


  \exercice
  Calculer les expressions suivantes et donner l'écriture scientifique du
  résultat.
  \begin{multicols}{2}
    \noindent%
    \[ \thenocalcul = \cfrac{\nombre{5,6} \times 10^{7} \times \nombre{8,1}
        \times 10^{-2}}{\nombre{7200} \times \big( 10^{-2} \big) ^3} \]
    \columnbreak\stepcounter{nocalcul}%
    \[ \thenocalcul = \cfrac{\nombre{480} \times 10^{9} \times \nombre{0,3}
        \times 10^{-6}}{\nombre{192} \times \big( 10^{6} \big) ^3} \]
  \end{multicols}


  \exercice
  Compléter par un nombre de la forme $a^n$ avec $a$ et $n$ entiers :
  \begin{multicols}{4}
    \noindent%
    \begin{enumerate}
    \item $(8^{10})^{3}=\dotfill$
    \item $\dfrac{5^{8}}{5^{4}}=\dotfill$
    \item $4^{7} \times 2^{7} = \dotfill$
    \item $5^{2}\times5^{8}=\dotfill$
    \item $(8^{6})^{5}=\dotfill$
    \item $11^{2} \times 9^{2} = \dotfill$
    \item $8^{7}\times8^{11}=\dotfill$
    \item $\dfrac{7^{9}}{7^{3}}=\dotfill$
    \end{enumerate}
  \end{multicols}


  \exercice
  Écrire sous la forme d'une puissance de 10 puis donner l'écriture
  décimale de ces nombres :
  \begin{multicols}{2}
    \noindent%
    \begin{enumerate}
    \item $\dfrac{10^{4}}{10^{-3}}=\dotfill$
    \item $(10^{2})^{4}=\dotfill$
    \item $10^{5} \times 10^{5} = \dotfill$
    \item $10^{2} \times 10^{5} = \dotfill$
    \item $\dfrac{10^{5}}{10^{-1}}=\dotfill$
    \item $(10^{-2})^{0}=\dotfill$
    \end{enumerate}
  \end{multicols}

  \label{LastPage}
  \newpage
  \currentpdfbookmark{Le corrigé des exercices}{Corrigé}
  \lhead{\textsl{{\footnotesize Page \thepage/ \pageref{LastCorPage}}}}
  \setcounter{page}{1} \setcounter{exo}{0}

  \exercice*
  Effectuer sans calculatrice :
  \begin{multicols}{3}\noindent
    \begin{enumerate}
    \item $8 \times \left( -4\right) = \mathbf{-32}$
    \item $\mathbf{2} + 9 = 11$
    \item $\mathbf{-2} + \left( -7\right) = -9$
    \item $\mathbf{-48} \div \left( -6\right) = 8$
    \item $27 \div 3 = \mathbf{9}$
    \item $-9 + \mathbf{10} = 1$
    \item $24 \div 6 = \mathbf{4}$
    \item $-1 \times \left( -4\right) = \mathbf{4}$
    \item $-12 - \left( -7\right) = \mathbf{-5}$
    \item $\mathbf{8} \times \left( -7\right) = -56$
    \item $30 \div 10 = \mathbf{3}$
    \item $\mathbf{-28} \div \left( -7\right) = 4$
    \item $1 - \mathbf{\left( -7\right)} = 8$
    \item $\mathbf{-10} \times 9 = -90$
    \item $\mathbf{3} \times 9 = 27$
    \item $-5 + \left( -1\right) = \mathbf{-6}$
    \item $0 - \left( -8\right) = \mathbf{8}$
    \item $1 + 6 = \mathbf{7}$
    \item $\mathbf{15} - 8 = 7$
    \item $-9 - \left( -6\right) = \mathbf{-3}$
    \end{enumerate}
  \end{multicols}

  \exercice*
  Calculer en détaillant les étapes. Donner le résultat sous la forme d'une
  fraction la plus simple possible (ou d'un entier lorsque c'est possible).
  \begin{multicols}{4}
    \begin{enumerate}
    \item $\thenocalcul = \dfrac{7}{9}+\dfrac{10}{8}$
      \[\thenocalcul = \dfrac{7_{\times 8}}{9_{\times 8}}+\dfrac{10_{\times
            9}}{8_{\times 9}}\]
      \[\thenocalcul = \dfrac{56}{72}+\dfrac{90}{72}\]
      \[\thenocalcul = \dfrac{146}{72}\]
      \[\thenocalcul = \dfrac{73\times \cancel{2}}{36\times \cancel{2}}\]
      \[\thenocalcul = \dfrac{73}{36}\]
      \stepcounter{nocalcul}
    \item $\thenocalcul = 5,8-\dfrac{9}{2}$
      \[\thenocalcul = \dfrac{58}{10}-\dfrac{9_{\times 5}}{2_{\times 5}}\]
      \[\thenocalcul = \dfrac{58}{10}-\dfrac{45}{10}\]
      \[\thenocalcul = \dfrac{13}{10}\]
      \stepcounter{nocalcul}
    \item $\thenocalcul = \dfrac{8}{3}-\dfrac{8}{24}$
      \[\thenocalcul = \dfrac{8_{\times 8}}{3_{\times 8}}-\dfrac{8}{24}\]
      \[\thenocalcul = \dfrac{64}{24}-\dfrac{8}{24}\]
      \[\thenocalcul = \dfrac{56}{24}\]
      \[\thenocalcul = \dfrac{7\times \cancel{8}}{3\times \cancel{8}}\]
      \[\thenocalcul = \dfrac{7}{3}\]
      \stepcounter{nocalcul}
    \item $\thenocalcul = \dfrac{4}{10}-\dfrac{4}{10}$
      \[\thenocalcul = 0\]
      \stepcounter{nocalcul}
    \item $\thenocalcul = \dfrac{5}{5}+\dfrac{8}{2}$
      \[\thenocalcul = \dfrac{5_{\times 2}}{5_{\times 2}}+\dfrac{8_{\times
            5}}{2_{\times 5}}\]
      \[\thenocalcul = \dfrac{10}{10}+\dfrac{40}{10}\]
      \[\thenocalcul = \dfrac{50}{10}\]
      \[\thenocalcul = \dfrac{5\times \cancel{10}}{1\times \cancel{10}}\]
      \[\thenocalcul = 5\]
      \stepcounter{nocalcul}
    \item $\thenocalcul = \dfrac{1}{7}-\dfrac{7}{8}$
      \[\thenocalcul = \dfrac{1_{\times 8}}{7_{\times 8}}-\dfrac{7_{\times
            7}}{8_{\times 7}}\]
      \[\thenocalcul = \dfrac{8}{56}-\dfrac{49}{56}\]
      \[\thenocalcul = \dfrac{-41}{56}\]
      \stepcounter{nocalcul}
    \item $\thenocalcul = 5-\dfrac{1}{7}$
      \[\thenocalcul = \dfrac{5_{\times 7}}{1_{\times 7}}-\dfrac{1}{7}\]
      \[\thenocalcul = \dfrac{35}{7}-\dfrac{1}{7}\]
      \[\thenocalcul = \dfrac{34}{7}\]
      \stepcounter{nocalcul}
    \item $\thenocalcul = \dfrac{10}{6}-1$
      \[\thenocalcul = \dfrac{10}{6}-\dfrac{1_{\times 6}}{1_{\times 6}}\]
      \[\thenocalcul = \dfrac{10}{6}-\dfrac{6}{6}\]
      \[\thenocalcul = \dfrac{4}{6}\]
      \[\thenocalcul = \dfrac{\cancel{2}\times 2}{3\times \cancel{2}}\]
      \[\thenocalcul = \dfrac{2}{3}\]
      \stepcounter{nocalcul}
    \end{enumerate}
  \end{multicols}

  \exercice*
  \begin{multicols}{2}
    \begin{enumerate}
    \item Soit $YGC$ un triangle rectangle en $Y$ tel que :\par
      $GY=\unit[2{,}1]{cm}$ et $GC=\unit[2{,}9]{cm}$.\par
      Calculer la longueur $CY$.
      \par\dotfill{}\par

      Le triangle $YGC$ est rectangle en $Y$.\par
      Son hypoténuse est $[GC]$.\par
      D'après le \textbf{théorème de Pythagore} :
      \[GC^2=CY^2+GY^2\]
      \[CY^2=GC^2-GY^2\kern1cm\text{(On cherche }CY)\]
      \[CY^2=2{,}9^2-2{,}1^2\]
      \[CY^2=8{,}41-4{,}41\]
      \[CY^2=4\]
      \[ \boxed{\text{Donc }CY=\sqrt{4}=\unit[2]{cm}}\]
      \columnbreak
    \item Soit $IVT$ un triangle rectangle en $T$ tel que :\par
      $IT=\unit[4{,}4]{cm}$ et $VT=\unit[11{,}7]{cm}$.\par
      Calculer la longueur $VI$.
      \par\dotfill{}\par

      Le triangle $IVT$ est rectangle en $T$.\par
      Son hypoténuse est $[VI]$.\par
      D'après le \textbf{théorème de Pythagore} :
      \[VI^2=IT^2+VT^2\]
      \[VI^2=4{,}4^2+11{,}7^2\]
      \[VI^2=19{,}36+136{,}89\]
      \[VI^2=156{,}25\]
      \[\boxed{\text{Donc }VI=\sqrt{156{,}25}=\unit[12{,}5]{cm}}\]
    \end{enumerate}
  \end{multicols}


  \exercice*
  Soit $XEQ$ un triangle tel que : $\quad QX=\unit[2{,}4]{cm}\quad$, $\quad
  EX=\unit[1]{cm}\quad$ et $\quad QE=\unit[2{,}6]{cm}$.\par
  Quelle est la nature du triangle $XEQ$?

  \par\dotfill{}\\

  Le triangle XEQ n'est ni isocèle, ni équilatéral.\par

  $\left.
  \renewcommand{\arraystretch}{2}
  \begin{array}{l}
    \bullet QE^2=2{,}6^2=6{,}76\qquad\text{(}[QE]\text{ est le plus grand
      côté.)}\\

    \bullet  EX^2+QX^2=1^2+2{,}4^2=6{,}76

  \end{array}
  \right\rbrace$
  Donc $QE^2=EX^2+QX^2$.\par
  D'après la \textbf{réciproque du théorème de Pythagore},
  \fbox{le triangle $XEQ$ est rectangle en $X$.}

  \exercice*
  \figureadroite{
    \psset{PointSymbol=none,unit=0.48}
    \begin{pspicture}(-1.5,-1.5)(6.82,6.34)
      \SpecialCoor
      \pstTriangle[PosAngleA=225,PosAngleB=-45,PosAngleC=97.02,PointNameA=I,
      PointNameB=K,PointNameC=R](0,0){a}(5.32,0){b}(7.42;52.02){c}
      \pstTriangle[PosAngleB=-45,PosAngleC=142.02,PointSymbolA=none,
      PointName=none,PointNameB=Z,PointNameC=T](0,0){a}(2.8,0){b}(3.9;52.02){c}
    \end{pspicture}
    }
  {Sur la figure ci-contre, les droites $(KR)$ et $(ZT)$ sont parallèles.\par
    On donne $KR=\unit[5{,}9]{cm}$,$\quad IZ=\unit[2{,}8]{cm}$, $\quad
    IT=\unit[3{,}9]{cm}\quad$ et $\quad ZT~=~\unit[3{,}1]{cm}$.\par
    Calculer $IK$ et $IR$, arrondies au millième}

  Dans le triangle $IKR$,~ $Z$ est sur le côté $[IK]$,~
  $T$ est sur le côté $[IR]$ et les droites $(KR)$ et $(ZT)$ sont
  parallèles.\par
  D'après le \textbf{théorème de Thalès} :
  $\qquad\mathbf{\cfrac{IK}{IZ}=\cfrac{IR}{IT}=\cfrac{KR}{ZT}}$

  \[\frac{IK}{2{,}8}=\frac{IR}{3{,}9}=\frac{5{,}9}{3{,}1}\]
  $\cfrac{5{,}9}{3{,}1}=\cfrac{IK}{2{,}8}\quad$ donc $\quad
  \boxed{IK=\cfrac{2{,}8\times 5{,}9}{3{,}1}\simeq\unit[5{,}329]{cm}}$\par
  $\cfrac{5{,}9}{3{,}1}=\cfrac{IR}{3{,}9}\quad$ donc
  $\quad\boxed{IR=\cfrac{3{,}9\times 5{,}9}{3{,}1}\simeq\unit[7{,}423]{cm}}$\par


  \exercice*
  \begin{minipage}{4cm}
    \begin{pspicture}(-2,-2)(2,2)
      \SpecialCoor\psset{PointSymbol=none}
      \pstGeonode[PointName=V,PosAngle=138](1.5;138){a}
      \pstGeonode[PointName=Z,PosAngle=318](1.5;318){b}
      \pstGeonode[PointName=W,PosAngle=244.26](1.5;244.26){c}
      \pspolygon(a)(b)(c)\pscircle(0,0){1.5}
      \rput(1.8;48){$\big(\mathcal{C}\big)$}
    \end{pspicture}
  \end{minipage}\hfill
  \begin{minipage}{13cm}
    $\big(\mathcal{C}\big)$ est un cercle de diamètre $[VZ]$ et $W$ est un
    point de $\big(\mathcal{C}\big)$.\par
    On donne $ZW=\unit[3{,}6]{cm}$ et $VZ=\unit[6]{cm}$.\par
    Calculer la longueur $VW$.
    \par\dotfill{}\\

    $[VZ]$ est le diamètre du cercle circonscrit au triangle $VWZ$.\par
    \fbox{Donc le triangle VWZ est rectangle en W.}\\

    D'après le \textbf{théorème de Pythagore} :
    \[VZ^2=ZW^2+VW^2\kern1cm\text{(car }[VZ]\text{ est \emph{l'hypoténuse})}\]
    \[VW^2=VZ^2-ZW^2\kern1cm\text{(On cherche }VW)\]
    \[VW^2=6^2-3{,}6^2\]
    \[VW^2=36-12{,}96\]
    \[VW^2=23{,}04\]
    \[\boxed{\text{Donc }VW=\sqrt{23{,}04}=\unit[4{,}8]{cm}}\]
  \end{minipage}


  \exercice*
  \begin{multicols}{2}
    \begin{enumerate}
    \item $DCB$ est un triangle rectangle en $D$ tel que :\par
      $DB=\unit[3{,}5]{cm}$ et $BC=\unit[11{,}5]{cm}$.\par
      Calculer la mesure de l'angle $\widehat{DBC}$, arrondie au millième.\par
      Dans le triangle $DCB$ rectangle en $D$,
      \[ \cos\widehat{DBC}=\cfrac{DB}{BC} \]
      \[ \cos\widehat{DBC}=\cfrac{3{,}5}{11{,}5} \]
      \[ \boxed{\widehat{DBC}=\cos^{-1}\left(\cfrac{3{,}5}{11{,}5}\right)
          \simeq72{,}281\degres} \]
      \columnbreak
    \item $JMZ$ est un triangle rectangle en $J$ tel que :\par
      $JZ=\unit[2]{cm}$ et $\widehat{JZM}=59\degres$.\par
      Calculer la longueur $ZM$, arrondie au millième.\par
      Dans le triangle $JMZ$ rectangle en $J$,
      \[ \cos\widehat{JZM}=\cfrac{JZ}{ZM} \]
      \[ \cos59=\cfrac{2}{ZM} \]
      \[ \boxed{ZM=\cfrac{2}{\cos59} \simeq\unit[3{,}883]{cm}} \]
    \end{enumerate}
  \end{multicols}

  \exercice*
  Développer et réduire chacune des expressions littérales suivantes :
  \begin{multicols}{2}
    $A=4\times 5\,x$\\
    $A=4\times 5\times x$
    \\
    \fbox{$A=20\,x$}\\

    $B=4\,x\times 2$\\
    $B=4\times x\times 2$\\
    $B=4\times 2\times x$
    \\
    \fbox{$B=8\,x$}\\

    $C=-8\,x+4+7\times \left( 3\,x+9\right) $\\
    $C=-8\,x+4+7\times 3\,x+7\times 9$\\
    $C=-8\,x+4+7\times 3\times x+63$\\
    $C=-8\,x+4+21\,x+63$\\
    $C=-8\,x+21\,x+4+63$\\
    $C=\left( -8+21\right) \,x+67$
    \\
    \fbox{$C=13\,x+67$}\\

    $D=8\times \left( -7\,x+5\right) -10\,x$\\
    $D=8\times \left( -7\,x\right) +8\times 5-10\,x$\\
    $D=8\times \left( -7\right) \times x+40-10\,x$\\
    $D=-56\,x-10\,x+40$\\
    $D=\left( -56-10\right) \,x+40$
    \\
    \fbox{$D=-66\,x+40$}\\

    $E=9+2\times \left( -4\,x-8\right) $\\
    $E=9+2\times \left( -4\,x\right) +2\times \left( -8\right) $\\
    $E=9+2\times \left( -4\right) \times x-16$\\
    $E=9-8\,x-16$\\
    $E=-8\,x+9-16$
    \\
    \fbox{$E=-8\,x-7$}\\

  \end{multicols}

  \exercice*
  Développer et réduire chacune des expressions littérales suivantes :
  \begin{multicols}{2}
    $A=7\,x\times x$\\
    $A=7\times x\times x$
    \\
    \fbox{$A=7\,x^{2}$}\\

    $B=4\,x\times 8\,x$\\
    $B=4\times x\times 8\times x$\\
    $B=4\times 8\times x\times x$
    \\
    \fbox{$B=32\,x^{2}$}\\

  \end{multicols}
  $C=\left( 10\,x-4\right) \times \left( 5\,x+3\right) +8$\\
  $C=10\,x\times 5\,x+10\,x\times 3-4\times 5\,x-4\times 3+8$\\
  $C=10\times x\times 5\times x+10\times x\times 3-4\times 5\times x-12+8$\\
  $C=10\times 5\times x\times x+10\times 3\times x-20\,x-4$\\
  $C=50\,x^{2}+30\,x-20\,x-4$\\
  $C=50\,x^{2}+\left( 30-20\right) \,x-4$
  \\
  \fbox{$C=50\,x^{2}+10\,x-4$}\\

  $D=2\,x^{2}+\left( -2\,x-9\right) \times \left( -8\,x-6\right) $\\
  $D=2\,x^{2}-2\,x\times \left( -8\,x\right) -2\,x\times \left( -6\right)
  -9\times \left( -8\,x\right) -9\times \left( -6\right) $\\
  $D=2\,x^{2}-2\times x\times \left( -8\right) \times x-2\times x\times \left(
  -6\right) -9\times \left( -8\right) \times x+54$\\
  $D=2\,x^{2}-2\times \left( -8\right) \times x\times x-2\times \left(
  -6\right) \times x+72\,x+54$\\
  $D=2\,x^{2}-\left( -16\,x^{2}\right) -\left( -12\,x\right) +72\,x+54$\\
  $D=18\,x^{2}+12\,x+72\,x+54$\\
  $D=18\,x^{2}+\left( 12+72\right) \,x+54$
  \\
  \fbox{$D=18\,x^{2}+84\,x+54$}\\

  $E=\left( 4\,x+8\right) \times \left( -4\,x-1\right) -10\,x+5$\\
  $E=4\,x\times \left( -4\,x\right) +4\,x\times \left( -1\right) +8\times
  \left( -4\,x\right) +8\times \left( -1\right) -10\,x+5$\\
  $E=4\times x\times \left( -4\right) \times x+4\times x\times \left( -1\right)
  +8\times \left( -4\right) \times x-8-10\,x+5$\\
  $E=4\times \left( -4\right) \times x\times x+4\times \left( -1\right) \times
  x-32\,x-10\,x-8+5$\\
  $E=-16\,x^{2}-4\,x+\left( -32-10\right) \,x-3$\\
  $E=-16\,x^{2}+\left( -4+\left (  -32\right )  -10\right) \,x-3$
  \\
  \fbox{$E=-16\,x^{2}-46\,x-3$}\\


  \exercice*
  Le principe est le suivant : l'extrémité de chaque flèche indique la somme de
  la ligne ou de la colonne correspondante. Compléter, sachant que $x$
  représente un nombre quelconque et que le contenu des deux cases grises doit
  être le même.\par
  \begin{center}
    \begin{pspicture}(16,6)
      \psframe[fillstyle=solid,fillcolor=Gray](0,0)(2,1)
      \psframe[fillstyle=solid,fillcolor=Gray](14,5)(16,6)
      \psframe(14,2)(16,3)
      \psframe(14,3)(16,4)
      \multido{\i=4+2}{4}{
        \rput(\i,0){\psframe(0,0)(2,1)\psline[linewidth=2pt]{<-}(1,1.2)(1,1.8)}
        \rput(\i,2){\psframe(0,0)(2,1)}
        \rput(\i,3){\psframe(0,0)(2,1)}}
      \psline[linewidth=2pt]{<-}(2.2,.5)(3.8,.5)
      \psline[linewidth=2pt]{->}(12.2,2.5)(13.8,2.5)
      \psline[linewidth=2pt]{->}(12.2,3.5)(13.8,3.5)
      \psline[linewidth=2pt]{<-}(15,4.8)(15,4.2)
      \rput(5,2.5){$-7\,x-7$}
      \rput(7,2.5){$-2\,x-10$}
      \rput(9,2.5){$9\,x-9$}
      \rput(11,2.5){$-5\,x+5$}
      \rput(5,3.5){$9\,x-4$}
      \rput(7,3.5){$5\,x+2$}
      \rput(9,3.5){$-3\,x-2$}
      \rput(11,3.5){$7\,x-7$}
      \rput(5,.5){$2\,x-11$}
      \rput(7,.5){$3\,x-8$}
      \rput(9,.5){$6\,x-11$}
      \rput(11,.5){$2\,x-2$}
      \rput(15,2.5){$-5\,x-21$}
      \rput(15,3.5){$18\,x-11$}
      \rput(1,.5){$13\,x-32$}
      \rput(15,5.5){$13\,x-32$}
    \end{pspicture}
  \end{center}
  \subsubsection*{Ligne du bas :}
  \begin{multicols}{4}
    $A=9\,x-4-7\,x-7$\\
    $A=9\,x-7\,x-4-7$\\
    $A=\left( 9-7\right) \,x-11$
    \\
    \fbox{$A=2\,x-11$}\\

    $B=5\,x+2-2\,x-10$\\
    $B=5\,x-2\,x+2-10$\\
    $B=\left( 5-2\right) \,x-8$
    \\
    \fbox{$B=3\,x-8$}\\

    $C=-3\,x-2+9\,x-9$\\
    $C=-3\,x+9\,x-2-9$\\
    $C=\left( -3+9\right) \,x-11$
    \\
    \fbox{$C=6\,x-11$}\\

    $D=7\,x-7-5\,x+5$\\
    $D=7\,x-5\,x-7+5$\\
    $D=\left( 7-5\right) \,x-2$
    \\
    \fbox{$D=2\,x-2$}\\

  \end{multicols}
  \subsubsection*{Colonne de droite :}
  \begin{multicols}{2}
    $E=-7\,x-7-2\,x-10+9\,x-9-5\,x+5$\\
    $E=-7\,x-2\,x+9\,x-5\,x-7-10-9+5$\\
    $E=\left( -7-2+9-5\right) \,x-21$
    \\
    \fbox{$E=-5\,x-21$}\\

    $F=9\,x-4+5\,x+2-3\,x-2+7\,x-7$\\
    $F=9\,x+5\,x-3\,x+7\,x-4+2-2-7$\\
    $F=\left( 9+5-3+7\right) \,x-11$
    \\
    \fbox{$F=18\,x-11$}\\

  \end{multicols}
  \subsubsection*{Cases grises :}
  \begin{multicols}{2}
    $G=2\,x-11+3\,x-8+6\,x-11+2\,x-2$\\
    $G=2\,x+3\,x+6\,x+2\,x-11-8-11-2$\\
    $G=\left( 2+3+6+2\right) \,x-32$
    \\
    \fbox{$G=13\,x-32$}\\

    $H=-5\,x-21+18\,x-11$\\
    $H=-5\,x+18\,x-21-11$\\
    $H=\left( -5+18\right) \,x-32$
    \\
    \fbox{$H=13\,x-32$}\\

  \end{multicols}

  \exercice*
  Réduire, si possible, les expressions suivantes :
  \begin{multicols}{3}\noindent
    \begin{enumerate}
    \item $\thenocalcul = 2\,a^{2}+a^{2}$
      \[\thenocalcul = \left( 2+1\right) \,a^{2}\]
      \[\thenocalcul = 3\,a^{2}\]
      \stepcounter{nocalcul}
    \item $\thenocalcul = 10\,a\times 4$
      \[\thenocalcul = 10\times a\times 4\]
      \[\thenocalcul = 10\times 4\times a\]
      \[\thenocalcul = 40\,a\]
      \stepcounter{nocalcul}
    \item $\thenocalcul = -8\,y^{2}\times 4$
      \[\thenocalcul = -8\times y^{2}\times 4\]
      \[\thenocalcul = -8\times 4\times y^{2}\]
      \[\thenocalcul = -32\,y^{2}\]
      \stepcounter{nocalcul}
    \item $\thenocalcul = 3\times 10\,x^{2}$
      \[\thenocalcul = 3\times 10\times x^{2}\]
      \[\thenocalcul = 30\,x^{2}\]
      \stepcounter{nocalcul}
    \item $\thenocalcul = -10\,a^{2}+4\,a^{2}$
      \[\thenocalcul = \left( -10+4\right) \,a^{2}\]
      \[\thenocalcul = -6\,a^{2}\]
      \stepcounter{nocalcul}
    \item $\thenocalcul = 8\,y^{2}+9\,y^{2}$
      \[\thenocalcul = \left( 8+9\right) \,y^{2}\]
      \[\thenocalcul = 17\,y^{2}\]
      \stepcounter{nocalcul}
    \item $\thenocalcul = 3\,y^{2}-3\,y$
      \stepcounter{nocalcul}
    \item $\thenocalcul = y^{2}-10\,y^{2}$
      \[\thenocalcul = \left( 1-10\right) \,y^{2}\]
      \[\thenocalcul = -9\,y^{2}\]
      \stepcounter{nocalcul}
    \item $\thenocalcul = 7\,a^{2}\times \left( -3\right) $
      \[\thenocalcul = 7\times a^{2}\times \left( -3\right) \]
      \[\thenocalcul = 7\times \left( -3\right) \times a^{2}\]
      \[\thenocalcul = -21\,a^{2}\]
      \stepcounter{nocalcul}
    \end{enumerate}
  \end{multicols}

  \exercice*
  Réduire chacune des expressions littérales suivantes :
  \begin{multicols}{2}
    $A=-4\,x-\left( -2\,x-4\right) +2$\\
    $A=-4\,x+2\,x+4+2$\\
    $A=\left( -4+2\right) \,x+6$
    \\
    \fbox{$A=-2\,x+6$}\\

    $B=-\left( 7\,x-1\right) +2\,x-8$\\
    $B=-7\,x+1+2\,x-8$\\
    $B=-7\,x+2\,x+1-8$\\
    $B=\left( -7+2\right) \,x-7$
    \\
    \fbox{$B=-5\,x-7$}\\

    $C=-\left( 5\,x+3\right) +4-7\,x$\\
    $C=-5\,x-3+4-7\,x$\\
    $C=-5\,x-7\,x-3+4$\\
    $C=\left( -5-7\right) \,x+1$
    \\
    \fbox{$C=-12\,x+1$}\\

    $D=\left( -4\,x-10\right) +10-6\,x$\\
    $D=-4\,x-10-6\,x+10$\\
    $D=-4\,x-6\,x-10+10$\\
    $D=\left( -4-6\right) \,x$
    \\
    \fbox{$D=-10\,x$}\\

    $E=8\,x+3-\left( x-8\right) $\\
    $E=8\,x+3-x+8$\\
    $E=8\,x-x+3+8$\\
    $E=\left( 8-1\right) \,x+11$
    \\
    \fbox{$E=7\,x+11$}\\

    $F=10+\left( 7\,x-10\right) +3\,x$\\
    $F=10+7\,x-10+3\,x$\\
    $F=7\,x+3\,x+10-10$\\
    $F=\left( 7+3\right) \,x$
    \\
    \fbox{$F=10\,x$}\\

  \end{multicols}

  \exercice*
  Compléter par le nombre qui convient :
  \begin{multicols}{3}
    \noindent%
    \begin{enumerate}
    \item $0{,}000\,308\,1=3{,}081\times\mathbf{10^{-4}}$
    \item $4{,}083\times\mathbf{10^{-6}}=0{,}000\,004\,083$
    \item $9{,}609\times\mathbf{10^{4}}=96\,090$
    \item $8{,}082\times\mathbf{10^{6}}=8\,082\,000$
    \item $0{,}039=3{,}9\times\mathbf{10^{-2}}$
    \item $3{,}098\times\mathbf{10^{2}}=309{,}8$
    \end{enumerate}
  \end{multicols}


  \exercice*
  Calculer les expressions suivantes et donner l'écriture scientifique du
  résultat.
  \begin{multicols}{2}
    \noindent%
    \[ \thenocalcul = \cfrac{\nombre{5,6} \times 10^{7} \times \nombre{8,1}
        \times 10^{-2}}{\nombre{7200} \times \big( 10^{-2} \big) ^3} \]
    \[ \thenocalcul = \cfrac{\nombre{5,6} \times \nombre{8,1}}{\nombre{7200}}
      \times \cfrac{10^{7+(-2)}}{10^{-2 \times 3}} \]
    \[ \thenocalcul = \nombre{0,0063} \times 10^{5-(-6)} \]
    \[ \thenocalcul = \nombre{6,3}  \times 10^{-3} \times 10^{11} \]
    \[ \boxed{\thenocalcul = \nombre{6,3}  \times 10^{8}} \]
    \columnbreak\stepcounter{nocalcul}%
    \[ \thenocalcul = \cfrac{\nombre{480} \times 10^{9} \times \nombre{0,3}
        \times 10^{-6}}{\nombre{192} \times \big( 10^{6} \big) ^3} \]
    \[ \thenocalcul = \cfrac{\nombre{480} \times \nombre{0,3}}{\nombre{192}}
      \times \cfrac{10^{9+(-6)}}{10^{6 \times 3}} \]
    \[ \thenocalcul = \nombre{0,75} \times 10^{3-18} \]
    \[ \thenocalcul = \nombre{7,5}  \times 10^{-1} \times 10^{-15} \]
    \[ \boxed{\thenocalcul = \nombre{7,5}  \times 10^{-16}} \]
  \end{multicols}


  \exercice*
  Compléter par un nombre de la forme $a^n$ avec $a$ et $n$ entiers :
  \begin{multicols}{4}
    \noindent%
    \begin{enumerate}
    \item $(8^{10})^{3}=
      8^{30}$
    \item $\dfrac{5^{8}}{5^{4}}=
      5^{4}$
    \item $4^{7}\times2^{7}=
      8^{7}$
    \item $5^{2}\times5^{8}=
      5^{10}$
    \item $(8^{6})^{5}=
      8^{30}$
    \item $11^{2}\times9^{2}=
      99^{2}$
    \item $8^{7}\times8^{11}=
      8^{18}$
    \item $\dfrac{7^{9}}{7^{3}}=
      7^{6}$
    \end{enumerate}
  \end{multicols}


  \exercice*
  Écrire sous la forme d'une puissance de 10 puis donner l'écriture
  décimale de ces nombres :
  \begin{multicols}{2}
    \noindent%
    \begin{enumerate}
    \item $\dfrac{10^{4}}{10^{-3}}=
      10^{4-\left( -3\right)}=
      10^{7}=10\,000\,000$
    \item $(10^{2})^{4}=
      10^{2 \times 4}=
      10^{8}=100\,000\,000$
    \item $10^{5}\times 10^{5}=
      10^{5+5}=
      10^{10}=10\,000\,000\,000$
    \item $10^{2}\times 10^{5}=
      10^{2+5}=
      10^{7}=10\,000\,000$
    \item $\dfrac{10^{5}}{10^{-1}}=
      10^{5-\left( -1\right)}=
      10^{6}=1\,000\,000$
    \item $(10^{-2})^{0}=
      10^{-2 \times }=
      10^{0}=1$
    \end{enumerate}
  \end{multicols}

  \label{LastCorPage}
\end{document}