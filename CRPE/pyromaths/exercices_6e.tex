%%!TEX TS-program = latex
\documentclass[a4paper,11pt]{article}
\usepackage[utf8]{inputenc} % UTF-8
\usepackage[T1]{fontenc}
\usepackage{lmodern} % Prévient un bug d'affichage evince lié à [T1]{fontenc}
\usepackage[frenchb]{babel} % francisation
\usepackage{adjustbox}
\usepackage[fleqn]{amsmath} % aligne le mode maths à gauche
\usepackage{amssymb} % the amsfont symbols
\usepackage[table, usenames, svgnames]{xcolor} % Couleurs
\usepackage{multicol} % Multi-colonnes
\usepackage{fancyhdr} % Mise en page, en-tête et pied de page
\usepackage{calc} % Opérations
\usepackage{marvosym} % Martin Vogels Symbole (\EUR)
\usepackage{cancel} % draw diagonal lines
\usepackage{units} % typesetting units and nice fractions
\usepackage[autolanguage]{numprint} % écrituredes virgules
\usepackage{tabularx} % creates a paragraph-like column whose width
% automatically expands
\usepackage{wrapfig} % allows figures or tables to have text wrapped around
\usepackage{pst-eucl, pst-plot} % figures géométriques
\usepackage{enumitem}
\usepackage{wasysym} % Symbole Euro
\usepackage{mathtools} % Encadrement dans align*
\usepackage[inline]{asymptote}
\usepackage{tkz-tab}
\usetikzlibrary{external} % set up externalization
\tikzexternalize[shell escape=-enable-write18] % activate externalisation
\tikzset{%
  external/system call={%
    latex \tikzexternalcheckshellescape -halt-on-error
    -interaction=batchmode -jobname "\image" "\texsource" &&
    dvips -o "\image".ps "\image".dvi &&
    ps2eps -f "\image.ps"
    }
  }

%\usepackage{textcomp}

\usepackage[a4paper, dvips, left=1.5cm, right=1.5cm, top=2cm,%
bottom=2cm, marginpar=5mm, marginparsep=5pt]{geometry}
\newcounter{exo}
\frenchbsetup{StandardItemLabels} % remet \textbullet pour les listes
\setlength{\headheight}{18pt}
\setlength{\fboxsep}{1em}
\setlength\parindent{0em}
\setlength\mathindent{0em}
\setlength{\columnsep}{30pt}
\usepackage[bookmarks=true, bookmarksnumbered=true, ps2pdf, pagebackref=true,%
colorlinks=true,linkcolor=blue,plainpages=true,unicode]{hyperref}
\hypersetup{pdfauthor={Jérôme Ortais},pdfsubject={Exercices de
    mathématiques},pdftitle={Exercices créés par Pyromaths, un logiciel libre
    en Python sous licence GPL}}

\def\pshlabel#1{\psframebox*[fillcolor=White,framearc=.2]{\footnotesize $#1$}}
\def\psvlabel#1{\psframebox*[fillcolor=White,framearc=.2]{\footnotesize $#1$}}

\makeatletter
\newcommand\styleexo[1][]{
  \renewcommand{\theenumi}{\arabic{enumi}}
  \renewcommand{\labelenumi}{$\blacktriangleright$\textbf{\theenumi.}}
  \renewcommand{\theenumii}{\alph{enumii}}
  \renewcommand{\labelenumii}{\textbf{\theenumii)}}
  {\fontfamily{pag}\fontseries{b}\selectfont \underline{#1 \theexo}}
  \par\@afterheading\vspace{0.5\baselineskip minus 0.2\baselineskip}}
\newcommand*\exercice{%
  \psset{unit=1cm, dash=4pt 4pt, PointName=default,linecolor=Maroon,
    dotstyle=x, linestyle=solid, hatchcolor=Peru, gridcolor=Olive,
    subgridcolor=Olive, fillcolor=Peru}
  %\ifthenelse{\equal{\theexo}{0}}{}{\filbreak}
  \refstepcounter{exo}%
  \stepcounter{nocalcul}%
  \par\addvspace{1.5\baselineskip minus 1\baselineskip}%
  \@ifstar%
  {\penalty-130\styleexo[Corrigé de l'exercice]}%
  {\penalty-130\styleexo[Exercice]}%
  }
\makeatother
\newlength{\ltxt}
\newcounter{fig}
\newcommand{\figureadroite}[2]{
  \setlength{\ltxt}{\linewidth}
  \setbox\thefig=\hbox{#1}
  \addtolength{\ltxt}{-\wd\thefig}
  \addtolength{\ltxt}{-10pt}
  \begin{minipage}{\ltxt}
    #2
  \end{minipage}
  \hfill
  \begin{minipage}{\wd\thefig}
    #1
  \end{minipage}
  \refstepcounter{fig}
  }
\count1=\year \count2=\year
\ifnum\month<8\advance\count1by-1\else\advance\count2by1\fi
\pagestyle{fancy}
\cfoot{\textsl{\footnotesize{Année \number\count1/\number\count2}}}
\rfoot{\textsl{\tiny{http://www.pyromaths.org}}}
\lhead{\textsl{\footnotesize{Page \thepage/ \pageref{LastPage}}}}
\chead{\Large{\textsc{Révisions}}}
\rhead{\textsl{\footnotesize{Classe de 6\ieme}}}

\begin{document}
  \currentpdfbookmark{Les énoncés des exercices}{Énoncés}
  \newcounter{nocalcul}[exo]
  \renewcommand{\thenocalcul}{\Alph{nocalcul}}
  \raggedcolumns
  \setlength{\columnseprule}{0.5pt}

  \exercice
  Calculer l'aire de chacune des figures suivantes dans l'unité d'aire donnée
  :\par
  \begin{pspicture}(0,0)(18,9)
    \psgrid[subgriddiv=2, gridlabels=0pt]
    \psframe[fillstyle=vlines, hatchsep=1pt](0,0)(.5,.5)
    \rput[l](0.6,0.25){\psframebox[linecolor=white, fillcolor=white,
      fillstyle=solid]{unité d'aire}}
    \psset{unit=5mm}

    \rput(0,12){
      \pspolygon[fillstyle=hlines](0, 0)(13, 0)(21, 6)(8, 6)
      \rput(10.50,3.00){\psframebox[linecolor=white, fillcolor=white,
        fillstyle=solid]{figure 1}}
      }
    \rput(22,11){
      \psframe[fillstyle=hlines](0, 0)(7, 7)
      \rput(3.50,3.50){\psframebox[linecolor=white, fillcolor=white,
        fillstyle=solid]{figure 2}}
      }
    \rput(30,11){
      \pspolygon[fillstyle=hlines](0, 7)(0, 0)(6, 0)
      \rput(2.00,2.33){\psframebox[linecolor=white, fillcolor=white,
        fillstyle=solid]{figure 3}}
      }
    \rput(0,2){
      \pspolygon[fillstyle=hlines](0, 9)(17, 0)(21, 2)
      \rput(12.67,3.67){\psframebox[linecolor=white, fillcolor=white,
        fillstyle=solid]{figure 4}}
      }
    \rput(22,2){
      \psframe[fillstyle=hlines](0, 0)(7, 8)
      \rput(3.50,4.00){\psframebox[linecolor=white, fillcolor=white,
        fillstyle=solid]{figure 5}}
      }
    \rput(30,2){
      \pspolygon[fillstyle=hlines](6, 0)(6, 8)(0, 2)
      \rput(4.00,3.33){\psframebox[linecolor=white, fillcolor=white,
        fillstyle=solid]{figure 6}}
      }
  \end{pspicture}

  \exercice
  Voici deux exemples de zigzags :\par
  \psset{unit=3.5mm,PointSymbol=none}
  \begin{pspicture}(-.4,-1)(16.4, 7.5)
    %\psgrid
    \pstGeonode[PosAngle=-96.00,PointSymbol=x](0.20, 0.20){A}
    \pstGeonode[PosAngle=-265.00](0.83, 6.17){B} \pstSegmentMark{A}{B}
    \pstGeonode[PosAngle=-87.50](2.48, 0.40){C} \pstSegmentMark{B}{C}
    \pstGeonode[PosAngle=-241.00](3.63, 6.29){D} \pstSegmentMark{C}{D}
    \pstGeonode[PosAngle=-90.00](9.23, 4.14){E} \pstSegmentMark{D}{E}
    \pstGeonode[PosAngle=-299.50](14.83, 6.29){F} \pstSegmentMark{E}{F}
    \pstGeonode[PosAngle=-80.00,PointSymbol=x](15.87, 0.38){G}
    \pstSegmentMark{F}{G}
    \pstMarkAngle[MarkAngleRadius=1.5]{A}{B}{C}{}
    \pstMarkAngle[MarkAngleRadius=1.5]{D}{C}{B}{}
    \pstMarkAngle[MarkAngleRadius=1.5]{C}{D}{E}{}
    \pstMarkAngle[MarkAngleRadius=1.5]{F}{E}{D}{}
    \pstMarkAngle[MarkAngleRadius=1.5]{E}{F}{G}{}
  \end{pspicture}
  \hspace{2cm}
  \begin{pspicture}(-.4,-1)(16.4, 7.5)
    \pstGeonode[PosAngle=-91.00,PointSymbol=x](0.20, 0.20){A}
    \pstGeonode[PosAngle=-263.50](0.30, 6.20){B} \pstSegmentMark{A}{B}
    \pstGeonode[PosAngle=-101.00](1.76, 0.38){C} \pstSegmentMark{B}{C}
    \pstGeonode[PosAngle=-259.00](5.28, 5.23){D} \pstSegmentMark{C}{D}
    \pstGeonode[PosAngle=-82.50](10.37, 2.05){E} \pstSegmentMark{D}{E}
    \pstGeonode[PosAngle=-286.00](14.46, 6.44){F} \pstSegmentMark{E}{F}
    \pstGeonode[PosAngle=-79.00,PointSymbol=x](15.61, 0.55){G}
    \pstSegmentMark{F}{G}
    \pstMarkAngle[MarkAngleRadius=1.5]{A}{B}{C}{}
    \pstMarkAngle[MarkAngleRadius=1.5]{D}{C}{B}{}
    \pstMarkAngle[MarkAngleRadius=1.5]{C}{D}{E}{}
    \pstMarkAngle[MarkAngleRadius=1.5]{F}{E}{D}{}
    \pstMarkAngle[MarkAngleRadius=1.5]{E}{F}{G}{}
  \end{pspicture}\par
  \psset{unit=1cm}
  Construire sur la figure ci-dessous les points $C$, $D$, $E$, $F$ et $G$ pour
  obtenir un zigzag tel que :\par
  $\widehat{ABC}=21\degres \qquad \widehat{BCD}=51\degres \qquad
  \widehat{CDE}=80\degres \qquad \widehat{DEF}=87\degres \qquad
  \widehat{EFG}=158\degres \qquad $\par
  Quand le travail est fait avec une bonne précision, les
  droites $(AG)$ et $(BF)$ se coupent au c\oe ur de la cible.\par
  \begin{center}
    \fbox{
      \begin{pspicture}(-.4,-.4)(16.4, 5.5)

        \pstGeonode[PosAngle=-100.00, PointSymbol=x](0.20, 0.20){A}
        \pstGeonode[PosAngle=-248.50, PointSymbol=x](0.89, 4.14){B}
        \pstSegmentMark{A}{B}
        \pscircle[linecolor=Gray](9.15, 3.02){0.1}
        \pscircle[linecolor=Gray](9.15, 3.02){0.2}
        \pscircle[linecolor=Gray](9.15, 3.02){0.3}
        \pscircle[linecolor=Gray](9.15, 3.02){0.4}

      \end{pspicture}
      }
  \end{center}

  \exercice
  Nommer, mesurer et donner la nature de chacun des angles suivants :\par
  \begin{pspicture}(18,8)
    \psframe(0,0)(18,8)
    \psset{PointSymbol=none,MarkAngleRadius=0.6}
    \pstGeonode[PointName={T,J,B},PosAngle={289.5,329,250}](7.60,2.00){a00}(10.69,7.14){a10}(1.96,4.05){a20}
    \pstMarkAngle{a10}{a00}{a20}{}
    \pstLineAB[nodesepB=-.5]{a00}{a10}\pstLineAB[arrows=-|,linestyle=none]{a00}{a10}
    \pstLineAB[nodesepB=-.5]{a00}{a20}\pstLineAB[arrows=-|,linestyle=none]{a00}{a20}
    \pstGeonode[PointName={U,E,K},PosAngle={274.5,295,254}](7.50,0.50){a01}(12.94,3.04){a11}(1.73,2.15){a21}
    \pstMarkAngle{a11}{a01}{a21}{}
    \pstLineAB[nodesepB=-.5]{a01}{a11}\pstLineAB[arrows=-|,linestyle=none]{a01}{a11}
    \pstLineAB[nodesepB=-.5]{a01}{a21}\pstLineAB[arrows=-|,linestyle=none]{a01}{a21}
    \pstGeonode[PointName={H,Q,X},PosAngle={172.0,249,95}](1.60,6.60){a02}(7.20,4.45){a12}(7.58,7.12){a22}
    \pstMarkAngle{a12}{a02}{a22}{}
    \pstLineAB[nodesepB=-.5]{a02}{a12}\pstLineAB[arrows=-|,linestyle=none]{a02}{a12}
    \pstLineAB[nodesepB=-.5]{a02}{a22}\pstLineAB[arrows=-|,linestyle=none]{a02}{a22}
    \pstGeonode[PointName={G,S,M},PosAngle={36.0,98,334}](17.40,6.10){a03}(11.46,5.26){a13}(14.77,0.71){a23}
    \pstMarkAngle{a13}{a03}{a23}{}
    \pstLineAB[nodesepB=-.5]{a03}{a13}\pstLineAB[arrows=-|,linestyle=none]{a03}{a13}
    \pstLineAB[nodesepB=-.5]{a03}{a23}\pstLineAB[arrows=-|,linestyle=none]{a03}{a23}
  \end{pspicture}\par
  \begin{tabularx}{\textwidth}{|*{4}{X|}}
    \hline angle 1 : & angle 2 : & angle 3 : & angle 4 : \\
    \hline &&& \\ &&& \\ &&& \\ \hline
  \end{tabularx}

  \exercice
  \begin{enumerate}
  \item Arrondir 894\,076 à la dizaine.
  \item Arrondir 6\,766\,140 à la centaine par excès.
  \item Arrondir 8\,125,72 au dixième.
  \item Arrondir 88\,223,3 à l'unité.
  \end{enumerate}

  \exercice
  \begin{enumerate}
  \item Classer les nombres suivants dans l'ordre décroissant.\par
    5,1 \kern1cm ; \kern1cm 1,293 \kern1cm ; \kern1cm 1,21 \kern1cm ; \kern1cm
    1,5
  \item Classer les nombres suivants dans l'ordre croissant.\par
    1,6 \kern1cm ; \kern1cm 6,322 \kern1cm ; \kern1cm 6,43 \kern1cm ; \kern1cm
    6,5
  \end{enumerate}

  \exercice
  Compléter avec un nombre décimal :
  \begin{multicols}{2}\noindent
    \begin{enumerate}
    \item $1\times \cfrac{1}{1\,000} + 7\times 1 + 5\times \cfrac{1}{10} =
      \dotfill$
    \item $7\times \cfrac{1}{100} + 6\times \cfrac{1}{10} + 3\times 1 =
      \dotfill$
    \item $2\times \cfrac{1}{1\,000} + 3\times 100 + 5\times 10 = \dotfill$
    \item $6\times 10 + 8\times 1 + 9\times 100 = \dotfill$
    \item $9\times \cfrac{1}{100} + 9\times 1 + 8\times 1\,000 = \dotfill$
    \item $4\times \cfrac{1}{1\,000} + 9\times 1\,000 + 5\times \cfrac{1}{100}
      = \dotfill$
    \end{enumerate}
  \end{multicols}

  \exercice
  \begin{enumerate}
  \item Écrire en chiffres les nombres suivants.
    \begin{enumerate}
    \item neuf-cent-quarante-quatre-millions-neuf-cent-soixante-dix-mille :
      \dotfill
    \item quatre-vingts-mille-quatre-cent-deux unités et six dixièmes : \dotfill
    \item neuf unités et sept dixièmes : \dotfill
    \item deux-cent-cinquante-huit millièmes : \dotfill
    \item six-cent-quarante-neuf-millions-huit-cent-quatre-mille-vingt-six :
      \dotfill
    \item sept-cent-vingt-sept-millions-vingt-deux-mille-six-cent-trente-quatre
      : \dotfill
    \item trente-millions-deux-cent-cinquante-huit : \dotfill
    \item deux-mille-deux-cent-soixante-trois unités et quatre dixièmes :
      \dotfill
    \end{enumerate}
  \item Écrire en lettres les nombres suivants (sans utiliser le mot
    ``virgule").
    \begin{enumerate}
    \item 320,63 : \dotfill
    \item 826\,000\,202 : \dotfill
    \item 0,202 : \dotfill
    \item 633\,500\,000 : \dotfill
    \item 994\,000\,358 : \dotfill
    \item 445\,032\,063 : \dotfill
    \item 3,58 : \dotfill
    \item 0,5 : \dotfill
    \end{enumerate}
  \end{enumerate}

  \exercice
  Compléter :
  \begin{multicols}{3}\noindent
    \begin{enumerate}
    \item $\cfrac{231}{1\,000}=\ldots$
    \item $\cfrac{5\,626}{\ldots}=5{,}626$
    \item $\cfrac{84\,020}{10\,000}=\ldots$
    \item $\cfrac{\ldots}{10\,000}=1{,}22$
    \item $\cfrac{1\,780}{10\,000}=\ldots$
    \item $\cfrac{\ldots}{100}=20{,}22$
    \end{enumerate}
  \end{multicols}

  \exercice
  Placer une virgule (en ajoutant éventuellement des zéros) dans            le
  nombre 821349 de telle sorte que :
  \begin{enumerate}
  \item le chiffre 3 soit le chiffre des milliers :
    \dotfill
  \item le chiffre 9 soit le chiffre des dixièmes :
    \dotfill
  \item le chiffre 3 soit le chiffre des unités :
    \dotfill
  \item le chiffre 4 soit le chiffre des centièmes :
    \dotfill
  \item le chiffre 3 soit le chiffre des centaines :
    \dotfill
  \item le chiffre 9 soit le chiffre des millièmes :
    \dotfill
  \end{enumerate}

  \exercice
  Effectuer les conversions suivantes :
  \begin{multicols}{3}\noindent
    \begin{enumerate}
    \item $\unit[5,32]{dm^2}=\unit[\dotfill]{dam^2}$
    \item $\unit[3,08]{cm^2}=\unit[\dotfill]{m^2}$
    \item $\unit[6,73]{km^2}=\unit[\dotfill]{hm^2}$
    \item $\unit[91,8]{dm^2}=\unit[\dotfill]{mm^2}$
    \item $\unit[76,1]{dam^2}=\unit[\dotfill]{m^2}$
    \item $\unit[12,8]{hm^2}=\unit[\dotfill]{dam^2}$
    \end{enumerate}
  \end{multicols}

  \exercice
  Effectuer les conversions suivantes :
  \begin{multicols}{3}\noindent
    \begin{enumerate}
    \item $\unit[2,08]{dam^3}=\unit[\dotfill]{m^3}$
    \item $\unit[9,66]{dam^3}=\unit[\dotfill]{m^3}$
    \item $\unit[4,07]{dm^3}=\unit[\dotfill]{m^3}$
    \item $\unit[9,06]{dm^3}=\unit[\dotfill]{cm^3}$
    \item $\unit[8,56]{m^3}=\unit[\dotfill]{hm^3}$
    \item $\unit[93,4]{dm^3}=\unit[\dotfill]{m^3}$
    \end{enumerate}
  \end{multicols}

  \exercice
  Effectuer les conversions suivantes :
  \begin{multicols}{3}\noindent
    \begin{enumerate}
    \item 1,33~hL=\dotfill~daL
    \item 7,87~dg=\dotfill~cg
    \item 9,61~dam=\dotfill~mm
    \item 9,97~hL=\dotfill~daL
    \item 97,7~L=\dotfill~mL
    \item 22,9~hL=\dotfill~daL
    \end{enumerate}
  \end{multicols}

  \exercice
  Compléter les pointillés et les figures :\par
  \renewcommand{\tabularxcolumn}[1]{m{#1}}
  \begin{tabularx}{\linewidth}{|X|>{\centering}m{5cm}|}
    \hline
    \textbf{phrase} & \textbf{Figure} \tabularnewline \hline
    $[ QA ]$ est \dotfill &
    \begin{pspicture*}(-0.5,0.2)(4.5,2.2)
      \psset{PointSymbol=x}
      \pstGeonode[PosAngle=90](0.5,1.5){Z}(2,1.0){A}(3.5,0.8){Q}
    \end{pspicture*}\tabularnewline
    \hline
    $\ldots \ldots\ldots \ldots$ est \dotfill &
    \begin{pspicture*}(-0.5,0.2)(4.5,2.2)
      \psset{PointSymbol=x}
      \pstGeonode[PosAngle=90](0.5,1.3){P}(2,0.6){L}(3.5,1.0){M}
      \pstLineAB[nodesepA=0, nodesepB=0]{L}{P}
    \end{pspicture*}\tabularnewline
    \hline
    $\ldots MY \ldots$ est une droite &
    \begin{pspicture*}(-0.5,0.2)(4.5,2.2)
      \psset{PointSymbol=x}
      \pstGeonode[PosAngle=90](0.5,0.5){M}(2,0.9){Y}(3.5,1.3){G}
    \end{pspicture*}\tabularnewline
    \hline
    $\ldots \ldots\ldots \ldots$ est \dotfill &
    \begin{pspicture*}(-0.5,0.2)(4.5,2.2)
      \psset{PointSymbol=x}
      \pstGeonode[PosAngle=90](0.5,1.4){Z}(2,0.6){C}(3.5,0.5){P}
      \pstLineAB[nodesepA=0, nodesepB=-6]{P}{C}
    \end{pspicture*}\tabularnewline
    \hline
    $( HP )$ est \dotfill &
    \begin{pspicture*}(-0.5,0.2)(4.5,2.2)
      \psset{PointSymbol=x}
      \pstGeonode[PosAngle=90](0.5,1.4){H}(2,1.4){N}(3.5,0.7){P}
    \end{pspicture*}\tabularnewline
    \hline
  \end{tabularx}

  \exercice
  Réaliser les figures suivantes :\par
  \begin{multicols}{2}
    \begin{pspicture*}(-4,-4)(4,4)
      \psset{PointSymbol=x}
      \pstGeonode[PointName={I,U}](3;75){a}(3;255){b}
      \pstGeonode[PointName={V,W}](2.1; 175){c}(2.0; 326){d}
    \end{pspicture*}\par
    \begin{enumerate}
    \item Tracer la droite parallèle à la droite $(IV)$ passant par $U$
    \item Tracer la droite perpendiculaire à la droite $(VU)$ passant par $I$
    \end{enumerate}
    \columnbreak
    \begin{pspicture*}(-4,-4)(4,4)
      \psset{PointSymbol=x}
      \pstGeonode[PointName={F,L}](3;80){a}(3;260){b}
      \pstGeonode[PointName={X,Z}](2.4; 157){c}(2.7; 315){d}
    \end{pspicture*}\par
    \begin{enumerate}
    \item Tracer la droite perpendiculaire à la droite $(ZX)$ passant par $F$
    \item Tracer la droite parallèle à la droite $(XF)$ passant par $Z$
    \end{enumerate}
  \end{multicols}

  \exercice
  Compléter le tableau suivant :\par Les droites en gras sont parallèles.\par
  \renewcommand{\tabularxcolumn}[1]{m{#1}}
  \begin{tabularx}{\textwidth}[t]{|m{3cm}|m{4cm}|X|m{3cm}|}
    \hline
    \multicolumn{1}{|c|}{\bf Données} & \multicolumn{1}{|c|}{\bf Figure codée}
    & \multicolumn{1}{|c|}{\bf Propriété} & \multicolumn{1}{|c|}{\bf
      Conclusion}\\
    \hline
    $(d_2)//(d_1)$\par et\par $(d_2)//(d_3)$ &
    \begin{pspicture*}[shift=-1.5](-2.1,-1.6)(2.1,1.6)
    \end{pspicture*}
    & & \\
    \hline
    $(d_1)\perp(d_2)$\par et\par $(d_3)\perp(d_2)$ &
    \begin{pspicture*}[shift=-1.5](-2.1,-1.6)(2.1,1.6)
    \end{pspicture*}
    & & \\
    \hline
    $(IJ)//(KM)$\par et\par $(IJ)\perp(IK)$ &
    \begin{pspicture*}[shift=-1.5](-2.1,-1.6)(2.1,1.6)
    \end{pspicture*}
    & & \\
    \hline
  \end{tabularx}


  \exercice
  Les figures 1 et 2 représentent le même cube CEGLNQST.\\
  \psset{xunit=1.0cm,yunit=1.0cm,dotstyle=*,dotsize=3pt
    0,linewidth=0.8pt,arrowsize=3pt 2,arrowinset=0.25}
  \begin{pspicture*}(-2,-0.38)(15,4.5)
    \psframe[fillstyle=solid,fillcolor=darkgray,framearc=0.2](-1,3.5)(-0.5,4)
    \rput[bl](-0.85,3.65){\white{\textbf{$1$}}}
    \psline[linestyle=dashed,linecolor=Maroon,dash=4pt 4pt](3.5,1)(1.5,1)
    \psline[linestyle=dashed,linecolor=Maroon,dash=4pt 4pt](1.5,1)(1.5,3)
    \psline[linestyle=dashed,linecolor=Maroon,dash=4pt 4pt](1.5,1)(0.5,0)
    \psframe[linecolor=Maroon](0.5,0)(2.5,2)
    \psline[linecolor=Maroon](1.5,3)(3.5,3)
    \psline[linecolor=Maroon](3.5,3)(3.5,1)
    \psline[linecolor=Maroon](0.5,2)(1.5,3)
    \psline[linecolor=Maroon](2.5,2)(3.5,3)
    \psline[linecolor=Maroon](2.5,0)(3.5,1)
    \rput[bl](0.22,2.08){C}
    \rput[bl](2.28,2.16){E}
    \rput[bl](2.6,-0.22){G}
    \rput[bl](0.15,-0.22){L}
    \rput[bl](1.4,3.08){N}
    \rput[bl](3.58,3.04){Q}
    \rput[bl](3.58,1.04){S}
    \rput[bl](1.58,1.04){T}
    \psline[linecolor=Maroon](5.2,-04)(5.2,4)
    \psframe[fillstyle=solid,fillcolor=darkgray,framearc=0.2](5.7,3.5)(6.2,4)
    \rput[bl](5.85,3.65){\white{\textbf{$2$}}}
    \psline[linecolor=Maroon](8.48,2.71)(7.18,2.42)
    \psline[linecolor=Maroon](7.18,2.42)(7.18,0.46)
    \psline[linestyle=dashed, linecolor=Maroon, dash=4pt
    4pt](7.18,0.46)(8.48,0.75)
    \psline[linestyle=dashed, linecolor=Maroon, dash=4pt
    4pt](8.48,0.75)(8.48,2.71)
    \psline[linecolor=Maroon](10,2.46)(8.7,2.17)
    \psline[linecolor=Maroon](8.7,2.17)(8.7,0.21)
    \psline[linecolor=Maroon](8.7,0.21)(10,0.5)
    \psline[linecolor=Maroon](10,0.5)(10,2.46)
    \psline[linecolor=Maroon](10,2.46)(8.48,2.71)
    \psline[linecolor=Maroon](7.18,2.42)(8.7,2.17)
    \psline[linecolor=Maroon](8.7,0.21)(7.18,0.46)
    \psline[linestyle=dashed,dash=4pt 4pt, linecolor=Maroon](10,0.5)(8.48,0.75)
    \rput[bl](8.68,2.27){N}
    \rput[bl](10.06,2.5){T}
    \rput[bl](8.71,-0.1){Q}
  \end{pspicture*}
  \begin{enumerate}
  \item Compléter les sommets manquants de la figure 2.
  \item Donner toutes les arêtes perpendiculaires à [TL].
  \item Donner toutes les arêtes parallèles à [NC].
  \end{enumerate}

  \exercice
  \begin{multicols}{2}
    \begin{enumerate}
    \item Colorer $\frac{6}{4}$ de ce rectangle.\par
      \psset{unit=4mm}
      \begin{pspicture}(16,4)
        \psgrid[gridcolor=Olive,subgriddiv=0,gridlabels=0pt]
        \psframe[linewidth=1.5\pslinewidth,linecolor=Maroon](0,0)(7,4)
      \end{pspicture}
    \item Colorer $\frac{4}{7}$ de ce rectangle.\par
      \psset{unit=4mm}
      \begin{pspicture}(16,5)
        \psgrid[gridcolor=Olive,subgriddiv=0,gridlabels=0pt]
        \psframe[linewidth=1.5\pslinewidth,linecolor=Maroon](0,0)(7,5)
      \end{pspicture}
      \columnbreak
    \item Colorer $\frac{9}{9}$ de ce rectangle.\par
      \psset{unit=4mm}
      \begin{pspicture}(16,4)
        \psgrid[gridcolor=Olive,subgriddiv=0,gridlabels=0pt]
        \psframe[linewidth=1.5\pslinewidth,linecolor=Maroon](0,0)(6,4)
      \end{pspicture}
    \item Colorer $\frac{5}{3}$ de ce rectangle.\par
      \psset{unit=4mm}
      \begin{pspicture}(16,6)
        \psgrid[gridcolor=Olive,subgriddiv=0,gridlabels=0pt]
        \psframe[linewidth=1.5\pslinewidth,linecolor=Maroon](0,0)(7,6)
      \end{pspicture}
    \end{enumerate}
  \end{multicols}

  \exercice
  \begin{enumerate}
  \item Compléter :
    \begin{multicols}{2}
      \begin{enumerate}
      \item 1 unité = \ldots neuvièmes
      \item 1 unité = \ldots~tiers
      \item 10 unités = \ldots~neuvièmes
      \item 10 unités = \ldots~tiers
      \end{enumerate}
    \end{multicols}
  \item Sur la demi-droite ci-dessous, placer les points d'abscisse donnée :
    \begin{multicols}{5}
      \begin{enumerate}
        \renewcommand{\theenumii}{\Alph{enumii}}
        \renewcommand{\labelenumii}{$\theenumii$}
      \item $\left(\cfrac{132}{9}\right)$
      \item $\left(\cfrac{101}{9}\right)$
      \item $\left(\cfrac{36}{3}\right)$
      \item $\left(\cfrac{39}{3}\right)$
      \item $\left(\cfrac{112}{7}\right)$
      \end{enumerate}
    \end{multicols}
  \item Compléter les abscisses des points suivants :
    \begin{multicols}{4}
      \begin{enumerate}
      \item $F~\left(\cfrac{\ldots}{9}\right)$
      \item $F~\left(\cfrac{\ldots}{3}\right)$
      \item $G~\left(\cfrac{\ldots}{9}\right)$
      \item $G~\left(\cfrac{\ldots}{3}\right)$
      \end{enumerate}
    \end{multicols}
  \end{enumerate}
  \begin{pspicture}(0,-.5)(18,.5)
    \psline[arrowscale=2,linecolor=Maroon]{->}(0,0)(18,0)
    \rput(2mm,0){%
      \multips(0,0)(3 mm,0){58}{\psline[linecolor=Maroon](0,-.1)(0,.1)}
      \multips(0,0)(27 mm,0){7}{\psline[linecolor=Maroon](0,-.2)(0,.2)}
      \rput[t](0 mm,-3mm){\centering 10}
      \rput[t](27 mm,-3mm){\centering 11}
      \rput[t](54 mm,-3mm){\centering 12}
      \rput[t](81 mm,-3mm){\centering 13}
      \rput[t](108 mm,-3mm){\centering 14}
      \rput[t](135 mm,-3mm){\centering 15}
      \rput[t](162 mm,-3mm){\centering 16}
      \rput[t](18.1 mm,4mm){\centering $F$}
      \rput[t](27.1 mm,4mm){\centering $G$}
      }
  \end{pspicture}

  \exercice
  Effectuer sans calculatrice :
  \begin{multicols}{4}\noindent
    \begin{enumerate}
    \item $19 - 9 = \ldots\ldots$
    \item $8 + \ldots\ldots = 13$
    \item $\ldots\ldots \times 7 = 42$
    \item $\ldots\ldots - 4 = 4$
    \item $\ldots\ldots \times 6 = 30$
    \item $\ldots\ldots \div 5 = 8$
    \item $1 + \ldots\ldots = 9$
    \item $25 \div 5 = \ldots\ldots$
    \item $5 \div \ldots\ldots = 5$
    \item $24 \div 4 = \ldots\ldots$
    \item $11 - 8 = \ldots\ldots$
    \item $11 - \ldots\ldots = 6$
    \item $\ldots\ldots \div 4 = 4$
    \item $1 + \ldots\ldots = 4$
    \item $11 - 9 = \ldots\ldots$
    \item $\ldots\ldots + 4 = 13$
    \item $\ldots\ldots + 7 = 12$
    \item $4 \times \ldots\ldots = 40$
    \item $\ldots\ldots \times 9 = 9$
    \item $6 \times \ldots\ldots = 60$
    \end{enumerate}
  \end{multicols}

  \exercice
  Poser et effectuer les opérations suivantes.
  \begin{multicols}{2}\noindent
    \begin{enumerate}
    \item La différence des termes 30\,371 et 20,97.\par
    \item Le produit des facteurs 96,7 et 13,4.\par
    \item La somme des termes 1\,126 et 501,6.\par
    \end{enumerate}
  \end{multicols}

  \exercice
  Compléter sans calculatrice :
  \begin{multicols}{2}\noindent
    \begin{enumerate}
    \item $10 \quad\times\quad 11{,}1 \quad = \quad \dotfill$
    \item $0{,}454 \quad\div\quad 1\,000 \quad = \quad \dotfill$
    \item $100 \quad\times\quad \dotfill \quad = \quad 868$
    \item $\dotfill \quad\times\quad 47{,}7 \quad = \quad 4{,}77$
    \item $0{,}000\,1 \quad\times\quad \dotfill \quad = \quad 0{,}000\,075\,7$
    \item $\dotfill \quad\times\quad 0{,}329 \quad = \quad 0{,}003\,29$
    \item $18{,}5 \quad\div\quad \dotfill \quad = \quad 0{,}001\,85$
    \item $0{,}148 \quad\div\quad 100 \quad = \quad \dotfill$
    \item $18{,}2 \quad\div\quad 10 \quad = \quad \dotfill$
    \item $0{,}001 \quad\times\quad \dotfill \quad = \quad 0{,}009\,48$
    \item $1\,000 \quad\times\quad 6{,}59 \quad = \quad \dotfill$
    \item $10\,000 \quad\times\quad 5{,}46 \quad = \quad \dotfill$
    \end{enumerate}
  \end{multicols}

  \exercice
  Cocher les bonnes réponses :\par
  \begin{tabular}{c@{ est divisible : \kern1cm}r@{ par 2\kern1cm}r@{ par
        3\kern1cm}r@{ par 5\kern1cm}r@{ par 9\kern1cm}r@{ par 10}}
    490 & $\square$ & $\square$ & $\square$ & $\square$ & $\square$ \\
    666 & $\square$ & $\square$ & $\square$ & $\square$ & $\square$ \\
    168 & $\square$ & $\square$ & $\square$ & $\square$ & $\square$ \\
    370 & $\square$ & $\square$ & $\square$ & $\square$ & $\square$ \\
    36 & $\square$ & $\square$ & $\square$ & $\square$ & $\square$ \\
  \end{tabular}

  \exercice
  Construire la symétrique de chacune des figures par rapport à la droite en
  utilisant le quadrillage :\par
  \psset{unit=.9cm}
  \begin{pspicture*}(-3,-3)(3,3)
    \psgrid[subgriddiv=2,gridlabels=0pt]
    \pstGeonode[PointSymbol=x,PointName=none](0.5,1.0){a}(-0.5,1.5){b}(-1.5,-1.0){c}(0.5,-1.5){d}(2.0,-0.5){e}
    \pstGeonode[PointSymbol=none,PointName=none](-4.5;0){A}(4.5;0){B}
    \psline[linecolor=Black, linewidth=1pt, nodesep=-4.5](A)(B)
    \pspolygon[linecolor=Maroon, linewidth=1pt](a)(b)(c)(d)(e)
  \end{pspicture*}
  \hfill
  \begin{pspicture*}(-3,-3)(3,3)
    \psgrid[subgriddiv=2,gridlabels=0pt]
    \pstGeonode[PointSymbol=x,PointName=none](2.5,0.5){a}(-1.0,2.5){b}(-2.5,1.0){c}(-2.5,-2.5){d}(2.0,-2.0){e}
    \pstGeonode[PointSymbol=none,PointName=none](-4.5;90){A}(4.5;90){B}
    \psline[linecolor=Black, linewidth=1pt, nodesep=-4.5](A)(B)
    \pspolygon[linecolor=Maroon, linewidth=1pt](a)(b)(c)(d)(e)
  \end{pspicture*}
  \hfill
  \begin{pspicture*}(-3,-3)(3,3)
    \psgrid[subgriddiv=2,gridlabels=0pt]
    \pstGeonode[PointSymbol=x,PointName=none](2.0,2.0){a}(-0.5,3.0){b}(-1.5,-1.0){c}(-2.0,-3.0){d}(1.0,-2.0){e}
    \pstGeonode[PointSymbol=none,PointName=none](-4.5;45){A}(4.5;45){B}
    \psline[linecolor=Black, linewidth=1pt, nodesep=-4.5](A)(B)
    \pspolygon[linecolor=Maroon, linewidth=1pt](a)(b)(c)(d)(e)
  \end{pspicture*}
  \label{LastPage}
  \newpage
  \currentpdfbookmark{Le corrigé des exercices}{Corrigé}
  \lhead{\textsl{{\footnotesize Page \thepage/ \pageref{LastCorPage}}}}
  \setcounter{page}{1} \setcounter{exo}{0}

  \exercice*
  Calculer l'aire de chacune des figures suivantes dans l'unité d'aire donnée
  :\par
  \begin{pspicture}(0,0)(18,9)
    \psgrid[subgriddiv=2, gridlabels=0pt]
    \psframe[fillstyle=vlines, hatchsep=1pt](0,0)(.5,.5)
    \rput[l](0.6,0.25){\psframebox[linecolor=white, fillcolor=white,
      fillstyle=solid]{unité d'aire}}
    \psset{unit=5mm}

    \rput(0,12){
      \pspolygon(0, 0)(13, 0)(21, 6)(8, 6)
      \psframe[linestyle=dashed, fillstyle=hlines](0, 0)(13, 6)
      \rput(10.50,3.00){\psframebox[linecolor=white, fillcolor=white,
        fillstyle=solid]{figure 1}}
      }
    \rput(22,11){
      \psframe[fillstyle=hlines](0, 0)(7, 7)
      \rput(3.50,3.50){\psframebox[linecolor=white, fillcolor=white,
        fillstyle=solid]{figure 2}}
      }
    \rput(30,11){
      \psframe[linestyle=dashed](0, 7)(6, 0)
      \pspolygon[fillstyle=hlines](0, 7)(0, 0)(6, 0)
      \rput(2.00,2.33){\psframebox[linecolor=white, fillcolor=white,
        fillstyle=solid]{figure 3}}
      }
    \rput(0,2){
      \psframe[linestyle=dashed](0,0)(21,9) \rput(5.67,3.00){\pscirclebox{1}}
      \rput(19.67,0.67){\pscirclebox{2}} \rput(14.00,6.67){\pscirclebox{3}}
      \pspolygon[fillstyle=hlines](0, 9)(17, 0)(21, 2)
      \rput(12.67,3.67){\psframebox[linecolor=white, fillcolor=white,
        fillstyle=solid]{figure 4}}
      }
    \rput(22,2){
      \psframe[fillstyle=hlines](0, 0)(7, 8)
      \rput(3.50,4.00){\psframebox[linecolor=white, fillcolor=white,
        fillstyle=solid]{figure 5}}
      }
    \rput(30,2){
      \psframe[linestyle=dashed](6, 0)(0, 2)
      \psframe[linestyle=dashed](0, 2)(6, 8)
      \pspolygon[fillstyle=hlines](6, 0)(6, 8)(0, 2)
      \rput(4.00,3.33){\psframebox[linecolor=white, fillcolor=white,
        fillstyle=solid]{figure 6}}
      }
  \end{pspicture}
  \begin{enumerate}
  \item Aire de la figure 1 : c'est l'aire du rectangle en pointillés.\par
    $13 \times 6 = 78$~unités d'aire
  \item Aire de la figure 2 : $7 \times 7 = 49$~unités d'aire
  \item Aire de la figure 3 : c'est la moitié de l'aire du rectangle en
    pointillés.\par
    $(6 \times 7) \div 2= 21$~unités d'aire
  \item Aire de la figure 4 : on calcule l'aire du rectangle en pointillés et
    on soustrait les aires des triangles rectangles \pscirclebox{1},
    \pscirclebox{2} et \pscirclebox{3}.\par
    $(21 \times 9) - (17 \times 9) \div 2 - (4 \times 2) \div 2 - (21 \times 7)
    \div 2 = 35$~unités d'aire
  \item Aire de la figure 5 : $7 \times 8 = 56$~unités d'aire
  \item Aire de la figure 6 : c'est la moitié de l'aire du rectangle en
    pointillés.\par
    $(6 \times 8) \div 2= 24$~unités d'aire
  \end{enumerate}

  \exercice*
  Construire sur la figure ci-dessous les points $C$, $D$, $E$, $F$ et $G$ pour
  obtenir un zigzag tel que :\par
  $\widehat{ABC}=21\degres \qquad \widehat{BCD}=51\degres \qquad
  \widehat{CDE}=80\degres \qquad \widehat{DEF}=87\degres \qquad
  \widehat{EFG}=158\degres \qquad $\par
  Quand le travail est fait avec une bonne précision, les
  droites $(AG)$ et $(BF)$ se coupent au c\oe ur de la cible.\par
  \begin{center}
    \fbox{
      \begin{pspicture}(-.4,-.4)(16.4, 5.5)

        \pstGeonode[PosAngle=-100.00, PointSymbol=x](0.20, 0.20){A}
        \pscircle[linecolor=Gray](9.15, 3.02){0.1}
        \pscircle[linecolor=Gray](9.15, 3.02){0.2}
        \pscircle[linecolor=Gray](9.15, 3.02){0.3}
        \pscircle[linecolor=Gray](9.15, 3.02){0.4}
        \pstGeonode[PosAngle=-269.50](0.89, 4.14){B} \pstSegmentMark{A}{B}
        \pstGeonode[PosAngle=-104.50](1.66, 0.21){C} \pstSegmentMark{B}{C}
        \pstGeonode[PosAngle=-270.00](4.23, 3.28){D} \pstSegmentMark{C}{D}
        \pstGeonode[PosAngle=-93.50](6.80, 0.21){E} \pstSegmentMark{D}{E}
        \pstGeonode[PosAngle=-238.00](9.73, 2.94){F} \pstSegmentMark{E}{F}
        \pstGeonode[PosAngle=21.00, PointSymbol=x](13.46, 4.37){G}
        \pstSegmentMark{F}{G}
        \pstMarkAngle{A}{B}{C}{21\degres}
        \pstMarkAngle{D}{C}{B}{51\degres}
        \pstMarkAngle{C}{D}{E}{80\degres}
        \pstMarkAngle{F}{E}{D}{87\degres}
        \pstMarkAngle{E}{F}{G}{158\degres}
        \psline[linestyle=dotted](B)(F) \psline[linestyle=dotted](A)(G)
      \end{pspicture}
      }
  \end{center}

  \exercice*
  Nommer, mesurer et donner la nature de chacun des angles suivants :\par
  \begin{pspicture}(18,8)
    \psset{PointSymbol=none,MarkAngleRadius=0.6}
    \psframe(0,0)(18,8)
    \pstGeonode[PointName={T,J,B},PosAngle={289.5,329,250}](7.60,2.00){a00}(10.69,7.14){a10}(1.96,4.05){a20}
    \pstMarkAngle{a10}{a00}{a20}{}
    \pstLineAB[nodesepB=-.5]{a00}{a10}\pstLineAB[arrows=-|,linestyle=none]{a00}{a10}
    \pstLineAB[nodesepB=-.5]{a00}{a20}\pstLineAB[arrows=-|,linestyle=none]{a00}{a20}
    \pstGeonode[PointName={U,E,K},PosAngle={274.5,295,254}](7.50,0.50){a01}(12.94,3.04){a11}(1.73,2.15){a21}
    \pstMarkAngle{a11}{a01}{a21}{}
    \pstLineAB[nodesepB=-.5]{a01}{a11}\pstLineAB[arrows=-|,linestyle=none]{a01}{a11}
    \pstLineAB[nodesepB=-.5]{a01}{a21}\pstLineAB[arrows=-|,linestyle=none]{a01}{a21}
    \pstGeonode[PointName={H,Q,X},PosAngle={172.0,249,95}](1.60,6.60){a02}(7.20,4.45){a12}(7.58,7.12){a22}
    \pstMarkAngle{a12}{a02}{a22}{}
    \pstLineAB[nodesepB=-.5]{a02}{a12}\pstLineAB[arrows=-|,linestyle=none]{a02}{a12}
    \pstLineAB[nodesepB=-.5]{a02}{a22}\pstLineAB[arrows=-|,linestyle=none]{a02}{a22}
    \pstGeonode[PointName={G,S,M},PosAngle={36.0,98,334}](17.40,6.10){a03}(11.46,5.26){a13}(14.77,0.71){a23}
    \pstMarkAngle{a13}{a03}{a23}{}
    \pstLineAB[nodesepB=-.5]{a03}{a13}\pstLineAB[arrows=-|,linestyle=none]{a03}{a13}
    \pstLineAB[nodesepB=-.5]{a03}{a23}\pstLineAB[arrows=-|,linestyle=none]{a03}{a23}
  \end{pspicture}\par
  \begin{multicols}{4}
    $\widehat{JTB}=101\degres$\par
    angle obtus\par
    $\widehat{EUK}=139\degres$\par
    angle obtus\par
    $\widehat{QHX}=26\degres$\par
    angle aigu\par
    $\widehat{SGM}=56\degres$\par
    angle aigu\par
  \end{multicols}

  \exercice*
  \begin{enumerate}
  \item L'encadrement de 894\,076 à la dizaine est :\par
    894\,070 < 894\,076 < 894\,080\par
    On en déduit que son arrondi à la dizaine  est : 894\,080.
  \item L'encadrement de 6\,766\,140 à la centaine est :\par
    6\,766\,100 < 6\,766\,140 < 6\,766\,200\par
    On en déduit que son arrondi à la centaine  par excès est : 6\,766\,200.
  \item L'encadrement de 8\,125,72 au dixième est :\par
    8\,125,7 < 8\,125,72 < 8\,125,8\par
    On en déduit que son arrondi au dixième  est : 8\,125,7.
  \item L'encadrement de 88\,223,3 à l'unité est :\par
    88\,223 < 88\,223,3 < 88\,224\par
    On en déduit que son arrondi à l'unité  est : 88\,223.
  \end{enumerate}

  \exercice*
  \begin{enumerate}
  \item Classer les nombres suivants dans l'ordre décroissant.\par
    5,1 \kern1cm ; \kern1cm 1,293 \kern1cm ; \kern1cm 1,21 \kern1cm ; \kern1cm
    1,5\par
    5,1 \kern1cm \textgreater \kern1cm 1,5 \kern1cm \textgreater \kern1cm 1,293
    \kern1cm \textgreater \kern1cm 1,21
  \item Classer les nombres suivants dans l'ordre croissant.\par
    1,6 \kern1cm ; \kern1cm 6,322 \kern1cm ; \kern1cm 6,43 \kern1cm ; \kern1cm
    6,5\par
    1,6 \kern1cm \textless \kern1cm 6,322 \kern1cm \textless \kern1cm 6,43
    \kern1cm \textless \kern1cm 6,5
  \end{enumerate}

  \exercice*
  Compléter avec un nombre décimal :
  \begin{multicols}{2}\noindent
    \begin{enumerate}
    \item $1\times \cfrac{1}{1\,000} + 7\times 1 + 5\times \cfrac{1}{10} =
      7{,}501$
    \item $7\times \cfrac{1}{100} + 6\times \cfrac{1}{10} + 3\times 1 = 3{,}67$
    \item $2\times \cfrac{1}{1\,000} + 3\times 100 + 5\times 10 = 350{,}002$
    \item $6\times 10 + 8\times 1 + 9\times 100 = 968$
    \item $9\times \cfrac{1}{100} + 9\times 1 + 8\times 1\,000 = 8\,009{,}09$
    \item $4\times \cfrac{1}{1\,000} + 9\times 1\,000 + 5\times \cfrac{1}{100}
      = 9\,000{,}054$
    \end{enumerate}
  \end{multicols}

  \exercice*
  \begin{enumerate}
  \item Écrire en chiffres les nombres suivants.
    \begin{enumerate}
    \item neuf-cent-quarante-quatre-millions-neuf-cent-soixante-dix-mille :
      944\,970\,000
    \item quatre-vingts-mille-quatre-cent-deux unités et six dixièmes :
      80\,402,6
    \item neuf unités et sept dixièmes :
      9,7
    \item deux-cent-cinquante-huit millièmes :
      0,258
    \item six-cent-quarante-neuf-millions-huit-cent-quatre-mille-vingt-six :
      649\,804\,026
    \item sept-cent-vingt-sept-millions-vingt-deux-mille-six-cent-trente-quatre
      :
      727\,022\,634
    \item trente-millions-deux-cent-cinquante-huit :
      30\,000\,258
    \item deux-mille-deux-cent-soixante-trois unités et quatre dixièmes :
      2\,263,4
    \end{enumerate}
  \item Écrire en lettres les nombres suivants (sans utiliser le mot
    ``virgule").
    \begin{enumerate}
    \item 320,63 :
      trois-cent-vingt unités et soixante-trois centièmes
    \item 826\,000\,202 :
      huit-cent-vingt-six-millions-deux-cent-deux
    \item 0,202 :
      deux-cent-deux millièmes
    \item 633\,500\,000 :
      six-cent-trente-trois-millions-cinq-cent-mille
    \item 994\,000\,358 :
      neuf-cent-quatre-vingt-quatorze-millions-trois-cent-cinquante-huit
    \item 445\,032\,063 :
      quatre-cent-quarante-cinq-millions-trente-deux-mille-soixante-trois
    \item 3,58 :
      trois unités et cinquante-huit centièmes
    \item 0,5 :
      cinq dixièmes
    \end{enumerate}
  \end{enumerate}

  \exercice*
  Compléter :
  \begin{multicols}{3}\noindent
    \begin{enumerate}
    \item $\cfrac{231}{1\,000}=\mathbf{0{,}231}$
    \item $\cfrac{5\,626}{\mathbf{1\,000}}=5{,}626$
    \item $\cfrac{84\,020}{10\,000}=\mathbf{8{,}402}$
    \item $\cfrac{\mathbf{12\,200}}{10\,000}=1{,}22$
    \item $\cfrac{1\,780}{10\,000}=\mathbf{0{,}178}$
    \item $\cfrac{\mathbf{2\,022}}{100}=20{,}22$
    \end{enumerate}
  \end{multicols}

  \exercice*
  Placer une virgule (en ajoutant éventuellement des zéros) dans            le
  nombre 821349 de telle sorte que :
  \begin{enumerate}
  \item le chiffre 3 soit le chiffre des milliers :
    8\,213\,490
  \item le chiffre 9 soit le chiffre des dixièmes :
    82\,134,9
  \item le chiffre 3 soit le chiffre des unités :
    8\,213,49
  \item le chiffre 4 soit le chiffre des centièmes :
    821,349
  \item le chiffre 3 soit le chiffre des centaines :
    821\,349
  \item le chiffre 9 soit le chiffre des millièmes :
    821,349
  \end{enumerate}

  \exercice*
  {
    \psset{nodesepA = -1.5mm, linewidth = 0.6pt, linestyle = dotted, vref =
      -0.8mm}
    \def\virgule{\textcolor{red}{ \LARGE ,}}
    Effectuer les conversions suivantes :
    \begin{multicols}{2}\noindent
      \begin{enumerate}
      \item $\unit[5,32]{dm^2}=\unit[0,000\,532]{dam^2}$\vspace{1ex}\par
      \item $\unit[3,08]{cm^2}=\unit[0,000\,308]{m^2}$\vspace{1ex}\par
      \item $\unit[6,73]{km^2}=\unit[673]{hm^2}$\vspace{1ex}\par
      \item $\unit[91,8]{dm^2}=\unit[918\,000]{mm^2}$\vspace{1ex}\par
      \item $\unit[76,1]{dam^2}=\unit[7\,610]{m^2}$\vspace{1ex}\par
      \item $\unit[12,8]{hm^2}=\unit[1\,280]{dam^2}$\vspace{1ex}\par
      \end{enumerate}
    \end{multicols}
    \begin{tabular}{*{13}{p{3.5mm}|}p{3.5mm}}
      \multicolumn{2}{c|}{$\rm km^2$} & \multicolumn{2}{c|}{$\rm hm^2$} &
      \multicolumn{2}{c|}{$\rm dam^2$} & \multicolumn{2}{c|}{$\rm m^2$} &
      \multicolumn{2}{c|}{$\rm dm^2$} & \multicolumn{2}{c|}{$\rm cm^2$}
      &\multicolumn{2}{c}{$\rm mm^2$}\\ \hline
      & & & & & {\textcolor{red}{0}\Rnode{virg1}{\virgule}}&
      \textcolor{red}{0}& \textcolor{red}{0}& \textcolor{red}{0}&
      \textcolor{blue}{5}\Rnode{virg0}{\ }& \textcolor{blue}{3}&
      \textcolor{blue}{2}& &
      \ncline{->}{virg0}{virg1} \\
      & & & & & & & {\textcolor{red}{0}\Rnode{virg1}{\virgule}}&
      \textcolor{red}{0}& \textcolor{red}{0}& \textcolor{red}{0}&
      \textcolor{blue}{3}\Rnode{virg0}{\ }& \textcolor{blue}{0}&
      \textcolor{blue}{8}
      \ncline{->}{virg0}{virg1} \\
      & \textcolor{blue}{6}\Rnode{virg0}{\ }& \textcolor{blue}{7}&
      {\textcolor{blue}{3}\Rnode{virg1}{\virgule}}& & & & & & & & & &
      \ncline{->}{virg0}{virg1} \\
      & & & & & & & & \textcolor{blue}{9}& \textcolor{blue}{1}\Rnode{virg0}{\
        }& \textcolor{blue}{8}& \textcolor{red}{0}& \textcolor{red}{0}&
      {\textcolor{red}{0}\Rnode{virg1}{\virgule}}
      \ncline{->}{virg0}{virg1} \\
      & & & & \textcolor{blue}{7}& \textcolor{blue}{6}\Rnode{virg0}{\ }&
      \textcolor{blue}{1}& {\textcolor{red}{0}\Rnode{virg1}{\virgule}}& & & & &
      &
      \ncline{->}{virg0}{virg1} \\
      & & \textcolor{blue}{1}& \textcolor{blue}{2}\Rnode{virg0}{\ }&
      \textcolor{blue}{8}& {\textcolor{red}{0}\Rnode{virg1}{\virgule}}& & & & &
      & & &
      \ncline{->}{virg0}{virg1} \\
    \end{tabular}
    }

  \exercice*
  {
    \psset{nodesepA = -1.5mm, linewidth = 0.6pt, linestyle = dotted, vref =
      -0.8mm}
    \def\virgule{\textcolor{red}{ \LARGE ,}}
    Effectuer les conversions suivantes :
    \begin{multicols}{2}\noindent
      \begin{enumerate}
      \item $\unit[2,08]{dam^3}=\unit[2\,080]{m^3}$\vspace{1ex}\par
      \item $\unit[9,66]{dam^3}=\unit[9\,660]{m^3}$\vspace{1ex}\par
      \item $\unit[4,07]{dm^3}=\unit[0,004\,07]{m^3}$\vspace{1ex}\par
      \item $\unit[9,06]{dm^3}=\unit[9\,060]{cm^3}$\vspace{1ex}\par
      \item $\unit[8,56]{m^3}=\unit[0,000\,008\,56]{hm^3}$\vspace{1ex}\par
      \item $\unit[93,4]{dm^3}=\unit[0,093\,4]{m^3}$\vspace{1ex}\par
      \end{enumerate}
    \end{multicols}
    \begin{tabular}{*{20}{p{3.5mm}|}p{3.5mm}}
      \multicolumn{3}{c|}{$\rm km^3$} & \multicolumn{3}{c|}{$\rm hm^3$} &
      \multicolumn{3}{c|}{$\rm dam^3$} & \multicolumn{3}{c|}{$\rm m^3$} &
      \multicolumn{3}{c|}{$\rm dm^3$} & \multicolumn{3}{c|}{$\rm cm^3$}
      &\multicolumn{3}{c}{$\rm mm^3$}\\ \hline
      & & & & & & & & \textcolor{blue}{2}\Rnode{virg0}{\ }&
      \textcolor{blue}{0}& \textcolor{blue}{8}&
      {\textcolor{red}{0}\Rnode{virg1}{\virgule}}& & & & & & & & &
      \ncline{->}{virg0}{virg1} \\
      & & & & & & & & \textcolor{blue}{9}\Rnode{virg0}{\ }&
      \textcolor{blue}{6}& \textcolor{blue}{6}&
      {\textcolor{red}{0}\Rnode{virg1}{\virgule}}& & & & & & & & &
      \ncline{->}{virg0}{virg1} \\
      & & & & & & & & & & & {\textcolor{red}{0}\Rnode{virg1}{\virgule}}&
      \textcolor{red}{0}& \textcolor{red}{0}&
      \textcolor{blue}{4}\Rnode{virg0}{\ }& \textcolor{blue}{0}&
      \textcolor{blue}{7}& & & &
      \ncline{->}{virg0}{virg1} \\
      & & & & & & & & & & & & & & \textcolor{blue}{9}\Rnode{virg0}{\ }&
      \textcolor{blue}{0}& \textcolor{blue}{6}&
      {\textcolor{red}{0}\Rnode{virg1}{\virgule}}& & &
      \ncline{->}{virg0}{virg1} \\
      & & & & & {\textcolor{red}{0}\Rnode{virg1}{\virgule}}&
      \textcolor{red}{0}& \textcolor{red}{0}& \textcolor{red}{0}&
      \textcolor{red}{0}& \textcolor{red}{0}&
      \textcolor{blue}{8}\Rnode{virg0}{\ }& \textcolor{blue}{5}&
      \textcolor{blue}{6}& & & & & & &
      \ncline{->}{virg0}{virg1} \\
      & & & & & & & & & & & {\textcolor{red}{0}\Rnode{virg1}{\virgule}}&
      \textcolor{red}{0}& \textcolor{blue}{9}&
      \textcolor{blue}{3}\Rnode{virg0}{\ }& \textcolor{blue}{4}& & & & &
      \ncline{->}{virg0}{virg1} \\
    \end{tabular}
    }

  \exercice*
  {
    \psset{nodesepA = -1.5mm, linewidth = 0.6pt, linestyle = dotted, vref =
      -0.8mm}
    Effectuer les conversions suivantes :
    \begin{multicols}{2}\noindent
      \begin{enumerate}
      \item 1,33~hL=13,3~daL\par
        \begin{tabular}{c|c|c|c|c|c}
          hL & daL & L & dL & cL & mL \\ \hline
          1\Rnode{virg0}{\ } & {3\Rnode{virg1}{\textcolor{red}{ \LARGE ,}}}& 3&
          0& 0& 0
        \end{tabular}
        \ncline{->}{virg0}{virg1}
      \item 7,87~dg=78,7~cg\par
        \begin{tabular}{c|c|c|c|c|c|c}
          kg & hg & dag & g & dg & cg & mg \\ \hline
          0 & 0& 0& 0& 7\Rnode{virg0}{\ }& {8\Rnode{virg1}{\textcolor{red}{
                \LARGE ,}}}& 7
        \end{tabular}
        \ncline{->}{virg0}{virg1}
      \item 9,61~dam=96\,100~mm\par
        \begin{tabular}{c|c|c|c|c|c|c}
          km & hm & dam & m & dm & cm & mm \\ \hline
          0 & 0& 9\Rnode{virg0}{\ }& 6& 1& 0& {0\Rnode{virg1}{\textcolor{red}{
                \LARGE ,}}}
        \end{tabular}
        \ncline{->}{virg0}{virg1}
      \item 9,97~hL=99,7~daL\par
        \begin{tabular}{c|c|c|c|c|c}
          hL & daL & L & dL & cL & mL \\ \hline
          9\Rnode{virg0}{\ } & {9\Rnode{virg1}{\textcolor{red}{ \LARGE ,}}}& 7&
          0& 0& 0
        \end{tabular}
        \ncline{->}{virg0}{virg1}
      \item 97,7~L=97\,700~mL\par
        \begin{tabular}{c|c|c|c|c|c}
          hL & daL & L & dL & cL & mL \\ \hline
          0 & 9& 7\Rnode{virg0}{\ }& 7& 0& {0\Rnode{virg1}{\textcolor{red}{
                \LARGE ,}}}
        \end{tabular}
        \ncline{->}{virg0}{virg1}
      \item 22,9~hL=229~daL\par
        \begin{tabular}{c|c|c|c|c|c}
          hL & daL & L & dL & cL & mL \\ \hline
          22\Rnode{virg0}{\ } & {9\Rnode{virg1}{\textcolor{red}{ \LARGE ,}}}&
          0& 0& 0& 0
        \end{tabular}
        \ncline{->}{virg0}{virg1}
      \end{enumerate}
    \end{multicols}
    {

      \exercice*
      Compléter les pointillés et les figures :\par
      \renewcommand{\tabularxcolumn}[1]{m{#1}}
      \begin{tabularx}{\linewidth}{|X|>{\centering}m{5cm}|}
        \hline
        \textbf{Phrase} & \textbf{Figure} \tabularnewline \hline
        $[ QA ]$ est un segment &
        \begin{pspicture*}(-0.5,0.2)(4.5,2.2)
          \psset{PointSymbol=x}
          \pstGeonode[PosAngle=90](0.5,1.5){Z}(2,1.0){A}(3.5,0.8){Q}
          \pstLineAB[nodesepA=0, nodesepB=0]{Q}{A}
        \end{pspicture*}\tabularnewline
        \hline
        $[ LP ]$ est un segment &
        \begin{pspicture*}(-0.5,0.2)(4.5,2.2)
          \psset{PointSymbol=x}
          \pstGeonode[PosAngle=90](0.5,1.3){P}(2,0.6){L}(3.5,1.0){M}
          \pstLineAB[nodesepA=0, nodesepB=0]{L}{P}
        \end{pspicture*}\tabularnewline
        \hline
        $( MY )$ est une droite &
        \begin{pspicture*}(-0.5,0.2)(4.5,2.2)
          \psset{PointSymbol=x}
          \pstGeonode[PosAngle=90](0.5,0.5){M}(2,0.9){Y}(3.5,1.3){G}
          \pstLineAB[nodesepA=-6, nodesepB=-6]{M}{Y}
        \end{pspicture*}\tabularnewline
        \hline
        $[ PC )$ est une demi-droite &
        \begin{pspicture*}(-0.5,0.2)(4.5,2.2)
          \psset{PointSymbol=x}
          \pstGeonode[PosAngle=90](0.5,1.4){Z}(2,0.6){C}(3.5,0.5){P}
          \pstLineAB[nodesepA=0, nodesepB=-6]{P}{C}
        \end{pspicture*}\tabularnewline
        \hline
        $( HP )$ est une droite &
        \begin{pspicture*}(-0.5,0.2)(4.5,2.2)
          \psset{PointSymbol=x}
          \pstGeonode[PosAngle=90](0.5,1.4){H}(2,1.4){N}(3.5,0.7){P}
          \pstLineAB[nodesepA=-6, nodesepB=-6]{H}{P}
        \end{pspicture*}\tabularnewline
        \hline
      \end{tabularx}

      \exercice*
      Réaliser les figures suivantes :\par
      \begin{multicols}{2}
        \begin{pspicture*}(-4,-4)(4,4)
          \psset{PointSymbol=x}
          \pstGeonode[PointName={I,U}](3;75){a}(3;255){b}
          \pstGeonode[PointName={V,W}](2.1; 175){c}(2.0; 326){d}   
          \pstLineAB[nodesep=-4, linecolor=DarkBlue]{c}{b}
          \pstProjection[PointName=none]{c}{b}{a}[e]\pstLineAB[nodesep=-7,
          linecolor=DarkBlue]{a}{e}
          \pstRightAngle[, linecolor=DarkBlue]{a}{e}{c}
          \pstLineAB[nodesep=-4, linecolor=DarkRed]{a}{c}
          \pstTranslation[PointName=none,PointSymbol=none]{a}{c}{b}[f]
          \pstLineAB[nodesep=-7, linecolor=DarkRed]{b}{f}
        \end{pspicture*}
        \par
        \begin{enumerate}
        \item Tracer la droite parallèle à la droite $(IV)$ passant par $U$
        \item Tracer la droite perpendiculaire à la droite $(VU)$ passant par
          $I$
        \end{enumerate}
        \columnbreak
        \begin{pspicture*}(-4,-4)(4,4)
          \psset{PointSymbol=x}
          \pstGeonode[PointName={F,L}](3;80){a}(3;260){b}
          \pstGeonode[PointName={X,Z}](2.4; 157){c}(2.7; 315){d}   
          \pstLineAB[nodesep=-4, linecolor=DarkBlue]{d}{c}
          \pstProjection[PointName=none]{d}{c}{a}[e]\pstLineAB[nodesep=-7,
          linecolor=DarkBlue]{a}{e}
          \pstRightAngle[, linecolor=DarkBlue]{a}{e}{d}
          \pstLineAB[nodesep=-4, linecolor=DarkRed]{c}{a}
          \pstTranslation[PointName=none,PointSymbol=none]{c}{a}{d}[f]
          \pstLineAB[nodesep=-7, linecolor=DarkRed]{d}{f}
        \end{pspicture*}
        \par
        \begin{enumerate}
        \item Tracer la droite perpendiculaire à la droite $(ZX)$ passant par
          $F$
        \item Tracer la droite parallèle à la droite $(XF)$ passant par $Z$
        \end{enumerate}
      \end{multicols}

      \exercice*
      Compléter le tableau suivant :\par Les droites en gras sont
      parallèles.\par
      \renewcommand{\tabularxcolumn}[1]{m{#1}}
      \begin{tabularx}{\textwidth}[t]{|m{3cm}|m{4cm}|X|m{3cm}|}
        \hline
        \multicolumn{1}{|c|}{\bf Données} & \multicolumn{1}{|c|}{\bf Figure
          codée}
        & \multicolumn{1}{|c|}{\bf Propriété} & \multicolumn{1}{|c|}{\bf
          Conclusion}\\
        \hline
        $(d_2)//(d_1)$\par et\par $(d_2)//(d_3)$ &
        \begin{pspicture*}[shift=-1.5](-2.1,-1.6)(2.1,1.6)
          \footnotesize
          \psplot[linewidth=1.5\pslinewidth]{-2.1}{2.1}{x 0.383 mul 0.061 add}
          \psplot[linewidth=1.5\pslinewidth]{-2.1}{2.1}{x 0.383 mul 0.597 add}
          \psplot[linewidth=1.5\pslinewidth]{-2.1}{2.1}{x 0.383 mul -0.582 add}
          \pstGeonode[PointSymbol=none,PosAngle=66,PointName={(d_2),none}](-1.5,-0.514){a1}(1.5,0.635){a2}
          \pstGeonode[PointSymbol=none,PosAngle=66.0,PointName={(d_1),none}](-1.5,0.022){b1}(1.5,1.171){b2}
          \pstGeonode[PointSymbol=none,PosAngle=-24.0,PointName={(d_3),none}](-1.5,-1.157){c1}(1.5,-0.008){c2}
        \end{pspicture*} &
        Si deux droites sont parallèles, alors toute parallèle à l'une est
        parallèle à l'autre. &
        $(d_1)//(d_3)$ \\
        \hline
        \hline
        $(d_1)\perp(d_2)$\par et\par $(d_3)\perp(d_2)$ &
        \begin{pspicture*}[shift=-1.5](-2.1,-1.6)(2.1,1.6)
          \footnotesize
          \psplot{-2.1}{2.1}{x 0.809 mul 0.104 add}
          \psplot{-2.1}{2.1}{x 0.809 mul 0.876 add}
          \psplot{-2.1}{2.1}{x -1.235 mul}
          \pstGeonode[PointSymbol=none,PointName={none,(d_1)}
          ,PosAngle=84](-0.051,0.062){i1}(1.228,1.1){a1}
          \pstGeonode[PointSymbol=none,PointName={none,(d_3)}
          ,PosAngle=84](-0.429,0.529){i2}(-1.5,-0.338){b1}
          \pstGeonode[PointSymbol=none,PointName={none,(d_2)}](-0.891,1.1){c1}(0.89,-1.1){c2}
          \pstRightAngle[RightAngleSize=.2]{c1}{i1}{a1}
          \pstRightAngle[RightAngleSize=.2]{c1}{i2}{b1}
        \end{pspicture*} &
        Si deux droites sont perpendiculaires à une même troisième alors elles
        sont parallèles entre elles. &
        $(d_1)//(d_3)$ \\
        \hline
        \hline
        $(IJ)//(KM)$\par et\par $(IJ)\perp(IK)$ &
        \begin{pspicture*}[shift=-1.5](-2.1,-1.6)(2.1,1.6)
          \footnotesize
          \psplot[linewidth=1.5\pslinewidth]{-2.1}{2.1}{x 0.034 mul 0.206 add}
          \psplot[linewidth=1.5\pslinewidth]{-2.1}{2.1}{x 0.034 mul 0.907 add}
          \psplot{-2.1}{2.1}{x -28.637 mul}
          \pstGeonode[PointSymbol={none,x},PointName={I,J}
          ,PosAngle=47](-0.008,0.205){i1}(-1.5,0.154){a1}
          \pstGeonode[PointSymbol={none,x},PointName={K,M}
          ,PosAngle=47](-0.032,0.905){i2}(-1.5,0.856){b1}
          \pstGeonode[PointSymbol=none,PointName=none](-0.039,1.1){c1}(0.038,-1.1){c2}
          \pstRightAngle[RightAngleSize=.2]{c1}{i1}{a1}
        \end{pspicture*} &
        Si deux droites sont parallèles, alors toute perpendiculaire à l'une
        est perpendiculaire à l'autre. &
        $(KM)\perp(IK)$ \\
        \hline
      \end{tabularx}

      \exercice*
      Les figures 1 et 2 représentent le même cube CEGLNQST.\\
      \psset{xunit=1.0cm,yunit=1.0cm,dotstyle=*,dotsize=3pt
        0,linewidth=0.8pt,arrowsize=3pt 2,arrowinset=0.25}
      \begin{pspicture*}(-2,-0.38)(15,4.5)
        \psframe[fillstyle=solid,fillcolor=darkgray,framearc=0.2](-1,3.5)(-0.5,4)
        \rput[bl](-0.85,3.65){\white{\textbf{$1$}}}
        \psline[linestyle=dashed,linecolor=Maroon,dash=4pt 4pt](3.5,1)(1.5,1)
        \psline[linestyle=dashed,linecolor=Maroon,dash=4pt 4pt](1.5,1)(1.5,3)
        \psline[linestyle=dashed,linecolor=Maroon,dash=4pt 4pt](1.5,1)(0.5,0)
        \psframe[linecolor=Maroon](0.5,0)(2.5,2)
        \psline[linecolor=Maroon](1.5,3)(3.5,3)
        \psline[linecolor=Maroon](3.5,3)(3.5,1)
        \psline[linecolor=Maroon](0.5,2)(1.5,3)
        \psline[linecolor=Maroon](2.5,2)(3.5,3)
        \psline[linecolor=Maroon](2.5,0)(3.5,1)
        \rput[bl](0.22,2.08){C}
        \rput[bl](2.28,2.16){E}
        \rput[bl](2.6,-0.22){G}
        \rput[bl](0.15,-0.22){L}
        \rput[bl](1.4,3.08){N}
        \rput[bl](3.58,3.04){Q}
        \rput[bl](3.58,1.04){S}
        \rput[bl](1.58,1.04){T}
        \psline[linecolor=Maroon](5.2,-04)(5.2,4)
        \psframe[fillstyle=solid,fillcolor=darkgray,framearc=0.2](5.7,3.5)(6.2,4)
        \rput[bl](5.85,3.65){\white{\textbf{$2$}}}
        \psline[linecolor=Maroon](8.48,2.71)(7.18,2.42)
        \psline[linecolor=Maroon](7.18,2.42)(7.18,0.46)
        \psline[linestyle=dashed,dash=4pt 4pt,
        linecolor=Maroon](7.18,0.46)(8.48,0.75)
        \psline[linestyle=dashed,dash=4pt 4pt,
        linecolor=Maroon](8.48,0.75)(8.48,2.71)
        \psline[linecolor=Maroon](10,2.46)(8.7,2.17)
        \psline[linecolor=Maroon](8.7,2.17)(8.7,0.21)
        \psline[linecolor=Maroon](8.7,0.21)(10,0.5)
        \psline[linecolor=Maroon](10,0.5)(10,2.46)
        \psline[linecolor=Maroon](10,2.46)(8.48,2.71)
        \psline[linecolor=Maroon](7.18,2.42)(8.7,2.17)
        \psline[linecolor=Maroon](8.7,0.21)(7.18,0.46)
        \psline[linestyle=dashed,dash=4pt 4pt,
        linecolor=Maroon](10,0.5)(8.48,0.75)
        \rput[bl](10.05,0.2){S}
        \rput[bl](8.71,-0.1){Q}
        \rput[bl](8.1,0.79){G}
        \rput[bl](6.8,0.2){E}
        \rput[bl](10.06,2.5){T}
        \rput[bl](8.68,2.27){N}
        \rput[bl](7.08,2.54){C}
        \rput[bl](8.5,2.83){L}
      \end{pspicture*}
      \begin{enumerate}
      \item Compléter les sommets manquants de la figure 2.
      \item Donner toutes les arêtes perpendiculaires à [TL].\par
        [TS], [TN], [LG] et [LC] sont les arêtes perpendiculaires à [TL].
      \item Donner toutes les arêtes parallèles à [NC]. \par
        [EQ], [GS] et [LT] sont les arêtes parallèles à [NC].
      \end{enumerate}

      \exercice*
      \begin{multicols}{2}
        \begin{enumerate}
        \item Colorer $\frac{6}{4}$ de ce rectangle.\par
          \psset{unit=4mm}
          \begin{pspicture}(16,4)
            \psframe[fillstyle=solid](0,0)(7,1)
            \psframe[fillstyle=solid](0,1)(7,2)
            \psframe[fillstyle=solid](0,2)(7,3)
            \psframe[fillstyle=solid](0,3)(7,4)
            \rput(8,0){\psframe[fillstyle=solid](0,0)(7,1)}
            \rput(8,0){\psframe[fillstyle=solid](0,1)(7,2)}
            \psgrid[gridcolor=Olive,subgriddiv=0,gridlabels=0pt]
            \psframe[linewidth=1.5\pslinewidth,linecolor=Maroon](0,0)(7,4)
            \psframe[linewidth=1.5\pslinewidth,linecolor=Maroon](8,0)(15,4)
            \multips(0,1)(0,1){3}{\psline[linecolor=Maroon](0,0)(7,0)}
            \rput(8,0){\multips(0,1)(0,1){3}{\psline[linecolor=Maroon](0,0)(7,0)}}
          \end{pspicture}
        \item Colorer $\frac{4}{7}$ de ce rectangle.\par
          \psset{unit=4mm}
          \begin{pspicture}(16,5)
            \psframe[fillstyle=solid](0,0)(1,5)
            \psframe[fillstyle=solid](1,0)(2,5)
            \psframe[fillstyle=solid](2,0)(3,5)
            \psframe[fillstyle=solid](3,0)(4,5)
            \psgrid[gridcolor=Olive,subgriddiv=0,gridlabels=0pt]
            \psframe[linewidth=1.5\pslinewidth,linecolor=Maroon](0,0)(7,5)
            \multips(1,0)(1,0){6}{\psline[linecolor=Maroon](0,0)(0,5)}
          \end{pspicture}
          \columnbreak
        \item Colorer $\frac{9}{9}$ de ce rectangle.\par
          \psset{unit=4mm}
          \begin{pspicture}(16,4)
            \psframe[fillstyle=solid](0,0)(6,4)
            \psgrid[gridcolor=Olive,subgriddiv=0,gridlabels=0pt]
            \psframe[linewidth=1.5\pslinewidth,linecolor=Maroon](0,0)(6,4)
          \end{pspicture}
        \item Colorer $\frac{5}{3}$ de ce rectangle.\par
          \psset{unit=4mm}
          \begin{pspicture}(16,6)
            \psframe[fillstyle=solid](0,0)(7,2)
            \psframe[fillstyle=solid](0,2)(7,4)
            \psframe[fillstyle=solid](0,4)(7,6)
            \rput(8,0){\psframe[fillstyle=solid](0,0)(7,2)}
            \rput(8,0){\psframe[fillstyle=solid](0,2)(7,4)}
            \psgrid[gridcolor=Olive,subgriddiv=0,gridlabels=0pt]
            \psframe[linewidth=1.5\pslinewidth,linecolor=Maroon](0,0)(7,6)
            \psframe[linewidth=1.5\pslinewidth,linecolor=Maroon](8,0)(15,6)
            \multips(0,2)(0,2){2}{\psline[linecolor=Maroon](0,0)(7,0)}
            \rput(8,0){\multips(0,2)(0,2){2}{\psline[linecolor=Maroon](0,0)(7,0)}}
          \end{pspicture}
        \end{enumerate}
      \end{multicols}

      \exercice*
      \begin{enumerate}
      \item Compléter :
        \begin{multicols}{2}
          \begin{enumerate}
          \item 1 unité = 9 neuvièmes
          \item 1 unité = 3 tiers
          \item 10 unités = 90 neuvièmes
          \item 10 unités = 30 tiers
          \end{enumerate}
        \end{multicols}
      \item Sur la demi-droite ci-dessous, placer les points d'abscisse donnée :
        \begin{multicols}{5}
          \begin{enumerate}
            \renewcommand{\theenumii}{\Alph{enumii}}
            \renewcommand{\labelenumii}{$\theenumii$}
          \item $\left(\cfrac{132}{9}\right)$
          \item $\left(\cfrac{101}{9}\right)$
          \item $\left(\cfrac{36}{3}\right)$
          \item $\left(\cfrac{39}{3}\right)$
          \item $\left(\cfrac{112}{7}\right)$
          \end{enumerate}
        \end{multicols}
      \item Compléter les abscisses des points suivants :
        \begin{multicols}{4}
          \begin{enumerate}
          \item $F~\left(\cfrac{96}{9}\right)$
          \item $F~\left(\cfrac{32}{3}\right)$
          \item $G~\left(\cfrac{99}{9}\right)$
          \item $G~\left(\cfrac{33}{3}\right)$
          \end{enumerate}
        \end{multicols}
      \end{enumerate}
      \begin{pspicture}(0,-.5)(18,.5)
        \psline[arrowscale=2,linecolor=Maroon]{->}(0,0)(18,0)
        \rput(2mm,0){%
          \multips(0,0)(3 mm,0){58}{\psline[linecolor=Maroon](0,-.1)(0,.1)}
          \multips(0,0)(27 mm,0){7}{\psline[linecolor=Maroon](0,-.2)(0,.2)}
          \rput[t](0 mm,-3mm){\centering 10}
          \rput[t](27 mm,-3mm){\centering 11}
          \rput[t](54 mm,-3mm){\centering 12}
          \rput[t](81 mm,-3mm){\centering 13}
          \rput[t](108 mm,-3mm){\centering 14}
          \rput[t](135 mm,-3mm){\centering 15}
          \rput[t](162 mm,-3mm){\centering 16}
          \rput[t](126.1 mm,4mm){\centering $A$}
          \rput[t](33.1 mm,4mm){\centering $B$}
          \rput[t](54.1 mm,4mm){\centering $C$}
          \rput[t](81.1 mm,4mm){\centering $D$}
          \rput[t](162.1 mm,4mm){\centering $E$}
          \rput[t](18.1 mm,4mm){\centering $F$}
          \rput[t](27.1 mm,4mm){\centering $G$}
          }
      \end{pspicture}

      \exercice*
      Effectuer sans calculatrice :
      \begin{multicols}{4}\noindent
        \begin{enumerate}
        \item $19 - 9 = \mathbf{10}$
        \item $8 + \mathbf{5} = 13$
        \item $\mathbf{6} \times 7 = 42$
        \item $\mathbf{8} - 4 = 4$
        \item $\mathbf{5} \times 6 = 30$
        \item $\mathbf{40} \div 5 = 8$
        \item $1 + \mathbf{8} = 9$
        \item $25 \div 5 = \mathbf{5}$
        \item $5 \div \mathbf{1} = 5$
        \item $24 \div 4 = \mathbf{6}$
        \item $11 - 8 = \mathbf{3}$
        \item $11 - \mathbf{5} = 6$
        \item $\mathbf{16} \div 4 = 4$
        \item $1 + \mathbf{3} = 4$
        \item $11 - 9 = \mathbf{2}$
        \item $\mathbf{9} + 4 = 13$
        \item $\mathbf{5} + 7 = 12$
        \item $4 \times \mathbf{10} = 40$
        \item $\mathbf{1} \times 9 = 9$
        \item $6 \times \mathbf{10} = 60$
        \end{enumerate}
      \end{multicols}

      \exercice*
      Poser et effectuer les opérations suivantes.
      \begin{multicols}{2}\noindent
        \begin{enumerate}
        \item La différence des termes 30\,371 et 20,97.\par
          \begin{tabular}[t]{*{9}{c}}
            & 3 & 0 & 3 & 7 & 1 & , & $_1$0 & $_1$0 \\
            - &  &  &  & 2 & 0$_1$ & , & 9$_1$ & 7 \\
            \hline
            & 3 & 0 & 3 & 5 & 0 & , & 0 & 3 \\
          \end{tabular}\par
          \[ \boxed{30\,371-20{,}97 = 30\,350{,}03} \]
        \item Le produit des facteurs 96,7 et 13,4.\par
          \begin{enumerate}
          \item Première méthode :\par
            \begin{tabular}[t]{*{6}{c}}
              &  & 9 & 6 & , & 7 \\
              $\times$ &  & 1 & 3 & , & 4 \\
              \hline
              &  & 3 & 8 & 6 & 8 \\
              & 2 & 9 & 0 & 1 & 0 \\
              & 9 & 6 & 7 & 0 & 0 \\
              \hline \\
              1 & 2 & 9 & 5\Huge , & 7 & 8 \\
            \end{tabular}
          \item Seconde méthode :\par
            \begin{tabular}[t]{*{6}{c}}
              &  & 1 & 3 & , & 4 \\
              $\times$ &  & 9 & 6 & , & 7 \\
              \hline
              &  &  & 9 & 3 & 8 \\
              &  & 8 & 0 & 4 & 0 \\
              1 & 2 & 0 & 6 & 0 & 0 \\
              \hline \\
              1 & 2 & 9 & 5\Huge , & 7 & 8 \\
            \end{tabular}
          \end{enumerate}
          \[ \boxed{96{,}7\times13{,}4 = 1\,295{,}78} \]
        \item La somme des termes 1\,126 et 501,6.\par
          \begin{tabular}[t]{*{7}{c}}
            & \tiny  & \tiny  & \tiny  & \tiny  & \tiny  & \tiny  \\
            & 1 & 1 & 2 & 6 & , & 0 \\
            + &  & 5 & 0 & 1 & , & 6 \\
            \hline
            & 1 & 6 & 2 & 7 & , & 6 \\
          \end{tabular}\par
          \[ \boxed{1\,126+501{,}6 = 1\,627{,}6} \]
        \end{enumerate}
      \end{multicols}

      \exercice*
      Compléter sans calculatrice :
      \begin{multicols}{2}\noindent
        \begin{enumerate}
        \item $10 \times 11{,}1 = \mathbf{111}$
        \item $0{,}454 \div 1\,000 = \mathbf{0{,}000\,454}$
        \item $100 \times \mathbf{8{,}68} = 868$
        \item $\mathbf{0{,}1} \times 47{,}7 = 4{,}77$
        \item $0{,}000\,1 \times \mathbf{0{,}757} = 0{,}000\,075\,7$
        \item $\mathbf{0{,}01} \times 0{,}329 = 0{,}003\,29$
        \item $18{,}5 \div \mathbf{10\,000} = 0{,}001\,85$
        \item $0{,}148 \div 100 = \mathbf{0{,}001\,48}$
        \item $18{,}2 \div 10 = \mathbf{1{,}82}$
        \item $0{,}001 \times \mathbf{9{,}48} = 0{,}009\,48$
        \item $1\,000 \times 6{,}59 = \mathbf{6\,590}$
        \item $10\,000 \times 5{,}46 = \mathbf{54\,600}$
        \end{enumerate}
      \end{multicols}

      \exercice*
      Cocher les bonnes réponses :\par
      \begin{tabular}{c@{ est divisible : \kern1cm}r@{ par 2\kern1cm}r@{ par
            3\kern1cm}r@{ par 5\kern1cm}r@{ par 9\kern1cm}r@{ par 10}}
        490 & $\CheckedBox$ & $\Square$ & $\CheckedBox$ & $\Square$ &
        $\CheckedBox$ \\
        666 & $\CheckedBox$ & $\CheckedBox$ & $\Square$ & $\CheckedBox$ &
        $\Square$ \\
        168 & $\CheckedBox$ & $\CheckedBox$ & $\Square$ & $\Square$ & $\Square$
        \\
        370 & $\CheckedBox$ & $\Square$ & $\CheckedBox$ & $\Square$ &
        $\CheckedBox$ \\
        36 & $\CheckedBox$ & $\CheckedBox$ & $\Square$ & $\CheckedBox$ &
        $\Square$ \\
      \end{tabular}

      \exercice*
      Construire la symétrique de chacune des figures par rapport à la droite en
      utilisant le quadrillage :\par
      \psset{unit=.9cm}
      \begin{pspicture*}(-3,-3)(3,3)
        \psgrid[subgriddiv=2,gridlabels=0pt]
        \pstGeonode[PointSymbol=x,PointName=none](0.5,1.0){a}(-0.5,1.5){b}(-1.5,-1.0){c}(0.5,-1.5){d}(2.0,-0.5){e}
        \pstGeonode[PointSymbol=none,PointName=none](-4.5;0){A}(4.5;0){B}
        \psline[linecolor=Black, linewidth=1pt, nodesep=-4.5](A)(B)
        \pspolygon[linecolor=Maroon, linewidth=1pt](a)(b)(c)(d)(e)
        \pstOrtSym[PointSymbol=x,PointName=none]{A}{B}{a,b,c,d,e}[a1,b1,c1,d1,e1]
        \pspolygon[linecolor=Black, linestyle=dashed,
        linewidth=1pt](a1)(b1)(c1)(d1)(e1)
      \end{pspicture*}
      \hfill
      \begin{pspicture*}(-3,-3)(3,3)
        \psgrid[subgriddiv=2,gridlabels=0pt]
        \pstGeonode[PointSymbol=x,PointName=none](2.5,0.5){a}(-1.0,2.5){b}(-2.5,1.0){c}(-2.5,-2.5){d}(2.0,-2.0){e}
        \pstGeonode[PointSymbol=none,PointName=none](-4.5;90){A}(4.5;90){B}
        \psline[linecolor=Black, linewidth=1pt, nodesep=-4.5](A)(B)
        \pspolygon[linecolor=Maroon, linewidth=1pt](a)(b)(c)(d)(e)
        \pstOrtSym[PointSymbol=x,PointName=none]{A}{B}{a,b,c,d,e}[a1,b1,c1,d1,e1]
        \pspolygon[linecolor=Black, linestyle=dashed,
        linewidth=1pt](a1)(b1)(c1)(d1)(e1)
      \end{pspicture*}
      \hfill
      \begin{pspicture*}(-3,-3)(3,3)
        \psgrid[subgriddiv=2,gridlabels=0pt]
        \pstGeonode[PointSymbol=x,PointName=none](2.0,2.0){a}(-0.5,3.0){b}(-1.5,-1.0){c}(-2.0,-3.0){d}(1.0,-2.0){e}
        \pstGeonode[PointSymbol=none,PointName=none](-4.5;45){A}(4.5;45){B}
        \psline[linecolor=Black, linewidth=1pt, nodesep=-4.5](A)(B)
        \pspolygon[linecolor=Maroon, linewidth=1pt](a)(b)(c)(d)(e)
        \pstOrtSym[PointSymbol=x,PointName=none]{A}{B}{a,b,c,d,e}[a1,b1,c1,d1,e1]
        \pspolygon[linecolor=Black, linestyle=dashed,
        linewidth=1pt](a1)(b1)(c1)(d1)(e1)
      \end{pspicture*}
      \label{LastCorPage}
    \end{document}