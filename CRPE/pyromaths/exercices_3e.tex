%%!TEX TS-program = latex
\documentclass[a4paper,11pt]{article}
\usepackage[utf8]{inputenc} % UTF-8
\usepackage[T1]{fontenc}
\usepackage{lmodern} % Prévient un bug d'affichage evince lié à [T1]{fontenc}
\usepackage[frenchb]{babel} % francisation
\usepackage{adjustbox}
\usepackage[fleqn]{amsmath} % aligne le mode maths à gauche
\usepackage{amssymb} % the amsfont symbols
\usepackage[table, usenames, svgnames]{xcolor} % Couleurs
\usepackage{multicol} % Multi-colonnes
\usepackage{fancyhdr} % Mise en page, en-tête et pied de page
\usepackage{calc} % Opérations
\usepackage{marvosym} % Martin Vogels Symbole (\EUR)
\usepackage{cancel} % draw diagonal lines
\usepackage{units} % typesetting units and nice fractions
\usepackage[autolanguage]{numprint} % écrituredes virgules
\usepackage{tabularx} % creates a paragraph-like column whose width
% automatically expands
\usepackage{wrapfig} % allows figures or tables to have text wrapped around
\usepackage{pst-eucl, pst-plot} % figures géométriques
\usepackage{enumitem}
\usepackage{wasysym} % Symbole Euro
\usepackage{mathtools} % Encadrement dans align*
\usepackage[inline]{asymptote}
\usepackage{tkz-tab}
\usetikzlibrary{external} % set up externalization
\tikzexternalize[shell escape=-enable-write18] % activate externalisation
\tikzset{%
  external/system call={%
    latex \tikzexternalcheckshellescape -halt-on-error
    -interaction=batchmode -jobname "\image" "\texsource" &&
    dvips -o "\image".ps "\image".dvi &&
    ps2eps -f "\image.ps"
    }
  }

%\usepackage{textcomp}

\usepackage[a4paper, dvips, left=1.5cm, right=1.5cm, top=2cm,%
bottom=2cm, marginpar=5mm, marginparsep=5pt]{geometry}
\newcounter{exo}
\frenchbsetup{StandardItemLabels} % remet \textbullet pour les listes
\setlength{\headheight}{18pt}
\setlength{\fboxsep}{1em}
\setlength\parindent{0em}
\setlength\mathindent{0em}
\setlength{\columnsep}{30pt}
\usepackage[bookmarks=true, bookmarksnumbered=true, ps2pdf, pagebackref=true,%
colorlinks=true,linkcolor=blue,plainpages=true,unicode]{hyperref}
\hypersetup{pdfauthor={Jérôme Ortais},pdfsubject={Exercices de
    mathématiques},pdftitle={Exercices créés par Pyromaths, un logiciel libre
    en Python sous licence GPL}}

\def\pshlabel#1{\psframebox*[fillcolor=White,framearc=.2]{\footnotesize $#1$}}
\def\psvlabel#1{\psframebox*[fillcolor=White,framearc=.2]{\footnotesize $#1$}}

\makeatletter
\newcommand\styleexo[1][]{
  \renewcommand{\theenumi}{\arabic{enumi}}
  \renewcommand{\labelenumi}{$\blacktriangleright$\textbf{\theenumi.}}
  \renewcommand{\theenumii}{\alph{enumii}}
  \renewcommand{\labelenumii}{\textbf{\theenumii)}}
  {\fontfamily{pag}\fontseries{b}\selectfont \underline{#1 \theexo}}
  \par\@afterheading\vspace{0.5\baselineskip minus 0.2\baselineskip}}
\newcommand*\exercice{%
  \psset{unit=1cm, dash=4pt 4pt, PointName=default,linecolor=Maroon,
    dotstyle=x, linestyle=solid, hatchcolor=Peru, gridcolor=Olive,
    subgridcolor=Olive, fillcolor=Peru}
  %\ifthenelse{\equal{\theexo}{0}}{}{\filbreak}
  \refstepcounter{exo}%
  \stepcounter{nocalcul}%
  \par\addvspace{1.5\baselineskip minus 1\baselineskip}%
  \@ifstar%
  {\penalty-130\styleexo[Corrigé de l'exercice]}%
  {\penalty-130\styleexo[Exercice]}%
  }
\makeatother
\newlength{\ltxt}
\newcounter{fig}
\newcommand{\figureadroite}[2]{
  \setlength{\ltxt}{\linewidth}
  \setbox\thefig=\hbox{#1}
  \addtolength{\ltxt}{-\wd\thefig}
  \addtolength{\ltxt}{-10pt}
  \begin{minipage}{\ltxt}
    #2
  \end{minipage}
  \hfill
  \begin{minipage}{\wd\thefig}
    #1
  \end{minipage}
  \refstepcounter{fig}
  }
\count1=\year \count2=\year
\ifnum\month<8\advance\count1by-1\else\advance\count2by1\fi
\pagestyle{fancy}
\cfoot{\textsl{\footnotesize{Année \number\count1/\number\count2}}}
\rfoot{\textsl{\tiny{http://www.pyromaths.org}}}
\lhead{\textsl{\footnotesize{Page \thepage/ \pageref{LastPage}}}}
\chead{\Large{\textsc{Révisions}}}
\rhead{\textsl{\footnotesize{Classe de 3\ieme}}}

\begin{document}
  \currentpdfbookmark{Les énoncés des exercices}{Énoncés}
  \newcounter{nocalcul}[exo]
  \renewcommand{\thenocalcul}{\Alph{nocalcul}}
  \raggedcolumns
  \setlength{\columnseprule}{0.5pt}

  \exercice
  \parbox{0.5\linewidth}{
    ($d_1$) est la droite représentative de la fonction $f$.
    \begin{enumerate}
    \item Donner  l'image de $3$ par la fonction \textit{f}.
    \item Donner  un antécédent de $0,5$ par la fonction \textit{f}.
    \item Tracer la droite représentative ($d_2$) de la fonction
      $g:x\longmapsto -x-1$.
    \item Déterminer l'expression de la fonction $h$ représentée ci-contre par
      la droite ($d_3$).
    \end{enumerate}}\hfill
  \parbox{0.45\linewidth}{
    \psset{unit=0.8cm}
    \begin{pspicture}(-5, -5)(5, 5)
      \psgrid[subgriddiv=2, gridlabels=8pt](0,0)(-5, -5)(5, 5)
      \psline[linewidth=1.2pt]{->}(-5, 0)(5, 0)
      \psline[linewidth=1.2pt]{->}(0, -5)(0, 5)
      \psline (-5, 0.2142857142857144)(5, 3.071428571428571)
      \rput(-5.5, 0.7142857142857144){($d_1$)}
      \psline (-4.5, -5)(0.5, 5)
      \rput(-5.0, -5.5){($d_3$)}
    \end{pspicture}}

  \exercice
  \begin{enumerate}
  \item Donner la décomposition en facteurs premiers des nombres suivants, et
    préciser quand il s'agit d'un nombre premier :\par
    262 ; 867 ; 1\,152 ; 139 ; 1\,512 ;
  \item En déduire le PGCD et le PPCM des nombres 1\,512 et 1\,152.
  \item Quel est le plus petit nombre par lequel il faut multiplier 262 pour
    obtenir un carré parfait ?
  \item Rendre la fraction $\dfrac{1\,512}{1\,152}$ irréductible.
  \item Calculer $\dfrac{27}{1\,512} + \dfrac{10}{1\,152}$.
  \end{enumerate}

  \exercice
  Développer chacune des expressions littérales suivantes :
  \begin{multicols}{2}
    $A=\left( 10\,x-5\right) \times \left( 5\,x+10\right) $\\
    $B=\left( 7\,x+6\right) \times \left( 7\,x-6\right) $\\
    $C=\left( 8\,x+5\right) ^{2}$\\
    $D=\left( 4\,x-6\right) ^{2}$\\
    $E=\left( \dfrac{4}{7}\,x+\dfrac{4}{7}\right) \times \left(
    \dfrac{4}{7}\,x-\dfrac{4}{7}\right) $\\
    $F=-\left( 6\,x-6\right) ^{2}$
  \end{multicols}

  \exercice
  Résoudre l'équation :
  \[ \cfrac{3\,x+3}{6}-\cfrac{-6\,x+6}{3}=\cfrac{-6\,x+5}{8}\]

  \exercice
  Factoriser chacune des expressions littérales suivantes :
  \begin{multicols}{2}
    $A=-100\,x^{2}+100$\\
    $B=\left( 5\,x+2\right) \times \left( 6\,x+6\right) +\left( 6\,x+10\right)
    \times \left( 6\,x+6\right) $\\
    $C=\left( 5\,x-9\right) ^{2}-4\,x^{2}$\\
    $D=9\,x^{2}-54\,x+81$\\
    $E=\left( -2\,x+10\right) ^{2}+\left( -2\,x+10\right) \times \left(
    -5\,x+2\right) $\\
    $F=\left( 2\,x-6\right) \times \left( 9\,x+3\right) -\left( 9\,x+3\right) $
  \end{multicols}


  \exercice
  Calculer les expressions suivantes et donner le résultat sous la forme d'une
  fraction irréductible.
  \begin{multicols}{3}\noindent
    \[ \thenocalcul = \cfrac{6}{5}\div\left(\cfrac{-13}{9}-\cfrac{9}{11}\right)
      \]
    \columnbreak\stepcounter{nocalcul}
    \[ \thenocalcul = \cfrac{-80}{7}+\cfrac{5}{7}\times\cfrac{7}{15}\]
    \columnbreak\stepcounter{nocalcul}
    \[ \thenocalcul = \cfrac{\cfrac{-1}{3}-3}{\cfrac{7}{10}-10} \]
  \end{multicols}

  \exercice
  {\begin{wrapfigure}{r}{4cm}
      \psset{PointSymbol=none,unit=0.19}
      \begin{pspicture}(-1.5,-1.5)(19.1,4.06)
        \SpecialCoor
        \pstTriangle[PosAngleA=225,PosAngleB=-45,PosAngleC=93,PointNameA=U,PointNameB=H,PointNameC=F](0,0){a}(17.6,0){b}(4.8;48){c}
        \pstTriangle[PosAngleB=-45,PosAngleC=138,PointSymbolA=none,PointNameA=none,PointNameB=Z,PointNameC=W](0,0){a}(6.6,0){b}(1.8;48){c}
      \end{pspicture}
    \end{wrapfigure}\par

    Sur la figure ci-contre, on donne $UW=\unit[1{,}8]{cm}$, $WF=\unit[3]{cm}$,
    $UH=\unit[17{,}6]{cm}$ et $UZ=\unit[6{,}6]{cm}$.\par
    Démontrer que les droites $(HF)$ et $(ZW)$ sont parallèles.
    \vspace{2cm}}

  \exercice
  \begin{multicols}{2}
    Sur la figure ci-dessous, les droites $(VB)\text{ et }(YH)$ sont
    parallèles.\par
    On donne $ZY=\unit[6{,}8]{cm},\quad ZH=\unit[2{,}7]{cm}, \quad
    YH=\unit[6]{cm}\quad\text{et}\quad YV~=~\unit[4{,}5]{cm}$.\par
    Calculer $ZB$ et $VB$, arrondies au centième.
    \begin{center}
      \psset{PointSymbol=none,unit=0.27}
      \begin{pspicture}(-1.5,-1.5)(12.8,4.43)
        \SpecialCoor
        \pstTriangle[PosAngleA=225, PosAngleB=-45, PosAngleC=106.48,
        PointNameA=Z, PointNameB=V,
        PointNameC=B](0,0){a}(11.3,0){b}(4.48;61.48){c}
        \pstTriangle[PosAngleB=-45, PosAngleC=151.48, PointSymbolA=none,
        PointNameA=none, PointNameB=Y,
        PointNameC=H](0,0){a}(6.8,0){b}(2.7;61.48){c}
      \end{pspicture}
    \end{center}

    \columnbreak
    Sur la figure ci-dessous, les droites $(WS)\text{ et }(OX)$ sont
    parallèles.\par
    On donne $FW=\unit[2{,}7]{cm},\quad FS=\unit[4{,}9]{cm}, \quad
    WS=\unit[3{,}7]{cm}\quad\text{et}\quad OX~=~\unit[2{,}3]{cm}$.\par
    Calculer $FO$ et $FX$, arrondies au millième.
    \begin{center}
      \psset{PointSymbol=none,unit=0.48}
      \begin{pspicture}(-3.52,-3.76)(4.76,4.15)
        \SpecialCoor
        \pstTriangle[PosAngleA=135, PosAngleB=-45, PosAngleC=93.27,
        PointNameA=F, PointNameB=W,
        PointNameC=S](0,0){a}(2.7,0){b}(4.9;48.27){c}
        \pstTriangle[PosAngleB=135, PosAngleC=228.27, PointSymbolA=none,
        PointNameA=none, PointNameB=O,
        PointNameC=X](0,0){a}(-1.67,0){b}(-3.04;48.27){c}
      \end{pspicture}
    \end{center}

  \end{multicols}

  \exercice
  \begin{multicols}{2}
    \begin{enumerate}
    \item $QLZ$ est un triangle rectangle en $L$ tel que :\par
      $LZ=\unit[7{,}6]{cm}$ et $QZ=\unit[11{,}1]{cm}$.\par
      Calculer la mesure de l'angle $\widehat{LQZ}$, arrondie au centième.\par
      \columnbreak
    \item $TCJ$ est un triangle rectangle en $T$ tel que :\par
      $TC=\unit[7{,}5]{cm}$ et $\widehat{TCJ}=49\degres$.\par
      Calculer la longueur $CJ$, arrondie au centième.\par
    \end{enumerate}
  \end{multicols}

  \exercice
  \begin{multicols}{2}
    \begin{enumerate}
    \item On donne
      $\begin{array}[t]{l}
        f:~x \longmapsto -9\,x^{2}-8\,x-3 \\ g:~x \longmapsto x-1
      \end{array}$
      \begin{enumerate}
      \item Quelle est l'image de $-3$ par la fonction $f$ ?
      \item Quelle est l'image de $3$ par la fonction $g$ ?
      \item Calculer $f\,(4)$.
      \item Calculer $g\,(-3)$.
      \end{enumerate}
    \item Voici un tableau de valeurs correspondant à une fonction $h$.\par
      \renewcommand{\arraystretch}{1.5}
      \begin{tabularx}{\linewidth}[t]{|c|*7{>{\centering}X|}}
        \hline
        $x$ & $-4$ & $-3$ & $-2$ & $-1$ & $0$ & $1$ & $2$ \tabularnewline \hline
        $h\,(x)$ & $2$ & $-4$ & $-1$ & $0$ & $-2$ & $-3$ & $1$ \tabularnewline
        \hline
      \end{tabularx} \medskip
      \begin{enumerate}
      \item Quelle est l'image de $1$ par la fonction $h$ ?
      \item Compléter : $h\,(2)=\ldots\ldots$
      \item Compléter : $h\,(\ldots\ldots)=-2$
      \item Quel est l'antécédent de $-1$ par la fonction $h$ ?
      \end{enumerate}
      \columnbreak
    \item Le graphique ci-dessous représente une fonction $k$ : \par
      \begin{center}
        \psset{unit=8mm, algebraic, dotsize=4pt 4}
        \begin{pspicture*}(-4.2,-4.2)(4.2,4.2)
          \psgrid[subgriddiv=2, gridwidth=.6pt,subgridcolor=lightgray,
          gridlabels=0pt]
          \psaxes[linewidth=1.2pt,]{->}(0,0)(-4.2,-4.2)(4.2,4.2)
          \psplot[plotpoints=200, linewidth=1.5pt,
          linecolor=DarkRed]{-4.2}{4.2}{2/15*x^4+13/30*x^3-2/15*x^2+17/30*x^1-1}
          \psdots (-4, 1) (-2, -4) (-1, -2) (0, -1) (1, 0)
        \end{pspicture*}
      \end{center}
      \begin{enumerate}
      \item Compléter : $k\,(\ldots\ldots)=1$
      \item Donner un antécédent de 0 par la fonction $k$.
      \item Compléter : $k\,(-2)=\ldots\ldots$
      \item Quelle est l'image de $0$ par la fonction $k$ ?
      \end{enumerate}
    \end{enumerate}
  \end{multicols}

  \exercice
  \begin{enumerate}
  \item Les nombres \nombre{19034} et \nombre{3100} sont-ils premiers entre eux
    ?
  \item Calculer le plus grand commun diviseur (\textsc{pgcd}) de
    \nombre{19034} et \nombre{3100}.
  \item Simplifier la fraction $\cfrac{\nombre{19034}}{\nombre{3100}}$ pour la
    rendre irréductible en indiquant la méthode.

  \end{enumerate}

  \exercice
  Dans une urne, il y a 1 boule verte (V), 2 boules bleues (B) et 3 boules
  rouges (R), indiscernables au toucher. On tire successivement et sans remise
  deux boules.
  \begin{enumerate}
  \item Quelle est la probabilité de tirer une boule bleue au premier tirage?
  \item Construire un arbre des probabilités décrivant l'expérience aléatoire.
  \item Quelle est la probabilité que la première boule soit rouge et la
    deuxième soit bleue?
  \item Quelle est la probabilité que la deuxième boule soit verte ?
  \end{enumerate}

  \exercice
  Calculer les expressions suivantes et donner l'écriture scientifique du
  résultat.
  \begin{multicols}{2}\noindent
    \[ \thenocalcul = \cfrac{\nombre{3,2} \times 10^{-4} \times \nombre{30}
        \times 10^{3}}{\nombre{24} \times \big( 10^{-2} \big) ^2}\]
    \columnbreak\stepcounter{nocalcul}%
    \[ \thenocalcul = \cfrac{\nombre{10} \times 10^{-5} \times \nombre{810}
        \times 10^{-7}}{\nombre{0,9} \times \big( 10^{9} \big) ^3}\]
  \end{multicols}

  \exercice
  \begin{enumerate}
  \item Calculer les expressions suivantes et donner le résultat sous la forme
    $a\,\sqrt{b}$ avec $a$ et $b$ entiers, $b$ le plus petit possible.
    \begin{multicols}{2}\noindent
      \[ \thenocalcul = 2\,\sqrt{28}+4\,\sqrt{112}+3\,\sqrt{63}\]
      \columnbreak\stepcounter{nocalcul}%
      \[ \thenocalcul = \sqrt{80}\times\sqrt{45}\times\sqrt{20}\]
    \end{multicols}\vspace{-3ex}
  \item Calculer les expressions suivantes et donner le résultat sous la forme
    $a+b\,\sqrt{c}$ avec $a$, $b$ et $c$ entiers.
    \stepcounter{nocalcul}%
    \begin{multicols}{2}\noindent
      \[ \thenocalcul = \left( 4\,\sqrt{10}-2\,\sqrt{6} \right)^2\]
      \columnbreak\stepcounter{nocalcul}%
      \[ \thenocalcul = \left( 2\,\sqrt{5}+4\,\sqrt{3} \right)^2\]
    \end{multicols}\vspace{-3ex}
  \item Calculer les expressions suivantes et donner le résultat sous la forme
    d'un nombre entier.

    \stepcounter{nocalcul}%
    \begin{multicols}{2}\noindent
      \[ \thenocalcul = \left( 3-3\,\sqrt{3} \right)\left( 3+3\,\sqrt{3}
        \right)\]
      \columnbreak\stepcounter{nocalcul}%
      \[ \thenocalcul = \frac{32\,\sqrt{45}}{12\,\sqrt{80}}\]
    \end{multicols}\vspace{-3ex}
  \end{enumerate}

  \exercice
  Résoudre le système d'équations suivant :
  $\left\lbrace
  \begin{array}{rcrcl}
    7\,x & + & 4\,y & = & 17 \\
    4\,x & + & 8\,y & = & 4
  \end{array}
  \right.$
  \label{LastPage}
  \newpage
  \currentpdfbookmark{Le corrigé des exercices}{Corrigé}
  \lhead{\textsl{{\footnotesize Page \thepage/ \pageref{LastCorPage}}}}
  \setcounter{page}{1} \setcounter{exo}{0}

  \exercice*
  \setlength{\columnsep}{2mm}
  \begin{multicols}{2}\noindent \small
    ($d_1$) est la droite représentative de la fonction $f$.
    \begin{enumerate}
    \item $2,5$ est l'image de $3$ par la \hbox{fonction \textit{f}}.
    \item $-4$ est un antécédent de $0,5$ par la \hbox{fonction \textit{f}}.
    \item
      \begin{flushleft}
        On sait que $g(0)=-1$ et $g(3)=-3-1=-4$.
      \end{flushleft}
    \item
      \begin{flushleft}
        On lit l'ordonnée à l'origine et le coefficient de la fonction affine
        sur le graphique.\\
        $h(x)=a\,x+b$ avec $b=4$ et $a=\dfrac{+4}{+2}=2$.\\
        L'expression de la fonction $h$ est $h(x)=2\,x+4$.
      \end{flushleft}
    \end{enumerate}
  \end{multicols}
  \vspace{0.45cm}
  \begin{minipage}{0.5\linewidth}
    \psset{unit=0.7cm}
    \begin{center}
      \begin{pspicture}(-5, -5)(5, 5)
        \psgrid[subgriddiv=2, gridlabels=8pt](0,0)(-5, -5)(5, 5)
        \psline[linewidth=1.2pt]{->}(-5, 0)(5, 0)
        \psline[linewidth=1.2pt]{->}(0, -5)(0, 5)
        \psline (-5, 0.2142857142857144)(5, 3.071428571428571)
        \rput(-5.5, 0.7142857142857144){($d_1$)}
        \psline (4.0, -5)(-5, 4.0)
        \rput(4.5, -5.5){($d_2$)}
        \psline[linestyle=dashed,linewidth=1.1pt]{->}(3.0, 0)(3.0, 1.25)
        \psline[linestyle=dashed,linewidth=1.1pt]{->}(3.0, 1.25)(3.0, 2.5)
        \psline[linestyle=dashed,linewidth=1.1pt]{->}(3.0, 2.5)(1.5, 2.5)
        \psline[linestyle=dashed,linewidth=1.1pt]{->}(1.5, 2.5)(0, 2.5)
        \psline[linestyle=dashed,linewidth=1.1pt]{->}(0, 0.5)(-2.0, 0.5)
        \psline[linestyle=dashed,linewidth=1.1pt]{->}(-2.0, 0.5)(-4.0, 0.5)
        \psline[linestyle=dashed,linewidth=1.1pt]{->}(-4.0, 0.5)(-4.0, 0.25)
        \psline[linestyle=dashed,linewidth=1.1pt]{->}(-4.0, 0.25)(-4.0, 0)
        \psdot [dotsize=4.5pt,dotstyle=x](0, -1.0)
        \psdot [dotsize=4.5pt,dotstyle=x](3.0, -4.0)
      \end{pspicture}
    \end{center}
  \end{minipage}
  \begin{minipage}{0.5\linewidth}
    \psset{unit=0.7cm}
    \begin{center}
      \begin{pspicture}(-5, -5)(5, 5)
        \psgrid[subgriddiv=2, gridlabels=8pt](0,0)(-5, -5)(5, 5)
        \psline[linewidth=1.2pt]{->}(-5, 0)(5, 0)
        \psline[linewidth=1.2pt]{->}(0, -5)(0, 5)
        \psline (-4.5, -5)(0.5, 5)
        \rput(-5.0, -5.5){($d_3$)}
        \psline[linestyle=dashed,linewidth=1.1pt]{->}(-2.0, 0.0)(-2.0, 2.0)
        \psline[linestyle=dashed,linewidth=1.1pt]{->}(-2.0, 2.0)(-2.0, 4.0)
        \psline[linestyle=dashed,linewidth=1.1pt]{->}(-2.0, 4.0)(-1.0, 4.0)
        \psline[linestyle=dashed,linewidth=1.1pt]{->}(-1.0, 4.0)(0, 4.0)
        \rput(-2.75, 2.0){(+4)}
        \rput(-1.0, 4.5){(+2)}
      \end{pspicture}
    \end{center}
  \end{minipage}
  \vspace{0.45cm}

  \exercice*
  \begin{enumerate}
  \item Donner la décomposition en facteurs premiers des nombres suivants, et
    préciser quand il s'agit d'un nombre premier :\par
    \begin{multicols}{2}
      \begin{align*}
        262
        & = 2 \times 131\\
      \end{align*}
      \begin{align*}
        867
        & = 3 \times 289\\
        & = 3 \times 17 \times 17\\
      \end{align*}
      \begin{align*}
        1152
        & = 2 \times 576\\
        & = 2 \times 2 \times 288\\
        & = 2 \times 2 \times 2 \times 144\\
        & = 2 \times 2 \times 2 \times 2 \times 72\\
        & = 2 \times 2 \times 2 \times 2 \times 2 \times 36\\
        & = 2 \times 2 \times 2 \times 2 \times 2 \times 2 \times 18\\
        & = 2 \times 2 \times 2 \times 2 \times 2 \times 2 \times 2 \times 9\\
        & = 2 \times 2 \times 2 \times 2 \times 2 \times 2 \times 2 \times 3
        \times 3\\
      \end{align*}
      139 est un nombre premier.\par
      \begin{align*}
        1512
        & = 2 \times 756\\
        & = 2 \times 2 \times 378\\
        & = 2 \times 2 \times 2 \times 189\\
        & = 2 \times 2 \times 2 \times 3 \times 63\\
        & = 2 \times 2 \times 2 \times 3 \times 3 \times 21\\
        & = 2 \times 2 \times 2 \times 3 \times 3 \times 3 \times 7\\
      \end{align*}
    \end{multicols}
  \item En déduire le PGCD et le PPCM des nombres 1\,512 et 1\,152.\par
    D'après la question 1), on sait que les nombres 1\,512 et 1\,152 ont comme
    facteurs premiers communs :
    $2 , 2 , 2 , 3 , 3$.\par
    On en déduit que le PGCD des nombres 1\,512 et 1\,152 est :
    $2 \times 2 \times 2 \times 3 \times 3 = 72.$\par
    Il existe plusieurs méthodes pour calculer le PPCM de 1\,512 et de
    1\,152.\par
    En voici deux :
    \begin{enumerate}
    \item On peut simplement utiliser la formule :
      $a \times b = PGCD(a;~b) \times PPCM(a;~b)$.\par
      Donc : $PPCM(1\,512;~1\,152) = \dfrac{1\,512\times1\,152}{72} = 24\,192.$
    \item On peut aussi multiplier un nombre par les "facteurs complémentaires"
      de l'autre.
      Ces "facteurs complémentaires" sont les facteurs qui complètent le PGCD
      pour former le nombre.\par
      Comme $PGCD(1\,512;~1\,152) = 72 = 2 \times 2 \times 2 \times 3 \times
      3$, alors les "facteurs complémentaires" de $1\,512 = 2 \times 2 \times 2
      \times 3 \times 3 \times 3 \times 7$ sont : 3 , 7.
      On en déduit que $PPCM(1\,512;~1\,152) = 1\,152 \times 3 \times 7 =
      24\,192.$
    \end{enumerate}
  \item Pour obtenir un carré parfait, il faut que sa décomposition en facteurs
    premiers ne contienne que des facteurs apparaissant un nombre pair de fois.
    D'après la question 1, la décomposition en facteurs premiers de 262
    est : \par
    $262 = 2 \times 131.$\par
    Il faut donc encore multiplier ce nombre par
    les facteurs
    2 et
    131.\par
    Le nombre cherché est par conséquent 262 et le carré parfait obtenu est
    68\,644.
  \item Le moyen le plus rapide de simplifier cette fraction estde diviser le
    numérateur et le dénominateur par leur PGCD. D'après la question 2), 
    PGCD(1\,512;~1\,152) = 72, donc on obtient :\par
    $\dfrac{1\,512{\scriptstyle \div 72}}{1\,152{\scriptstyle \div 72}} =
    \dfrac{21}{16}.$
  \item Il faut mettre les fractions au même dénominateur. Grâceà la question
    2), nous avons déjà un dénominateur commun : le PPCM des nombres 1\,512 et
    1\,152, qui est par définition le plus petitmultiple commun de ces deux
    nombres.\par
    $\dfrac{27{\scriptstyle \times 16}}{1\,512{\scriptstyle \times 16}} +
    \dfrac{10{\scriptstyle \times 21}}{1\,152{\scriptstyle \times 21}} =
    \dfrac{432}{24\,192} + \dfrac{210}{24\,192} = \dfrac{642{\scriptstyle \div
        6}}{24\,192{\scriptstyle \div 6}} = \dfrac{107}{4\,032}.$
  \end{enumerate}

  \exercice*
  Développer chacune des expressions littérales suivantes :
  \begin{multicols}{2}
    $A=\left( 10\,x-5\right) \times \left( 5\,x+10\right) $\\
    $A=10\,x\times 5\,x+10\,x\times 10-5\times 5\,x-5\times 10$\\
    $A=50\,x^{2}+100\,x-25\,x-50$\\
    $A=50\,x^{2}+\left( 100-25\right) \,x-50$
    \\
    \fbox{$A=50\,x^{2}+75\,x-50$}\\

    $B=\left( 7\,x+6\right) \times \left( 7\,x-6\right) $\\
    $B=\left( 7\,x\right) ^{2}-6^{2}$
    \\
    \fbox{$B=49\,x^{2}-36$}\\

    $C=\left( 8\,x+5\right) ^{2}$\\
    $C=\left( 8\,x\right) ^{2}+2\times 8\,x\times 5+5^{2}$
    \\
    \fbox{$C=64\,x^{2}+80\,x+25$}\\

    $D=\left( 4\,x-6\right) ^{2}$\\
    $D=\left( 4\,x\right) ^{2}-2\times 4\,x\times 6+6^{2}$
    \\
    \fbox{$D=16\,x^{2}-48\,x+36$}\\

    $E=\left( \dfrac{4}{7}\,x+\dfrac{4}{7}\right) \times \left(
    \dfrac{4}{7}\,x-\dfrac{4}{7}\right) $\\
    $E=\left( \dfrac{4}{7}\,x\right) ^{2}-\left( \dfrac{4}{7}\right) ^{2}$
    \\
    \fbox{$E=\dfrac{16}{49}\,x^{2}-\dfrac{16}{49}$}\\

    $F=-\left( 6\,x-6\right) ^{2}$\\
    $F=-\left( \left( 6\,x\right) ^{2}-2\times 6\,x\times 6+6^{2}\right) $\\
    $F=-\left( 36\,x^{2}-72\,x+36\right) $
    \\
    \fbox{$F=-36\,x^{2}+72\,x-36$}\\

  \end{multicols}

  \exercice*
  Résoudre l'équation :
  \[\cfrac{3\,x+3}{6}-\cfrac{-6\,x+6}{3}=\cfrac{-6\,x+5}{8}\]
  \[\cfrac{(3\,x+3)_{\times4}}{6_{\times4}}-\cfrac{(-6\,x+6)_{\times8}}{3_{\times8}}=\cfrac{(-6\,x+5)_{\times3}}{8_{\times3}}\]
  \[\cfrac{12\,x+12-\left( -48\,x+48\right)
      }{\cancel{24}}=\cfrac{-18\,x+15}{\cancel{24}}\]
  \[ 12\,x+1248\,x-48=-18\,x+15\]
  \[60\,x-36=-18\,x+15\]
  \[60\,x+18\,x=15+36\]
  \[78\,x=51\]
  \[x=\cfrac{51}{78}=\cfrac{17}{26}\]
  \fbox{La solution de cette équation est $\cfrac{17}{26}$\,.}

  \exercice*
  Factoriser chacune des expressions littérales suivantes :
  \begin{multicols}{2}
    $A=-100\,x^{2}+100$\\
    $A=\sqrt{100}^{2}-\left( \sqrt{100}\,x\right) ^{2}$\\
    $A=\left( \sqrt{100}+\sqrt{100}\,x\right) \times \left(
    \sqrt{100}-\sqrt{100}\,x\right) $\\
    $A=\left( \sqrt{100}\,x+\sqrt{100}\right) \times \left( 10-10\,x\right) $\\
    $A=\left( \sqrt{100}\,x+\sqrt{100}\right) \times \left( -10\,x+10\right) $
    \\
    \fbox{$A=\left( 10\,x+10\right) \times \left( -10\,x+10\right) $}\\

    $B=\left( 5\,x+2\right) \times \left( 6\,x+6\right) +\left( 6\,x+10\right)
    \times \left( 6\,x+6\right) $\\
    $B=\left( 6\,x+6\right) \times \left( 5\,x+2+6\,x+10\right) $\\
    $B=\left( 6\,x+6\right) \times \left( 5\,x+6\,x+2+10\right) $
    \\
    \fbox{$B=\left( 6\,x+6\right) \times \left( 11\,x+12\right) $}\\

    $C=\left( 5\,x-9\right) ^{2}-4\,x^{2}$\\
    $C=\left( 5\,x-9\right) ^{2}-\left( 2\,x\right) ^{2}$\\
    $C=\left( 5\,x-9+2\,x\right) \times \left( 5\,x-9-2\,x\right) $\\
    $C=\left( 5\,x+2\,x-9\right) \times \left( 5\,x-2\,x-9\right) $
    \\
    \fbox{$C=\left( 7\,x-9\right) \times \left( 3\,x-9\right) $}\\

    $D=9\,x^{2}-54\,x+81$\\
    $D=\left( 3\,x\right) ^{2}-2\times 3\,x\times 9+9^{2}$
    \\
    \fbox{$D=\left( 3\,x-9\right) ^{2}$}\\

    $E=\left( -2\,x+10\right) ^{2}+\left( -2\,x+10\right) \times \left(
    -5\,x+2\right) $\\
    $E=\left( -2\,x+10\right) \times \left( -2\,x+10\right) +\left(
    -2\,x+10\right) \times \left( -5\,x+2\right) $\\
    $E=\left( -2\,x+10\right) \times \left( -2\,x+10-5\,x+2\right) $\\
    $E=\left( -2\,x+10\right) \times \left( -2\,x-5\,x+10+2\right) $
    \\
    \fbox{$E=\left( -2\,x+10\right) \times \left( -7\,x+12\right) $}\\

    $F=\left( 2\,x-6\right) \times \left( 9\,x+3\right) -\left( 9\,x+3\right)
    $\\
    $F=\left( 2\,x-6\right) \times \left( 9\,x+3\right) -\left( 9\,x+3\right)
    \times 1$\\
    $F=\left( 9\,x+3\right) \times \left( 2\,x-6-1\right) $
    \\
    \fbox{$F=\left( 9\,x+3\right) \times \left( 2\,x-7\right) $}\\

  \end{multicols}

  \exercice*
  Calculer les expressions suivantes et donner le résultat sous la forme d'une
  fraction irréductible.
  \begin{multicols}{3}\noindent
    \[ \thenocalcul = \cfrac{6}{5}\div\left(\cfrac{-13}{9}-\cfrac{9}{11}\right)
      \]
    \[ \thenocalcul = \cfrac{6}{5}\div\left(\dfrac{-13_{\times 11}}{9_{\times
          11}}-\dfrac{9_{\times 9}}{11_{\times 9}}\right) \]
    \[ \thenocalcul =
      \cfrac{6}{5}\div\left(\dfrac{-143}{99}-\dfrac{81}{99}\right) \]
    \[ \thenocalcul = \cfrac{6}{5}\div\cfrac{-224}{99}\]
    \[ \thenocalcul = \cfrac{6}{5}\times\cfrac{-99}{224}\]
    \[ \thenocalcul =
      \cfrac{3\times\cancel{2}}{-5\times\bcancel{-1}}\times\cfrac{99\times\bcancel{-1}}{112\times\cancel{2}}\]
    \[ \boxed{\thenocalcul = \cfrac{-297}{560}} \]
    \columnbreak\stepcounter{nocalcul}
    \[ \thenocalcul = \cfrac{-80}{7}+\cfrac{5}{7}\times\cfrac{7}{15}\]
    \[ \thenocalcul =
      \cfrac{-80}{7}+\cfrac{1\times\cancel{5}}{1\times\bcancel{7}}\times\cfrac{1\times\bcancel{7}}{3\times\cancel{5}}\]
    \[ \thenocalcul = \cfrac{-80}{7}+\cfrac{1}{3}\]
    \[ \thenocalcul = \dfrac{-80_{\times 3}}{7_{\times 3}}+\dfrac{1_{\times
          7}}{3_{\times 7}}\]
    \[ \thenocalcul = \dfrac{-240}{21}+\dfrac{7}{21}\]
    \[ \boxed{\thenocalcul = \cfrac{-233}{21}} \]
    \columnbreak\stepcounter{nocalcul}
    \[ \thenocalcul =  \cfrac{\cfrac{-1}{3}-3}{\cfrac{7}{10}-10} \]
    \[ \thenocalcul = \cfrac{\dfrac{-1}{3}-\dfrac{3_{\times 3}}{1_{\times
            3}}}{\dfrac{7}{10}-\dfrac{10_{\times 10}}{1_{\times 10}}} \]
    \[ \thenocalcul =
      \cfrac{\dfrac{-1}{3}-\dfrac{9}{3}}{\dfrac{7}{10}-\dfrac{100}{10}} \]
    \[ \thenocalcul = \cfrac{-10}{3}\div\cfrac{-93}{10} \]
    \[ \thenocalcul = \cfrac{-10}{3}\times\cfrac{-10}{93} \]
    \[ \thenocalcul =
      \cfrac{-10}{-3\times\bcancel{-1}}\times\cfrac{10\times\bcancel{-1}}{93} \]
    \[ \boxed{\thenocalcul = \cfrac{100}{279}} \]
  \end{multicols}

  \exercice*
  {\begin{wrapfigure}{r}{4cm}
      \psset{PointSymbol=none,unit=0.19}
      \begin{pspicture}(-1.5,-1.5)(19.1,4.06)
        \SpecialCoor
        \pstTriangle[PosAngleA=225,PosAngleB=-45,PosAngleC=93,PointNameA=U,PointNameB=H,PointNameC=F](0,0){a}(17.6,0){b}(4.8;48){c}
        \pstTriangle[PosAngleB=-45,PosAngleC=138,PointSymbolA=none,PointNameA=none,PointNameB=Z,PointNameC=W](0,0){a}(6.6,0){b}(1.8;48){c}
      \end{pspicture}
    \end{wrapfigure}\par

    Sur la figure ci-contre, on donne $UZ=\unit[6{,}6]{cm}$,
    $UH=\unit[17{,}6]{cm}$, $UW=\unit[1{,}8]{cm}$ et $WF=\unit[3]{cm}$.\par
    Démontrer que les droites $(HF)$ et $(ZW)$ sont parallèles.
    \par\dotfill{}\\}

  Les points $U$, $Z$, $H$~ et $U$, $W$, $F$ sont alignés dans le même
  ordre.\par
  De plus $UF=WF+UW=\unit[4{,}8]{cm}$.\par

  $\left.
  \renewcommand{\arraystretch}{2}
  \begin{array}{l}
    \bullet\cfrac{UH}{UZ}=\cfrac{17{,}6}{6{,}6}=\cfrac{176_{\div22}}{66_{\div22}}=\cfrac{8}{3}\\
    \bullet\cfrac{UF}{UW}=\cfrac{4{,}8}{1{,}8}=\cfrac{48_{\div6}}{18_{\div6}}=\cfrac{8}{3}
  \end{array}
  \right\rbrace$
  Donc $\cfrac{UH}{UZ}=\cfrac{UF}{UW}$\,.\par
  D'après la \textbf{réciproque du théorème de Thalès}, \fbox{les droites
    $(HF)$ et $(ZW)$ sont parallèles.}


  \exercice*
  \begin{multicols}{2}
    Sur la figure ci-dessous, les droites $(VB)\text{ et }(YH)$ sont
    parallèles.\par
    On donne $ZY=\unit[6{,}8]{cm},\quad ZH=\unit[2{,}7]{cm}, \quad
    YH=\unit[6]{cm}\quad\text{et}\quad YV~=~\unit[4{,}5]{cm}$.\par
    Calculer $ZB$ et $VB$, arrondies au centième.
    \begin{center}
      \psset{PointSymbol=none,unit=0.27}
      \begin{pspicture}(-1.5,-1.5)(12.8,4.43)
        \SpecialCoor
        \pstTriangle[PosAngleA=225, PosAngleB=-45, PosAngleC=106.48,
        PointNameA=Z, PointNameB=V,
        PointNameC=B](0,0){a}(11.3,0){b}(4.48;61.48){c}
        \pstTriangle[PosAngleB=-45, PosAngleC=151.48, PointSymbolA=none,
        PointNameA=none, PointNameB=Y,
        PointNameC=H](0,0){a}(6.8,0){b}(2.7;61.48){c}
      \end{pspicture}
    \end{center}

    \par\dotfill{}
    Les points $Z$,~ $Y$,~ $V$ et $Z$, $H$, $B$ sont alignés et les droites
    $(VB)$ et $(YH)$ sont parallèles.\par
    D'après le \textbf{théorème de Thalès} :
    $\qquad\mathbf{\cfrac{ZV}{ZY}=\cfrac{ZB}{ZH}=\cfrac{VB}{YH}}$

    \vspace{1ex}\par
    De plus $ZV=YV+ZY=\unit[11{,}3]{cm}$
    \[\frac{11{,}3}{6{,}8}=\frac{ZB}{2{,}7}=\frac{VB}{6}\]
    $\cfrac{11{,}3}{6{,}8}=\cfrac{ZB}{2{,}7}\quad$ donc
    $\quad\boxed{ZB=\cfrac{2{,}7\times
        11{,}3}{6{,}8}\simeq\unit[4{,}49]{cm}}$\par
    $\cfrac{11{,}3}{6{,}8}=\cfrac{VB}{6}\quad$ donc
    $\quad\boxed{VB=\cfrac{6\times 11{,}3}{6{,}8}\simeq\unit[9{,}97]{cm}}$\par

    \columnbreak
    Sur la figure ci-dessous, les droites $(WS)\text{ et }(OX)$ sont
    parallèles.\par
    On donne $FW=\unit[2{,}7]{cm},\quad FS=\unit[4{,}9]{cm}, \quad
    WS=\unit[3{,}7]{cm}\quad\text{et}\quad OX~=~\unit[2{,}3]{cm}$.\par
    Calculer $FO$ et $FX$, arrondies au millième.
    \begin{center}
      \psset{PointSymbol=none,unit=0.48}
      \begin{pspicture}(-3.52,-3.76)(4.76,4.15)
        \SpecialCoor
        \pstTriangle[PosAngleA=135, PosAngleB=-45, PosAngleC=93.27,
        PointNameA=F, PointNameB=W,
        PointNameC=S](0,0){a}(2.7,0){b}(4.9;48.27){c}
        \pstTriangle[PosAngleB=135, PosAngleC=228.27, PointSymbolA=none,
        PointNameA=none, PointNameB=O,
        PointNameC=X](0,0){a}(-1.67,0){b}(-3.04;48.27){c}
      \end{pspicture}
    \end{center}

    \par\dotfill{}
    Les points $F$,~ $O$,~ $W$ et $F$, $X$, $S$ sont alignés et les droites
    $(WS)$ et $(OX)$ sont parallèles.\par
    D'après le \textbf{théorème de Thalès} :
    $\qquad\mathbf{\cfrac{FW}{FO}=\cfrac{FS}{FX}=\cfrac{WS}{OX}}$


    \[\frac{2{,}7}{FO}=\frac{4{,}9}{FX}=\frac{3{,}7}{2{,}3}\]
    $\cfrac{3{,}7}{2{,}3}=\cfrac{2{,}7}{FO}\quad$ donc
    $\quad\boxed{FO=\cfrac{2{,}7\times
        2{,}3}{3{,}7}\simeq\unit[1{,}678]{cm}}$\par
    $\cfrac{3{,}7}{2{,}3}=\cfrac{4{,}9}{FX}\quad$ donc
    $\quad\boxed{FX=\cfrac{4{,}9\times
        2{,}3}{3{,}7}\simeq\unit[3{,}046]{cm}}$\par

  \end{multicols}

  \exercice*
  \begin{multicols}{2}
    \begin{enumerate}
    \item $QLZ$ est un triangle rectangle en $L$ tel que :\par
      $LZ=\unit[7{,}6]{cm}$ et $QZ=\unit[11{,}1]{cm}$.\par
      Calculer la mesure de l'angle $\widehat{LQZ}$, arrondie au centième.\par
      \dotfill{}\par\vspace{2ex}
      Dans le triangle $QLZ$ rectangle en $L$,
      \[\sin\widehat{LQZ}=\cfrac{LZ}{QZ}\]
      \[ \sin\widehat{LQZ}=\cfrac{7{,}6}{11{,}1}\]
      \[ \boxed{\widehat{LQZ}=\sin^{-1}\left(\cfrac{7{,}6}{11.1}\right)
          \simeq43{,}21\degres} \]
      \columnbreak
    \item $TCJ$ est un triangle rectangle en $T$ tel que :\par
      $TC=\unit[7{,}5]{cm}$ et $\widehat{TCJ}=49\degres$.\par
      Calculer la longueur $CJ$, arrondie au centième.\par
      \dotfill{}\par\vspace{2ex}
      Dans le triangle $TCJ$ rectangle en $T$,
      \[\cos\widehat{TCJ}=\cfrac{TC}{CJ}\]
      \[ \cos49=\cfrac{7{,}5}{CJ}\]
      \[ \boxed{CJ=\cfrac{7{,}5}{\cos49} \simeq\unit[11{,}43]{cm}} \]
    \end{enumerate}
  \end{multicols}

  \exercice
  \begin{multicols}{2}
    \begin{enumerate}
    \item On donne
      $\begin{array}[t]{l}
        f:~x \longmapsto -9\,x^{2}-8\,x-3 \\ g:~x \longmapsto x-1
      \end{array}$
      \begin{enumerate}
      \item Quelle est l'image de $-3$ par la fonction $f$ ?
        \par $f\,(-3)=-9\times \left( -3\right) ^{2}-8\times \left( -3\right)
        -3$\par
        $f\,(-3)=-9\times 9-\left( -24\right) -3$\par
        $f\,(-3)=-81+24-3$\par
        $f\,(-3)=-57-3$
        \par
        \fbox{$f\,(-3)=-60$}\\

      \item Quelle est l'image de $3$ par la fonction $g$ ?
        \par $g\,(3)=3-1$\par

        \par
        \fbox{$g\,(3)=2$}\\

      \item Calculer $f\,(4)$.
        \par $f\,(4)=-9\times 4^{2}-8\times 4-3$\par
        $f\,(4)=-9\times 16-32-3$\par
        $f\,(4)=-144-32-3$
        \par
        \fbox{$f\,(4)=-179$}\\

      \item Calculer $g\,(-3)$.
        \par $g\,(-3)=-3-1$\par

        \par
        \fbox{$g\,(-3)=-4$}\\

      \end{enumerate}
    \item Voici un tableau de valeurs correspondant à une fonction $h$.\par
      \renewcommand{\arraystretch}{1.5}
      \begin{tabularx}{\linewidth}[t]{|c|*7{>{\centering}X|}}
        \hline
        $x$ & $-4$ & $-3$ & $-2$ & $-1$ & $0$ & $1$ & $2$ \tabularnewline \hline
        $h\,(x)$ & $2$ & $-4$ & $-1$ & $0$ & $-2$ & $-3$ & $1$ \tabularnewline
        \hline
      \end{tabularx} \medskip
      \begin{enumerate}
      \item L'image de $1$ par la fonction $h$ est $\mathbf{-3}$.
      \item $h\,(2)=\mathbf{1}$.
      \item Un antécédent de $-1$ par la fonction $h$ est $\mathbf{-2}$.
      \item $h\,(\mathbf{0})=-2$.
      \end{enumerate}
    \item Le graphique ci-après représente une fonction $k$ : \par
      \begin{center}
        \psset{unit=8mm, algebraic, dotsize=4pt 4}
        \begin{pspicture*}(-4.2,-4.2)(4.2,4.2)
          \psgrid[subgriddiv=2, gridwidth=.6pt,subgridcolor=lightgray,
          gridlabels=0pt]
          \psaxes[linewidth=1.2pt,]{->}(0,0)(-4.2,-4.2)(4.2,4.2)
          \psplot[plotpoints=200, linewidth=1.5pt,
          linecolor=DarkRed]{-4.2}{4.2}{2/15*x^4+13/30*x^3-2/15*x^2+17/30*x^1-1}
          \psdots (-4, 1) (-2, -4) (-1, -2) (0, -1) (1, 0)
          \psline[linestyle=dashed, linecolor=DarkRed](0, 1)(-4, 1)(-4, 0)
          \psline[linestyle=dashed, linecolor=DarkRed](0, -4)(-2, -4)(-2, 0)
          \psline[linestyle=dashed, linecolor=DarkRed](0, -1)(0, -1)(0, 0)
          \psline[linestyle=dashed, linecolor=DarkRed](0, 0)(1, 0)(1, 0)
        \end{pspicture*}
      \end{center}
      \begin{enumerate}
      \item Un antécédent de $0$ par la fonction $k$ est $\mathbf{1}$.
      \item $k\,(\mathbf{-4})=1$.
      \item $k\,(-2)=\mathbf{-4}$.
      \item L'image de $0$ par la fonction $k$ est $\mathbf{-1}$.
      \end{enumerate}
    \end{enumerate}
  \end{multicols}

  \exercice*
  \begin{enumerate}
  \item Les nombres \nombre{19034} et \nombre{3100} sont-ils premiers entre eux
    ?\par
    \nombre{19034} et \nombre{3100} sont deux nombres pairs donc ils sont
    divisibles par 2.\par
    \nombre{19034} et \nombre{3100} ne sont donc pas premiers entre eux
  \item Calculer le plus grand commun diviseur (\textsc{pgcd}) de
    \nombre{19034} et \nombre{3100}.\par
    On calcule le \textsc{pgcd} des nombres \nombre{19034} et \nombre{3100} en
    utilisant l'algorithme d'Euclide.
    \[ \nombre{19034}=\nombre{3100}\times\nombre{6}+\nombre{434}\]
    \[ \nombre{3100}=\nombre{434}\times\nombre{7}+\nombre{62}\]
    \[ \nombre{434}=\nombre{62}\times\nombre{7}+\nombre{0}\]
    \fbox{Donc le \textsc{pgcd} de \nombre{19034} et \nombre{3100} est 62}.

  \item Simplifier la fraction $\cfrac{\nombre{19034}}{\nombre{3100}}$ pour la
    rendre irréductible en indiquant la méthode.
    \begin{align*}
      \cfrac{\nombre{19034}}{\nombre{3100}} &=
      \cfrac{\nombre{19034}\div62}{\nombre{3100}\div62}\\
      &= \boxed{\cfrac{\nombre{307}}{\nombre{50}}}
    \end{align*}
  \end{enumerate}

  \exercice*
  Dans une urne, il y a 1 boule verte (V), 2 boules bleues (B) et 3 boules
  rouges (R), indiscernables au toucher. On tire successivement et sans remise
  deux boules.
  \begin{enumerate}
  \item Quelle est la probabilité de tirer une boule bleue au premier
    tirage?\par
    Il y a 6 boules dans l'urne dont 2 boules bleues. \par
    La probabilité de tirer une boule bleue au premier tirage est donc
    $\dfrac{2}{6}$.
  \item Construire un arbre des probabilités décrivant l'expérience
    aléatoire.\\ [0,3cm]
    \psset{unit=1 mm}
    \psset{linewidth=0.3,dotsep=1,hatchwidth=0.3,hatchsep=1.5,shadowsize=1,dimen=middle}
    \psset{dotsize=0.7 2.5,dotscale=1 1,fillcolor=black}
    \psset{arrowsize=1 2,arrowlength=1,arrowinset=0.25,tbarsize=0.7
      5,bracketlength=0.15,rbracketlength=0.15}
    \begin{pspicture}(0,0)(80,53)
      \psline(0,3)(10,23)
      \psline(10,3)(10,23)
      \psline(20,3)(10,23)
      \psline(30,3)(40,23)
      \psline(40,3)(40,23)
      \psline(50,3)(40,23)
      \psline(60,3)(70,23)
      \psline(80,3)(70,23)
      \psline(70,3)(70,23)
      \psline(15,28)(40,53)
      \psline(40,53)(65,28)
      \psline(40,53)(40,28)
      \rput(20,40){$\dfrac{1}{6}$} \rput(37,40){$\dfrac{2}{6}$}
      \rput(60,40){$\dfrac{3}{6}$}
      \rput(10,26){V} \rput(40,26){B} \rput(70,26){R}
      \rput(0,10){$\dfrac{0}{5}$} \rput(7,10){$\dfrac{2}{5}$}
      \rput(20,10){$\dfrac{3}{5}$}
      \rput(0,0){V} \rput(10,0){B} \rput(20,0){R}
      \rput(30,10){$\dfrac{1}{5}$} \rput(37,10){$\dfrac{1}{5}$}
      \rput(50,10){$\dfrac{3}{5}$}
      \rput(30,0){V} \rput(40,0){B} \rput(50,0){R}
      \rput(60,10){$\dfrac{1}{5}$} \rput(67,10){$\dfrac{2}{5}$}
      \rput(80,10){$\dfrac{2}{5}$}
      \rput(60,0){V} \rput(70,0){B} \rput(80,0){R}
    \end{pspicture}
    \vspace{0.3cm}
  \item Quelle est la probabilité que la première boule soit rouge et la
    deuxième soit bleue?\par
    On utilise l'arbre construit précédemment.\par
    $p(R,B)=\dfrac{3}{6} \times \dfrac{2}{5} = \dfrac{6}{30}$\par
    La probabilité que la première boule soit rouge et la deuxième soit bleue
    est égale à $\dfrac{6}{30}$.
  \item Quelle est la probabilité que la deuxième boule soit verte ?\par
    On note (?, V) l'évènement: la deuxième boule tirée est verte. \par
    $p(?,V)=p(V,V)+p(B,V)+p(R,V,)=\dfrac{1}{6}\times
    \dfrac{0}{5}+\dfrac{2}{6}\times \dfrac{1}{5}+\dfrac{3}{6}\times
    \dfrac{1}{5}=\dfrac{5}{30}$
  \end{enumerate}

  \exercice*
  Calculer les expressions suivantes et donner l'écriture scientifique du
  résultat.
  \begin{multicols}{2}\noindent
    \[ \thenocalcul = \cfrac{\nombre{3,2} \times 10^{-4} \times \nombre{30}
        \times 10^{3}}{\nombre{24} \times \big( 10^{-2} \big) ^2}\]
    \[ \thenocalcul = \cfrac{\nombre{3,2} \times \nombre{30}}{\nombre{24}}
      \times \cfrac{10^{-4+3}}{10^{-2 \times 2}}\]
    \[ \thenocalcul = \nombre{4} \times 10^{-1-(-4)}\]
    \[ \boxed{\thenocalcul = \nombre{4}  \times 10^{3}} \]
    \columnbreak\stepcounter{nocalcul}%
    \[ \thenocalcul = \cfrac{\nombre{10} \times 10^{-5} \times \nombre{810}
        \times 10^{-7}}{\nombre{0,9} \times \big( 10^{9} \big) ^3}\]
    \[ \thenocalcul = \cfrac{\nombre{10} \times \nombre{810}}{\nombre{0,9}}
      \times \cfrac{10^{-5+(-7)}}{10^{9 \times 3}}\]
    \[ \thenocalcul = \nombre{9000} \times 10^{-12-27}\]
    \[ \thenocalcul = \nombre{9}  \times 10^{3} \times 10^{-39}\]
    \[ \boxed{\thenocalcul = \nombre{9}  \times 10^{-36}} \]
  \end{multicols}

  \exercice*
  \begin{enumerate}
  \item Calculer les expressions suivantes et donner le résultat sous la forme
    $a\,\sqrt{b}$ avec $a$ et $b$ entiers, $b$ le plus petit possible.
    \begin{multicols}{2}\noindent
      \[ \thenocalcul = 2\,\sqrt{28}+4\,\sqrt{112}+3\,\sqrt{63}\]
      \[ \thenocalcul =
        2\,\sqrt{4}\times\sqrt{7}+4\,\sqrt{16}\times\sqrt{7}+3\,\sqrt{9}\times\sqrt{7}\]
      \[ \thenocalcul =
        2\times2\times\sqrt{7}+4\times4\times\sqrt{7}+3\times3\times\sqrt{7}\]
      \[ \thenocalcul = 4\,\sqrt{7}+16\,\sqrt{7}+9\,\sqrt{7}\]
      \[ \boxed{\thenocalcul = 29\,\sqrt{7}} \]
      \columnbreak\stepcounter{nocalcul}%
      \[ \thenocalcul = \sqrt{80}\times\sqrt{45}\times\sqrt{20}\]
      \[ \thenocalcul =
        \sqrt{16}\times\sqrt{5}\times\sqrt{9}\times\sqrt{5}\times\sqrt{4}\times\sqrt{5}\]
      \[ \thenocalcul =
        4\times\sqrt{5}\times3\times\sqrt{5}\times2\times\sqrt{5}\]
      \[ \thenocalcul = 24\times\left(\sqrt{5}\right)^2\times\sqrt{5}\]
      \[ \thenocalcul = 24\times5\times\sqrt{5}\]
      \[ \boxed{\thenocalcul = 120\,\sqrt{5}} \]
    \end{multicols}
  \item Calculer les expressions suivantes et donner le résultat sous la forme
    $a+b\,\sqrt{c}$ avec $a$, $b$ et $c$ entiers.
    \stepcounter{nocalcul}%
    \begin{multicols}{2}\noindent
      \[ \thenocalcul = \left( 4\,\sqrt{10}-2\,\sqrt{6} \right)^2\]
      \[ \thenocalcul = \left(
        4\,\sqrt{10}\right)^2-2\times4\,\sqrt{10}\times2\,\sqrt{6}+\left(
        2\,\sqrt{6}\right)^2\]
      \[ \thenocalcul = 16\times 10 -16\,\sqrt{60}+4\times 6\]
      \[ \boxed{\thenocalcul = 184-16\,\sqrt{60}} \]
      \columnbreak\stepcounter{nocalcul}%
      \[ \thenocalcul = \left( 2\,\sqrt{5}+4\,\sqrt{3} \right)^2\]
      \[ \thenocalcul = \left(
        2\,\sqrt{5}\right)^2+2\times2\,\sqrt{5}\times4\,\sqrt{3}+\left(
        4\,\sqrt{3}\right)^2\]
      \[ \thenocalcul = 4\times 5 +16\,\sqrt{15}+16\times 3\]
      \[ \boxed{\thenocalcul = 68+16\,\sqrt{15}} \]
    \end{multicols}
  \item Calculer les expressions suivantes et donner le résultat sous la forme
    d'un nombre entier.

    \stepcounter{nocalcul}%
    \begin{multicols}{2}\noindent
      \[ \thenocalcul = \left( 3-3\,\sqrt{3} \right)\left( 3+3\,\sqrt{3}
        \right)\]
      \[ \thenocalcul = 3^2-\left( 3\,\sqrt{3}\right)^2\]
      \[ \thenocalcul = 9-9\times 3\]
      \[ \boxed{\thenocalcul = -18} \]
      \columnbreak\stepcounter{nocalcul}%
      \[ \thenocalcul = \frac{32\,\sqrt{45}}{12\,\sqrt{80}}\]
      \[ \thenocalcul =
        \frac{32\times\sqrt{9}\times\cancel{\sqrt{5}}}{12\times\sqrt{16}\times\cancel{\sqrt{5}}}\]
      \[ \thenocalcul = \frac{32\times 3}{12\times 4}\]
      \[ \boxed{\thenocalcul = 2} \]
    \end{multicols}
  \end{enumerate}

  \exercice*
  Résoudre le système d'équations suivant :
  $\left\lbrace
  \begin{array}{rcrcll}
    7\,x & + & 4\,y & = & 17 & \qquad\hbox{\footnotesize$\mathit{(\times 2)}$}
    \\
    4\,x & + & 8\,y & = & 4 & \qquad\hbox{\footnotesize$\mathit{(\times \left(
        -1\right))}$}
  \end{array}
  \right.$
  \vspace{2ex}
  \begin{multicols}{2}\noindent

    \[ \left\lbrace
      \begin{array}{rcrcl}
        14\,x & + & 8\,y & = & 34 \\
        -4\,x & - & 8\,y & = & -4
      \end{array}
      \right.\quad\text{\footnotesize On ajoute les deux lignes}\]
    \[ 14\,x\cancel{+8\,y}-4\,x\cancel{-8\,y}=34-4\]
    \[ 10\,x=30\]
    \[ \boxed{x=\frac{30}{10}=3} \]
    \columnbreak\par
    $7\,x+4\,y=17\quad\text{et}\quad x=3\quad\text{donc :}$
    \[7\times 3 +4\,y=17\]

    \[ 4\,y=17-21\]
    \[ \boxed{y=\frac{-4}{4}=-1} \]
  \end{multicols}
  \underline{La solution de ce système d'équations est $(x;~y)=(3;~-1)$.}\par
  {Vérification : $\left\lbrace
    \begin{array}{l}
      7\,\times 3 +4\,\times \left( -1\right)=21 -4=17 \\
      4\,\times 3 +8\,\times \left( -1\right)=12 -8=4
    \end{array}
    \right.$}
  \label{LastCorPage}
\end{document}