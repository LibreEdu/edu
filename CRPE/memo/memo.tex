\documentclass[10pt,french,twocolumn,landscape,a4paper]{article}
%\documentclass[10pt,french,twocolumn,landscape]{article}
\usepackage[cm]{fullpage}
\usepackage[utf8]{inputenc}
\usepackage{lmodern}
\usepackage{hyperref}
\usepackage{babel}

\begin{document}
\pagestyle{empty}

\paragraph{\href{http://eduscol.education.fr/cid73652/charte-de-la-laicite-a-l-ecole.html}{Charte de la laïcité à l’école (2013)}}
\begin{itemize}
\item La république est laïque :
	\begin{enumerate}
	\item République indivisible, laïque, démocratique et sociale
	\item Séparation des religions et de l’État
	\item Liberté de conscience : chacun est libre de croire ou de ne pas croire
	\item Liberté de chacun, l’égalité et la fraternité de tous
	\item La République assure dans les établissements scolaires le respect de chacun de ces principes
	\end{enumerate}
\item L’école est laïque :
	\begin{enumerate}
	\setcounter{enumi}{5}
	\item Protège de tout prosélytisme et de toute pression
	\item Culture commune et partagée
	\item Liberté d’expression
	\item Rejet de toutes les violences et de toutes les discriminations, l’égalité entre les filles et les garçons
	\item Il appartient à tous les personnels de transmettre aux élèves le sens et la valeur de la laïcité
	\item Les personnels ont un devoir de stricte neutralité
	\item Les enseignements sont laïques, aucun sujet n’est à priori exclu du questionnement scientifique et pédagogique
	\item Nul ne peut se prévaloir de son appartenance religieuse pour refuser de se conformer aux règles applicables dans l'École de la République
	\item Le port de signes ou tenues par lesquels les élèves manifestent ostensiblement une appartenance religieuse est interdit
	\item Les élèves contribuent à faire vivre la laïcité
	\end{enumerate}
\end{itemize}

\paragraph{\href{http://www.education.gouv.fr/cid2770/le-socle-commun-de-connaissances-et-de-competences.html}{Socle commun de connaissances et de compétences (2006)}}
\begin{enumerate}
\item La maitrise de la langue française
\item La pratique d’une langue vivante étrangère
\item Les principaux éléments de mathématiques et la culture scientifique et technologique
\item La maitrise des techniques usuelles de l’information et de la communication
\item La culture humaniste
\item Les compétences sociales et civiques
\item L’autonomie et l’initiative
\end{enumerate}

\paragraph{\href{http://www.legifrance.gouv.fr/jopdf//jopdf/2013/0718/joe_20130718_0004.pdf}{Compétences communes à tous les professeurs et personnels d’éducation (2013)}}
\begin{itemize}
\item Les professeurs et les personnels d'éducation, acteurs du service public d'éducation
	\begin{enumerate}
	\item Faire partager les valeurs de la République
	\item Inscrire son action dans le cadre des principes fondamentaux du système éducatif et dans le cadre règlementaire de l’école
	\end{enumerate}
\item Les professeurs et les personnels d'éducation, pédagogues et éducateurs au service de la réussite de tous les élèves
	\begin{enumerate}\setcounter{enumi}{2}
	\item Connaitre les élèves et les processus d’apprentissage
	\item Prendre en compte la diversité des élèves
	\item Accompagner les élèves dans leur parcours de formation
	\item Agir en éducateur responsable et selon des principes éthiques
	\item Maitriser la langue française à des fins de communication
	\item Utiliser une langue vivante étrangère dans les situations exigées par son métier
	\item Intégrer les éléments de la culture numérique nécessaires à l’exercice de son métier
	\end{enumerate}
\item Les professeurs et les personnels d'éducation, acteurs de la communauté éducative
	\begin{enumerate}\setcounter{enumi}{9}
	\item Coopérer au sein d’une équipe
	\item Contribuer à l’action de la communauté éducative
	\item Coopérer avec les parents d’élèves
	\item Coopérer avec les partenaires de l’école
	\item S’engager dans une démarche individuelle et collective de développement professionnel
	\end{enumerate}
\end{itemize}

\renewcommand{\theenumi}{P\arabic{enumi}}
\paragraph{\href{http://www.education.gouv.fr/pid25535/bulletin_officiel.html?cid_bo=73066}{Compétences communes à tous les professeurs (2013)}}
\begin{itemize}
\item Les professeurs, professionnels porteurs de savoirs et d'une culture commune
	\begin{enumerate}
	\item Maitriser les savoirs disciplinaires et leur didactique
	\item Maitriser la langue française dans le cadre de son enseignement
	\end{enumerate}
\item Les professeurs, praticiens experts des apprentissages
	\begin{enumerate}\setcounter{enumi}{9}
	\item Construire, mettre en œuvre et animer des situations d’enseignement et d’apprentissage prenant en compte la diversité des élèves
	\item Organiser et assurer un mode de fonctionnement du groupe favorisant l’apprentissage et la socialisation des élèves
	\item Évaluer les progrès et les acquisitions des élèves
	\end{enumerate}
\end{itemize}
\renewcommand{\theenumi}{\arabic{enumi}}

\paragraph{\href{http://www.education.gouv.fr/pid25535/bulletin_officiel.html&cid_bo=87834}{Socle commun de connaissances, de
compétences et de culture (2015-04-23)}}
\begin{enumerate}
\item Les langages pour penser et communiquer
	\begin{description}
	\item [Comprendre, s'exprimer en utilisant les langages des arts et du corps] Il s'exprime par des activités, physiques, sportives ou artistiques, impliquant le corps. Il apprend ainsi le contrôle et la maitrise de soi.
	\end{description}
\item Les méthodes et outils pour apprendre
\item La formation de la personne et du citoyen
	\begin{description}
	\item [Expression de la sensibilité et des opinions, respect des autres] Il exploite ses facultés intellectuelles et physiques en ayant confiance en sa capacité à réussir et à progresser.
	\end{description}
\item Les systèmes naturels et les systèmes techniques
	\begin{description}
	\item [Responsabilités individuelles et collectives] Il est conscient des enjeux de bien-être et de santé des pratiques alimentaires et physiques. Il observe les règles élémentaires de sécurité liées aux techniques et produits rencontrés dans la vie quotidienne.
	\end{description}
\item Les représentations du monde et l'activité humaine
	\begin{description}
	\item [Invention, élaboration, production] Il connait les contraintes et les libertés qui s'exercent dans le cadre des activités physiques et sportives ou artistiques personnelles et collectives. Il sait en tirer parti et gère son activité physique et sa production ou sa performance artistiques pour les améliorer, progresser et se perfectionner. Il cherche et utilise des techniques pertinentes, il construit des stratégies pour réaliser une performance sportive. Dans le cadre d'activités et de projets collectifs, il prend sa place dans le groupe en étant attentif aux autres pour coopérer ou s'affronter dans un cadre règlementé.
	\end{description}
\end{enumerate}

\paragraph{\href{http://www.education.gouv.fr/cid87300/rentree-2015-le-nouveau-programme-de-l-ecole-maternelle.html}{Cycle des apprentissages premiers (2015-03-26)}}
\begin{enumerate}
\item Mobiliser le langage dans toutes ses dimensions
\item Agir, s'exprimer, comprendre à travers l'activité physique
	\begin{itemize}
	\item Agir dans l’espace, dans la durée et sur les objets
	\item Adapter ses équilibres et ses déplacements à des environnements ou des contraintes variés
	\item Communiquer avec les autres au travers d’actions à visée expressive ou artistique
	\item Collaborer, coopérer, s’opposer
	\end{itemize}
\item Agir, s'exprimer, comprendre à travers les activités artistiques
\item Construire les premiers outils pour structurer sa pensée
\item Explorer le monde
\end{enumerate}

\paragraph{\href{http://www.education.gouv.fr/bo/2008/hs3/programme_CP_CE1.htm}{Cycle des apprentissages fondamentaux (2008)}}
\begin{enumerate}
\item Français
%	\begin{itemize}
%	\item Langage oral
%	\item Lecture
%	\item Écriture
%	\item Vocabulaire
%	\item Grammaire
%	\item Orthographe
%	\end{itemize}
\item Mathématiques
%	\begin{itemize}
%	\item Nombres et calcul
%	\item Géométrie
%	\item Grandeurs et mesures
%	\item Organisation et gestion des données
%	\end{itemize}
\item Éducation physique et sportive
	\begin{itemize}
	\item Réaliser une performance
	\item Adapter ses déplacements à différents types d’environnement
	\item Coopérer et s’opposer individuellement et collectivement
	\item Concevoir et réaliser des actions à visées expressive, artistique, esthétique
	\end{itemize}
\item Langue vivante
\item Découverte du monde
\item Instruction civique et morale
\end{enumerate}

\paragraph{\href{http://www.education.gouv.fr/bo/2008/hs3/programme_CE2_CM1_CM2.htm}{Cycle des approfondissements (2008)}}
\begin{enumerate}
\item Français
%	\begin{itemize}
%	\item Langage oral
%	\item Lecture
%	\item Littérature
%	\item Écriture
%	\item Rédaction
%	\item Vocabulaire
%	\item Grammaire
%	\item Orthographe
%	\end{itemize}
\item Mathématiques
%	\begin{itemize}
%	\item Nombres et calcul
%	\item Géométrie
%	\item Grandeurs et mesures
%	\item Organisation et gestion des données
%	\end{itemize}
\item Éducation physique et sportive
	\begin{itemize}
	\item Réaliser une performance mesurée (en distance, en temps)
	\item Adapter ses déplacements à différents types d’environnement
	\item Coopérer ou s’opposer individuellement et collectivement
	\item Concevoir et réaliser des actions à visées expressive, artistique, esthétique
	\end{itemize}
\item Langue vivante
\item Sciences expérimentales et technologie
\item Histoire
\item Géographie
\item Instruction civique et morale
\end{enumerate}

%\noindent Compétence :  2015)
\begin{description}
\item[Compétence] aptitude à mobiliser ses ressources (connaissances, capacités, attitudes) pour accomplir une tâche ou faire face à une situation complexe ou inédite. (Socle commun de connaissances, de compétences et de culture.
\end{description}

%\paragraph{\href{http://www.education.gouv.fr/bo/2008/hs3/programme_maternelle.htm}{
Cycle des apprentissages premiers (2008)%}}
%\begin{enumerate}
%\item S’approprier le langage
%\item Découvrir l’écrit
%%	\begin{itemize}
%%	\item Se familiariser avec l’écrit
%%	\item Se préparer à apprendre à lire et à écrire
%%	\end{itemize}
%\item Devenir élève
%\item
, agir et s’exprimer avec son corps :
	\begin{itemize}
	\item Adapter ses déplacements à des environnements ou contraintes variés
	\item Coopérer et s’opposer individuellement ou collectivement ; accepter les contraintes collectives
	\item S’exprimer sur un rythme musical ou non, avec un engin ou non ; exprimer des sentiments et des émotions par le geste et le déplacement
	\item Se repérer et se déplacer dans l’espace
	\item Décrire ou représenter un parcours simple
	\end{itemize}
%\item Découvrir le monde
%\item Percevoir, sentir, imaginer, créer
%\end{enumerate}


\end{document}
